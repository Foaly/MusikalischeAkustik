\chapter{Berechnung der Tonsysteme}

\begin{enumerate}[a)]
\item

\item
\textbf{Berechnungsvorschriften} \\

\underline{Gleichstufig:}
\\
\begin{align*}
    f(n) &= a_n \cdot f_0 \\ 
    a_n &= \sqrt[12]{2}^n \\
    n &= 0, \,... \, ,127
\end{align*} 
\underline{Pythagoräisch:}
\\
\\
Zunächst werden 12 Faktoren berechnet, die den Multiplikatoren für die Referenzfrequenz entsprechen um die erste Oktave zu erhalten.
Die Faktoren ergeben sich aus Aufeinanderschichtung reiner Quinten und anschließender gegebenfalls mehrfacher Oktavierung nach unten.
\\
Von C bis Fis wandert man im Uhrzeigersinn den Quintenzirkel entlang:

\begin{align*}
    a_n &= \frac{1.5^n}{2^m} \\
    n &= 0,  \,... \, ,6
\end{align*} 
$m$ gibt an wie oft man durch 2 teilen muss um die durch Quintschichtung erhaltenen Frequenz auf die erste Oktave zurückzuführen, d.h. ein $a_n$ zwischen 1 und 2 zu erhalten.
\\

Für die restlichen 5 Faktoren, wandert man entgegen dem Uhrzeigersinn den Quintenzirkel entlang, erniedrigt jeweils um eine reine Quinte und oktaviert gegegebenfalls mehrfach, diesmal nach oben.

\begin{align*}
    a_n &= 1.5^{-(n-6)} \cdot 2^m \\
    n &= 7, \, ... \, ,11 
\end{align*}

Die Faktoren $a_n$ werden nun in aufsteigender Reihenfolge sortiert. Die Frequenzen können dann folgendermaßen berechnet werden :

\[
    f_n = 
\begin{cases}
    a_n \cdot f_0,& \text{if } n\leq 11 \\
    f_{n-12},&\text{if } n> 11
\end{cases}  
\]





\end{enumerate}