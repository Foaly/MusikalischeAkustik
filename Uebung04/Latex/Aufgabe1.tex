\section{Partialtonverläufe}
\label{sec:1}

\subsection{}
Die Funktion \texttt{loadRecordings()} liest die Datensätze ein und stellt sie zur weiteren Verarbeitung zur Verfügung.
\files{main.py}

\subsection{}
Der Verlauf der Amplitude für die ersten fünf Partialtöne beider Aufnahmen ist Abbildung \ref{fig:ampl} zu sehen.

\paragraph{BuK\_04}
\label{par:04}
Diese Aufnahme teilt sich in zwei signifikante Bereiche auf;
Einen leisen Linken Teil der bei Frame 1279 abrupt in einen deutlich lauteren rechten Teil wechselt.
Der Amplitudenverlauf des linken Teils ist linear wohingegen der rechte Teil eine Dynamikbogen, von leise zu laut und wieder zurück, aufweist.

\paragraph{BuK\_23}
Auch in dieser Aufnahme sind zwei Bereiche erkennbar, die bei Frame 1279 wechseln.
Im Gegensatz zur Aufnahme \ref{par:04} ist in dieser Aufnahme eine Oszillation der Amplitude erkennbar.
Auffällig ist, dass 

\begin{figure}[tbh]
    \centering
    \begin{subfigure}{.5\textwidth}
        \centering
        \caption{BuK\_04}
        \scalebox{0.5}{%% Creator: Matplotlib, PGF backend
%%
%% To include the figure in your LaTeX document, write
%%   \input{<filename>.pgf}
%%
%% Make sure the required packages are loaded in your preamble
%%   \usepackage{pgf}
%%
%% Figures using additional raster images can only be included by \input if
%% they are in the same directory as the main LaTeX file. For loading figures
%% from other directories you can use the `import` package
%%   \usepackage{import}
%% and then include the figures with
%%   \import{<path to file>}{<filename>.pgf}
%%
%% Matplotlib used the following preamble
%%   \usepackage{fontspec}
%%   \setmainfont{DejaVu Serif}
%%   \setsansfont{DejaVu Sans}
%%   \setmonofont{DejaVu Sans Mono}
%%
\begingroup%
\makeatletter%
\begin{pgfpicture}%
\pgfpathrectangle{\pgfpointorigin}{\pgfqpoint{6.400000in}{4.800000in}}%
\pgfusepath{use as bounding box, clip}%
\begin{pgfscope}%
\pgfsetbuttcap%
\pgfsetmiterjoin%
\definecolor{currentfill}{rgb}{1.000000,1.000000,1.000000}%
\pgfsetfillcolor{currentfill}%
\pgfsetlinewidth{0.000000pt}%
\definecolor{currentstroke}{rgb}{1.000000,1.000000,1.000000}%
\pgfsetstrokecolor{currentstroke}%
\pgfsetdash{}{0pt}%
\pgfpathmoveto{\pgfqpoint{0.000000in}{0.000000in}}%
\pgfpathlineto{\pgfqpoint{6.400000in}{0.000000in}}%
\pgfpathlineto{\pgfqpoint{6.400000in}{4.800000in}}%
\pgfpathlineto{\pgfqpoint{0.000000in}{4.800000in}}%
\pgfpathclose%
\pgfusepath{fill}%
\end{pgfscope}%
\begin{pgfscope}%
\pgfsetbuttcap%
\pgfsetmiterjoin%
\definecolor{currentfill}{rgb}{1.000000,1.000000,1.000000}%
\pgfsetfillcolor{currentfill}%
\pgfsetlinewidth{0.000000pt}%
\definecolor{currentstroke}{rgb}{0.000000,0.000000,0.000000}%
\pgfsetstrokecolor{currentstroke}%
\pgfsetstrokeopacity{0.000000}%
\pgfsetdash{}{0pt}%
\pgfpathmoveto{\pgfqpoint{0.800000in}{0.528000in}}%
\pgfpathlineto{\pgfqpoint{5.760000in}{0.528000in}}%
\pgfpathlineto{\pgfqpoint{5.760000in}{4.224000in}}%
\pgfpathlineto{\pgfqpoint{0.800000in}{4.224000in}}%
\pgfpathclose%
\pgfusepath{fill}%
\end{pgfscope}%
\begin{pgfscope}%
\pgfsetbuttcap%
\pgfsetroundjoin%
\definecolor{currentfill}{rgb}{0.000000,0.000000,0.000000}%
\pgfsetfillcolor{currentfill}%
\pgfsetlinewidth{0.803000pt}%
\definecolor{currentstroke}{rgb}{0.000000,0.000000,0.000000}%
\pgfsetstrokecolor{currentstroke}%
\pgfsetdash{}{0pt}%
\pgfsys@defobject{currentmarker}{\pgfqpoint{0.000000in}{-0.048611in}}{\pgfqpoint{0.000000in}{0.000000in}}{%
\pgfpathmoveto{\pgfqpoint{0.000000in}{0.000000in}}%
\pgfpathlineto{\pgfqpoint{0.000000in}{-0.048611in}}%
\pgfusepath{stroke,fill}%
}%
\begin{pgfscope}%
\pgfsys@transformshift{1.025455in}{0.528000in}%
\pgfsys@useobject{currentmarker}{}%
\end{pgfscope}%
\end{pgfscope}%
\begin{pgfscope}%
\pgftext[x=1.025455in,y=0.430778in,,top]{\sffamily\fontsize{10.000000}{12.000000}\selectfont 0}%
\end{pgfscope}%
\begin{pgfscope}%
\pgfsetbuttcap%
\pgfsetroundjoin%
\definecolor{currentfill}{rgb}{0.000000,0.000000,0.000000}%
\pgfsetfillcolor{currentfill}%
\pgfsetlinewidth{0.803000pt}%
\definecolor{currentstroke}{rgb}{0.000000,0.000000,0.000000}%
\pgfsetstrokecolor{currentstroke}%
\pgfsetdash{}{0pt}%
\pgfsys@defobject{currentmarker}{\pgfqpoint{0.000000in}{-0.048611in}}{\pgfqpoint{0.000000in}{0.000000in}}{%
\pgfpathmoveto{\pgfqpoint{0.000000in}{0.000000in}}%
\pgfpathlineto{\pgfqpoint{0.000000in}{-0.048611in}}%
\pgfusepath{stroke,fill}%
}%
\begin{pgfscope}%
\pgfsys@transformshift{1.913071in}{0.528000in}%
\pgfsys@useobject{currentmarker}{}%
\end{pgfscope}%
\end{pgfscope}%
\begin{pgfscope}%
\pgftext[x=1.913071in,y=0.430778in,,top]{\sffamily\fontsize{10.000000}{12.000000}\selectfont 500}%
\end{pgfscope}%
\begin{pgfscope}%
\pgfsetbuttcap%
\pgfsetroundjoin%
\definecolor{currentfill}{rgb}{0.000000,0.000000,0.000000}%
\pgfsetfillcolor{currentfill}%
\pgfsetlinewidth{0.803000pt}%
\definecolor{currentstroke}{rgb}{0.000000,0.000000,0.000000}%
\pgfsetstrokecolor{currentstroke}%
\pgfsetdash{}{0pt}%
\pgfsys@defobject{currentmarker}{\pgfqpoint{0.000000in}{-0.048611in}}{\pgfqpoint{0.000000in}{0.000000in}}{%
\pgfpathmoveto{\pgfqpoint{0.000000in}{0.000000in}}%
\pgfpathlineto{\pgfqpoint{0.000000in}{-0.048611in}}%
\pgfusepath{stroke,fill}%
}%
\begin{pgfscope}%
\pgfsys@transformshift{2.800687in}{0.528000in}%
\pgfsys@useobject{currentmarker}{}%
\end{pgfscope}%
\end{pgfscope}%
\begin{pgfscope}%
\pgftext[x=2.800687in,y=0.430778in,,top]{\sffamily\fontsize{10.000000}{12.000000}\selectfont 1000}%
\end{pgfscope}%
\begin{pgfscope}%
\pgfsetbuttcap%
\pgfsetroundjoin%
\definecolor{currentfill}{rgb}{0.000000,0.000000,0.000000}%
\pgfsetfillcolor{currentfill}%
\pgfsetlinewidth{0.803000pt}%
\definecolor{currentstroke}{rgb}{0.000000,0.000000,0.000000}%
\pgfsetstrokecolor{currentstroke}%
\pgfsetdash{}{0pt}%
\pgfsys@defobject{currentmarker}{\pgfqpoint{0.000000in}{-0.048611in}}{\pgfqpoint{0.000000in}{0.000000in}}{%
\pgfpathmoveto{\pgfqpoint{0.000000in}{0.000000in}}%
\pgfpathlineto{\pgfqpoint{0.000000in}{-0.048611in}}%
\pgfusepath{stroke,fill}%
}%
\begin{pgfscope}%
\pgfsys@transformshift{3.688304in}{0.528000in}%
\pgfsys@useobject{currentmarker}{}%
\end{pgfscope}%
\end{pgfscope}%
\begin{pgfscope}%
\pgftext[x=3.688304in,y=0.430778in,,top]{\sffamily\fontsize{10.000000}{12.000000}\selectfont 1500}%
\end{pgfscope}%
\begin{pgfscope}%
\pgfsetbuttcap%
\pgfsetroundjoin%
\definecolor{currentfill}{rgb}{0.000000,0.000000,0.000000}%
\pgfsetfillcolor{currentfill}%
\pgfsetlinewidth{0.803000pt}%
\definecolor{currentstroke}{rgb}{0.000000,0.000000,0.000000}%
\pgfsetstrokecolor{currentstroke}%
\pgfsetdash{}{0pt}%
\pgfsys@defobject{currentmarker}{\pgfqpoint{0.000000in}{-0.048611in}}{\pgfqpoint{0.000000in}{0.000000in}}{%
\pgfpathmoveto{\pgfqpoint{0.000000in}{0.000000in}}%
\pgfpathlineto{\pgfqpoint{0.000000in}{-0.048611in}}%
\pgfusepath{stroke,fill}%
}%
\begin{pgfscope}%
\pgfsys@transformshift{4.575920in}{0.528000in}%
\pgfsys@useobject{currentmarker}{}%
\end{pgfscope}%
\end{pgfscope}%
\begin{pgfscope}%
\pgftext[x=4.575920in,y=0.430778in,,top]{\sffamily\fontsize{10.000000}{12.000000}\selectfont 2000}%
\end{pgfscope}%
\begin{pgfscope}%
\pgfsetbuttcap%
\pgfsetroundjoin%
\definecolor{currentfill}{rgb}{0.000000,0.000000,0.000000}%
\pgfsetfillcolor{currentfill}%
\pgfsetlinewidth{0.803000pt}%
\definecolor{currentstroke}{rgb}{0.000000,0.000000,0.000000}%
\pgfsetstrokecolor{currentstroke}%
\pgfsetdash{}{0pt}%
\pgfsys@defobject{currentmarker}{\pgfqpoint{0.000000in}{-0.048611in}}{\pgfqpoint{0.000000in}{0.000000in}}{%
\pgfpathmoveto{\pgfqpoint{0.000000in}{0.000000in}}%
\pgfpathlineto{\pgfqpoint{0.000000in}{-0.048611in}}%
\pgfusepath{stroke,fill}%
}%
\begin{pgfscope}%
\pgfsys@transformshift{5.463536in}{0.528000in}%
\pgfsys@useobject{currentmarker}{}%
\end{pgfscope}%
\end{pgfscope}%
\begin{pgfscope}%
\pgftext[x=5.463536in,y=0.430778in,,top]{\sffamily\fontsize{10.000000}{12.000000}\selectfont 2500}%
\end{pgfscope}%
\begin{pgfscope}%
\pgftext[x=3.280000in,y=0.240809in,,top]{\sffamily\fontsize{10.000000}{12.000000}\selectfont Frame}%
\end{pgfscope}%
\begin{pgfscope}%
\pgfsetbuttcap%
\pgfsetroundjoin%
\definecolor{currentfill}{rgb}{0.000000,0.000000,0.000000}%
\pgfsetfillcolor{currentfill}%
\pgfsetlinewidth{0.803000pt}%
\definecolor{currentstroke}{rgb}{0.000000,0.000000,0.000000}%
\pgfsetstrokecolor{currentstroke}%
\pgfsetdash{}{0pt}%
\pgfsys@defobject{currentmarker}{\pgfqpoint{-0.048611in}{0.000000in}}{\pgfqpoint{0.000000in}{0.000000in}}{%
\pgfpathmoveto{\pgfqpoint{0.000000in}{0.000000in}}%
\pgfpathlineto{\pgfqpoint{-0.048611in}{0.000000in}}%
\pgfusepath{stroke,fill}%
}%
\begin{pgfscope}%
\pgfsys@transformshift{0.800000in}{0.696000in}%
\pgfsys@useobject{currentmarker}{}%
\end{pgfscope}%
\end{pgfscope}%
\begin{pgfscope}%
\pgftext[x=0.393533in,y=0.643238in,left,base]{\sffamily\fontsize{10.000000}{12.000000}\selectfont 0.00}%
\end{pgfscope}%
\begin{pgfscope}%
\pgfsetbuttcap%
\pgfsetroundjoin%
\definecolor{currentfill}{rgb}{0.000000,0.000000,0.000000}%
\pgfsetfillcolor{currentfill}%
\pgfsetlinewidth{0.803000pt}%
\definecolor{currentstroke}{rgb}{0.000000,0.000000,0.000000}%
\pgfsetstrokecolor{currentstroke}%
\pgfsetdash{}{0pt}%
\pgfsys@defobject{currentmarker}{\pgfqpoint{-0.048611in}{0.000000in}}{\pgfqpoint{0.000000in}{0.000000in}}{%
\pgfpathmoveto{\pgfqpoint{0.000000in}{0.000000in}}%
\pgfpathlineto{\pgfqpoint{-0.048611in}{0.000000in}}%
\pgfusepath{stroke,fill}%
}%
\begin{pgfscope}%
\pgfsys@transformshift{0.800000in}{1.108927in}%
\pgfsys@useobject{currentmarker}{}%
\end{pgfscope}%
\end{pgfscope}%
\begin{pgfscope}%
\pgftext[x=0.393533in,y=1.056165in,left,base]{\sffamily\fontsize{10.000000}{12.000000}\selectfont 0.02}%
\end{pgfscope}%
\begin{pgfscope}%
\pgfsetbuttcap%
\pgfsetroundjoin%
\definecolor{currentfill}{rgb}{0.000000,0.000000,0.000000}%
\pgfsetfillcolor{currentfill}%
\pgfsetlinewidth{0.803000pt}%
\definecolor{currentstroke}{rgb}{0.000000,0.000000,0.000000}%
\pgfsetstrokecolor{currentstroke}%
\pgfsetdash{}{0pt}%
\pgfsys@defobject{currentmarker}{\pgfqpoint{-0.048611in}{0.000000in}}{\pgfqpoint{0.000000in}{0.000000in}}{%
\pgfpathmoveto{\pgfqpoint{0.000000in}{0.000000in}}%
\pgfpathlineto{\pgfqpoint{-0.048611in}{0.000000in}}%
\pgfusepath{stroke,fill}%
}%
\begin{pgfscope}%
\pgfsys@transformshift{0.800000in}{1.521853in}%
\pgfsys@useobject{currentmarker}{}%
\end{pgfscope}%
\end{pgfscope}%
\begin{pgfscope}%
\pgftext[x=0.393533in,y=1.469092in,left,base]{\sffamily\fontsize{10.000000}{12.000000}\selectfont 0.04}%
\end{pgfscope}%
\begin{pgfscope}%
\pgfsetbuttcap%
\pgfsetroundjoin%
\definecolor{currentfill}{rgb}{0.000000,0.000000,0.000000}%
\pgfsetfillcolor{currentfill}%
\pgfsetlinewidth{0.803000pt}%
\definecolor{currentstroke}{rgb}{0.000000,0.000000,0.000000}%
\pgfsetstrokecolor{currentstroke}%
\pgfsetdash{}{0pt}%
\pgfsys@defobject{currentmarker}{\pgfqpoint{-0.048611in}{0.000000in}}{\pgfqpoint{0.000000in}{0.000000in}}{%
\pgfpathmoveto{\pgfqpoint{0.000000in}{0.000000in}}%
\pgfpathlineto{\pgfqpoint{-0.048611in}{0.000000in}}%
\pgfusepath{stroke,fill}%
}%
\begin{pgfscope}%
\pgfsys@transformshift{0.800000in}{1.934780in}%
\pgfsys@useobject{currentmarker}{}%
\end{pgfscope}%
\end{pgfscope}%
\begin{pgfscope}%
\pgftext[x=0.393533in,y=1.882018in,left,base]{\sffamily\fontsize{10.000000}{12.000000}\selectfont 0.06}%
\end{pgfscope}%
\begin{pgfscope}%
\pgfsetbuttcap%
\pgfsetroundjoin%
\definecolor{currentfill}{rgb}{0.000000,0.000000,0.000000}%
\pgfsetfillcolor{currentfill}%
\pgfsetlinewidth{0.803000pt}%
\definecolor{currentstroke}{rgb}{0.000000,0.000000,0.000000}%
\pgfsetstrokecolor{currentstroke}%
\pgfsetdash{}{0pt}%
\pgfsys@defobject{currentmarker}{\pgfqpoint{-0.048611in}{0.000000in}}{\pgfqpoint{0.000000in}{0.000000in}}{%
\pgfpathmoveto{\pgfqpoint{0.000000in}{0.000000in}}%
\pgfpathlineto{\pgfqpoint{-0.048611in}{0.000000in}}%
\pgfusepath{stroke,fill}%
}%
\begin{pgfscope}%
\pgfsys@transformshift{0.800000in}{2.347706in}%
\pgfsys@useobject{currentmarker}{}%
\end{pgfscope}%
\end{pgfscope}%
\begin{pgfscope}%
\pgftext[x=0.393533in,y=2.294945in,left,base]{\sffamily\fontsize{10.000000}{12.000000}\selectfont 0.08}%
\end{pgfscope}%
\begin{pgfscope}%
\pgfsetbuttcap%
\pgfsetroundjoin%
\definecolor{currentfill}{rgb}{0.000000,0.000000,0.000000}%
\pgfsetfillcolor{currentfill}%
\pgfsetlinewidth{0.803000pt}%
\definecolor{currentstroke}{rgb}{0.000000,0.000000,0.000000}%
\pgfsetstrokecolor{currentstroke}%
\pgfsetdash{}{0pt}%
\pgfsys@defobject{currentmarker}{\pgfqpoint{-0.048611in}{0.000000in}}{\pgfqpoint{0.000000in}{0.000000in}}{%
\pgfpathmoveto{\pgfqpoint{0.000000in}{0.000000in}}%
\pgfpathlineto{\pgfqpoint{-0.048611in}{0.000000in}}%
\pgfusepath{stroke,fill}%
}%
\begin{pgfscope}%
\pgfsys@transformshift{0.800000in}{2.760633in}%
\pgfsys@useobject{currentmarker}{}%
\end{pgfscope}%
\end{pgfscope}%
\begin{pgfscope}%
\pgftext[x=0.393533in,y=2.707871in,left,base]{\sffamily\fontsize{10.000000}{12.000000}\selectfont 0.10}%
\end{pgfscope}%
\begin{pgfscope}%
\pgfsetbuttcap%
\pgfsetroundjoin%
\definecolor{currentfill}{rgb}{0.000000,0.000000,0.000000}%
\pgfsetfillcolor{currentfill}%
\pgfsetlinewidth{0.803000pt}%
\definecolor{currentstroke}{rgb}{0.000000,0.000000,0.000000}%
\pgfsetstrokecolor{currentstroke}%
\pgfsetdash{}{0pt}%
\pgfsys@defobject{currentmarker}{\pgfqpoint{-0.048611in}{0.000000in}}{\pgfqpoint{0.000000in}{0.000000in}}{%
\pgfpathmoveto{\pgfqpoint{0.000000in}{0.000000in}}%
\pgfpathlineto{\pgfqpoint{-0.048611in}{0.000000in}}%
\pgfusepath{stroke,fill}%
}%
\begin{pgfscope}%
\pgfsys@transformshift{0.800000in}{3.173559in}%
\pgfsys@useobject{currentmarker}{}%
\end{pgfscope}%
\end{pgfscope}%
\begin{pgfscope}%
\pgftext[x=0.393533in,y=3.120798in,left,base]{\sffamily\fontsize{10.000000}{12.000000}\selectfont 0.12}%
\end{pgfscope}%
\begin{pgfscope}%
\pgfsetbuttcap%
\pgfsetroundjoin%
\definecolor{currentfill}{rgb}{0.000000,0.000000,0.000000}%
\pgfsetfillcolor{currentfill}%
\pgfsetlinewidth{0.803000pt}%
\definecolor{currentstroke}{rgb}{0.000000,0.000000,0.000000}%
\pgfsetstrokecolor{currentstroke}%
\pgfsetdash{}{0pt}%
\pgfsys@defobject{currentmarker}{\pgfqpoint{-0.048611in}{0.000000in}}{\pgfqpoint{0.000000in}{0.000000in}}{%
\pgfpathmoveto{\pgfqpoint{0.000000in}{0.000000in}}%
\pgfpathlineto{\pgfqpoint{-0.048611in}{0.000000in}}%
\pgfusepath{stroke,fill}%
}%
\begin{pgfscope}%
\pgfsys@transformshift{0.800000in}{3.586486in}%
\pgfsys@useobject{currentmarker}{}%
\end{pgfscope}%
\end{pgfscope}%
\begin{pgfscope}%
\pgftext[x=0.393533in,y=3.533724in,left,base]{\sffamily\fontsize{10.000000}{12.000000}\selectfont 0.14}%
\end{pgfscope}%
\begin{pgfscope}%
\pgfsetbuttcap%
\pgfsetroundjoin%
\definecolor{currentfill}{rgb}{0.000000,0.000000,0.000000}%
\pgfsetfillcolor{currentfill}%
\pgfsetlinewidth{0.803000pt}%
\definecolor{currentstroke}{rgb}{0.000000,0.000000,0.000000}%
\pgfsetstrokecolor{currentstroke}%
\pgfsetdash{}{0pt}%
\pgfsys@defobject{currentmarker}{\pgfqpoint{-0.048611in}{0.000000in}}{\pgfqpoint{0.000000in}{0.000000in}}{%
\pgfpathmoveto{\pgfqpoint{0.000000in}{0.000000in}}%
\pgfpathlineto{\pgfqpoint{-0.048611in}{0.000000in}}%
\pgfusepath{stroke,fill}%
}%
\begin{pgfscope}%
\pgfsys@transformshift{0.800000in}{3.999413in}%
\pgfsys@useobject{currentmarker}{}%
\end{pgfscope}%
\end{pgfscope}%
\begin{pgfscope}%
\pgftext[x=0.393533in,y=3.946651in,left,base]{\sffamily\fontsize{10.000000}{12.000000}\selectfont 0.16}%
\end{pgfscope}%
\begin{pgfscope}%
\pgftext[x=0.337977in,y=2.376000in,,bottom,rotate=90.000000]{\sffamily\fontsize{10.000000}{12.000000}\selectfont Amplitude}%
\end{pgfscope}%
\begin{pgfscope}%
\pgfpathrectangle{\pgfqpoint{0.800000in}{0.528000in}}{\pgfqpoint{4.960000in}{3.696000in}} %
\pgfusepath{clip}%
\pgfsetrectcap%
\pgfsetroundjoin%
\pgfsetlinewidth{1.505625pt}%
\definecolor{currentstroke}{rgb}{0.121569,0.466667,0.705882}%
\pgfsetstrokecolor{currentstroke}%
\pgfsetdash{}{0pt}%
\pgfpathmoveto{\pgfqpoint{1.025455in}{0.696000in}}%
\pgfpathlineto{\pgfqpoint{1.121317in}{0.696000in}}%
\pgfpathlineto{\pgfqpoint{1.123092in}{0.738372in}}%
\pgfpathlineto{\pgfqpoint{1.124868in}{0.736399in}}%
\pgfpathlineto{\pgfqpoint{1.130193in}{0.738162in}}%
\pgfpathlineto{\pgfqpoint{1.137294in}{0.745300in}}%
\pgfpathlineto{\pgfqpoint{1.139069in}{0.743888in}}%
\pgfpathlineto{\pgfqpoint{1.144395in}{0.728113in}}%
\pgfpathlineto{\pgfqpoint{1.149721in}{0.713493in}}%
\pgfpathlineto{\pgfqpoint{1.153271in}{0.760935in}}%
\pgfpathlineto{\pgfqpoint{1.160372in}{0.823028in}}%
\pgfpathlineto{\pgfqpoint{1.163923in}{0.845515in}}%
\pgfpathlineto{\pgfqpoint{1.167473in}{0.854717in}}%
\pgfpathlineto{\pgfqpoint{1.169248in}{0.852463in}}%
\pgfpathlineto{\pgfqpoint{1.172799in}{0.832965in}}%
\pgfpathlineto{\pgfqpoint{1.174574in}{0.818737in}}%
\pgfpathlineto{\pgfqpoint{1.178125in}{0.811032in}}%
\pgfpathlineto{\pgfqpoint{1.179900in}{0.813171in}}%
\pgfpathlineto{\pgfqpoint{1.183450in}{0.835854in}}%
\pgfpathlineto{\pgfqpoint{1.187001in}{0.898687in}}%
\pgfpathlineto{\pgfqpoint{1.194102in}{1.066818in}}%
\pgfpathlineto{\pgfqpoint{1.195877in}{0.748391in}}%
\pgfpathlineto{\pgfqpoint{1.201203in}{0.753200in}}%
\pgfpathlineto{\pgfqpoint{1.204753in}{0.754687in}}%
\pgfpathlineto{\pgfqpoint{1.211854in}{0.755915in}}%
\pgfpathlineto{\pgfqpoint{1.220730in}{0.757797in}}%
\pgfpathlineto{\pgfqpoint{1.226056in}{0.760187in}}%
\pgfpathlineto{\pgfqpoint{1.231382in}{0.763217in}}%
\pgfpathlineto{\pgfqpoint{1.245583in}{0.772094in}}%
\pgfpathlineto{\pgfqpoint{1.250909in}{0.778629in}}%
\pgfpathlineto{\pgfqpoint{1.266886in}{0.800480in}}%
\pgfpathlineto{\pgfqpoint{1.279313in}{0.811829in}}%
\pgfpathlineto{\pgfqpoint{1.291739in}{0.814175in}}%
\pgfpathlineto{\pgfqpoint{1.304166in}{0.814173in}}%
\pgfpathlineto{\pgfqpoint{1.321918in}{0.817923in}}%
\pgfpathlineto{\pgfqpoint{1.330795in}{0.821940in}}%
\pgfpathlineto{\pgfqpoint{1.336120in}{0.824595in}}%
\pgfpathlineto{\pgfqpoint{1.339671in}{0.827385in}}%
\pgfpathlineto{\pgfqpoint{1.344996in}{0.833435in}}%
\pgfpathlineto{\pgfqpoint{1.359198in}{0.849542in}}%
\pgfpathlineto{\pgfqpoint{1.368074in}{0.856275in}}%
\pgfpathlineto{\pgfqpoint{1.400029in}{0.879323in}}%
\pgfpathlineto{\pgfqpoint{1.407130in}{0.884437in}}%
\pgfpathlineto{\pgfqpoint{1.423107in}{0.894515in}}%
\pgfpathlineto{\pgfqpoint{1.431983in}{0.894704in}}%
\pgfpathlineto{\pgfqpoint{1.460387in}{0.885800in}}%
\pgfpathlineto{\pgfqpoint{1.465712in}{0.885221in}}%
\pgfpathlineto{\pgfqpoint{1.471038in}{0.884527in}}%
\pgfpathlineto{\pgfqpoint{1.478139in}{0.884761in}}%
\pgfpathlineto{\pgfqpoint{1.485240in}{0.883944in}}%
\pgfpathlineto{\pgfqpoint{1.492341in}{0.882044in}}%
\pgfpathlineto{\pgfqpoint{1.497666in}{0.880822in}}%
\pgfpathlineto{\pgfqpoint{1.513644in}{0.879267in}}%
\pgfpathlineto{\pgfqpoint{1.518969in}{0.880151in}}%
\pgfpathlineto{\pgfqpoint{1.527845in}{0.884496in}}%
\pgfpathlineto{\pgfqpoint{1.538497in}{0.889821in}}%
\pgfpathlineto{\pgfqpoint{1.552699in}{0.893951in}}%
\pgfpathlineto{\pgfqpoint{1.568676in}{0.895244in}}%
\pgfpathlineto{\pgfqpoint{1.579327in}{0.896296in}}%
\pgfpathlineto{\pgfqpoint{1.588203in}{0.894702in}}%
\pgfpathlineto{\pgfqpoint{1.605956in}{0.890794in}}%
\pgfpathlineto{\pgfqpoint{1.616607in}{0.891268in}}%
\pgfpathlineto{\pgfqpoint{1.646786in}{0.894837in}}%
\pgfpathlineto{\pgfqpoint{1.657437in}{0.897815in}}%
\pgfpathlineto{\pgfqpoint{1.662763in}{0.899641in}}%
\pgfpathlineto{\pgfqpoint{1.669864in}{0.902925in}}%
\pgfpathlineto{\pgfqpoint{1.680515in}{0.906132in}}%
\pgfpathlineto{\pgfqpoint{1.685841in}{0.906826in}}%
\pgfpathlineto{\pgfqpoint{1.691167in}{0.904375in}}%
\pgfpathlineto{\pgfqpoint{1.698268in}{0.900892in}}%
\pgfpathlineto{\pgfqpoint{1.710694in}{0.894780in}}%
\pgfpathlineto{\pgfqpoint{1.716020in}{0.893551in}}%
\pgfpathlineto{\pgfqpoint{1.723121in}{0.894293in}}%
\pgfpathlineto{\pgfqpoint{1.733772in}{0.892311in}}%
\pgfpathlineto{\pgfqpoint{1.740873in}{0.887726in}}%
\pgfpathlineto{\pgfqpoint{1.753300in}{0.883072in}}%
\pgfpathlineto{\pgfqpoint{1.760401in}{0.885852in}}%
\pgfpathlineto{\pgfqpoint{1.778153in}{0.894408in}}%
\pgfpathlineto{\pgfqpoint{1.785254in}{0.891899in}}%
\pgfpathlineto{\pgfqpoint{1.803006in}{0.879427in}}%
\pgfpathlineto{\pgfqpoint{1.806557in}{0.879808in}}%
\pgfpathlineto{\pgfqpoint{1.818984in}{0.885580in}}%
\pgfpathlineto{\pgfqpoint{1.824309in}{0.887200in}}%
\pgfpathlineto{\pgfqpoint{1.834961in}{0.885219in}}%
\pgfpathlineto{\pgfqpoint{1.847387in}{0.880886in}}%
\pgfpathlineto{\pgfqpoint{1.854488in}{0.881720in}}%
\pgfpathlineto{\pgfqpoint{1.877566in}{0.882661in}}%
\pgfpathlineto{\pgfqpoint{1.884667in}{0.877224in}}%
\pgfpathlineto{\pgfqpoint{1.891768in}{0.869308in}}%
\pgfpathlineto{\pgfqpoint{1.900644in}{0.858542in}}%
\pgfpathlineto{\pgfqpoint{1.909520in}{0.851897in}}%
\pgfpathlineto{\pgfqpoint{1.916621in}{0.850196in}}%
\pgfpathlineto{\pgfqpoint{1.921947in}{0.852278in}}%
\pgfpathlineto{\pgfqpoint{1.932598in}{0.859041in}}%
\pgfpathlineto{\pgfqpoint{1.950351in}{0.863832in}}%
\pgfpathlineto{\pgfqpoint{1.964553in}{0.861691in}}%
\pgfpathlineto{\pgfqpoint{1.984080in}{0.872511in}}%
\pgfpathlineto{\pgfqpoint{2.001832in}{0.886258in}}%
\pgfpathlineto{\pgfqpoint{2.023135in}{0.893130in}}%
\pgfpathlineto{\pgfqpoint{2.030236in}{0.893129in}}%
\pgfpathlineto{\pgfqpoint{2.035562in}{0.892504in}}%
\pgfpathlineto{\pgfqpoint{2.049764in}{0.887006in}}%
\pgfpathlineto{\pgfqpoint{2.065741in}{0.879257in}}%
\pgfpathlineto{\pgfqpoint{2.072842in}{0.878538in}}%
\pgfpathlineto{\pgfqpoint{2.085268in}{0.883968in}}%
\pgfpathlineto{\pgfqpoint{2.094145in}{0.891114in}}%
\pgfpathlineto{\pgfqpoint{2.099470in}{0.892435in}}%
\pgfpathlineto{\pgfqpoint{2.103021in}{0.891702in}}%
\pgfpathlineto{\pgfqpoint{2.111897in}{0.887349in}}%
\pgfpathlineto{\pgfqpoint{2.118998in}{0.884969in}}%
\pgfpathlineto{\pgfqpoint{2.126099in}{0.885415in}}%
\pgfpathlineto{\pgfqpoint{2.131424in}{0.885871in}}%
\pgfpathlineto{\pgfqpoint{2.136750in}{0.885582in}}%
\pgfpathlineto{\pgfqpoint{2.147402in}{0.887218in}}%
\pgfpathlineto{\pgfqpoint{2.154503in}{0.886593in}}%
\pgfpathlineto{\pgfqpoint{2.159828in}{0.886204in}}%
\pgfpathlineto{\pgfqpoint{2.165154in}{0.886891in}}%
\pgfpathlineto{\pgfqpoint{2.170480in}{0.887190in}}%
\pgfpathlineto{\pgfqpoint{2.181131in}{0.890426in}}%
\pgfpathlineto{\pgfqpoint{2.186457in}{0.891960in}}%
\pgfpathlineto{\pgfqpoint{2.200659in}{0.894015in}}%
\pgfpathlineto{\pgfqpoint{2.205984in}{0.897032in}}%
\pgfpathlineto{\pgfqpoint{2.229062in}{0.913550in}}%
\pgfpathlineto{\pgfqpoint{2.234388in}{0.917321in}}%
\pgfpathlineto{\pgfqpoint{2.239714in}{0.918944in}}%
\pgfpathlineto{\pgfqpoint{2.264567in}{0.915850in}}%
\pgfpathlineto{\pgfqpoint{2.275218in}{0.917087in}}%
\pgfpathlineto{\pgfqpoint{2.285870in}{0.916601in}}%
\pgfpathlineto{\pgfqpoint{2.289420in}{0.917451in}}%
\pgfpathlineto{\pgfqpoint{2.292971in}{0.917001in}}%
\pgfpathlineto{\pgfqpoint{2.319599in}{0.910811in}}%
\pgfpathlineto{\pgfqpoint{2.328475in}{0.914592in}}%
\pgfpathlineto{\pgfqpoint{2.335576in}{0.921079in}}%
\pgfpathlineto{\pgfqpoint{2.342677in}{0.923497in}}%
\pgfpathlineto{\pgfqpoint{2.348003in}{0.926229in}}%
\pgfpathlineto{\pgfqpoint{2.360429in}{0.924410in}}%
\pgfpathlineto{\pgfqpoint{2.365755in}{0.922238in}}%
\pgfpathlineto{\pgfqpoint{2.376407in}{0.919084in}}%
\pgfpathlineto{\pgfqpoint{2.385283in}{0.921026in}}%
\pgfpathlineto{\pgfqpoint{2.397709in}{0.926339in}}%
\pgfpathlineto{\pgfqpoint{2.410136in}{0.934361in}}%
\pgfpathlineto{\pgfqpoint{2.419012in}{0.938454in}}%
\pgfpathlineto{\pgfqpoint{2.422563in}{0.940245in}}%
\pgfpathlineto{\pgfqpoint{2.427888in}{0.942529in}}%
\pgfpathlineto{\pgfqpoint{2.434989in}{0.947420in}}%
\pgfpathlineto{\pgfqpoint{2.440315in}{0.951373in}}%
\pgfpathlineto{\pgfqpoint{2.445641in}{0.954527in}}%
\pgfpathlineto{\pgfqpoint{2.459843in}{0.963104in}}%
\pgfpathlineto{\pgfqpoint{2.468719in}{0.962722in}}%
\pgfpathlineto{\pgfqpoint{2.486471in}{0.958477in}}%
\pgfpathlineto{\pgfqpoint{2.491797in}{0.956586in}}%
\pgfpathlineto{\pgfqpoint{2.500673in}{0.957379in}}%
\pgfpathlineto{\pgfqpoint{2.513099in}{0.961016in}}%
\pgfpathlineto{\pgfqpoint{2.532627in}{0.966930in}}%
\pgfpathlineto{\pgfqpoint{2.537953in}{0.967110in}}%
\pgfpathlineto{\pgfqpoint{2.541503in}{0.966816in}}%
\pgfpathlineto{\pgfqpoint{2.550379in}{0.964438in}}%
\pgfpathlineto{\pgfqpoint{2.553930in}{0.963471in}}%
\pgfpathlineto{\pgfqpoint{2.568132in}{0.964618in}}%
\pgfpathlineto{\pgfqpoint{2.571682in}{0.964696in}}%
\pgfpathlineto{\pgfqpoint{2.577008in}{0.966424in}}%
\pgfpathlineto{\pgfqpoint{2.584109in}{0.969202in}}%
\pgfpathlineto{\pgfqpoint{2.589435in}{0.971035in}}%
\pgfpathlineto{\pgfqpoint{2.596535in}{0.972284in}}%
\pgfpathlineto{\pgfqpoint{2.600086in}{0.972004in}}%
\pgfpathlineto{\pgfqpoint{2.607187in}{0.970489in}}%
\pgfpathlineto{\pgfqpoint{2.623164in}{0.963284in}}%
\pgfpathlineto{\pgfqpoint{2.628490in}{0.960631in}}%
\pgfpathlineto{\pgfqpoint{2.632040in}{0.960749in}}%
\pgfpathlineto{\pgfqpoint{2.644467in}{0.965925in}}%
\pgfpathlineto{\pgfqpoint{2.663994in}{0.975178in}}%
\pgfpathlineto{\pgfqpoint{2.672870in}{0.973683in}}%
\pgfpathlineto{\pgfqpoint{2.678196in}{0.971656in}}%
\pgfpathlineto{\pgfqpoint{2.681747in}{0.971181in}}%
\pgfpathlineto{\pgfqpoint{2.687072in}{0.969755in}}%
\pgfpathlineto{\pgfqpoint{2.690623in}{0.969780in}}%
\pgfpathlineto{\pgfqpoint{2.694173in}{0.968921in}}%
\pgfpathlineto{\pgfqpoint{2.711926in}{0.958237in}}%
\pgfpathlineto{\pgfqpoint{2.720802in}{0.956943in}}%
\pgfpathlineto{\pgfqpoint{2.736779in}{0.953633in}}%
\pgfpathlineto{\pgfqpoint{2.747430in}{0.950729in}}%
\pgfpathlineto{\pgfqpoint{2.752756in}{0.954215in}}%
\pgfpathlineto{\pgfqpoint{2.765183in}{0.966820in}}%
\pgfpathlineto{\pgfqpoint{2.770508in}{0.972648in}}%
\pgfpathlineto{\pgfqpoint{2.777609in}{0.977824in}}%
\pgfpathlineto{\pgfqpoint{2.781160in}{0.978280in}}%
\pgfpathlineto{\pgfqpoint{2.788261in}{0.977387in}}%
\pgfpathlineto{\pgfqpoint{2.816664in}{0.952862in}}%
\pgfpathlineto{\pgfqpoint{2.825540in}{0.953987in}}%
\pgfpathlineto{\pgfqpoint{2.832641in}{0.957070in}}%
\pgfpathlineto{\pgfqpoint{2.845068in}{0.959133in}}%
\pgfpathlineto{\pgfqpoint{2.852169in}{0.958228in}}%
\pgfpathlineto{\pgfqpoint{2.861045in}{0.951370in}}%
\pgfpathlineto{\pgfqpoint{2.868146in}{0.945443in}}%
\pgfpathlineto{\pgfqpoint{2.873472in}{0.942838in}}%
\pgfpathlineto{\pgfqpoint{2.882348in}{0.939298in}}%
\pgfpathlineto{\pgfqpoint{2.887674in}{0.939630in}}%
\pgfpathlineto{\pgfqpoint{2.891224in}{0.941551in}}%
\pgfpathlineto{\pgfqpoint{2.898325in}{0.948073in}}%
\pgfpathlineto{\pgfqpoint{2.903651in}{0.953971in}}%
\pgfpathlineto{\pgfqpoint{2.908976in}{0.958012in}}%
\pgfpathlineto{\pgfqpoint{2.921403in}{0.963031in}}%
\pgfpathlineto{\pgfqpoint{2.926729in}{0.962858in}}%
\pgfpathlineto{\pgfqpoint{2.937380in}{0.957315in}}%
\pgfpathlineto{\pgfqpoint{2.948031in}{0.954237in}}%
\pgfpathlineto{\pgfqpoint{2.955132in}{0.953543in}}%
\pgfpathlineto{\pgfqpoint{2.965784in}{0.958236in}}%
\pgfpathlineto{\pgfqpoint{2.969334in}{0.960985in}}%
\pgfpathlineto{\pgfqpoint{2.983536in}{0.973446in}}%
\pgfpathlineto{\pgfqpoint{2.990637in}{0.975785in}}%
\pgfpathlineto{\pgfqpoint{2.994188in}{0.974870in}}%
\pgfpathlineto{\pgfqpoint{2.997738in}{0.972753in}}%
\pgfpathlineto{\pgfqpoint{3.003064in}{0.967667in}}%
\pgfpathlineto{\pgfqpoint{3.017266in}{0.956145in}}%
\pgfpathlineto{\pgfqpoint{3.020816in}{0.954627in}}%
\pgfpathlineto{\pgfqpoint{3.026142in}{0.955453in}}%
\pgfpathlineto{\pgfqpoint{3.029692in}{0.956447in}}%
\pgfpathlineto{\pgfqpoint{3.033243in}{0.959222in}}%
\pgfpathlineto{\pgfqpoint{3.038568in}{0.964798in}}%
\pgfpathlineto{\pgfqpoint{3.049220in}{0.968956in}}%
\pgfpathlineto{\pgfqpoint{3.061646in}{0.972252in}}%
\pgfpathlineto{\pgfqpoint{3.072298in}{0.970244in}}%
\pgfpathlineto{\pgfqpoint{3.079399in}{0.972234in}}%
\pgfpathlineto{\pgfqpoint{3.090050in}{0.981249in}}%
\pgfpathlineto{\pgfqpoint{3.098926in}{0.985769in}}%
\pgfpathlineto{\pgfqpoint{3.106027in}{0.985815in}}%
\pgfpathlineto{\pgfqpoint{3.107802in}{0.986567in}}%
\pgfpathlineto{\pgfqpoint{3.114903in}{0.984165in}}%
\pgfpathlineto{\pgfqpoint{3.116679in}{0.984318in}}%
\pgfpathlineto{\pgfqpoint{3.122004in}{0.982392in}}%
\pgfpathlineto{\pgfqpoint{3.125555in}{0.982263in}}%
\pgfpathlineto{\pgfqpoint{3.127330in}{0.982406in}}%
\pgfpathlineto{\pgfqpoint{3.137981in}{0.991116in}}%
\pgfpathlineto{\pgfqpoint{3.145082in}{0.995259in}}%
\pgfpathlineto{\pgfqpoint{3.148633in}{0.997097in}}%
\pgfpathlineto{\pgfqpoint{3.152183in}{0.996999in}}%
\pgfpathlineto{\pgfqpoint{3.164610in}{0.988085in}}%
\pgfpathlineto{\pgfqpoint{3.168160in}{0.985328in}}%
\pgfpathlineto{\pgfqpoint{3.177037in}{0.980470in}}%
\pgfpathlineto{\pgfqpoint{3.180587in}{0.980120in}}%
\pgfpathlineto{\pgfqpoint{3.191238in}{0.984910in}}%
\pgfpathlineto{\pgfqpoint{3.207215in}{0.991658in}}%
\pgfpathlineto{\pgfqpoint{3.223193in}{0.994123in}}%
\pgfpathlineto{\pgfqpoint{3.235619in}{1.007201in}}%
\pgfpathlineto{\pgfqpoint{3.240945in}{1.011562in}}%
\pgfpathlineto{\pgfqpoint{3.244495in}{1.012854in}}%
\pgfpathlineto{\pgfqpoint{3.248046in}{1.012559in}}%
\pgfpathlineto{\pgfqpoint{3.253372in}{1.008509in}}%
\pgfpathlineto{\pgfqpoint{3.256922in}{1.003574in}}%
\pgfpathlineto{\pgfqpoint{3.260472in}{0.995317in}}%
\pgfpathlineto{\pgfqpoint{3.269349in}{0.961690in}}%
\pgfpathlineto{\pgfqpoint{3.271124in}{0.952464in}}%
\pgfpathlineto{\pgfqpoint{3.272899in}{0.935645in}}%
\pgfpathlineto{\pgfqpoint{3.274674in}{0.935278in}}%
\pgfpathlineto{\pgfqpoint{3.276450in}{0.924155in}}%
\pgfpathlineto{\pgfqpoint{3.283550in}{0.974372in}}%
\pgfpathlineto{\pgfqpoint{3.285326in}{0.973969in}}%
\pgfpathlineto{\pgfqpoint{3.288876in}{0.939457in}}%
\pgfpathlineto{\pgfqpoint{3.295977in}{0.862977in}}%
\pgfpathlineto{\pgfqpoint{3.297752in}{0.864421in}}%
\pgfpathlineto{\pgfqpoint{3.299528in}{0.877905in}}%
\pgfpathlineto{\pgfqpoint{3.301303in}{0.915113in}}%
\pgfpathlineto{\pgfqpoint{3.304853in}{1.069258in}}%
\pgfpathlineto{\pgfqpoint{3.310179in}{1.453104in}}%
\pgfpathlineto{\pgfqpoint{3.324381in}{2.662718in}}%
\pgfpathlineto{\pgfqpoint{3.327931in}{2.822497in}}%
\pgfpathlineto{\pgfqpoint{3.331482in}{2.910129in}}%
\pgfpathlineto{\pgfqpoint{3.335032in}{2.940516in}}%
\pgfpathlineto{\pgfqpoint{3.336807in}{2.940194in}}%
\pgfpathlineto{\pgfqpoint{3.340358in}{2.921870in}}%
\pgfpathlineto{\pgfqpoint{3.345684in}{2.884191in}}%
\pgfpathlineto{\pgfqpoint{3.365211in}{2.833919in}}%
\pgfpathlineto{\pgfqpoint{3.368762in}{2.812804in}}%
\pgfpathlineto{\pgfqpoint{3.374087in}{2.760829in}}%
\pgfpathlineto{\pgfqpoint{3.381188in}{2.685291in}}%
\pgfpathlineto{\pgfqpoint{3.388289in}{2.633354in}}%
\pgfpathlineto{\pgfqpoint{3.393615in}{2.610223in}}%
\pgfpathlineto{\pgfqpoint{3.397165in}{2.603297in}}%
\pgfpathlineto{\pgfqpoint{3.398941in}{2.602310in}}%
\pgfpathlineto{\pgfqpoint{3.400716in}{2.605358in}}%
\pgfpathlineto{\pgfqpoint{3.409592in}{2.629478in}}%
\pgfpathlineto{\pgfqpoint{3.413142in}{2.632942in}}%
\pgfpathlineto{\pgfqpoint{3.416693in}{2.632142in}}%
\pgfpathlineto{\pgfqpoint{3.420243in}{2.628241in}}%
\pgfpathlineto{\pgfqpoint{3.423794in}{2.620902in}}%
\pgfpathlineto{\pgfqpoint{3.427344in}{2.608500in}}%
\pgfpathlineto{\pgfqpoint{3.432670in}{2.579349in}}%
\pgfpathlineto{\pgfqpoint{3.443321in}{2.510684in}}%
\pgfpathlineto{\pgfqpoint{3.446872in}{2.499210in}}%
\pgfpathlineto{\pgfqpoint{3.448647in}{2.497074in}}%
\pgfpathlineto{\pgfqpoint{3.450422in}{2.497428in}}%
\pgfpathlineto{\pgfqpoint{3.452198in}{2.500174in}}%
\pgfpathlineto{\pgfqpoint{3.455748in}{2.512704in}}%
\pgfpathlineto{\pgfqpoint{3.459298in}{2.534836in}}%
\pgfpathlineto{\pgfqpoint{3.464624in}{2.587775in}}%
\pgfpathlineto{\pgfqpoint{3.473500in}{2.710539in}}%
\pgfpathlineto{\pgfqpoint{3.484152in}{2.849808in}}%
\pgfpathlineto{\pgfqpoint{3.491253in}{2.915233in}}%
\pgfpathlineto{\pgfqpoint{3.501904in}{3.002583in}}%
\pgfpathlineto{\pgfqpoint{3.509005in}{3.083176in}}%
\pgfpathlineto{\pgfqpoint{3.521432in}{3.231997in}}%
\pgfpathlineto{\pgfqpoint{3.526757in}{3.270465in}}%
\pgfpathlineto{\pgfqpoint{3.530308in}{3.279781in}}%
\pgfpathlineto{\pgfqpoint{3.532083in}{3.279417in}}%
\pgfpathlineto{\pgfqpoint{3.533858in}{3.275947in}}%
\pgfpathlineto{\pgfqpoint{3.537409in}{3.261979in}}%
\pgfpathlineto{\pgfqpoint{3.544510in}{3.230642in}}%
\pgfpathlineto{\pgfqpoint{3.546285in}{3.226831in}}%
\pgfpathlineto{\pgfqpoint{3.548060in}{3.225991in}}%
\pgfpathlineto{\pgfqpoint{3.549835in}{3.228650in}}%
\pgfpathlineto{\pgfqpoint{3.553386in}{3.244802in}}%
\pgfpathlineto{\pgfqpoint{3.558712in}{3.291520in}}%
\pgfpathlineto{\pgfqpoint{3.565812in}{3.355183in}}%
\pgfpathlineto{\pgfqpoint{3.571138in}{3.383991in}}%
\pgfpathlineto{\pgfqpoint{3.581790in}{3.431934in}}%
\pgfpathlineto{\pgfqpoint{3.588890in}{3.465026in}}%
\pgfpathlineto{\pgfqpoint{3.592441in}{3.475153in}}%
\pgfpathlineto{\pgfqpoint{3.595991in}{3.479098in}}%
\pgfpathlineto{\pgfqpoint{3.599542in}{3.480448in}}%
\pgfpathlineto{\pgfqpoint{3.603092in}{3.485265in}}%
\pgfpathlineto{\pgfqpoint{3.606643in}{3.497601in}}%
\pgfpathlineto{\pgfqpoint{3.611969in}{3.531779in}}%
\pgfpathlineto{\pgfqpoint{3.622620in}{3.607810in}}%
\pgfpathlineto{\pgfqpoint{3.626170in}{3.618876in}}%
\pgfpathlineto{\pgfqpoint{3.627946in}{3.620744in}}%
\pgfpathlineto{\pgfqpoint{3.629721in}{3.619970in}}%
\pgfpathlineto{\pgfqpoint{3.631496in}{3.616873in}}%
\pgfpathlineto{\pgfqpoint{3.635047in}{3.604153in}}%
\pgfpathlineto{\pgfqpoint{3.640372in}{3.571135in}}%
\pgfpathlineto{\pgfqpoint{3.656349in}{3.456590in}}%
\pgfpathlineto{\pgfqpoint{3.659900in}{3.449355in}}%
\pgfpathlineto{\pgfqpoint{3.661675in}{3.449663in}}%
\pgfpathlineto{\pgfqpoint{3.665225in}{3.455762in}}%
\pgfpathlineto{\pgfqpoint{3.675877in}{3.482369in}}%
\pgfpathlineto{\pgfqpoint{3.677652in}{3.483607in}}%
\pgfpathlineto{\pgfqpoint{3.679427in}{3.483079in}}%
\pgfpathlineto{\pgfqpoint{3.682978in}{3.477213in}}%
\pgfpathlineto{\pgfqpoint{3.690079in}{3.459804in}}%
\pgfpathlineto{\pgfqpoint{3.691854in}{3.457698in}}%
\pgfpathlineto{\pgfqpoint{3.693629in}{3.457467in}}%
\pgfpathlineto{\pgfqpoint{3.695404in}{3.459232in}}%
\pgfpathlineto{\pgfqpoint{3.698955in}{3.468280in}}%
\pgfpathlineto{\pgfqpoint{3.713157in}{3.519544in}}%
\pgfpathlineto{\pgfqpoint{3.718482in}{3.533138in}}%
\pgfpathlineto{\pgfqpoint{3.722033in}{3.549100in}}%
\pgfpathlineto{\pgfqpoint{3.727359in}{3.586926in}}%
\pgfpathlineto{\pgfqpoint{3.738010in}{3.671677in}}%
\pgfpathlineto{\pgfqpoint{3.741560in}{3.681661in}}%
\pgfpathlineto{\pgfqpoint{3.743336in}{3.681341in}}%
\pgfpathlineto{\pgfqpoint{3.745111in}{3.677563in}}%
\pgfpathlineto{\pgfqpoint{3.748661in}{3.661432in}}%
\pgfpathlineto{\pgfqpoint{3.753987in}{3.621954in}}%
\pgfpathlineto{\pgfqpoint{3.768189in}{3.506083in}}%
\pgfpathlineto{\pgfqpoint{3.775290in}{3.469386in}}%
\pgfpathlineto{\pgfqpoint{3.780616in}{3.453169in}}%
\pgfpathlineto{\pgfqpoint{3.784166in}{3.447020in}}%
\pgfpathlineto{\pgfqpoint{3.787717in}{3.444435in}}%
\pgfpathlineto{\pgfqpoint{3.791267in}{3.445808in}}%
\pgfpathlineto{\pgfqpoint{3.794817in}{3.451147in}}%
\pgfpathlineto{\pgfqpoint{3.800143in}{3.465711in}}%
\pgfpathlineto{\pgfqpoint{3.809019in}{3.491982in}}%
\pgfpathlineto{\pgfqpoint{3.812570in}{3.498314in}}%
\pgfpathlineto{\pgfqpoint{3.816120in}{3.500981in}}%
\pgfpathlineto{\pgfqpoint{3.821446in}{3.501118in}}%
\pgfpathlineto{\pgfqpoint{3.824996in}{3.499866in}}%
\pgfpathlineto{\pgfqpoint{3.828547in}{3.496055in}}%
\pgfpathlineto{\pgfqpoint{3.839198in}{3.477923in}}%
\pgfpathlineto{\pgfqpoint{3.840974in}{3.478603in}}%
\pgfpathlineto{\pgfqpoint{3.844524in}{3.484801in}}%
\pgfpathlineto{\pgfqpoint{3.855175in}{3.511416in}}%
\pgfpathlineto{\pgfqpoint{3.867602in}{3.527252in}}%
\pgfpathlineto{\pgfqpoint{3.874703in}{3.539891in}}%
\pgfpathlineto{\pgfqpoint{3.883579in}{3.549228in}}%
\pgfpathlineto{\pgfqpoint{3.887130in}{3.558062in}}%
\pgfpathlineto{\pgfqpoint{3.890680in}{3.573832in}}%
\pgfpathlineto{\pgfqpoint{3.897781in}{3.620643in}}%
\pgfpathlineto{\pgfqpoint{3.908432in}{3.708528in}}%
\pgfpathlineto{\pgfqpoint{3.920859in}{3.835662in}}%
\pgfpathlineto{\pgfqpoint{3.927960in}{3.904423in}}%
\pgfpathlineto{\pgfqpoint{3.931510in}{3.923399in}}%
\pgfpathlineto{\pgfqpoint{3.933286in}{3.926599in}}%
\pgfpathlineto{\pgfqpoint{3.935061in}{3.925009in}}%
\pgfpathlineto{\pgfqpoint{3.936836in}{3.918392in}}%
\pgfpathlineto{\pgfqpoint{3.940387in}{3.892052in}}%
\pgfpathlineto{\pgfqpoint{3.947487in}{3.811671in}}%
\pgfpathlineto{\pgfqpoint{3.958139in}{3.689796in}}%
\pgfpathlineto{\pgfqpoint{3.961689in}{3.664669in}}%
\pgfpathlineto{\pgfqpoint{3.965240in}{3.654503in}}%
\pgfpathlineto{\pgfqpoint{3.967015in}{3.655744in}}%
\pgfpathlineto{\pgfqpoint{3.968790in}{3.661723in}}%
\pgfpathlineto{\pgfqpoint{3.972341in}{3.685423in}}%
\pgfpathlineto{\pgfqpoint{3.977666in}{3.742271in}}%
\pgfpathlineto{\pgfqpoint{3.991868in}{3.907289in}}%
\pgfpathlineto{\pgfqpoint{3.995419in}{3.931643in}}%
\pgfpathlineto{\pgfqpoint{3.997194in}{3.937162in}}%
\pgfpathlineto{\pgfqpoint{3.998969in}{3.937156in}}%
\pgfpathlineto{\pgfqpoint{4.000744in}{3.931321in}}%
\pgfpathlineto{\pgfqpoint{4.004295in}{3.904093in}}%
\pgfpathlineto{\pgfqpoint{4.014946in}{3.798918in}}%
\pgfpathlineto{\pgfqpoint{4.018497in}{3.783264in}}%
\pgfpathlineto{\pgfqpoint{4.020272in}{3.780621in}}%
\pgfpathlineto{\pgfqpoint{4.022047in}{3.781319in}}%
\pgfpathlineto{\pgfqpoint{4.023822in}{3.785622in}}%
\pgfpathlineto{\pgfqpoint{4.027373in}{3.804668in}}%
\pgfpathlineto{\pgfqpoint{4.032699in}{3.854374in}}%
\pgfpathlineto{\pgfqpoint{4.043350in}{3.966552in}}%
\pgfpathlineto{\pgfqpoint{4.048676in}{3.996681in}}%
\pgfpathlineto{\pgfqpoint{4.052226in}{4.004169in}}%
\pgfpathlineto{\pgfqpoint{4.054001in}{4.004002in}}%
\pgfpathlineto{\pgfqpoint{4.055777in}{4.000969in}}%
\pgfpathlineto{\pgfqpoint{4.059327in}{3.986575in}}%
\pgfpathlineto{\pgfqpoint{4.062878in}{3.961198in}}%
\pgfpathlineto{\pgfqpoint{4.077079in}{3.834199in}}%
\pgfpathlineto{\pgfqpoint{4.080630in}{3.820917in}}%
\pgfpathlineto{\pgfqpoint{4.082405in}{3.818466in}}%
\pgfpathlineto{\pgfqpoint{4.084180in}{3.819329in}}%
\pgfpathlineto{\pgfqpoint{4.085956in}{3.823704in}}%
\pgfpathlineto{\pgfqpoint{4.089506in}{3.844895in}}%
\pgfpathlineto{\pgfqpoint{4.093057in}{3.882719in}}%
\pgfpathlineto{\pgfqpoint{4.105483in}{4.042243in}}%
\pgfpathlineto{\pgfqpoint{4.109034in}{4.056000in}}%
\pgfpathlineto{\pgfqpoint{4.110809in}{4.054852in}}%
\pgfpathlineto{\pgfqpoint{4.112584in}{4.048735in}}%
\pgfpathlineto{\pgfqpoint{4.116135in}{4.023112in}}%
\pgfpathlineto{\pgfqpoint{4.121460in}{3.960131in}}%
\pgfpathlineto{\pgfqpoint{4.135662in}{3.782025in}}%
\pgfpathlineto{\pgfqpoint{4.139213in}{3.766805in}}%
\pgfpathlineto{\pgfqpoint{4.140988in}{3.768448in}}%
\pgfpathlineto{\pgfqpoint{4.142763in}{3.776445in}}%
\pgfpathlineto{\pgfqpoint{4.146314in}{3.809225in}}%
\pgfpathlineto{\pgfqpoint{4.162291in}{3.997961in}}%
\pgfpathlineto{\pgfqpoint{4.165841in}{4.010861in}}%
\pgfpathlineto{\pgfqpoint{4.167616in}{4.011559in}}%
\pgfpathlineto{\pgfqpoint{4.169392in}{4.008619in}}%
\pgfpathlineto{\pgfqpoint{4.172942in}{3.991928in}}%
\pgfpathlineto{\pgfqpoint{4.176492in}{3.960835in}}%
\pgfpathlineto{\pgfqpoint{4.181818in}{3.888228in}}%
\pgfpathlineto{\pgfqpoint{4.192470in}{3.730645in}}%
\pgfpathlineto{\pgfqpoint{4.196020in}{3.703032in}}%
\pgfpathlineto{\pgfqpoint{4.199571in}{3.691093in}}%
\pgfpathlineto{\pgfqpoint{4.203121in}{3.689340in}}%
\pgfpathlineto{\pgfqpoint{4.206671in}{3.689270in}}%
\pgfpathlineto{\pgfqpoint{4.208447in}{3.687488in}}%
\pgfpathlineto{\pgfqpoint{4.211997in}{3.677728in}}%
\pgfpathlineto{\pgfqpoint{4.224424in}{3.621265in}}%
\pgfpathlineto{\pgfqpoint{4.229749in}{3.589343in}}%
\pgfpathlineto{\pgfqpoint{4.238626in}{3.527058in}}%
\pgfpathlineto{\pgfqpoint{4.242176in}{3.516319in}}%
\pgfpathlineto{\pgfqpoint{4.243951in}{3.515628in}}%
\pgfpathlineto{\pgfqpoint{4.245727in}{3.517550in}}%
\pgfpathlineto{\pgfqpoint{4.249277in}{3.528464in}}%
\pgfpathlineto{\pgfqpoint{4.254603in}{3.556788in}}%
\pgfpathlineto{\pgfqpoint{4.267029in}{3.629112in}}%
\pgfpathlineto{\pgfqpoint{4.270580in}{3.636958in}}%
\pgfpathlineto{\pgfqpoint{4.272355in}{3.637435in}}%
\pgfpathlineto{\pgfqpoint{4.274130in}{3.635853in}}%
\pgfpathlineto{\pgfqpoint{4.277681in}{3.626903in}}%
\pgfpathlineto{\pgfqpoint{4.281231in}{3.609558in}}%
\pgfpathlineto{\pgfqpoint{4.286557in}{3.566490in}}%
\pgfpathlineto{\pgfqpoint{4.295433in}{3.457000in}}%
\pgfpathlineto{\pgfqpoint{4.300759in}{3.402919in}}%
\pgfpathlineto{\pgfqpoint{4.304309in}{3.386392in}}%
\pgfpathlineto{\pgfqpoint{4.306084in}{3.385591in}}%
\pgfpathlineto{\pgfqpoint{4.307860in}{3.390362in}}%
\pgfpathlineto{\pgfqpoint{4.311410in}{3.416719in}}%
\pgfpathlineto{\pgfqpoint{4.316736in}{3.486372in}}%
\pgfpathlineto{\pgfqpoint{4.323837in}{3.578824in}}%
\pgfpathlineto{\pgfqpoint{4.329162in}{3.620860in}}%
\pgfpathlineto{\pgfqpoint{4.334488in}{3.643218in}}%
\pgfpathlineto{\pgfqpoint{4.338039in}{3.648679in}}%
\pgfpathlineto{\pgfqpoint{4.339814in}{3.647993in}}%
\pgfpathlineto{\pgfqpoint{4.341589in}{3.644996in}}%
\pgfpathlineto{\pgfqpoint{4.345140in}{3.631743in}}%
\pgfpathlineto{\pgfqpoint{4.348690in}{3.606922in}}%
\pgfpathlineto{\pgfqpoint{4.354016in}{3.550078in}}%
\pgfpathlineto{\pgfqpoint{4.361117in}{3.472145in}}%
\pgfpathlineto{\pgfqpoint{4.364667in}{3.457242in}}%
\pgfpathlineto{\pgfqpoint{4.366442in}{3.459047in}}%
\pgfpathlineto{\pgfqpoint{4.368218in}{3.467212in}}%
\pgfpathlineto{\pgfqpoint{4.371768in}{3.500864in}}%
\pgfpathlineto{\pgfqpoint{4.378869in}{3.606527in}}%
\pgfpathlineto{\pgfqpoint{4.384195in}{3.679560in}}%
\pgfpathlineto{\pgfqpoint{4.387745in}{3.708452in}}%
\pgfpathlineto{\pgfqpoint{4.389520in}{3.715391in}}%
\pgfpathlineto{\pgfqpoint{4.391296in}{3.717417in}}%
\pgfpathlineto{\pgfqpoint{4.393071in}{3.715379in}}%
\pgfpathlineto{\pgfqpoint{4.396621in}{3.702202in}}%
\pgfpathlineto{\pgfqpoint{4.401947in}{3.668553in}}%
\pgfpathlineto{\pgfqpoint{4.417924in}{3.547749in}}%
\pgfpathlineto{\pgfqpoint{4.419699in}{3.542902in}}%
\pgfpathlineto{\pgfqpoint{4.421475in}{3.541758in}}%
\pgfpathlineto{\pgfqpoint{4.423250in}{3.544657in}}%
\pgfpathlineto{\pgfqpoint{4.426800in}{3.562898in}}%
\pgfpathlineto{\pgfqpoint{4.432126in}{3.616254in}}%
\pgfpathlineto{\pgfqpoint{4.437452in}{3.665462in}}%
\pgfpathlineto{\pgfqpoint{4.439227in}{3.673397in}}%
\pgfpathlineto{\pgfqpoint{4.441002in}{3.675755in}}%
\pgfpathlineto{\pgfqpoint{4.442777in}{3.672984in}}%
\pgfpathlineto{\pgfqpoint{4.446328in}{3.656946in}}%
\pgfpathlineto{\pgfqpoint{4.458754in}{3.589125in}}%
\pgfpathlineto{\pgfqpoint{4.465855in}{3.562160in}}%
\pgfpathlineto{\pgfqpoint{4.469406in}{3.553136in}}%
\pgfpathlineto{\pgfqpoint{4.472956in}{3.549089in}}%
\pgfpathlineto{\pgfqpoint{4.478282in}{3.547667in}}%
\pgfpathlineto{\pgfqpoint{4.481832in}{3.548336in}}%
\pgfpathlineto{\pgfqpoint{4.485383in}{3.551658in}}%
\pgfpathlineto{\pgfqpoint{4.488933in}{3.561958in}}%
\pgfpathlineto{\pgfqpoint{4.492484in}{3.582770in}}%
\pgfpathlineto{\pgfqpoint{4.504911in}{3.671421in}}%
\pgfpathlineto{\pgfqpoint{4.506686in}{3.674152in}}%
\pgfpathlineto{\pgfqpoint{4.508461in}{3.673256in}}%
\pgfpathlineto{\pgfqpoint{4.512011in}{3.661403in}}%
\pgfpathlineto{\pgfqpoint{4.517337in}{3.628495in}}%
\pgfpathlineto{\pgfqpoint{4.529764in}{3.538277in}}%
\pgfpathlineto{\pgfqpoint{4.533314in}{3.523895in}}%
\pgfpathlineto{\pgfqpoint{4.536865in}{3.518291in}}%
\pgfpathlineto{\pgfqpoint{4.538640in}{3.518273in}}%
\pgfpathlineto{\pgfqpoint{4.542190in}{3.521543in}}%
\pgfpathlineto{\pgfqpoint{4.545741in}{3.524919in}}%
\pgfpathlineto{\pgfqpoint{4.549291in}{3.525486in}}%
\pgfpathlineto{\pgfqpoint{4.552842in}{3.523060in}}%
\pgfpathlineto{\pgfqpoint{4.556392in}{3.517827in}}%
\pgfpathlineto{\pgfqpoint{4.561718in}{3.505224in}}%
\pgfpathlineto{\pgfqpoint{4.570594in}{3.474833in}}%
\pgfpathlineto{\pgfqpoint{4.575920in}{3.449851in}}%
\pgfpathlineto{\pgfqpoint{4.586571in}{3.395070in}}%
\pgfpathlineto{\pgfqpoint{4.590122in}{3.389080in}}%
\pgfpathlineto{\pgfqpoint{4.591897in}{3.390096in}}%
\pgfpathlineto{\pgfqpoint{4.595447in}{3.400512in}}%
\pgfpathlineto{\pgfqpoint{4.598998in}{3.421107in}}%
\pgfpathlineto{\pgfqpoint{4.604324in}{3.470119in}}%
\pgfpathlineto{\pgfqpoint{4.613200in}{3.554647in}}%
\pgfpathlineto{\pgfqpoint{4.618525in}{3.581888in}}%
\pgfpathlineto{\pgfqpoint{4.622076in}{3.588910in}}%
\pgfpathlineto{\pgfqpoint{4.623851in}{3.589480in}}%
\pgfpathlineto{\pgfqpoint{4.627402in}{3.586149in}}%
\pgfpathlineto{\pgfqpoint{4.630952in}{3.577096in}}%
\pgfpathlineto{\pgfqpoint{4.636278in}{3.556080in}}%
\pgfpathlineto{\pgfqpoint{4.645154in}{3.511185in}}%
\pgfpathlineto{\pgfqpoint{4.659356in}{3.438239in}}%
\pgfpathlineto{\pgfqpoint{4.661131in}{3.435658in}}%
\pgfpathlineto{\pgfqpoint{4.662906in}{3.435795in}}%
\pgfpathlineto{\pgfqpoint{4.664681in}{3.438477in}}%
\pgfpathlineto{\pgfqpoint{4.668232in}{3.449630in}}%
\pgfpathlineto{\pgfqpoint{4.675333in}{3.474668in}}%
\pgfpathlineto{\pgfqpoint{4.677108in}{3.477226in}}%
\pgfpathlineto{\pgfqpoint{4.678883in}{3.477721in}}%
\pgfpathlineto{\pgfqpoint{4.680659in}{3.475564in}}%
\pgfpathlineto{\pgfqpoint{4.684209in}{3.463919in}}%
\pgfpathlineto{\pgfqpoint{4.689535in}{3.432266in}}%
\pgfpathlineto{\pgfqpoint{4.698411in}{3.374068in}}%
\pgfpathlineto{\pgfqpoint{4.701961in}{3.360834in}}%
\pgfpathlineto{\pgfqpoint{4.703737in}{3.357929in}}%
\pgfpathlineto{\pgfqpoint{4.705512in}{3.358055in}}%
\pgfpathlineto{\pgfqpoint{4.707287in}{3.361144in}}%
\pgfpathlineto{\pgfqpoint{4.710838in}{3.377987in}}%
\pgfpathlineto{\pgfqpoint{4.714388in}{3.409165in}}%
\pgfpathlineto{\pgfqpoint{4.730365in}{3.579123in}}%
\pgfpathlineto{\pgfqpoint{4.733916in}{3.588286in}}%
\pgfpathlineto{\pgfqpoint{4.735691in}{3.587840in}}%
\pgfpathlineto{\pgfqpoint{4.739241in}{3.579495in}}%
\pgfpathlineto{\pgfqpoint{4.748117in}{3.548063in}}%
\pgfpathlineto{\pgfqpoint{4.751668in}{3.544572in}}%
\pgfpathlineto{\pgfqpoint{4.753443in}{3.545517in}}%
\pgfpathlineto{\pgfqpoint{4.756994in}{3.552017in}}%
\pgfpathlineto{\pgfqpoint{4.762319in}{3.570144in}}%
\pgfpathlineto{\pgfqpoint{4.767645in}{3.598166in}}%
\pgfpathlineto{\pgfqpoint{4.780072in}{3.670512in}}%
\pgfpathlineto{\pgfqpoint{4.781847in}{3.674470in}}%
\pgfpathlineto{\pgfqpoint{4.783622in}{3.675468in}}%
\pgfpathlineto{\pgfqpoint{4.785397in}{3.673644in}}%
\pgfpathlineto{\pgfqpoint{4.788948in}{3.662857in}}%
\pgfpathlineto{\pgfqpoint{4.792498in}{3.643833in}}%
\pgfpathlineto{\pgfqpoint{4.797824in}{3.600282in}}%
\pgfpathlineto{\pgfqpoint{4.813801in}{3.442267in}}%
\pgfpathlineto{\pgfqpoint{4.815576in}{3.438303in}}%
\pgfpathlineto{\pgfqpoint{4.817351in}{3.437775in}}%
\pgfpathlineto{\pgfqpoint{4.819127in}{3.440234in}}%
\pgfpathlineto{\pgfqpoint{4.822677in}{3.453063in}}%
\pgfpathlineto{\pgfqpoint{4.828003in}{3.485243in}}%
\pgfpathlineto{\pgfqpoint{4.836879in}{3.542685in}}%
\pgfpathlineto{\pgfqpoint{4.840429in}{3.555483in}}%
\pgfpathlineto{\pgfqpoint{4.842205in}{3.558180in}}%
\pgfpathlineto{\pgfqpoint{4.843980in}{3.557480in}}%
\pgfpathlineto{\pgfqpoint{4.845755in}{3.553355in}}%
\pgfpathlineto{\pgfqpoint{4.849306in}{3.533557in}}%
\pgfpathlineto{\pgfqpoint{4.852856in}{3.496565in}}%
\pgfpathlineto{\pgfqpoint{4.863508in}{3.363132in}}%
\pgfpathlineto{\pgfqpoint{4.867058in}{3.348987in}}%
\pgfpathlineto{\pgfqpoint{4.868833in}{3.349991in}}%
\pgfpathlineto{\pgfqpoint{4.872384in}{3.363995in}}%
\pgfpathlineto{\pgfqpoint{4.893686in}{3.491282in}}%
\pgfpathlineto{\pgfqpoint{4.897237in}{3.496966in}}%
\pgfpathlineto{\pgfqpoint{4.899012in}{3.496123in}}%
\pgfpathlineto{\pgfqpoint{4.902563in}{3.488843in}}%
\pgfpathlineto{\pgfqpoint{4.914989in}{3.453709in}}%
\pgfpathlineto{\pgfqpoint{4.920315in}{3.439637in}}%
\pgfpathlineto{\pgfqpoint{4.929191in}{3.413059in}}%
\pgfpathlineto{\pgfqpoint{4.934517in}{3.405259in}}%
\pgfpathlineto{\pgfqpoint{4.939843in}{3.398297in}}%
\pgfpathlineto{\pgfqpoint{4.948719in}{3.380260in}}%
\pgfpathlineto{\pgfqpoint{4.950494in}{3.379118in}}%
\pgfpathlineto{\pgfqpoint{4.952269in}{3.379858in}}%
\pgfpathlineto{\pgfqpoint{4.955820in}{3.384798in}}%
\pgfpathlineto{\pgfqpoint{4.959370in}{3.390777in}}%
\pgfpathlineto{\pgfqpoint{4.961145in}{3.392039in}}%
\pgfpathlineto{\pgfqpoint{4.962921in}{3.391019in}}%
\pgfpathlineto{\pgfqpoint{4.964696in}{3.387253in}}%
\pgfpathlineto{\pgfqpoint{4.968246in}{3.371260in}}%
\pgfpathlineto{\pgfqpoint{4.975347in}{3.320037in}}%
\pgfpathlineto{\pgfqpoint{4.980673in}{3.287690in}}%
\pgfpathlineto{\pgfqpoint{4.984223in}{3.274720in}}%
\pgfpathlineto{\pgfqpoint{4.989549in}{3.265448in}}%
\pgfpathlineto{\pgfqpoint{4.994875in}{3.257454in}}%
\pgfpathlineto{\pgfqpoint{4.998425in}{3.246670in}}%
\pgfpathlineto{\pgfqpoint{5.001976in}{3.228450in}}%
\pgfpathlineto{\pgfqpoint{5.007301in}{3.183137in}}%
\pgfpathlineto{\pgfqpoint{5.014402in}{3.091775in}}%
\pgfpathlineto{\pgfqpoint{5.023278in}{2.977353in}}%
\pgfpathlineto{\pgfqpoint{5.026829in}{2.950276in}}%
\pgfpathlineto{\pgfqpoint{5.028604in}{2.943435in}}%
\pgfpathlineto{\pgfqpoint{5.030379in}{2.941385in}}%
\pgfpathlineto{\pgfqpoint{5.032155in}{2.944457in}}%
\pgfpathlineto{\pgfqpoint{5.035705in}{2.966080in}}%
\pgfpathlineto{\pgfqpoint{5.041031in}{3.026865in}}%
\pgfpathlineto{\pgfqpoint{5.051682in}{3.154195in}}%
\pgfpathlineto{\pgfqpoint{5.057008in}{3.190039in}}%
\pgfpathlineto{\pgfqpoint{5.058783in}{3.195120in}}%
\pgfpathlineto{\pgfqpoint{5.060558in}{3.196082in}}%
\pgfpathlineto{\pgfqpoint{5.062334in}{3.192647in}}%
\pgfpathlineto{\pgfqpoint{5.065884in}{3.171848in}}%
\pgfpathlineto{\pgfqpoint{5.071210in}{3.115008in}}%
\pgfpathlineto{\pgfqpoint{5.078311in}{3.038958in}}%
\pgfpathlineto{\pgfqpoint{5.081861in}{3.022426in}}%
\pgfpathlineto{\pgfqpoint{5.083636in}{3.020888in}}%
\pgfpathlineto{\pgfqpoint{5.085412in}{3.023353in}}%
\pgfpathlineto{\pgfqpoint{5.090737in}{3.042893in}}%
\pgfpathlineto{\pgfqpoint{5.094288in}{3.054257in}}%
\pgfpathlineto{\pgfqpoint{5.096063in}{3.057292in}}%
\pgfpathlineto{\pgfqpoint{5.097838in}{3.058058in}}%
\pgfpathlineto{\pgfqpoint{5.099613in}{3.056654in}}%
\pgfpathlineto{\pgfqpoint{5.103164in}{3.048820in}}%
\pgfpathlineto{\pgfqpoint{5.108490in}{3.027831in}}%
\pgfpathlineto{\pgfqpoint{5.113815in}{2.996152in}}%
\pgfpathlineto{\pgfqpoint{5.129792in}{2.874359in}}%
\pgfpathlineto{\pgfqpoint{5.135118in}{2.814666in}}%
\pgfpathlineto{\pgfqpoint{5.154646in}{2.556876in}}%
\pgfpathlineto{\pgfqpoint{5.159971in}{2.531762in}}%
\pgfpathlineto{\pgfqpoint{5.184825in}{2.442107in}}%
\pgfpathlineto{\pgfqpoint{5.188375in}{2.425226in}}%
\pgfpathlineto{\pgfqpoint{5.193701in}{2.385682in}}%
\pgfpathlineto{\pgfqpoint{5.200802in}{2.307153in}}%
\pgfpathlineto{\pgfqpoint{5.207903in}{2.197807in}}%
\pgfpathlineto{\pgfqpoint{5.236306in}{1.695557in}}%
\pgfpathlineto{\pgfqpoint{5.255834in}{1.424607in}}%
\pgfpathlineto{\pgfqpoint{5.266485in}{1.310781in}}%
\pgfpathlineto{\pgfqpoint{5.278912in}{1.206326in}}%
\pgfpathlineto{\pgfqpoint{5.293114in}{1.105763in}}%
\pgfpathlineto{\pgfqpoint{5.307316in}{1.019144in}}%
\pgfpathlineto{\pgfqpoint{5.319742in}{0.956597in}}%
\pgfpathlineto{\pgfqpoint{5.330394in}{0.912947in}}%
\pgfpathlineto{\pgfqpoint{5.341045in}{0.878409in}}%
\pgfpathlineto{\pgfqpoint{5.355247in}{0.835707in}}%
\pgfpathlineto{\pgfqpoint{5.367674in}{0.805049in}}%
\pgfpathlineto{\pgfqpoint{5.378325in}{0.785527in}}%
\pgfpathlineto{\pgfqpoint{5.390752in}{0.768130in}}%
\pgfpathlineto{\pgfqpoint{5.404953in}{0.751680in}}%
\pgfpathlineto{\pgfqpoint{5.419155in}{0.738088in}}%
\pgfpathlineto{\pgfqpoint{5.433357in}{0.728044in}}%
\pgfpathlineto{\pgfqpoint{5.456435in}{0.716545in}}%
\pgfpathlineto{\pgfqpoint{5.483064in}{0.706991in}}%
\pgfpathlineto{\pgfqpoint{5.484839in}{0.741984in}}%
\pgfpathlineto{\pgfqpoint{5.488389in}{0.744060in}}%
\pgfpathlineto{\pgfqpoint{5.490165in}{0.705464in}}%
\pgfpathlineto{\pgfqpoint{5.493715in}{0.704693in}}%
\pgfpathlineto{\pgfqpoint{5.495490in}{0.696000in}}%
\pgfpathlineto{\pgfqpoint{5.534545in}{0.696000in}}%
\pgfpathlineto{\pgfqpoint{5.534545in}{0.696000in}}%
\pgfusepath{stroke}%
\end{pgfscope}%
\begin{pgfscope}%
\pgfpathrectangle{\pgfqpoint{0.800000in}{0.528000in}}{\pgfqpoint{4.960000in}{3.696000in}} %
\pgfusepath{clip}%
\pgfsetrectcap%
\pgfsetroundjoin%
\pgfsetlinewidth{1.505625pt}%
\definecolor{currentstroke}{rgb}{1.000000,0.498039,0.054902}%
\pgfsetstrokecolor{currentstroke}%
\pgfsetdash{}{0pt}%
\pgfpathmoveto{\pgfqpoint{1.025455in}{0.696000in}}%
\pgfpathlineto{\pgfqpoint{1.121317in}{0.696000in}}%
\pgfpathlineto{\pgfqpoint{1.123092in}{0.699670in}}%
\pgfpathlineto{\pgfqpoint{1.130193in}{0.700462in}}%
\pgfpathlineto{\pgfqpoint{1.137294in}{0.697436in}}%
\pgfpathlineto{\pgfqpoint{1.142620in}{0.698102in}}%
\pgfpathlineto{\pgfqpoint{1.146170in}{0.697720in}}%
\pgfpathlineto{\pgfqpoint{1.149721in}{0.701287in}}%
\pgfpathlineto{\pgfqpoint{1.153271in}{0.716867in}}%
\pgfpathlineto{\pgfqpoint{1.162147in}{0.741984in}}%
\pgfpathlineto{\pgfqpoint{1.167473in}{0.754484in}}%
\pgfpathlineto{\pgfqpoint{1.174574in}{0.764990in}}%
\pgfpathlineto{\pgfqpoint{1.178125in}{0.768420in}}%
\pgfpathlineto{\pgfqpoint{1.179900in}{0.768054in}}%
\pgfpathlineto{\pgfqpoint{1.181675in}{0.770724in}}%
\pgfpathlineto{\pgfqpoint{1.188776in}{0.814224in}}%
\pgfpathlineto{\pgfqpoint{1.194102in}{0.872370in}}%
\pgfpathlineto{\pgfqpoint{1.195877in}{1.104697in}}%
\pgfpathlineto{\pgfqpoint{1.201203in}{1.194125in}}%
\pgfpathlineto{\pgfqpoint{1.204753in}{1.224238in}}%
\pgfpathlineto{\pgfqpoint{1.208304in}{1.234724in}}%
\pgfpathlineto{\pgfqpoint{1.210079in}{1.235055in}}%
\pgfpathlineto{\pgfqpoint{1.213629in}{1.229674in}}%
\pgfpathlineto{\pgfqpoint{1.218955in}{1.220088in}}%
\pgfpathlineto{\pgfqpoint{1.222505in}{1.218253in}}%
\pgfpathlineto{\pgfqpoint{1.226056in}{1.220342in}}%
\pgfpathlineto{\pgfqpoint{1.233157in}{1.226490in}}%
\pgfpathlineto{\pgfqpoint{1.238482in}{1.230142in}}%
\pgfpathlineto{\pgfqpoint{1.242033in}{1.238070in}}%
\pgfpathlineto{\pgfqpoint{1.266886in}{1.317367in}}%
\pgfpathlineto{\pgfqpoint{1.273987in}{1.335784in}}%
\pgfpathlineto{\pgfqpoint{1.277538in}{1.338823in}}%
\pgfpathlineto{\pgfqpoint{1.279313in}{1.338493in}}%
\pgfpathlineto{\pgfqpoint{1.282863in}{1.333599in}}%
\pgfpathlineto{\pgfqpoint{1.286414in}{1.323877in}}%
\pgfpathlineto{\pgfqpoint{1.295290in}{1.294495in}}%
\pgfpathlineto{\pgfqpoint{1.298840in}{1.289704in}}%
\pgfpathlineto{\pgfqpoint{1.302391in}{1.285740in}}%
\pgfpathlineto{\pgfqpoint{1.305941in}{1.275101in}}%
\pgfpathlineto{\pgfqpoint{1.309492in}{1.253640in}}%
\pgfpathlineto{\pgfqpoint{1.327244in}{1.110632in}}%
\pgfpathlineto{\pgfqpoint{1.332570in}{1.090187in}}%
\pgfpathlineto{\pgfqpoint{1.348547in}{1.043263in}}%
\pgfpathlineto{\pgfqpoint{1.352097in}{1.037168in}}%
\pgfpathlineto{\pgfqpoint{1.353873in}{1.036048in}}%
\pgfpathlineto{\pgfqpoint{1.357423in}{1.037585in}}%
\pgfpathlineto{\pgfqpoint{1.362749in}{1.040842in}}%
\pgfpathlineto{\pgfqpoint{1.364524in}{1.040272in}}%
\pgfpathlineto{\pgfqpoint{1.368074in}{1.035238in}}%
\pgfpathlineto{\pgfqpoint{1.378726in}{1.010433in}}%
\pgfpathlineto{\pgfqpoint{1.380501in}{1.009825in}}%
\pgfpathlineto{\pgfqpoint{1.382276in}{1.010742in}}%
\pgfpathlineto{\pgfqpoint{1.385827in}{1.017979in}}%
\pgfpathlineto{\pgfqpoint{1.391152in}{1.038733in}}%
\pgfpathlineto{\pgfqpoint{1.396478in}{1.069736in}}%
\pgfpathlineto{\pgfqpoint{1.410680in}{1.158338in}}%
\pgfpathlineto{\pgfqpoint{1.421331in}{1.208051in}}%
\pgfpathlineto{\pgfqpoint{1.426657in}{1.224513in}}%
\pgfpathlineto{\pgfqpoint{1.430208in}{1.228843in}}%
\pgfpathlineto{\pgfqpoint{1.431983in}{1.228648in}}%
\pgfpathlineto{\pgfqpoint{1.435533in}{1.223317in}}%
\pgfpathlineto{\pgfqpoint{1.440859in}{1.207633in}}%
\pgfpathlineto{\pgfqpoint{1.451510in}{1.171084in}}%
\pgfpathlineto{\pgfqpoint{1.458611in}{1.155665in}}%
\pgfpathlineto{\pgfqpoint{1.463937in}{1.149243in}}%
\pgfpathlineto{\pgfqpoint{1.471038in}{1.141043in}}%
\pgfpathlineto{\pgfqpoint{1.483465in}{1.123293in}}%
\pgfpathlineto{\pgfqpoint{1.488790in}{1.109151in}}%
\pgfpathlineto{\pgfqpoint{1.502992in}{1.066303in}}%
\pgfpathlineto{\pgfqpoint{1.510093in}{1.048933in}}%
\pgfpathlineto{\pgfqpoint{1.513644in}{1.043852in}}%
\pgfpathlineto{\pgfqpoint{1.517194in}{1.043103in}}%
\pgfpathlineto{\pgfqpoint{1.520744in}{1.047219in}}%
\pgfpathlineto{\pgfqpoint{1.526070in}{1.062078in}}%
\pgfpathlineto{\pgfqpoint{1.534946in}{1.090564in}}%
\pgfpathlineto{\pgfqpoint{1.540272in}{1.100208in}}%
\pgfpathlineto{\pgfqpoint{1.549148in}{1.109131in}}%
\pgfpathlineto{\pgfqpoint{1.554474in}{1.112673in}}%
\pgfpathlineto{\pgfqpoint{1.558024in}{1.111778in}}%
\pgfpathlineto{\pgfqpoint{1.566901in}{1.098829in}}%
\pgfpathlineto{\pgfqpoint{1.574001in}{1.090839in}}%
\pgfpathlineto{\pgfqpoint{1.579327in}{1.084746in}}%
\pgfpathlineto{\pgfqpoint{1.582878in}{1.076221in}}%
\pgfpathlineto{\pgfqpoint{1.591754in}{1.043765in}}%
\pgfpathlineto{\pgfqpoint{1.600630in}{1.008136in}}%
\pgfpathlineto{\pgfqpoint{1.607731in}{0.987344in}}%
\pgfpathlineto{\pgfqpoint{1.613057in}{0.977752in}}%
\pgfpathlineto{\pgfqpoint{1.614832in}{0.977325in}}%
\pgfpathlineto{\pgfqpoint{1.620157in}{0.972126in}}%
\pgfpathlineto{\pgfqpoint{1.629034in}{0.968160in}}%
\pgfpathlineto{\pgfqpoint{1.643236in}{0.956184in}}%
\pgfpathlineto{\pgfqpoint{1.646786in}{0.956918in}}%
\pgfpathlineto{\pgfqpoint{1.650336in}{0.961225in}}%
\pgfpathlineto{\pgfqpoint{1.653887in}{0.969874in}}%
\pgfpathlineto{\pgfqpoint{1.659213in}{0.990348in}}%
\pgfpathlineto{\pgfqpoint{1.669864in}{1.050134in}}%
\pgfpathlineto{\pgfqpoint{1.676965in}{1.090008in}}%
\pgfpathlineto{\pgfqpoint{1.682291in}{1.103161in}}%
\pgfpathlineto{\pgfqpoint{1.685841in}{1.105423in}}%
\pgfpathlineto{\pgfqpoint{1.687616in}{1.104544in}}%
\pgfpathlineto{\pgfqpoint{1.691167in}{1.098881in}}%
\pgfpathlineto{\pgfqpoint{1.696492in}{1.083576in}}%
\pgfpathlineto{\pgfqpoint{1.707144in}{1.047127in}}%
\pgfpathlineto{\pgfqpoint{1.717795in}{1.018518in}}%
\pgfpathlineto{\pgfqpoint{1.721346in}{1.013496in}}%
\pgfpathlineto{\pgfqpoint{1.724896in}{1.011279in}}%
\pgfpathlineto{\pgfqpoint{1.730222in}{1.011431in}}%
\pgfpathlineto{\pgfqpoint{1.733772in}{1.009218in}}%
\pgfpathlineto{\pgfqpoint{1.737323in}{1.004432in}}%
\pgfpathlineto{\pgfqpoint{1.740873in}{0.995857in}}%
\pgfpathlineto{\pgfqpoint{1.751525in}{0.963782in}}%
\pgfpathlineto{\pgfqpoint{1.753300in}{0.961965in}}%
\pgfpathlineto{\pgfqpoint{1.755075in}{0.962498in}}%
\pgfpathlineto{\pgfqpoint{1.758626in}{0.969013in}}%
\pgfpathlineto{\pgfqpoint{1.763951in}{0.986092in}}%
\pgfpathlineto{\pgfqpoint{1.772827in}{1.017756in}}%
\pgfpathlineto{\pgfqpoint{1.774603in}{1.020401in}}%
\pgfpathlineto{\pgfqpoint{1.776378in}{1.020851in}}%
\pgfpathlineto{\pgfqpoint{1.778153in}{1.019001in}}%
\pgfpathlineto{\pgfqpoint{1.781704in}{1.008728in}}%
\pgfpathlineto{\pgfqpoint{1.804782in}{0.910825in}}%
\pgfpathlineto{\pgfqpoint{1.808332in}{0.904844in}}%
\pgfpathlineto{\pgfqpoint{1.810107in}{0.903958in}}%
\pgfpathlineto{\pgfqpoint{1.811883in}{0.904948in}}%
\pgfpathlineto{\pgfqpoint{1.815433in}{0.911982in}}%
\pgfpathlineto{\pgfqpoint{1.831410in}{0.964675in}}%
\pgfpathlineto{\pgfqpoint{1.833185in}{0.965844in}}%
\pgfpathlineto{\pgfqpoint{1.834961in}{0.965314in}}%
\pgfpathlineto{\pgfqpoint{1.840286in}{0.958551in}}%
\pgfpathlineto{\pgfqpoint{1.847387in}{0.949076in}}%
\pgfpathlineto{\pgfqpoint{1.850938in}{0.948230in}}%
\pgfpathlineto{\pgfqpoint{1.854488in}{0.948907in}}%
\pgfpathlineto{\pgfqpoint{1.861589in}{0.954514in}}%
\pgfpathlineto{\pgfqpoint{1.868690in}{0.959235in}}%
\pgfpathlineto{\pgfqpoint{1.874016in}{0.960782in}}%
\pgfpathlineto{\pgfqpoint{1.875791in}{0.958662in}}%
\pgfpathlineto{\pgfqpoint{1.886442in}{0.929995in}}%
\pgfpathlineto{\pgfqpoint{1.891768in}{0.923038in}}%
\pgfpathlineto{\pgfqpoint{1.895319in}{0.919397in}}%
\pgfpathlineto{\pgfqpoint{1.900644in}{0.912803in}}%
\pgfpathlineto{\pgfqpoint{1.902419in}{0.912773in}}%
\pgfpathlineto{\pgfqpoint{1.905970in}{0.916074in}}%
\pgfpathlineto{\pgfqpoint{1.916621in}{0.930429in}}%
\pgfpathlineto{\pgfqpoint{1.918397in}{0.932450in}}%
\pgfpathlineto{\pgfqpoint{1.925497in}{0.930806in}}%
\pgfpathlineto{\pgfqpoint{1.930823in}{0.933508in}}%
\pgfpathlineto{\pgfqpoint{1.937924in}{0.937026in}}%
\pgfpathlineto{\pgfqpoint{1.946800in}{0.937185in}}%
\pgfpathlineto{\pgfqpoint{1.948576in}{0.937176in}}%
\pgfpathlineto{\pgfqpoint{1.952126in}{0.933212in}}%
\pgfpathlineto{\pgfqpoint{1.957452in}{0.929912in}}%
\pgfpathlineto{\pgfqpoint{1.962777in}{0.926777in}}%
\pgfpathlineto{\pgfqpoint{1.968103in}{0.921799in}}%
\pgfpathlineto{\pgfqpoint{1.969878in}{0.922176in}}%
\pgfpathlineto{\pgfqpoint{1.971654in}{0.924519in}}%
\pgfpathlineto{\pgfqpoint{1.975204in}{0.936118in}}%
\pgfpathlineto{\pgfqpoint{1.980530in}{0.969754in}}%
\pgfpathlineto{\pgfqpoint{1.991181in}{1.047242in}}%
\pgfpathlineto{\pgfqpoint{1.998282in}{1.086391in}}%
\pgfpathlineto{\pgfqpoint{2.003608in}{1.103752in}}%
\pgfpathlineto{\pgfqpoint{2.008933in}{1.112222in}}%
\pgfpathlineto{\pgfqpoint{2.016034in}{1.121364in}}%
\pgfpathlineto{\pgfqpoint{2.033787in}{1.147954in}}%
\pgfpathlineto{\pgfqpoint{2.035562in}{1.149106in}}%
\pgfpathlineto{\pgfqpoint{2.037337in}{1.148174in}}%
\pgfpathlineto{\pgfqpoint{2.039112in}{1.144456in}}%
\pgfpathlineto{\pgfqpoint{2.042663in}{1.129302in}}%
\pgfpathlineto{\pgfqpoint{2.049764in}{1.079287in}}%
\pgfpathlineto{\pgfqpoint{2.071067in}{0.918494in}}%
\pgfpathlineto{\pgfqpoint{2.074617in}{0.909014in}}%
\pgfpathlineto{\pgfqpoint{2.076392in}{0.908472in}}%
\pgfpathlineto{\pgfqpoint{2.078168in}{0.910501in}}%
\pgfpathlineto{\pgfqpoint{2.081718in}{0.921407in}}%
\pgfpathlineto{\pgfqpoint{2.088819in}{0.958810in}}%
\pgfpathlineto{\pgfqpoint{2.094145in}{0.982222in}}%
\pgfpathlineto{\pgfqpoint{2.099470in}{0.996852in}}%
\pgfpathlineto{\pgfqpoint{2.101246in}{0.998973in}}%
\pgfpathlineto{\pgfqpoint{2.103021in}{0.999330in}}%
\pgfpathlineto{\pgfqpoint{2.104796in}{0.997925in}}%
\pgfpathlineto{\pgfqpoint{2.108346in}{0.990960in}}%
\pgfpathlineto{\pgfqpoint{2.117223in}{0.971211in}}%
\pgfpathlineto{\pgfqpoint{2.118998in}{0.970614in}}%
\pgfpathlineto{\pgfqpoint{2.122548in}{0.973111in}}%
\pgfpathlineto{\pgfqpoint{2.133200in}{0.986615in}}%
\pgfpathlineto{\pgfqpoint{2.138525in}{0.991944in}}%
\pgfpathlineto{\pgfqpoint{2.142076in}{0.999384in}}%
\pgfpathlineto{\pgfqpoint{2.152727in}{1.026978in}}%
\pgfpathlineto{\pgfqpoint{2.161603in}{1.039839in}}%
\pgfpathlineto{\pgfqpoint{2.168704in}{1.046666in}}%
\pgfpathlineto{\pgfqpoint{2.184681in}{1.052394in}}%
\pgfpathlineto{\pgfqpoint{2.188232in}{1.049028in}}%
\pgfpathlineto{\pgfqpoint{2.195333in}{1.036839in}}%
\pgfpathlineto{\pgfqpoint{2.200659in}{1.037040in}}%
\pgfpathlineto{\pgfqpoint{2.204209in}{1.043112in}}%
\pgfpathlineto{\pgfqpoint{2.211310in}{1.066186in}}%
\pgfpathlineto{\pgfqpoint{2.225512in}{1.118310in}}%
\pgfpathlineto{\pgfqpoint{2.229062in}{1.125150in}}%
\pgfpathlineto{\pgfqpoint{2.236163in}{1.137346in}}%
\pgfpathlineto{\pgfqpoint{2.241489in}{1.142559in}}%
\pgfpathlineto{\pgfqpoint{2.245039in}{1.142761in}}%
\pgfpathlineto{\pgfqpoint{2.248590in}{1.138284in}}%
\pgfpathlineto{\pgfqpoint{2.261016in}{1.114702in}}%
\pgfpathlineto{\pgfqpoint{2.268117in}{1.111098in}}%
\pgfpathlineto{\pgfqpoint{2.284094in}{1.102400in}}%
\pgfpathlineto{\pgfqpoint{2.289420in}{1.095334in}}%
\pgfpathlineto{\pgfqpoint{2.308948in}{1.060360in}}%
\pgfpathlineto{\pgfqpoint{2.317824in}{1.042463in}}%
\pgfpathlineto{\pgfqpoint{2.319599in}{1.041013in}}%
\pgfpathlineto{\pgfqpoint{2.321374in}{1.042603in}}%
\pgfpathlineto{\pgfqpoint{2.323150in}{1.042602in}}%
\pgfpathlineto{\pgfqpoint{2.324925in}{1.043784in}}%
\pgfpathlineto{\pgfqpoint{2.328475in}{1.051446in}}%
\pgfpathlineto{\pgfqpoint{2.333801in}{1.070823in}}%
\pgfpathlineto{\pgfqpoint{2.348003in}{1.129236in}}%
\pgfpathlineto{\pgfqpoint{2.351553in}{1.134562in}}%
\pgfpathlineto{\pgfqpoint{2.353329in}{1.135249in}}%
\pgfpathlineto{\pgfqpoint{2.356879in}{1.131729in}}%
\pgfpathlineto{\pgfqpoint{2.360429in}{1.123202in}}%
\pgfpathlineto{\pgfqpoint{2.365755in}{1.104253in}}%
\pgfpathlineto{\pgfqpoint{2.372856in}{1.078687in}}%
\pgfpathlineto{\pgfqpoint{2.376407in}{1.071577in}}%
\pgfpathlineto{\pgfqpoint{2.378182in}{1.070484in}}%
\pgfpathlineto{\pgfqpoint{2.379957in}{1.070758in}}%
\pgfpathlineto{\pgfqpoint{2.383508in}{1.075197in}}%
\pgfpathlineto{\pgfqpoint{2.387058in}{1.083344in}}%
\pgfpathlineto{\pgfqpoint{2.392384in}{1.101521in}}%
\pgfpathlineto{\pgfqpoint{2.404810in}{1.148409in}}%
\pgfpathlineto{\pgfqpoint{2.419012in}{1.183154in}}%
\pgfpathlineto{\pgfqpoint{2.429664in}{1.221423in}}%
\pgfpathlineto{\pgfqpoint{2.438540in}{1.242194in}}%
\pgfpathlineto{\pgfqpoint{2.450966in}{1.265814in}}%
\pgfpathlineto{\pgfqpoint{2.458067in}{1.283721in}}%
\pgfpathlineto{\pgfqpoint{2.461618in}{1.286755in}}%
\pgfpathlineto{\pgfqpoint{2.465168in}{1.283672in}}%
\pgfpathlineto{\pgfqpoint{2.472269in}{1.271478in}}%
\pgfpathlineto{\pgfqpoint{2.484696in}{1.246777in}}%
\pgfpathlineto{\pgfqpoint{2.490021in}{1.243531in}}%
\pgfpathlineto{\pgfqpoint{2.495347in}{1.239611in}}%
\pgfpathlineto{\pgfqpoint{2.504223in}{1.230564in}}%
\pgfpathlineto{\pgfqpoint{2.507774in}{1.230905in}}%
\pgfpathlineto{\pgfqpoint{2.511324in}{1.235069in}}%
\pgfpathlineto{\pgfqpoint{2.523751in}{1.255221in}}%
\pgfpathlineto{\pgfqpoint{2.527301in}{1.255287in}}%
\pgfpathlineto{\pgfqpoint{2.530852in}{1.252327in}}%
\pgfpathlineto{\pgfqpoint{2.537953in}{1.241799in}}%
\pgfpathlineto{\pgfqpoint{2.543278in}{1.229437in}}%
\pgfpathlineto{\pgfqpoint{2.550379in}{1.205451in}}%
\pgfpathlineto{\pgfqpoint{2.566356in}{1.142712in}}%
\pgfpathlineto{\pgfqpoint{2.571682in}{1.131601in}}%
\pgfpathlineto{\pgfqpoint{2.575233in}{1.128410in}}%
\pgfpathlineto{\pgfqpoint{2.578783in}{1.129012in}}%
\pgfpathlineto{\pgfqpoint{2.584109in}{1.135027in}}%
\pgfpathlineto{\pgfqpoint{2.596535in}{1.151554in}}%
\pgfpathlineto{\pgfqpoint{2.598311in}{1.151839in}}%
\pgfpathlineto{\pgfqpoint{2.600086in}{1.150750in}}%
\pgfpathlineto{\pgfqpoint{2.603636in}{1.144761in}}%
\pgfpathlineto{\pgfqpoint{2.612513in}{1.126549in}}%
\pgfpathlineto{\pgfqpoint{2.619613in}{1.113011in}}%
\pgfpathlineto{\pgfqpoint{2.624939in}{1.094603in}}%
\pgfpathlineto{\pgfqpoint{2.633815in}{1.063139in}}%
\pgfpathlineto{\pgfqpoint{2.635591in}{1.060374in}}%
\pgfpathlineto{\pgfqpoint{2.639141in}{1.062204in}}%
\pgfpathlineto{\pgfqpoint{2.642691in}{1.070597in}}%
\pgfpathlineto{\pgfqpoint{2.648017in}{1.092876in}}%
\pgfpathlineto{\pgfqpoint{2.653343in}{1.118930in}}%
\pgfpathlineto{\pgfqpoint{2.660444in}{1.151036in}}%
\pgfpathlineto{\pgfqpoint{2.665770in}{1.165385in}}%
\pgfpathlineto{\pgfqpoint{2.667545in}{1.167561in}}%
\pgfpathlineto{\pgfqpoint{2.669320in}{1.167808in}}%
\pgfpathlineto{\pgfqpoint{2.672870in}{1.164935in}}%
\pgfpathlineto{\pgfqpoint{2.676421in}{1.158748in}}%
\pgfpathlineto{\pgfqpoint{2.683522in}{1.137377in}}%
\pgfpathlineto{\pgfqpoint{2.692398in}{1.106179in}}%
\pgfpathlineto{\pgfqpoint{2.699499in}{1.080284in}}%
\pgfpathlineto{\pgfqpoint{2.703049in}{1.072536in}}%
\pgfpathlineto{\pgfqpoint{2.706600in}{1.068506in}}%
\pgfpathlineto{\pgfqpoint{2.713701in}{1.063385in}}%
\pgfpathlineto{\pgfqpoint{2.719026in}{1.055731in}}%
\pgfpathlineto{\pgfqpoint{2.726127in}{1.038929in}}%
\pgfpathlineto{\pgfqpoint{2.736779in}{1.011415in}}%
\pgfpathlineto{\pgfqpoint{2.738554in}{1.010067in}}%
\pgfpathlineto{\pgfqpoint{2.740329in}{1.011146in}}%
\pgfpathlineto{\pgfqpoint{2.742105in}{1.014710in}}%
\pgfpathlineto{\pgfqpoint{2.745655in}{1.030018in}}%
\pgfpathlineto{\pgfqpoint{2.750981in}{1.071934in}}%
\pgfpathlineto{\pgfqpoint{2.763407in}{1.179681in}}%
\pgfpathlineto{\pgfqpoint{2.772283in}{1.229864in}}%
\pgfpathlineto{\pgfqpoint{2.784710in}{1.295910in}}%
\pgfpathlineto{\pgfqpoint{2.786485in}{1.299968in}}%
\pgfpathlineto{\pgfqpoint{2.788261in}{1.301028in}}%
\pgfpathlineto{\pgfqpoint{2.790036in}{1.299093in}}%
\pgfpathlineto{\pgfqpoint{2.793586in}{1.288831in}}%
\pgfpathlineto{\pgfqpoint{2.800687in}{1.257064in}}%
\pgfpathlineto{\pgfqpoint{2.809563in}{1.215591in}}%
\pgfpathlineto{\pgfqpoint{2.813114in}{1.204427in}}%
\pgfpathlineto{\pgfqpoint{2.816664in}{1.198769in}}%
\pgfpathlineto{\pgfqpoint{2.821990in}{1.196342in}}%
\pgfpathlineto{\pgfqpoint{2.825540in}{1.196377in}}%
\pgfpathlineto{\pgfqpoint{2.829091in}{1.198499in}}%
\pgfpathlineto{\pgfqpoint{2.834417in}{1.204610in}}%
\pgfpathlineto{\pgfqpoint{2.848618in}{1.227264in}}%
\pgfpathlineto{\pgfqpoint{2.850394in}{1.227511in}}%
\pgfpathlineto{\pgfqpoint{2.853944in}{1.224799in}}%
\pgfpathlineto{\pgfqpoint{2.859270in}{1.216633in}}%
\pgfpathlineto{\pgfqpoint{2.864596in}{1.209879in}}%
\pgfpathlineto{\pgfqpoint{2.868146in}{1.210081in}}%
\pgfpathlineto{\pgfqpoint{2.871696in}{1.210901in}}%
\pgfpathlineto{\pgfqpoint{2.875247in}{1.208979in}}%
\pgfpathlineto{\pgfqpoint{2.878797in}{1.202748in}}%
\pgfpathlineto{\pgfqpoint{2.882348in}{1.190690in}}%
\pgfpathlineto{\pgfqpoint{2.891224in}{1.154696in}}%
\pgfpathlineto{\pgfqpoint{2.894775in}{1.147833in}}%
\pgfpathlineto{\pgfqpoint{2.896550in}{1.146956in}}%
\pgfpathlineto{\pgfqpoint{2.898325in}{1.147534in}}%
\pgfpathlineto{\pgfqpoint{2.901875in}{1.152041in}}%
\pgfpathlineto{\pgfqpoint{2.908976in}{1.166789in}}%
\pgfpathlineto{\pgfqpoint{2.926729in}{1.215094in}}%
\pgfpathlineto{\pgfqpoint{2.940931in}{1.230194in}}%
\pgfpathlineto{\pgfqpoint{2.951582in}{1.249803in}}%
\pgfpathlineto{\pgfqpoint{2.962233in}{1.256720in}}%
\pgfpathlineto{\pgfqpoint{2.971110in}{1.259326in}}%
\pgfpathlineto{\pgfqpoint{2.974660in}{1.266320in}}%
\pgfpathlineto{\pgfqpoint{2.979986in}{1.289403in}}%
\pgfpathlineto{\pgfqpoint{2.985311in}{1.314181in}}%
\pgfpathlineto{\pgfqpoint{2.988862in}{1.323287in}}%
\pgfpathlineto{\pgfqpoint{2.990637in}{1.324960in}}%
\pgfpathlineto{\pgfqpoint{2.992412in}{1.325076in}}%
\pgfpathlineto{\pgfqpoint{2.994188in}{1.323309in}}%
\pgfpathlineto{\pgfqpoint{2.997738in}{1.314937in}}%
\pgfpathlineto{\pgfqpoint{3.011940in}{1.270313in}}%
\pgfpathlineto{\pgfqpoint{3.015490in}{1.266130in}}%
\pgfpathlineto{\pgfqpoint{3.019041in}{1.266290in}}%
\pgfpathlineto{\pgfqpoint{3.022591in}{1.271171in}}%
\pgfpathlineto{\pgfqpoint{3.027917in}{1.286097in}}%
\pgfpathlineto{\pgfqpoint{3.035018in}{1.315925in}}%
\pgfpathlineto{\pgfqpoint{3.047445in}{1.380569in}}%
\pgfpathlineto{\pgfqpoint{3.054545in}{1.426496in}}%
\pgfpathlineto{\pgfqpoint{3.061646in}{1.471598in}}%
\pgfpathlineto{\pgfqpoint{3.066972in}{1.489847in}}%
\pgfpathlineto{\pgfqpoint{3.070523in}{1.495646in}}%
\pgfpathlineto{\pgfqpoint{3.074073in}{1.497516in}}%
\pgfpathlineto{\pgfqpoint{3.079399in}{1.499188in}}%
\pgfpathlineto{\pgfqpoint{3.082949in}{1.503223in}}%
\pgfpathlineto{\pgfqpoint{3.086500in}{1.511723in}}%
\pgfpathlineto{\pgfqpoint{3.091825in}{1.532507in}}%
\pgfpathlineto{\pgfqpoint{3.098926in}{1.560834in}}%
\pgfpathlineto{\pgfqpoint{3.102477in}{1.569875in}}%
\pgfpathlineto{\pgfqpoint{3.106027in}{1.574020in}}%
\pgfpathlineto{\pgfqpoint{3.107802in}{1.574236in}}%
\pgfpathlineto{\pgfqpoint{3.111353in}{1.571016in}}%
\pgfpathlineto{\pgfqpoint{3.114903in}{1.562425in}}%
\pgfpathlineto{\pgfqpoint{3.120229in}{1.540259in}}%
\pgfpathlineto{\pgfqpoint{3.130880in}{1.484153in}}%
\pgfpathlineto{\pgfqpoint{3.141532in}{1.426720in}}%
\pgfpathlineto{\pgfqpoint{3.148633in}{1.406263in}}%
\pgfpathlineto{\pgfqpoint{3.152183in}{1.394434in}}%
\pgfpathlineto{\pgfqpoint{3.155734in}{1.376377in}}%
\pgfpathlineto{\pgfqpoint{3.161059in}{1.334844in}}%
\pgfpathlineto{\pgfqpoint{3.168160in}{1.256978in}}%
\pgfpathlineto{\pgfqpoint{3.178812in}{1.139530in}}%
\pgfpathlineto{\pgfqpoint{3.184137in}{1.099554in}}%
\pgfpathlineto{\pgfqpoint{3.187688in}{1.084618in}}%
\pgfpathlineto{\pgfqpoint{3.191238in}{1.079212in}}%
\pgfpathlineto{\pgfqpoint{3.193014in}{1.080317in}}%
\pgfpathlineto{\pgfqpoint{3.196564in}{1.090797in}}%
\pgfpathlineto{\pgfqpoint{3.201890in}{1.118810in}}%
\pgfpathlineto{\pgfqpoint{3.214316in}{1.189928in}}%
\pgfpathlineto{\pgfqpoint{3.219642in}{1.222901in}}%
\pgfpathlineto{\pgfqpoint{3.226743in}{1.286914in}}%
\pgfpathlineto{\pgfqpoint{3.235619in}{1.367432in}}%
\pgfpathlineto{\pgfqpoint{3.239170in}{1.382398in}}%
\pgfpathlineto{\pgfqpoint{3.240945in}{1.384398in}}%
\pgfpathlineto{\pgfqpoint{3.242720in}{1.384566in}}%
\pgfpathlineto{\pgfqpoint{3.244495in}{1.383006in}}%
\pgfpathlineto{\pgfqpoint{3.246271in}{1.383294in}}%
\pgfpathlineto{\pgfqpoint{3.249821in}{1.391792in}}%
\pgfpathlineto{\pgfqpoint{3.251596in}{1.392223in}}%
\pgfpathlineto{\pgfqpoint{3.253372in}{1.389369in}}%
\pgfpathlineto{\pgfqpoint{3.256922in}{1.372419in}}%
\pgfpathlineto{\pgfqpoint{3.260472in}{1.335711in}}%
\pgfpathlineto{\pgfqpoint{3.265798in}{1.255837in}}%
\pgfpathlineto{\pgfqpoint{3.267573in}{1.258578in}}%
\pgfpathlineto{\pgfqpoint{3.269349in}{1.270407in}}%
\pgfpathlineto{\pgfqpoint{3.271124in}{1.274720in}}%
\pgfpathlineto{\pgfqpoint{3.272899in}{1.269855in}}%
\pgfpathlineto{\pgfqpoint{3.274674in}{1.253525in}}%
\pgfpathlineto{\pgfqpoint{3.276450in}{1.224128in}}%
\pgfpathlineto{\pgfqpoint{3.280000in}{1.309093in}}%
\pgfpathlineto{\pgfqpoint{3.281775in}{1.312242in}}%
\pgfpathlineto{\pgfqpoint{3.283550in}{1.283329in}}%
\pgfpathlineto{\pgfqpoint{3.287101in}{1.144405in}}%
\pgfpathlineto{\pgfqpoint{3.290651in}{0.989701in}}%
\pgfpathlineto{\pgfqpoint{3.294202in}{0.911151in}}%
\pgfpathlineto{\pgfqpoint{3.295977in}{0.884866in}}%
\pgfpathlineto{\pgfqpoint{3.299528in}{0.868858in}}%
\pgfpathlineto{\pgfqpoint{3.301303in}{0.895637in}}%
\pgfpathlineto{\pgfqpoint{3.306628in}{1.061194in}}%
\pgfpathlineto{\pgfqpoint{3.313729in}{1.302117in}}%
\pgfpathlineto{\pgfqpoint{3.326156in}{1.585450in}}%
\pgfpathlineto{\pgfqpoint{3.333257in}{1.683977in}}%
\pgfpathlineto{\pgfqpoint{3.338583in}{1.733283in}}%
\pgfpathlineto{\pgfqpoint{3.343908in}{1.765659in}}%
\pgfpathlineto{\pgfqpoint{3.349234in}{1.784347in}}%
\pgfpathlineto{\pgfqpoint{3.352785in}{1.790199in}}%
\pgfpathlineto{\pgfqpoint{3.354560in}{1.790948in}}%
\pgfpathlineto{\pgfqpoint{3.356335in}{1.790287in}}%
\pgfpathlineto{\pgfqpoint{3.359885in}{1.784601in}}%
\pgfpathlineto{\pgfqpoint{3.363436in}{1.772907in}}%
\pgfpathlineto{\pgfqpoint{3.382963in}{1.694117in}}%
\pgfpathlineto{\pgfqpoint{3.393615in}{1.672135in}}%
\pgfpathlineto{\pgfqpoint{3.420243in}{1.622918in}}%
\pgfpathlineto{\pgfqpoint{3.425569in}{1.609093in}}%
\pgfpathlineto{\pgfqpoint{3.443321in}{1.557718in}}%
\pgfpathlineto{\pgfqpoint{3.446872in}{1.552540in}}%
\pgfpathlineto{\pgfqpoint{3.450422in}{1.550732in}}%
\pgfpathlineto{\pgfqpoint{3.452198in}{1.551648in}}%
\pgfpathlineto{\pgfqpoint{3.455748in}{1.558159in}}%
\pgfpathlineto{\pgfqpoint{3.459298in}{1.570346in}}%
\pgfpathlineto{\pgfqpoint{3.487702in}{1.691650in}}%
\pgfpathlineto{\pgfqpoint{3.496578in}{1.730958in}}%
\pgfpathlineto{\pgfqpoint{3.512555in}{1.815908in}}%
\pgfpathlineto{\pgfqpoint{3.516106in}{1.826392in}}%
\pgfpathlineto{\pgfqpoint{3.519656in}{1.829712in}}%
\pgfpathlineto{\pgfqpoint{3.523207in}{1.827924in}}%
\pgfpathlineto{\pgfqpoint{3.533858in}{1.816473in}}%
\pgfpathlineto{\pgfqpoint{3.539184in}{1.815195in}}%
\pgfpathlineto{\pgfqpoint{3.544510in}{1.815249in}}%
\pgfpathlineto{\pgfqpoint{3.548060in}{1.819927in}}%
\pgfpathlineto{\pgfqpoint{3.553386in}{1.834663in}}%
\pgfpathlineto{\pgfqpoint{3.572913in}{1.901588in}}%
\pgfpathlineto{\pgfqpoint{3.578239in}{1.913389in}}%
\pgfpathlineto{\pgfqpoint{3.581790in}{1.918354in}}%
\pgfpathlineto{\pgfqpoint{3.588890in}{1.922565in}}%
\pgfpathlineto{\pgfqpoint{3.595991in}{1.927001in}}%
\pgfpathlineto{\pgfqpoint{3.601317in}{1.934249in}}%
\pgfpathlineto{\pgfqpoint{3.615519in}{1.956726in}}%
\pgfpathlineto{\pgfqpoint{3.620845in}{1.960425in}}%
\pgfpathlineto{\pgfqpoint{3.626170in}{1.961754in}}%
\pgfpathlineto{\pgfqpoint{3.629721in}{1.958922in}}%
\pgfpathlineto{\pgfqpoint{3.636822in}{1.945183in}}%
\pgfpathlineto{\pgfqpoint{3.654574in}{1.906642in}}%
\pgfpathlineto{\pgfqpoint{3.659900in}{1.898141in}}%
\pgfpathlineto{\pgfqpoint{3.677652in}{1.881556in}}%
\pgfpathlineto{\pgfqpoint{3.691854in}{1.876292in}}%
\pgfpathlineto{\pgfqpoint{3.704281in}{1.881233in}}%
\pgfpathlineto{\pgfqpoint{3.709606in}{1.884200in}}%
\pgfpathlineto{\pgfqpoint{3.714932in}{1.891638in}}%
\pgfpathlineto{\pgfqpoint{3.718482in}{1.899611in}}%
\pgfpathlineto{\pgfqpoint{3.725583in}{1.917395in}}%
\pgfpathlineto{\pgfqpoint{3.730909in}{1.927250in}}%
\pgfpathlineto{\pgfqpoint{3.734460in}{1.930113in}}%
\pgfpathlineto{\pgfqpoint{3.738010in}{1.930342in}}%
\pgfpathlineto{\pgfqpoint{3.741560in}{1.926352in}}%
\pgfpathlineto{\pgfqpoint{3.746886in}{1.915856in}}%
\pgfpathlineto{\pgfqpoint{3.752212in}{1.896945in}}%
\pgfpathlineto{\pgfqpoint{3.759313in}{1.860519in}}%
\pgfpathlineto{\pgfqpoint{3.766414in}{1.824599in}}%
\pgfpathlineto{\pgfqpoint{3.771739in}{1.808617in}}%
\pgfpathlineto{\pgfqpoint{3.777065in}{1.799510in}}%
\pgfpathlineto{\pgfqpoint{3.780616in}{1.798657in}}%
\pgfpathlineto{\pgfqpoint{3.784166in}{1.801440in}}%
\pgfpathlineto{\pgfqpoint{3.791267in}{1.812195in}}%
\pgfpathlineto{\pgfqpoint{3.796593in}{1.819284in}}%
\pgfpathlineto{\pgfqpoint{3.807244in}{1.829427in}}%
\pgfpathlineto{\pgfqpoint{3.816120in}{1.840573in}}%
\pgfpathlineto{\pgfqpoint{3.819671in}{1.841396in}}%
\pgfpathlineto{\pgfqpoint{3.830322in}{1.841893in}}%
\pgfpathlineto{\pgfqpoint{3.835648in}{1.845706in}}%
\pgfpathlineto{\pgfqpoint{3.844524in}{1.855847in}}%
\pgfpathlineto{\pgfqpoint{3.849850in}{1.854986in}}%
\pgfpathlineto{\pgfqpoint{3.855175in}{1.852982in}}%
\pgfpathlineto{\pgfqpoint{3.858726in}{1.854963in}}%
\pgfpathlineto{\pgfqpoint{3.864052in}{1.861856in}}%
\pgfpathlineto{\pgfqpoint{3.869377in}{1.868737in}}%
\pgfpathlineto{\pgfqpoint{3.881804in}{1.873075in}}%
\pgfpathlineto{\pgfqpoint{3.887130in}{1.879032in}}%
\pgfpathlineto{\pgfqpoint{3.892455in}{1.890453in}}%
\pgfpathlineto{\pgfqpoint{3.899556in}{1.909461in}}%
\pgfpathlineto{\pgfqpoint{3.922634in}{1.987669in}}%
\pgfpathlineto{\pgfqpoint{3.926185in}{1.992399in}}%
\pgfpathlineto{\pgfqpoint{3.927960in}{1.992911in}}%
\pgfpathlineto{\pgfqpoint{3.931510in}{1.989977in}}%
\pgfpathlineto{\pgfqpoint{3.935061in}{1.984366in}}%
\pgfpathlineto{\pgfqpoint{3.943937in}{1.962633in}}%
\pgfpathlineto{\pgfqpoint{3.954588in}{1.929498in}}%
\pgfpathlineto{\pgfqpoint{3.958139in}{1.923943in}}%
\pgfpathlineto{\pgfqpoint{3.961689in}{1.922605in}}%
\pgfpathlineto{\pgfqpoint{3.965240in}{1.924007in}}%
\pgfpathlineto{\pgfqpoint{3.968790in}{1.928060in}}%
\pgfpathlineto{\pgfqpoint{3.972341in}{1.935765in}}%
\pgfpathlineto{\pgfqpoint{3.977666in}{1.954946in}}%
\pgfpathlineto{\pgfqpoint{3.988318in}{2.000140in}}%
\pgfpathlineto{\pgfqpoint{3.990093in}{2.002332in}}%
\pgfpathlineto{\pgfqpoint{3.991868in}{2.001853in}}%
\pgfpathlineto{\pgfqpoint{3.995419in}{1.994498in}}%
\pgfpathlineto{\pgfqpoint{4.004295in}{1.967906in}}%
\pgfpathlineto{\pgfqpoint{4.007845in}{1.963631in}}%
\pgfpathlineto{\pgfqpoint{4.018497in}{1.956774in}}%
\pgfpathlineto{\pgfqpoint{4.020272in}{1.957464in}}%
\pgfpathlineto{\pgfqpoint{4.023822in}{1.963180in}}%
\pgfpathlineto{\pgfqpoint{4.034474in}{1.991111in}}%
\pgfpathlineto{\pgfqpoint{4.043350in}{2.004353in}}%
\pgfpathlineto{\pgfqpoint{4.050451in}{2.013932in}}%
\pgfpathlineto{\pgfqpoint{4.052226in}{2.013901in}}%
\pgfpathlineto{\pgfqpoint{4.054001in}{2.012136in}}%
\pgfpathlineto{\pgfqpoint{4.057552in}{2.003107in}}%
\pgfpathlineto{\pgfqpoint{4.068203in}{1.963789in}}%
\pgfpathlineto{\pgfqpoint{4.071754in}{1.957966in}}%
\pgfpathlineto{\pgfqpoint{4.073529in}{1.957717in}}%
\pgfpathlineto{\pgfqpoint{4.077079in}{1.962482in}}%
\pgfpathlineto{\pgfqpoint{4.080630in}{1.972255in}}%
\pgfpathlineto{\pgfqpoint{4.089506in}{2.008261in}}%
\pgfpathlineto{\pgfqpoint{4.094832in}{2.026206in}}%
\pgfpathlineto{\pgfqpoint{4.100157in}{2.035649in}}%
\pgfpathlineto{\pgfqpoint{4.101933in}{2.036839in}}%
\pgfpathlineto{\pgfqpoint{4.103708in}{2.036666in}}%
\pgfpathlineto{\pgfqpoint{4.105483in}{2.035104in}}%
\pgfpathlineto{\pgfqpoint{4.109034in}{2.027190in}}%
\pgfpathlineto{\pgfqpoint{4.114359in}{2.004918in}}%
\pgfpathlineto{\pgfqpoint{4.126786in}{1.940371in}}%
\pgfpathlineto{\pgfqpoint{4.130336in}{1.929839in}}%
\pgfpathlineto{\pgfqpoint{4.133887in}{1.924237in}}%
\pgfpathlineto{\pgfqpoint{4.135662in}{1.923101in}}%
\pgfpathlineto{\pgfqpoint{4.137437in}{1.923630in}}%
\pgfpathlineto{\pgfqpoint{4.140988in}{1.928892in}}%
\pgfpathlineto{\pgfqpoint{4.144538in}{1.937141in}}%
\pgfpathlineto{\pgfqpoint{4.151639in}{1.963609in}}%
\pgfpathlineto{\pgfqpoint{4.156965in}{1.980644in}}%
\pgfpathlineto{\pgfqpoint{4.160515in}{1.987729in}}%
\pgfpathlineto{\pgfqpoint{4.164066in}{1.990173in}}%
\pgfpathlineto{\pgfqpoint{4.165841in}{1.989300in}}%
\pgfpathlineto{\pgfqpoint{4.169392in}{1.980983in}}%
\pgfpathlineto{\pgfqpoint{4.174717in}{1.957482in}}%
\pgfpathlineto{\pgfqpoint{4.187144in}{1.892661in}}%
\pgfpathlineto{\pgfqpoint{4.192470in}{1.877989in}}%
\pgfpathlineto{\pgfqpoint{4.196020in}{1.873333in}}%
\pgfpathlineto{\pgfqpoint{4.199571in}{1.872331in}}%
\pgfpathlineto{\pgfqpoint{4.211997in}{1.872428in}}%
\pgfpathlineto{\pgfqpoint{4.217323in}{1.868526in}}%
\pgfpathlineto{\pgfqpoint{4.220873in}{1.863085in}}%
\pgfpathlineto{\pgfqpoint{4.229749in}{1.845269in}}%
\pgfpathlineto{\pgfqpoint{4.235075in}{1.841185in}}%
\pgfpathlineto{\pgfqpoint{4.238626in}{1.840441in}}%
\pgfpathlineto{\pgfqpoint{4.243951in}{1.844412in}}%
\pgfpathlineto{\pgfqpoint{4.249277in}{1.849964in}}%
\pgfpathlineto{\pgfqpoint{4.252827in}{1.858320in}}%
\pgfpathlineto{\pgfqpoint{4.261704in}{1.881122in}}%
\pgfpathlineto{\pgfqpoint{4.268805in}{1.890760in}}%
\pgfpathlineto{\pgfqpoint{4.272355in}{1.889687in}}%
\pgfpathlineto{\pgfqpoint{4.275906in}{1.885224in}}%
\pgfpathlineto{\pgfqpoint{4.279456in}{1.876618in}}%
\pgfpathlineto{\pgfqpoint{4.284782in}{1.852606in}}%
\pgfpathlineto{\pgfqpoint{4.291883in}{1.814697in}}%
\pgfpathlineto{\pgfqpoint{4.295433in}{1.803152in}}%
\pgfpathlineto{\pgfqpoint{4.297208in}{1.800407in}}%
\pgfpathlineto{\pgfqpoint{4.298984in}{1.800180in}}%
\pgfpathlineto{\pgfqpoint{4.302534in}{1.806163in}}%
\pgfpathlineto{\pgfqpoint{4.307860in}{1.822486in}}%
\pgfpathlineto{\pgfqpoint{4.322062in}{1.880878in}}%
\pgfpathlineto{\pgfqpoint{4.325612in}{1.889456in}}%
\pgfpathlineto{\pgfqpoint{4.327387in}{1.891238in}}%
\pgfpathlineto{\pgfqpoint{4.329162in}{1.890890in}}%
\pgfpathlineto{\pgfqpoint{4.332713in}{1.886598in}}%
\pgfpathlineto{\pgfqpoint{4.338039in}{1.876221in}}%
\pgfpathlineto{\pgfqpoint{4.346915in}{1.847968in}}%
\pgfpathlineto{\pgfqpoint{4.352241in}{1.833990in}}%
\pgfpathlineto{\pgfqpoint{4.355791in}{1.830678in}}%
\pgfpathlineto{\pgfqpoint{4.357566in}{1.831942in}}%
\pgfpathlineto{\pgfqpoint{4.361117in}{1.841336in}}%
\pgfpathlineto{\pgfqpoint{4.373543in}{1.886652in}}%
\pgfpathlineto{\pgfqpoint{4.380644in}{1.902574in}}%
\pgfpathlineto{\pgfqpoint{4.391296in}{1.920686in}}%
\pgfpathlineto{\pgfqpoint{4.394846in}{1.922137in}}%
\pgfpathlineto{\pgfqpoint{4.396621in}{1.920407in}}%
\pgfpathlineto{\pgfqpoint{4.400172in}{1.911479in}}%
\pgfpathlineto{\pgfqpoint{4.410823in}{1.880088in}}%
\pgfpathlineto{\pgfqpoint{4.414374in}{1.876721in}}%
\pgfpathlineto{\pgfqpoint{4.417924in}{1.876059in}}%
\pgfpathlineto{\pgfqpoint{4.421475in}{1.880060in}}%
\pgfpathlineto{\pgfqpoint{4.428576in}{1.891734in}}%
\pgfpathlineto{\pgfqpoint{4.433901in}{1.899708in}}%
\pgfpathlineto{\pgfqpoint{4.437452in}{1.901871in}}%
\pgfpathlineto{\pgfqpoint{4.441002in}{1.900417in}}%
\pgfpathlineto{\pgfqpoint{4.444553in}{1.897898in}}%
\pgfpathlineto{\pgfqpoint{4.448103in}{1.896070in}}%
\pgfpathlineto{\pgfqpoint{4.455204in}{1.884232in}}%
\pgfpathlineto{\pgfqpoint{4.462305in}{1.873602in}}%
\pgfpathlineto{\pgfqpoint{4.467631in}{1.870112in}}%
\pgfpathlineto{\pgfqpoint{4.474732in}{1.871658in}}%
\pgfpathlineto{\pgfqpoint{4.478282in}{1.873442in}}%
\pgfpathlineto{\pgfqpoint{4.481832in}{1.877773in}}%
\pgfpathlineto{\pgfqpoint{4.485383in}{1.884541in}}%
\pgfpathlineto{\pgfqpoint{4.494259in}{1.905078in}}%
\pgfpathlineto{\pgfqpoint{4.496034in}{1.906135in}}%
\pgfpathlineto{\pgfqpoint{4.499585in}{1.904382in}}%
\pgfpathlineto{\pgfqpoint{4.503135in}{1.898292in}}%
\pgfpathlineto{\pgfqpoint{4.520888in}{1.849845in}}%
\pgfpathlineto{\pgfqpoint{4.524438in}{1.845899in}}%
\pgfpathlineto{\pgfqpoint{4.529764in}{1.844508in}}%
\pgfpathlineto{\pgfqpoint{4.535089in}{1.846227in}}%
\pgfpathlineto{\pgfqpoint{4.540415in}{1.846024in}}%
\pgfpathlineto{\pgfqpoint{4.547516in}{1.845185in}}%
\pgfpathlineto{\pgfqpoint{4.558168in}{1.847918in}}%
\pgfpathlineto{\pgfqpoint{4.561718in}{1.844022in}}%
\pgfpathlineto{\pgfqpoint{4.565268in}{1.834310in}}%
\pgfpathlineto{\pgfqpoint{4.570594in}{1.811631in}}%
\pgfpathlineto{\pgfqpoint{4.577695in}{1.781769in}}%
\pgfpathlineto{\pgfqpoint{4.581246in}{1.774128in}}%
\pgfpathlineto{\pgfqpoint{4.584796in}{1.772487in}}%
\pgfpathlineto{\pgfqpoint{4.588346in}{1.776404in}}%
\pgfpathlineto{\pgfqpoint{4.593672in}{1.787990in}}%
\pgfpathlineto{\pgfqpoint{4.604324in}{1.814430in}}%
\pgfpathlineto{\pgfqpoint{4.611424in}{1.825902in}}%
\pgfpathlineto{\pgfqpoint{4.620301in}{1.835131in}}%
\pgfpathlineto{\pgfqpoint{4.622076in}{1.835375in}}%
\pgfpathlineto{\pgfqpoint{4.623851in}{1.834094in}}%
\pgfpathlineto{\pgfqpoint{4.627402in}{1.827648in}}%
\pgfpathlineto{\pgfqpoint{4.634503in}{1.810295in}}%
\pgfpathlineto{\pgfqpoint{4.638053in}{1.806299in}}%
\pgfpathlineto{\pgfqpoint{4.650480in}{1.800929in}}%
\pgfpathlineto{\pgfqpoint{4.655805in}{1.802787in}}%
\pgfpathlineto{\pgfqpoint{4.659356in}{1.807707in}}%
\pgfpathlineto{\pgfqpoint{4.662906in}{1.813043in}}%
\pgfpathlineto{\pgfqpoint{4.666457in}{1.812860in}}%
\pgfpathlineto{\pgfqpoint{4.670007in}{1.809198in}}%
\pgfpathlineto{\pgfqpoint{4.677108in}{1.797621in}}%
\pgfpathlineto{\pgfqpoint{4.693085in}{1.769441in}}%
\pgfpathlineto{\pgfqpoint{4.696636in}{1.768969in}}%
\pgfpathlineto{\pgfqpoint{4.698411in}{1.770000in}}%
\pgfpathlineto{\pgfqpoint{4.701961in}{1.775540in}}%
\pgfpathlineto{\pgfqpoint{4.707287in}{1.788486in}}%
\pgfpathlineto{\pgfqpoint{4.721489in}{1.840268in}}%
\pgfpathlineto{\pgfqpoint{4.725039in}{1.845226in}}%
\pgfpathlineto{\pgfqpoint{4.728590in}{1.847242in}}%
\pgfpathlineto{\pgfqpoint{4.735691in}{1.843955in}}%
\pgfpathlineto{\pgfqpoint{4.742792in}{1.839205in}}%
\pgfpathlineto{\pgfqpoint{4.746342in}{1.839025in}}%
\pgfpathlineto{\pgfqpoint{4.753443in}{1.840400in}}%
\pgfpathlineto{\pgfqpoint{4.764094in}{1.850133in}}%
\pgfpathlineto{\pgfqpoint{4.776521in}{1.872411in}}%
\pgfpathlineto{\pgfqpoint{4.780072in}{1.874791in}}%
\pgfpathlineto{\pgfqpoint{4.781847in}{1.874595in}}%
\pgfpathlineto{\pgfqpoint{4.785397in}{1.870435in}}%
\pgfpathlineto{\pgfqpoint{4.788948in}{1.863334in}}%
\pgfpathlineto{\pgfqpoint{4.792498in}{1.851958in}}%
\pgfpathlineto{\pgfqpoint{4.804925in}{1.803946in}}%
\pgfpathlineto{\pgfqpoint{4.808475in}{1.798006in}}%
\pgfpathlineto{\pgfqpoint{4.812026in}{1.796295in}}%
\pgfpathlineto{\pgfqpoint{4.817351in}{1.798768in}}%
\pgfpathlineto{\pgfqpoint{4.824452in}{1.802312in}}%
\pgfpathlineto{\pgfqpoint{4.833329in}{1.803851in}}%
\pgfpathlineto{\pgfqpoint{4.836879in}{1.802092in}}%
\pgfpathlineto{\pgfqpoint{4.840429in}{1.796231in}}%
\pgfpathlineto{\pgfqpoint{4.854631in}{1.759431in}}%
\pgfpathlineto{\pgfqpoint{4.858182in}{1.758148in}}%
\pgfpathlineto{\pgfqpoint{4.861732in}{1.759419in}}%
\pgfpathlineto{\pgfqpoint{4.870608in}{1.766509in}}%
\pgfpathlineto{\pgfqpoint{4.877709in}{1.768921in}}%
\pgfpathlineto{\pgfqpoint{4.886586in}{1.780516in}}%
\pgfpathlineto{\pgfqpoint{4.890136in}{1.785290in}}%
\pgfpathlineto{\pgfqpoint{4.893686in}{1.785981in}}%
\pgfpathlineto{\pgfqpoint{4.897237in}{1.783470in}}%
\pgfpathlineto{\pgfqpoint{4.911439in}{1.768298in}}%
\pgfpathlineto{\pgfqpoint{4.916764in}{1.765186in}}%
\pgfpathlineto{\pgfqpoint{4.923865in}{1.759334in}}%
\pgfpathlineto{\pgfqpoint{4.927416in}{1.755793in}}%
\pgfpathlineto{\pgfqpoint{4.943393in}{1.735564in}}%
\pgfpathlineto{\pgfqpoint{4.946943in}{1.734091in}}%
\pgfpathlineto{\pgfqpoint{4.950494in}{1.733597in}}%
\pgfpathlineto{\pgfqpoint{4.954044in}{1.730438in}}%
\pgfpathlineto{\pgfqpoint{4.957595in}{1.724492in}}%
\pgfpathlineto{\pgfqpoint{4.962921in}{1.710499in}}%
\pgfpathlineto{\pgfqpoint{4.973572in}{1.682839in}}%
\pgfpathlineto{\pgfqpoint{4.982448in}{1.667604in}}%
\pgfpathlineto{\pgfqpoint{4.989549in}{1.659945in}}%
\pgfpathlineto{\pgfqpoint{4.993099in}{1.658916in}}%
\pgfpathlineto{\pgfqpoint{4.996650in}{1.657824in}}%
\pgfpathlineto{\pgfqpoint{5.000200in}{1.652062in}}%
\pgfpathlineto{\pgfqpoint{5.003751in}{1.640032in}}%
\pgfpathlineto{\pgfqpoint{5.021503in}{1.563185in}}%
\pgfpathlineto{\pgfqpoint{5.025054in}{1.558241in}}%
\pgfpathlineto{\pgfqpoint{5.026829in}{1.558769in}}%
\pgfpathlineto{\pgfqpoint{5.030379in}{1.565046in}}%
\pgfpathlineto{\pgfqpoint{5.049907in}{1.616123in}}%
\pgfpathlineto{\pgfqpoint{5.051682in}{1.617099in}}%
\pgfpathlineto{\pgfqpoint{5.053457in}{1.616666in}}%
\pgfpathlineto{\pgfqpoint{5.055233in}{1.614506in}}%
\pgfpathlineto{\pgfqpoint{5.058783in}{1.605116in}}%
\pgfpathlineto{\pgfqpoint{5.071210in}{1.563178in}}%
\pgfpathlineto{\pgfqpoint{5.074760in}{1.559687in}}%
\pgfpathlineto{\pgfqpoint{5.078311in}{1.559393in}}%
\pgfpathlineto{\pgfqpoint{5.083636in}{1.560064in}}%
\pgfpathlineto{\pgfqpoint{5.087187in}{1.558427in}}%
\pgfpathlineto{\pgfqpoint{5.092513in}{1.551638in}}%
\pgfpathlineto{\pgfqpoint{5.097838in}{1.542216in}}%
\pgfpathlineto{\pgfqpoint{5.110265in}{1.515985in}}%
\pgfpathlineto{\pgfqpoint{5.117366in}{1.504804in}}%
\pgfpathlineto{\pgfqpoint{5.122691in}{1.489694in}}%
\pgfpathlineto{\pgfqpoint{5.129792in}{1.458371in}}%
\pgfpathlineto{\pgfqpoint{5.140444in}{1.405756in}}%
\pgfpathlineto{\pgfqpoint{5.147545in}{1.383172in}}%
\pgfpathlineto{\pgfqpoint{5.156421in}{1.363742in}}%
\pgfpathlineto{\pgfqpoint{5.170623in}{1.341319in}}%
\pgfpathlineto{\pgfqpoint{5.177724in}{1.332430in}}%
\pgfpathlineto{\pgfqpoint{5.181274in}{1.330978in}}%
\pgfpathlineto{\pgfqpoint{5.184825in}{1.332957in}}%
\pgfpathlineto{\pgfqpoint{5.190150in}{1.336582in}}%
\pgfpathlineto{\pgfqpoint{5.191926in}{1.336579in}}%
\pgfpathlineto{\pgfqpoint{5.193701in}{1.335072in}}%
\pgfpathlineto{\pgfqpoint{5.197251in}{1.326659in}}%
\pgfpathlineto{\pgfqpoint{5.200802in}{1.311234in}}%
\pgfpathlineto{\pgfqpoint{5.207903in}{1.267610in}}%
\pgfpathlineto{\pgfqpoint{5.216779in}{1.200040in}}%
\pgfpathlineto{\pgfqpoint{5.252283in}{0.896427in}}%
\pgfpathlineto{\pgfqpoint{5.261160in}{0.848580in}}%
\pgfpathlineto{\pgfqpoint{5.270036in}{0.812566in}}%
\pgfpathlineto{\pgfqpoint{5.278912in}{0.785742in}}%
\pgfpathlineto{\pgfqpoint{5.287788in}{0.765782in}}%
\pgfpathlineto{\pgfqpoint{5.298440in}{0.747511in}}%
\pgfpathlineto{\pgfqpoint{5.307316in}{0.736328in}}%
\pgfpathlineto{\pgfqpoint{5.316192in}{0.728339in}}%
\pgfpathlineto{\pgfqpoint{5.328618in}{0.720148in}}%
\pgfpathlineto{\pgfqpoint{5.341045in}{0.714598in}}%
\pgfpathlineto{\pgfqpoint{5.367674in}{0.706118in}}%
\pgfpathlineto{\pgfqpoint{5.394302in}{0.700362in}}%
\pgfpathlineto{\pgfqpoint{5.420931in}{0.698056in}}%
\pgfpathlineto{\pgfqpoint{5.475963in}{0.696322in}}%
\pgfpathlineto{\pgfqpoint{5.483064in}{0.696290in}}%
\pgfpathlineto{\pgfqpoint{5.484839in}{0.700283in}}%
\pgfpathlineto{\pgfqpoint{5.486614in}{0.697333in}}%
\pgfpathlineto{\pgfqpoint{5.497266in}{0.696000in}}%
\pgfpathlineto{\pgfqpoint{5.534545in}{0.696000in}}%
\pgfpathlineto{\pgfqpoint{5.534545in}{0.696000in}}%
\pgfusepath{stroke}%
\end{pgfscope}%
\begin{pgfscope}%
\pgfpathrectangle{\pgfqpoint{0.800000in}{0.528000in}}{\pgfqpoint{4.960000in}{3.696000in}} %
\pgfusepath{clip}%
\pgfsetrectcap%
\pgfsetroundjoin%
\pgfsetlinewidth{1.505625pt}%
\definecolor{currentstroke}{rgb}{0.172549,0.627451,0.172549}%
\pgfsetstrokecolor{currentstroke}%
\pgfsetdash{}{0pt}%
\pgfpathmoveto{\pgfqpoint{1.025455in}{0.696000in}}%
\pgfpathlineto{\pgfqpoint{1.149721in}{0.696824in}}%
\pgfpathlineto{\pgfqpoint{1.151496in}{0.707575in}}%
\pgfpathlineto{\pgfqpoint{1.158597in}{0.720924in}}%
\pgfpathlineto{\pgfqpoint{1.160372in}{0.721822in}}%
\pgfpathlineto{\pgfqpoint{1.163923in}{0.720623in}}%
\pgfpathlineto{\pgfqpoint{1.171024in}{0.730951in}}%
\pgfpathlineto{\pgfqpoint{1.172799in}{0.727111in}}%
\pgfpathlineto{\pgfqpoint{1.174574in}{0.728908in}}%
\pgfpathlineto{\pgfqpoint{1.178125in}{0.726158in}}%
\pgfpathlineto{\pgfqpoint{1.181675in}{0.728363in}}%
\pgfpathlineto{\pgfqpoint{1.183450in}{0.733449in}}%
\pgfpathlineto{\pgfqpoint{1.194102in}{0.799184in}}%
\pgfpathlineto{\pgfqpoint{1.195877in}{0.756304in}}%
\pgfpathlineto{\pgfqpoint{1.199427in}{0.763070in}}%
\pgfpathlineto{\pgfqpoint{1.202978in}{0.763545in}}%
\pgfpathlineto{\pgfqpoint{1.208304in}{0.758037in}}%
\pgfpathlineto{\pgfqpoint{1.215404in}{0.748285in}}%
\pgfpathlineto{\pgfqpoint{1.217180in}{0.747708in}}%
\pgfpathlineto{\pgfqpoint{1.220730in}{0.750621in}}%
\pgfpathlineto{\pgfqpoint{1.227831in}{0.754037in}}%
\pgfpathlineto{\pgfqpoint{1.231382in}{0.753869in}}%
\pgfpathlineto{\pgfqpoint{1.236707in}{0.753603in}}%
\pgfpathlineto{\pgfqpoint{1.243808in}{0.757993in}}%
\pgfpathlineto{\pgfqpoint{1.250909in}{0.760955in}}%
\pgfpathlineto{\pgfqpoint{1.252684in}{0.759172in}}%
\pgfpathlineto{\pgfqpoint{1.256235in}{0.760540in}}%
\pgfpathlineto{\pgfqpoint{1.259785in}{0.764149in}}%
\pgfpathlineto{\pgfqpoint{1.272212in}{0.780568in}}%
\pgfpathlineto{\pgfqpoint{1.275762in}{0.781427in}}%
\pgfpathlineto{\pgfqpoint{1.286414in}{0.782793in}}%
\pgfpathlineto{\pgfqpoint{1.289964in}{0.784631in}}%
\pgfpathlineto{\pgfqpoint{1.295290in}{0.788584in}}%
\pgfpathlineto{\pgfqpoint{1.298840in}{0.788749in}}%
\pgfpathlineto{\pgfqpoint{1.302391in}{0.786856in}}%
\pgfpathlineto{\pgfqpoint{1.313042in}{0.777077in}}%
\pgfpathlineto{\pgfqpoint{1.314817in}{0.776905in}}%
\pgfpathlineto{\pgfqpoint{1.316593in}{0.778254in}}%
\pgfpathlineto{\pgfqpoint{1.318368in}{0.777578in}}%
\pgfpathlineto{\pgfqpoint{1.329019in}{0.787714in}}%
\pgfpathlineto{\pgfqpoint{1.337895in}{0.790918in}}%
\pgfpathlineto{\pgfqpoint{1.344996in}{0.793769in}}%
\pgfpathlineto{\pgfqpoint{1.364524in}{0.808430in}}%
\pgfpathlineto{\pgfqpoint{1.375175in}{0.809878in}}%
\pgfpathlineto{\pgfqpoint{1.378726in}{0.813688in}}%
\pgfpathlineto{\pgfqpoint{1.392928in}{0.822158in}}%
\pgfpathlineto{\pgfqpoint{1.396478in}{0.826993in}}%
\pgfpathlineto{\pgfqpoint{1.398253in}{0.828216in}}%
\pgfpathlineto{\pgfqpoint{1.400029in}{0.827348in}}%
\pgfpathlineto{\pgfqpoint{1.412455in}{0.833728in}}%
\pgfpathlineto{\pgfqpoint{1.421331in}{0.837219in}}%
\pgfpathlineto{\pgfqpoint{1.431983in}{0.846140in}}%
\pgfpathlineto{\pgfqpoint{1.439084in}{0.847204in}}%
\pgfpathlineto{\pgfqpoint{1.451510in}{0.845765in}}%
\pgfpathlineto{\pgfqpoint{1.456836in}{0.848577in}}%
\pgfpathlineto{\pgfqpoint{1.462162in}{0.849028in}}%
\pgfpathlineto{\pgfqpoint{1.467487in}{0.848716in}}%
\pgfpathlineto{\pgfqpoint{1.472813in}{0.849898in}}%
\pgfpathlineto{\pgfqpoint{1.478139in}{0.851426in}}%
\pgfpathlineto{\pgfqpoint{1.483465in}{0.850258in}}%
\pgfpathlineto{\pgfqpoint{1.494116in}{0.845136in}}%
\pgfpathlineto{\pgfqpoint{1.499442in}{0.845582in}}%
\pgfpathlineto{\pgfqpoint{1.506543in}{0.847238in}}%
\pgfpathlineto{\pgfqpoint{1.517194in}{0.847181in}}%
\pgfpathlineto{\pgfqpoint{1.522520in}{0.852018in}}%
\pgfpathlineto{\pgfqpoint{1.527845in}{0.856615in}}%
\pgfpathlineto{\pgfqpoint{1.533171in}{0.857412in}}%
\pgfpathlineto{\pgfqpoint{1.538497in}{0.858596in}}%
\pgfpathlineto{\pgfqpoint{1.549148in}{0.864184in}}%
\pgfpathlineto{\pgfqpoint{1.556249in}{0.862901in}}%
\pgfpathlineto{\pgfqpoint{1.561575in}{0.863340in}}%
\pgfpathlineto{\pgfqpoint{1.572226in}{0.866988in}}%
\pgfpathlineto{\pgfqpoint{1.584653in}{0.868676in}}%
\pgfpathlineto{\pgfqpoint{1.589979in}{0.866634in}}%
\pgfpathlineto{\pgfqpoint{1.595304in}{0.863727in}}%
\pgfpathlineto{\pgfqpoint{1.598855in}{0.863786in}}%
\pgfpathlineto{\pgfqpoint{1.607731in}{0.865301in}}%
\pgfpathlineto{\pgfqpoint{1.621933in}{0.862303in}}%
\pgfpathlineto{\pgfqpoint{1.653887in}{0.870581in}}%
\pgfpathlineto{\pgfqpoint{1.666314in}{0.876549in}}%
\pgfpathlineto{\pgfqpoint{1.684066in}{0.877463in}}%
\pgfpathlineto{\pgfqpoint{1.694717in}{0.882019in}}%
\pgfpathlineto{\pgfqpoint{1.705369in}{0.881490in}}%
\pgfpathlineto{\pgfqpoint{1.710694in}{0.878335in}}%
\pgfpathlineto{\pgfqpoint{1.716020in}{0.875445in}}%
\pgfpathlineto{\pgfqpoint{1.721346in}{0.875354in}}%
\pgfpathlineto{\pgfqpoint{1.731997in}{0.877762in}}%
\pgfpathlineto{\pgfqpoint{1.737323in}{0.876117in}}%
\pgfpathlineto{\pgfqpoint{1.746199in}{0.869597in}}%
\pgfpathlineto{\pgfqpoint{1.749749in}{0.870194in}}%
\pgfpathlineto{\pgfqpoint{1.753300in}{0.873021in}}%
\pgfpathlineto{\pgfqpoint{1.772827in}{0.893173in}}%
\pgfpathlineto{\pgfqpoint{1.778153in}{0.897731in}}%
\pgfpathlineto{\pgfqpoint{1.783479in}{0.904972in}}%
\pgfpathlineto{\pgfqpoint{1.785254in}{0.905759in}}%
\pgfpathlineto{\pgfqpoint{1.787029in}{0.904554in}}%
\pgfpathlineto{\pgfqpoint{1.792355in}{0.894519in}}%
\pgfpathlineto{\pgfqpoint{1.799456in}{0.880703in}}%
\pgfpathlineto{\pgfqpoint{1.803006in}{0.877979in}}%
\pgfpathlineto{\pgfqpoint{1.806557in}{0.878040in}}%
\pgfpathlineto{\pgfqpoint{1.813658in}{0.880517in}}%
\pgfpathlineto{\pgfqpoint{1.827860in}{0.886524in}}%
\pgfpathlineto{\pgfqpoint{1.831410in}{0.884752in}}%
\pgfpathlineto{\pgfqpoint{1.840286in}{0.877946in}}%
\pgfpathlineto{\pgfqpoint{1.847387in}{0.876655in}}%
\pgfpathlineto{\pgfqpoint{1.856263in}{0.877782in}}%
\pgfpathlineto{\pgfqpoint{1.865140in}{0.880815in}}%
\pgfpathlineto{\pgfqpoint{1.870465in}{0.879366in}}%
\pgfpathlineto{\pgfqpoint{1.875791in}{0.878418in}}%
\pgfpathlineto{\pgfqpoint{1.889993in}{0.876122in}}%
\pgfpathlineto{\pgfqpoint{1.895319in}{0.871041in}}%
\pgfpathlineto{\pgfqpoint{1.902419in}{0.864046in}}%
\pgfpathlineto{\pgfqpoint{1.916621in}{0.857648in}}%
\pgfpathlineto{\pgfqpoint{1.923722in}{0.859859in}}%
\pgfpathlineto{\pgfqpoint{1.945025in}{0.869704in}}%
\pgfpathlineto{\pgfqpoint{1.950351in}{0.870877in}}%
\pgfpathlineto{\pgfqpoint{1.957452in}{0.872848in}}%
\pgfpathlineto{\pgfqpoint{1.962777in}{0.870100in}}%
\pgfpathlineto{\pgfqpoint{1.966328in}{0.868048in}}%
\pgfpathlineto{\pgfqpoint{1.969878in}{0.868406in}}%
\pgfpathlineto{\pgfqpoint{1.973429in}{0.871842in}}%
\pgfpathlineto{\pgfqpoint{1.982305in}{0.884168in}}%
\pgfpathlineto{\pgfqpoint{1.985855in}{0.884945in}}%
\pgfpathlineto{\pgfqpoint{1.994732in}{0.885511in}}%
\pgfpathlineto{\pgfqpoint{2.003608in}{0.889415in}}%
\pgfpathlineto{\pgfqpoint{2.008933in}{0.888422in}}%
\pgfpathlineto{\pgfqpoint{2.014259in}{0.886207in}}%
\pgfpathlineto{\pgfqpoint{2.017810in}{0.886768in}}%
\pgfpathlineto{\pgfqpoint{2.021360in}{0.889533in}}%
\pgfpathlineto{\pgfqpoint{2.032011in}{0.899322in}}%
\pgfpathlineto{\pgfqpoint{2.035562in}{0.897529in}}%
\pgfpathlineto{\pgfqpoint{2.042663in}{0.887895in}}%
\pgfpathlineto{\pgfqpoint{2.047989in}{0.881886in}}%
\pgfpathlineto{\pgfqpoint{2.051539in}{0.880377in}}%
\pgfpathlineto{\pgfqpoint{2.058640in}{0.880218in}}%
\pgfpathlineto{\pgfqpoint{2.076392in}{0.882034in}}%
\pgfpathlineto{\pgfqpoint{2.090594in}{0.889926in}}%
\pgfpathlineto{\pgfqpoint{2.103021in}{0.890005in}}%
\pgfpathlineto{\pgfqpoint{2.113672in}{0.884655in}}%
\pgfpathlineto{\pgfqpoint{2.126099in}{0.886364in}}%
\pgfpathlineto{\pgfqpoint{2.131424in}{0.885289in}}%
\pgfpathlineto{\pgfqpoint{2.147402in}{0.887323in}}%
\pgfpathlineto{\pgfqpoint{2.152727in}{0.886449in}}%
\pgfpathlineto{\pgfqpoint{2.161603in}{0.887308in}}%
\pgfpathlineto{\pgfqpoint{2.170480in}{0.885231in}}%
\pgfpathlineto{\pgfqpoint{2.184681in}{0.889882in}}%
\pgfpathlineto{\pgfqpoint{2.195333in}{0.890058in}}%
\pgfpathlineto{\pgfqpoint{2.200659in}{0.892934in}}%
\pgfpathlineto{\pgfqpoint{2.207759in}{0.896645in}}%
\pgfpathlineto{\pgfqpoint{2.211310in}{0.898467in}}%
\pgfpathlineto{\pgfqpoint{2.216636in}{0.903956in}}%
\pgfpathlineto{\pgfqpoint{2.223737in}{0.909547in}}%
\pgfpathlineto{\pgfqpoint{2.237938in}{0.916394in}}%
\pgfpathlineto{\pgfqpoint{2.246815in}{0.916868in}}%
\pgfpathlineto{\pgfqpoint{2.268117in}{0.913840in}}%
\pgfpathlineto{\pgfqpoint{2.284094in}{0.912813in}}%
\pgfpathlineto{\pgfqpoint{2.292971in}{0.915863in}}%
\pgfpathlineto{\pgfqpoint{2.298296in}{0.917665in}}%
\pgfpathlineto{\pgfqpoint{2.303622in}{0.915768in}}%
\pgfpathlineto{\pgfqpoint{2.310723in}{0.909188in}}%
\pgfpathlineto{\pgfqpoint{2.317824in}{0.902642in}}%
\pgfpathlineto{\pgfqpoint{2.324925in}{0.899290in}}%
\pgfpathlineto{\pgfqpoint{2.328475in}{0.899965in}}%
\pgfpathlineto{\pgfqpoint{2.333801in}{0.904055in}}%
\pgfpathlineto{\pgfqpoint{2.348003in}{0.915555in}}%
\pgfpathlineto{\pgfqpoint{2.351553in}{0.914654in}}%
\pgfpathlineto{\pgfqpoint{2.372856in}{0.904183in}}%
\pgfpathlineto{\pgfqpoint{2.376407in}{0.904304in}}%
\pgfpathlineto{\pgfqpoint{2.381732in}{0.906899in}}%
\pgfpathlineto{\pgfqpoint{2.387058in}{0.911504in}}%
\pgfpathlineto{\pgfqpoint{2.394159in}{0.918167in}}%
\pgfpathlineto{\pgfqpoint{2.399485in}{0.920504in}}%
\pgfpathlineto{\pgfqpoint{2.406586in}{0.920315in}}%
\pgfpathlineto{\pgfqpoint{2.410136in}{0.920846in}}%
\pgfpathlineto{\pgfqpoint{2.417237in}{0.923942in}}%
\pgfpathlineto{\pgfqpoint{2.420787in}{0.925473in}}%
\pgfpathlineto{\pgfqpoint{2.433214in}{0.928402in}}%
\pgfpathlineto{\pgfqpoint{2.443865in}{0.938120in}}%
\pgfpathlineto{\pgfqpoint{2.452742in}{0.947106in}}%
\pgfpathlineto{\pgfqpoint{2.458067in}{0.950287in}}%
\pgfpathlineto{\pgfqpoint{2.468719in}{0.953713in}}%
\pgfpathlineto{\pgfqpoint{2.472269in}{0.954378in}}%
\pgfpathlineto{\pgfqpoint{2.491797in}{0.947820in}}%
\pgfpathlineto{\pgfqpoint{2.497122in}{0.946542in}}%
\pgfpathlineto{\pgfqpoint{2.502448in}{0.947779in}}%
\pgfpathlineto{\pgfqpoint{2.507774in}{0.949370in}}%
\pgfpathlineto{\pgfqpoint{2.511324in}{0.950763in}}%
\pgfpathlineto{\pgfqpoint{2.525526in}{0.952010in}}%
\pgfpathlineto{\pgfqpoint{2.532627in}{0.953162in}}%
\pgfpathlineto{\pgfqpoint{2.545054in}{0.948521in}}%
\pgfpathlineto{\pgfqpoint{2.552155in}{0.945294in}}%
\pgfpathlineto{\pgfqpoint{2.564581in}{0.943708in}}%
\pgfpathlineto{\pgfqpoint{2.575233in}{0.948793in}}%
\pgfpathlineto{\pgfqpoint{2.578783in}{0.950482in}}%
\pgfpathlineto{\pgfqpoint{2.592985in}{0.957690in}}%
\pgfpathlineto{\pgfqpoint{2.598311in}{0.956065in}}%
\pgfpathlineto{\pgfqpoint{2.607187in}{0.951347in}}%
\pgfpathlineto{\pgfqpoint{2.612513in}{0.952443in}}%
\pgfpathlineto{\pgfqpoint{2.617838in}{0.954145in}}%
\pgfpathlineto{\pgfqpoint{2.621389in}{0.952943in}}%
\pgfpathlineto{\pgfqpoint{2.630265in}{0.948191in}}%
\pgfpathlineto{\pgfqpoint{2.635591in}{0.950291in}}%
\pgfpathlineto{\pgfqpoint{2.639141in}{0.953266in}}%
\pgfpathlineto{\pgfqpoint{2.651568in}{0.965302in}}%
\pgfpathlineto{\pgfqpoint{2.655118in}{0.966038in}}%
\pgfpathlineto{\pgfqpoint{2.658669in}{0.964705in}}%
\pgfpathlineto{\pgfqpoint{2.665770in}{0.956747in}}%
\pgfpathlineto{\pgfqpoint{2.672870in}{0.947386in}}%
\pgfpathlineto{\pgfqpoint{2.690623in}{0.937547in}}%
\pgfpathlineto{\pgfqpoint{2.699499in}{0.938735in}}%
\pgfpathlineto{\pgfqpoint{2.713701in}{0.941539in}}%
\pgfpathlineto{\pgfqpoint{2.731453in}{0.934385in}}%
\pgfpathlineto{\pgfqpoint{2.735004in}{0.933481in}}%
\pgfpathlineto{\pgfqpoint{2.738554in}{0.934337in}}%
\pgfpathlineto{\pgfqpoint{2.743880in}{0.939448in}}%
\pgfpathlineto{\pgfqpoint{2.750981in}{0.946192in}}%
\pgfpathlineto{\pgfqpoint{2.765183in}{0.954362in}}%
\pgfpathlineto{\pgfqpoint{2.770508in}{0.961298in}}%
\pgfpathlineto{\pgfqpoint{2.779384in}{0.972257in}}%
\pgfpathlineto{\pgfqpoint{2.782935in}{0.973801in}}%
\pgfpathlineto{\pgfqpoint{2.788261in}{0.971174in}}%
\pgfpathlineto{\pgfqpoint{2.793586in}{0.968051in}}%
\pgfpathlineto{\pgfqpoint{2.806013in}{0.970164in}}%
\pgfpathlineto{\pgfqpoint{2.825540in}{0.974936in}}%
\pgfpathlineto{\pgfqpoint{2.832641in}{0.970340in}}%
\pgfpathlineto{\pgfqpoint{2.836192in}{0.966345in}}%
\pgfpathlineto{\pgfqpoint{2.846843in}{0.954201in}}%
\pgfpathlineto{\pgfqpoint{2.853944in}{0.949807in}}%
\pgfpathlineto{\pgfqpoint{2.861045in}{0.950142in}}%
\pgfpathlineto{\pgfqpoint{2.868146in}{0.949239in}}%
\pgfpathlineto{\pgfqpoint{2.873472in}{0.944381in}}%
\pgfpathlineto{\pgfqpoint{2.877022in}{0.940823in}}%
\pgfpathlineto{\pgfqpoint{2.880573in}{0.938678in}}%
\pgfpathlineto{\pgfqpoint{2.885898in}{0.939538in}}%
\pgfpathlineto{\pgfqpoint{2.891224in}{0.943591in}}%
\pgfpathlineto{\pgfqpoint{2.896550in}{0.948017in}}%
\pgfpathlineto{\pgfqpoint{2.903651in}{0.956675in}}%
\pgfpathlineto{\pgfqpoint{2.916077in}{0.971604in}}%
\pgfpathlineto{\pgfqpoint{2.921403in}{0.972571in}}%
\pgfpathlineto{\pgfqpoint{2.926729in}{0.973120in}}%
\pgfpathlineto{\pgfqpoint{2.932054in}{0.975741in}}%
\pgfpathlineto{\pgfqpoint{2.937380in}{0.979538in}}%
\pgfpathlineto{\pgfqpoint{2.942706in}{0.979272in}}%
\pgfpathlineto{\pgfqpoint{2.953357in}{0.975263in}}%
\pgfpathlineto{\pgfqpoint{2.965784in}{0.979012in}}%
\pgfpathlineto{\pgfqpoint{2.972885in}{0.982943in}}%
\pgfpathlineto{\pgfqpoint{2.976435in}{0.987893in}}%
\pgfpathlineto{\pgfqpoint{2.985311in}{1.001506in}}%
\pgfpathlineto{\pgfqpoint{2.994188in}{1.006701in}}%
\pgfpathlineto{\pgfqpoint{3.001288in}{1.006734in}}%
\pgfpathlineto{\pgfqpoint{3.004839in}{1.004799in}}%
\pgfpathlineto{\pgfqpoint{3.011940in}{0.999268in}}%
\pgfpathlineto{\pgfqpoint{3.017266in}{0.994807in}}%
\pgfpathlineto{\pgfqpoint{3.022591in}{0.993073in}}%
\pgfpathlineto{\pgfqpoint{3.042119in}{1.000086in}}%
\pgfpathlineto{\pgfqpoint{3.043894in}{1.001390in}}%
\pgfpathlineto{\pgfqpoint{3.054545in}{1.017460in}}%
\pgfpathlineto{\pgfqpoint{3.058096in}{1.018556in}}%
\pgfpathlineto{\pgfqpoint{3.059871in}{1.018227in}}%
\pgfpathlineto{\pgfqpoint{3.065197in}{1.014871in}}%
\pgfpathlineto{\pgfqpoint{3.074073in}{1.014040in}}%
\pgfpathlineto{\pgfqpoint{3.086500in}{1.024224in}}%
\pgfpathlineto{\pgfqpoint{3.098926in}{1.036661in}}%
\pgfpathlineto{\pgfqpoint{3.102477in}{1.036848in}}%
\pgfpathlineto{\pgfqpoint{3.104252in}{1.035145in}}%
\pgfpathlineto{\pgfqpoint{3.107802in}{1.034682in}}%
\pgfpathlineto{\pgfqpoint{3.114903in}{1.030920in}}%
\pgfpathlineto{\pgfqpoint{3.118454in}{1.029206in}}%
\pgfpathlineto{\pgfqpoint{3.123780in}{1.030014in}}%
\pgfpathlineto{\pgfqpoint{3.127330in}{1.031703in}}%
\pgfpathlineto{\pgfqpoint{3.134431in}{1.036442in}}%
\pgfpathlineto{\pgfqpoint{3.141532in}{1.043295in}}%
\pgfpathlineto{\pgfqpoint{3.150408in}{1.049635in}}%
\pgfpathlineto{\pgfqpoint{3.153958in}{1.048487in}}%
\pgfpathlineto{\pgfqpoint{3.162835in}{1.036915in}}%
\pgfpathlineto{\pgfqpoint{3.173486in}{1.019630in}}%
\pgfpathlineto{\pgfqpoint{3.184137in}{1.001552in}}%
\pgfpathlineto{\pgfqpoint{3.187688in}{0.998983in}}%
\pgfpathlineto{\pgfqpoint{3.193014in}{0.998834in}}%
\pgfpathlineto{\pgfqpoint{3.196564in}{1.000375in}}%
\pgfpathlineto{\pgfqpoint{3.203665in}{1.003799in}}%
\pgfpathlineto{\pgfqpoint{3.208991in}{1.003927in}}%
\pgfpathlineto{\pgfqpoint{3.221417in}{1.004467in}}%
\pgfpathlineto{\pgfqpoint{3.228518in}{1.008105in}}%
\pgfpathlineto{\pgfqpoint{3.232069in}{1.012367in}}%
\pgfpathlineto{\pgfqpoint{3.237394in}{1.021892in}}%
\pgfpathlineto{\pgfqpoint{3.246271in}{1.039437in}}%
\pgfpathlineto{\pgfqpoint{3.249821in}{1.040628in}}%
\pgfpathlineto{\pgfqpoint{3.253372in}{1.036160in}}%
\pgfpathlineto{\pgfqpoint{3.258697in}{1.023082in}}%
\pgfpathlineto{\pgfqpoint{3.264023in}{1.002098in}}%
\pgfpathlineto{\pgfqpoint{3.272899in}{0.944990in}}%
\pgfpathlineto{\pgfqpoint{3.278225in}{0.883515in}}%
\pgfpathlineto{\pgfqpoint{3.280000in}{0.855086in}}%
\pgfpathlineto{\pgfqpoint{3.283550in}{0.787301in}}%
\pgfpathlineto{\pgfqpoint{3.287101in}{0.761945in}}%
\pgfpathlineto{\pgfqpoint{3.288876in}{0.758161in}}%
\pgfpathlineto{\pgfqpoint{3.292427in}{0.745071in}}%
\pgfpathlineto{\pgfqpoint{3.295977in}{0.748709in}}%
\pgfpathlineto{\pgfqpoint{3.297752in}{0.759865in}}%
\pgfpathlineto{\pgfqpoint{3.301303in}{0.813691in}}%
\pgfpathlineto{\pgfqpoint{3.304853in}{0.928018in}}%
\pgfpathlineto{\pgfqpoint{3.319055in}{1.431832in}}%
\pgfpathlineto{\pgfqpoint{3.324381in}{1.541751in}}%
\pgfpathlineto{\pgfqpoint{3.333257in}{1.675017in}}%
\pgfpathlineto{\pgfqpoint{3.351009in}{1.910595in}}%
\pgfpathlineto{\pgfqpoint{3.354560in}{1.935563in}}%
\pgfpathlineto{\pgfqpoint{3.358110in}{1.945113in}}%
\pgfpathlineto{\pgfqpoint{3.359885in}{1.943295in}}%
\pgfpathlineto{\pgfqpoint{3.363436in}{1.928268in}}%
\pgfpathlineto{\pgfqpoint{3.368762in}{1.884166in}}%
\pgfpathlineto{\pgfqpoint{3.375863in}{1.821311in}}%
\pgfpathlineto{\pgfqpoint{3.381188in}{1.796836in}}%
\pgfpathlineto{\pgfqpoint{3.384739in}{1.789683in}}%
\pgfpathlineto{\pgfqpoint{3.388289in}{1.786845in}}%
\pgfpathlineto{\pgfqpoint{3.391840in}{1.785078in}}%
\pgfpathlineto{\pgfqpoint{3.393615in}{1.785987in}}%
\pgfpathlineto{\pgfqpoint{3.398941in}{1.796601in}}%
\pgfpathlineto{\pgfqpoint{3.400716in}{1.797165in}}%
\pgfpathlineto{\pgfqpoint{3.402491in}{1.795675in}}%
\pgfpathlineto{\pgfqpoint{3.406042in}{1.788074in}}%
\pgfpathlineto{\pgfqpoint{3.411367in}{1.768398in}}%
\pgfpathlineto{\pgfqpoint{3.432670in}{1.677098in}}%
\pgfpathlineto{\pgfqpoint{3.436220in}{1.672835in}}%
\pgfpathlineto{\pgfqpoint{3.439771in}{1.671365in}}%
\pgfpathlineto{\pgfqpoint{3.446872in}{1.669814in}}%
\pgfpathlineto{\pgfqpoint{3.452198in}{1.668894in}}%
\pgfpathlineto{\pgfqpoint{3.455748in}{1.670964in}}%
\pgfpathlineto{\pgfqpoint{3.459298in}{1.677652in}}%
\pgfpathlineto{\pgfqpoint{3.471725in}{1.707364in}}%
\pgfpathlineto{\pgfqpoint{3.484152in}{1.719535in}}%
\pgfpathlineto{\pgfqpoint{3.489477in}{1.731356in}}%
\pgfpathlineto{\pgfqpoint{3.498354in}{1.759373in}}%
\pgfpathlineto{\pgfqpoint{3.505455in}{1.783583in}}%
\pgfpathlineto{\pgfqpoint{3.510780in}{1.807782in}}%
\pgfpathlineto{\pgfqpoint{3.519656in}{1.851249in}}%
\pgfpathlineto{\pgfqpoint{3.523207in}{1.859231in}}%
\pgfpathlineto{\pgfqpoint{3.524982in}{1.859995in}}%
\pgfpathlineto{\pgfqpoint{3.526757in}{1.859407in}}%
\pgfpathlineto{\pgfqpoint{3.530308in}{1.854788in}}%
\pgfpathlineto{\pgfqpoint{3.537409in}{1.838119in}}%
\pgfpathlineto{\pgfqpoint{3.544510in}{1.821837in}}%
\pgfpathlineto{\pgfqpoint{3.546285in}{1.820329in}}%
\pgfpathlineto{\pgfqpoint{3.548060in}{1.820628in}}%
\pgfpathlineto{\pgfqpoint{3.549835in}{1.822676in}}%
\pgfpathlineto{\pgfqpoint{3.553386in}{1.832520in}}%
\pgfpathlineto{\pgfqpoint{3.564037in}{1.875803in}}%
\pgfpathlineto{\pgfqpoint{3.569363in}{1.905417in}}%
\pgfpathlineto{\pgfqpoint{3.576464in}{1.947777in}}%
\pgfpathlineto{\pgfqpoint{3.580014in}{1.960235in}}%
\pgfpathlineto{\pgfqpoint{3.583565in}{1.965635in}}%
\pgfpathlineto{\pgfqpoint{3.606643in}{1.979194in}}%
\pgfpathlineto{\pgfqpoint{3.610193in}{1.978105in}}%
\pgfpathlineto{\pgfqpoint{3.617294in}{1.971458in}}%
\pgfpathlineto{\pgfqpoint{3.622620in}{1.964194in}}%
\pgfpathlineto{\pgfqpoint{3.629721in}{1.959487in}}%
\pgfpathlineto{\pgfqpoint{3.633271in}{1.955724in}}%
\pgfpathlineto{\pgfqpoint{3.636822in}{1.947249in}}%
\pgfpathlineto{\pgfqpoint{3.647473in}{1.917643in}}%
\pgfpathlineto{\pgfqpoint{3.651024in}{1.916265in}}%
\pgfpathlineto{\pgfqpoint{3.652799in}{1.916509in}}%
\pgfpathlineto{\pgfqpoint{3.654574in}{1.918614in}}%
\pgfpathlineto{\pgfqpoint{3.658125in}{1.916739in}}%
\pgfpathlineto{\pgfqpoint{3.661675in}{1.910466in}}%
\pgfpathlineto{\pgfqpoint{3.674102in}{1.880201in}}%
\pgfpathlineto{\pgfqpoint{3.677652in}{1.879650in}}%
\pgfpathlineto{\pgfqpoint{3.681203in}{1.883099in}}%
\pgfpathlineto{\pgfqpoint{3.691854in}{1.901104in}}%
\pgfpathlineto{\pgfqpoint{3.695404in}{1.901634in}}%
\pgfpathlineto{\pgfqpoint{3.704281in}{1.898364in}}%
\pgfpathlineto{\pgfqpoint{3.707831in}{1.900776in}}%
\pgfpathlineto{\pgfqpoint{3.713157in}{1.909874in}}%
\pgfpathlineto{\pgfqpoint{3.720258in}{1.926794in}}%
\pgfpathlineto{\pgfqpoint{3.725583in}{1.942250in}}%
\pgfpathlineto{\pgfqpoint{3.738010in}{1.980822in}}%
\pgfpathlineto{\pgfqpoint{3.741560in}{1.986390in}}%
\pgfpathlineto{\pgfqpoint{3.746886in}{1.984409in}}%
\pgfpathlineto{\pgfqpoint{3.752212in}{1.975739in}}%
\pgfpathlineto{\pgfqpoint{3.757538in}{1.956214in}}%
\pgfpathlineto{\pgfqpoint{3.762863in}{1.930599in}}%
\pgfpathlineto{\pgfqpoint{3.771739in}{1.883599in}}%
\pgfpathlineto{\pgfqpoint{3.775290in}{1.873142in}}%
\pgfpathlineto{\pgfqpoint{3.777065in}{1.871037in}}%
\pgfpathlineto{\pgfqpoint{3.778840in}{1.870811in}}%
\pgfpathlineto{\pgfqpoint{3.782391in}{1.875054in}}%
\pgfpathlineto{\pgfqpoint{3.787717in}{1.891025in}}%
\pgfpathlineto{\pgfqpoint{3.794817in}{1.913640in}}%
\pgfpathlineto{\pgfqpoint{3.796593in}{1.916096in}}%
\pgfpathlineto{\pgfqpoint{3.798368in}{1.916302in}}%
\pgfpathlineto{\pgfqpoint{3.801918in}{1.912839in}}%
\pgfpathlineto{\pgfqpoint{3.805469in}{1.908988in}}%
\pgfpathlineto{\pgfqpoint{3.807244in}{1.908364in}}%
\pgfpathlineto{\pgfqpoint{3.809019in}{1.909366in}}%
\pgfpathlineto{\pgfqpoint{3.812570in}{1.915316in}}%
\pgfpathlineto{\pgfqpoint{3.821446in}{1.931652in}}%
\pgfpathlineto{\pgfqpoint{3.828547in}{1.941623in}}%
\pgfpathlineto{\pgfqpoint{3.837423in}{1.958887in}}%
\pgfpathlineto{\pgfqpoint{3.840974in}{1.961671in}}%
\pgfpathlineto{\pgfqpoint{3.849850in}{1.962467in}}%
\pgfpathlineto{\pgfqpoint{3.853400in}{1.967988in}}%
\pgfpathlineto{\pgfqpoint{3.860501in}{1.986000in}}%
\pgfpathlineto{\pgfqpoint{3.864052in}{1.998639in}}%
\pgfpathlineto{\pgfqpoint{3.867602in}{2.004922in}}%
\pgfpathlineto{\pgfqpoint{3.871152in}{2.006478in}}%
\pgfpathlineto{\pgfqpoint{3.878253in}{2.008140in}}%
\pgfpathlineto{\pgfqpoint{3.881804in}{2.011030in}}%
\pgfpathlineto{\pgfqpoint{3.887130in}{2.018226in}}%
\pgfpathlineto{\pgfqpoint{3.894230in}{2.030513in}}%
\pgfpathlineto{\pgfqpoint{3.899556in}{2.047351in}}%
\pgfpathlineto{\pgfqpoint{3.913758in}{2.101078in}}%
\pgfpathlineto{\pgfqpoint{3.920859in}{2.126879in}}%
\pgfpathlineto{\pgfqpoint{3.926185in}{2.138216in}}%
\pgfpathlineto{\pgfqpoint{3.935061in}{2.149226in}}%
\pgfpathlineto{\pgfqpoint{3.938611in}{2.151249in}}%
\pgfpathlineto{\pgfqpoint{3.945712in}{2.142092in}}%
\pgfpathlineto{\pgfqpoint{3.954588in}{2.127423in}}%
\pgfpathlineto{\pgfqpoint{3.961689in}{2.129996in}}%
\pgfpathlineto{\pgfqpoint{3.967015in}{2.138577in}}%
\pgfpathlineto{\pgfqpoint{3.972341in}{2.150228in}}%
\pgfpathlineto{\pgfqpoint{3.977666in}{2.170505in}}%
\pgfpathlineto{\pgfqpoint{3.988318in}{2.228280in}}%
\pgfpathlineto{\pgfqpoint{3.990093in}{2.229744in}}%
\pgfpathlineto{\pgfqpoint{3.991868in}{2.227061in}}%
\pgfpathlineto{\pgfqpoint{3.995419in}{2.211700in}}%
\pgfpathlineto{\pgfqpoint{4.000744in}{2.183967in}}%
\pgfpathlineto{\pgfqpoint{4.004295in}{2.174097in}}%
\pgfpathlineto{\pgfqpoint{4.007845in}{2.170412in}}%
\pgfpathlineto{\pgfqpoint{4.013171in}{2.165933in}}%
\pgfpathlineto{\pgfqpoint{4.018497in}{2.161169in}}%
\pgfpathlineto{\pgfqpoint{4.022047in}{2.161427in}}%
\pgfpathlineto{\pgfqpoint{4.025598in}{2.166873in}}%
\pgfpathlineto{\pgfqpoint{4.036249in}{2.189534in}}%
\pgfpathlineto{\pgfqpoint{4.039800in}{2.190969in}}%
\pgfpathlineto{\pgfqpoint{4.045125in}{2.189146in}}%
\pgfpathlineto{\pgfqpoint{4.057552in}{2.181417in}}%
\pgfpathlineto{\pgfqpoint{4.069979in}{2.167928in}}%
\pgfpathlineto{\pgfqpoint{4.071754in}{2.168080in}}%
\pgfpathlineto{\pgfqpoint{4.073529in}{2.166821in}}%
\pgfpathlineto{\pgfqpoint{4.075304in}{2.167325in}}%
\pgfpathlineto{\pgfqpoint{4.078855in}{2.172123in}}%
\pgfpathlineto{\pgfqpoint{4.082405in}{2.185793in}}%
\pgfpathlineto{\pgfqpoint{4.091281in}{2.228680in}}%
\pgfpathlineto{\pgfqpoint{4.094832in}{2.237952in}}%
\pgfpathlineto{\pgfqpoint{4.096607in}{2.239081in}}%
\pgfpathlineto{\pgfqpoint{4.100157in}{2.237539in}}%
\pgfpathlineto{\pgfqpoint{4.105483in}{2.230954in}}%
\pgfpathlineto{\pgfqpoint{4.109034in}{2.223498in}}%
\pgfpathlineto{\pgfqpoint{4.114359in}{2.217617in}}%
\pgfpathlineto{\pgfqpoint{4.117910in}{2.210426in}}%
\pgfpathlineto{\pgfqpoint{4.123236in}{2.185032in}}%
\pgfpathlineto{\pgfqpoint{4.128561in}{2.158071in}}%
\pgfpathlineto{\pgfqpoint{4.132112in}{2.151908in}}%
\pgfpathlineto{\pgfqpoint{4.133887in}{2.153668in}}%
\pgfpathlineto{\pgfqpoint{4.137437in}{2.165584in}}%
\pgfpathlineto{\pgfqpoint{4.144538in}{2.191214in}}%
\pgfpathlineto{\pgfqpoint{4.153414in}{2.221018in}}%
\pgfpathlineto{\pgfqpoint{4.158740in}{2.235772in}}%
\pgfpathlineto{\pgfqpoint{4.160515in}{2.236912in}}%
\pgfpathlineto{\pgfqpoint{4.162291in}{2.235505in}}%
\pgfpathlineto{\pgfqpoint{4.165841in}{2.225492in}}%
\pgfpathlineto{\pgfqpoint{4.172942in}{2.189223in}}%
\pgfpathlineto{\pgfqpoint{4.196020in}{2.048068in}}%
\pgfpathlineto{\pgfqpoint{4.199571in}{2.039327in}}%
\pgfpathlineto{\pgfqpoint{4.201346in}{2.037976in}}%
\pgfpathlineto{\pgfqpoint{4.203121in}{2.039358in}}%
\pgfpathlineto{\pgfqpoint{4.206671in}{2.049167in}}%
\pgfpathlineto{\pgfqpoint{4.213772in}{2.083705in}}%
\pgfpathlineto{\pgfqpoint{4.220873in}{2.120026in}}%
\pgfpathlineto{\pgfqpoint{4.224424in}{2.126121in}}%
\pgfpathlineto{\pgfqpoint{4.226199in}{2.125839in}}%
\pgfpathlineto{\pgfqpoint{4.243951in}{2.084672in}}%
\pgfpathlineto{\pgfqpoint{4.245727in}{2.083101in}}%
\pgfpathlineto{\pgfqpoint{4.249277in}{2.082529in}}%
\pgfpathlineto{\pgfqpoint{4.252827in}{2.088152in}}%
\pgfpathlineto{\pgfqpoint{4.259928in}{2.103458in}}%
\pgfpathlineto{\pgfqpoint{4.265254in}{2.107674in}}%
\pgfpathlineto{\pgfqpoint{4.268805in}{2.107819in}}%
\pgfpathlineto{\pgfqpoint{4.272355in}{2.105729in}}%
\pgfpathlineto{\pgfqpoint{4.275906in}{2.099671in}}%
\pgfpathlineto{\pgfqpoint{4.277681in}{2.095363in}}%
\pgfpathlineto{\pgfqpoint{4.281231in}{2.074020in}}%
\pgfpathlineto{\pgfqpoint{4.293658in}{1.980988in}}%
\pgfpathlineto{\pgfqpoint{4.297208in}{1.972258in}}%
\pgfpathlineto{\pgfqpoint{4.298984in}{1.972275in}}%
\pgfpathlineto{\pgfqpoint{4.302534in}{1.978513in}}%
\pgfpathlineto{\pgfqpoint{4.306084in}{1.990317in}}%
\pgfpathlineto{\pgfqpoint{4.311410in}{2.020645in}}%
\pgfpathlineto{\pgfqpoint{4.323837in}{2.102189in}}%
\pgfpathlineto{\pgfqpoint{4.325612in}{2.106361in}}%
\pgfpathlineto{\pgfqpoint{4.327387in}{2.106571in}}%
\pgfpathlineto{\pgfqpoint{4.329162in}{2.104330in}}%
\pgfpathlineto{\pgfqpoint{4.332713in}{2.084470in}}%
\pgfpathlineto{\pgfqpoint{4.343364in}{2.008175in}}%
\pgfpathlineto{\pgfqpoint{4.346915in}{1.995492in}}%
\pgfpathlineto{\pgfqpoint{4.348690in}{1.993214in}}%
\pgfpathlineto{\pgfqpoint{4.350465in}{1.994149in}}%
\pgfpathlineto{\pgfqpoint{4.352241in}{1.998491in}}%
\pgfpathlineto{\pgfqpoint{4.355791in}{2.017123in}}%
\pgfpathlineto{\pgfqpoint{4.359341in}{2.043851in}}%
\pgfpathlineto{\pgfqpoint{4.368218in}{2.115938in}}%
\pgfpathlineto{\pgfqpoint{4.373543in}{2.142469in}}%
\pgfpathlineto{\pgfqpoint{4.378869in}{2.157699in}}%
\pgfpathlineto{\pgfqpoint{4.384195in}{2.167425in}}%
\pgfpathlineto{\pgfqpoint{4.387745in}{2.170234in}}%
\pgfpathlineto{\pgfqpoint{4.391296in}{2.168629in}}%
\pgfpathlineto{\pgfqpoint{4.394846in}{2.163071in}}%
\pgfpathlineto{\pgfqpoint{4.398397in}{2.150457in}}%
\pgfpathlineto{\pgfqpoint{4.403722in}{2.119690in}}%
\pgfpathlineto{\pgfqpoint{4.410823in}{2.077005in}}%
\pgfpathlineto{\pgfqpoint{4.414374in}{2.068645in}}%
\pgfpathlineto{\pgfqpoint{4.416149in}{2.069118in}}%
\pgfpathlineto{\pgfqpoint{4.419699in}{2.079479in}}%
\pgfpathlineto{\pgfqpoint{4.426800in}{2.108270in}}%
\pgfpathlineto{\pgfqpoint{4.430351in}{2.115666in}}%
\pgfpathlineto{\pgfqpoint{4.439227in}{2.121493in}}%
\pgfpathlineto{\pgfqpoint{4.442777in}{2.127055in}}%
\pgfpathlineto{\pgfqpoint{4.448103in}{2.136433in}}%
\pgfpathlineto{\pgfqpoint{4.451654in}{2.136757in}}%
\pgfpathlineto{\pgfqpoint{4.455204in}{2.133590in}}%
\pgfpathlineto{\pgfqpoint{4.460530in}{2.125801in}}%
\pgfpathlineto{\pgfqpoint{4.464080in}{2.118688in}}%
\pgfpathlineto{\pgfqpoint{4.471181in}{2.106275in}}%
\pgfpathlineto{\pgfqpoint{4.478282in}{2.088282in}}%
\pgfpathlineto{\pgfqpoint{4.480057in}{2.088128in}}%
\pgfpathlineto{\pgfqpoint{4.481832in}{2.090377in}}%
\pgfpathlineto{\pgfqpoint{4.485383in}{2.100230in}}%
\pgfpathlineto{\pgfqpoint{4.490709in}{2.115780in}}%
\pgfpathlineto{\pgfqpoint{4.492484in}{2.116355in}}%
\pgfpathlineto{\pgfqpoint{4.494259in}{2.114191in}}%
\pgfpathlineto{\pgfqpoint{4.497810in}{2.102263in}}%
\pgfpathlineto{\pgfqpoint{4.506686in}{2.064969in}}%
\pgfpathlineto{\pgfqpoint{4.512011in}{2.054832in}}%
\pgfpathlineto{\pgfqpoint{4.517337in}{2.045633in}}%
\pgfpathlineto{\pgfqpoint{4.522663in}{2.034701in}}%
\pgfpathlineto{\pgfqpoint{4.526213in}{2.032379in}}%
\pgfpathlineto{\pgfqpoint{4.527989in}{2.033535in}}%
\pgfpathlineto{\pgfqpoint{4.531539in}{2.039784in}}%
\pgfpathlineto{\pgfqpoint{4.551067in}{2.086349in}}%
\pgfpathlineto{\pgfqpoint{4.556392in}{2.100102in}}%
\pgfpathlineto{\pgfqpoint{4.558168in}{2.102474in}}%
\pgfpathlineto{\pgfqpoint{4.559943in}{2.102727in}}%
\pgfpathlineto{\pgfqpoint{4.561718in}{2.100598in}}%
\pgfpathlineto{\pgfqpoint{4.565268in}{2.088671in}}%
\pgfpathlineto{\pgfqpoint{4.568819in}{2.065585in}}%
\pgfpathlineto{\pgfqpoint{4.577695in}{2.000426in}}%
\pgfpathlineto{\pgfqpoint{4.581246in}{1.989257in}}%
\pgfpathlineto{\pgfqpoint{4.583021in}{1.987960in}}%
\pgfpathlineto{\pgfqpoint{4.584796in}{1.989019in}}%
\pgfpathlineto{\pgfqpoint{4.588346in}{1.996039in}}%
\pgfpathlineto{\pgfqpoint{4.595447in}{2.020651in}}%
\pgfpathlineto{\pgfqpoint{4.600773in}{2.039808in}}%
\pgfpathlineto{\pgfqpoint{4.604324in}{2.046218in}}%
\pgfpathlineto{\pgfqpoint{4.607874in}{2.048636in}}%
\pgfpathlineto{\pgfqpoint{4.613200in}{2.050086in}}%
\pgfpathlineto{\pgfqpoint{4.616750in}{2.054520in}}%
\pgfpathlineto{\pgfqpoint{4.622076in}{2.061562in}}%
\pgfpathlineto{\pgfqpoint{4.623851in}{2.062243in}}%
\pgfpathlineto{\pgfqpoint{4.625626in}{2.061440in}}%
\pgfpathlineto{\pgfqpoint{4.627402in}{2.058884in}}%
\pgfpathlineto{\pgfqpoint{4.630952in}{2.048000in}}%
\pgfpathlineto{\pgfqpoint{4.643379in}{1.997248in}}%
\pgfpathlineto{\pgfqpoint{4.645154in}{1.995555in}}%
\pgfpathlineto{\pgfqpoint{4.646929in}{1.996650in}}%
\pgfpathlineto{\pgfqpoint{4.648704in}{2.001265in}}%
\pgfpathlineto{\pgfqpoint{4.652255in}{2.019709in}}%
\pgfpathlineto{\pgfqpoint{4.664681in}{2.100828in}}%
\pgfpathlineto{\pgfqpoint{4.668232in}{2.106273in}}%
\pgfpathlineto{\pgfqpoint{4.670007in}{2.105812in}}%
\pgfpathlineto{\pgfqpoint{4.673558in}{2.100387in}}%
\pgfpathlineto{\pgfqpoint{4.677108in}{2.091604in}}%
\pgfpathlineto{\pgfqpoint{4.684209in}{2.064997in}}%
\pgfpathlineto{\pgfqpoint{4.693085in}{2.034668in}}%
\pgfpathlineto{\pgfqpoint{4.696636in}{2.032031in}}%
\pgfpathlineto{\pgfqpoint{4.698411in}{2.033345in}}%
\pgfpathlineto{\pgfqpoint{4.701961in}{2.041775in}}%
\pgfpathlineto{\pgfqpoint{4.707287in}{2.066043in}}%
\pgfpathlineto{\pgfqpoint{4.717938in}{2.127384in}}%
\pgfpathlineto{\pgfqpoint{4.721489in}{2.138737in}}%
\pgfpathlineto{\pgfqpoint{4.725039in}{2.141773in}}%
\pgfpathlineto{\pgfqpoint{4.728590in}{2.138417in}}%
\pgfpathlineto{\pgfqpoint{4.733916in}{2.130927in}}%
\pgfpathlineto{\pgfqpoint{4.737466in}{2.130182in}}%
\pgfpathlineto{\pgfqpoint{4.746342in}{2.131410in}}%
\pgfpathlineto{\pgfqpoint{4.749893in}{2.125997in}}%
\pgfpathlineto{\pgfqpoint{4.756994in}{2.110729in}}%
\pgfpathlineto{\pgfqpoint{4.760544in}{2.107864in}}%
\pgfpathlineto{\pgfqpoint{4.762319in}{2.108334in}}%
\pgfpathlineto{\pgfqpoint{4.765870in}{2.112471in}}%
\pgfpathlineto{\pgfqpoint{4.783622in}{2.146734in}}%
\pgfpathlineto{\pgfqpoint{4.785397in}{2.147133in}}%
\pgfpathlineto{\pgfqpoint{4.787173in}{2.145922in}}%
\pgfpathlineto{\pgfqpoint{4.788948in}{2.143056in}}%
\pgfpathlineto{\pgfqpoint{4.792498in}{2.130480in}}%
\pgfpathlineto{\pgfqpoint{4.801374in}{2.090141in}}%
\pgfpathlineto{\pgfqpoint{4.804925in}{2.082666in}}%
\pgfpathlineto{\pgfqpoint{4.808475in}{2.079740in}}%
\pgfpathlineto{\pgfqpoint{4.812026in}{2.081279in}}%
\pgfpathlineto{\pgfqpoint{4.815576in}{2.086620in}}%
\pgfpathlineto{\pgfqpoint{4.819127in}{2.094791in}}%
\pgfpathlineto{\pgfqpoint{4.826228in}{2.112335in}}%
\pgfpathlineto{\pgfqpoint{4.828003in}{2.113636in}}%
\pgfpathlineto{\pgfqpoint{4.829778in}{2.112963in}}%
\pgfpathlineto{\pgfqpoint{4.831553in}{2.110338in}}%
\pgfpathlineto{\pgfqpoint{4.835104in}{2.098455in}}%
\pgfpathlineto{\pgfqpoint{4.840429in}{2.068769in}}%
\pgfpathlineto{\pgfqpoint{4.849306in}{2.011001in}}%
\pgfpathlineto{\pgfqpoint{4.852856in}{2.001834in}}%
\pgfpathlineto{\pgfqpoint{4.854631in}{2.003424in}}%
\pgfpathlineto{\pgfqpoint{4.858182in}{2.017593in}}%
\pgfpathlineto{\pgfqpoint{4.867058in}{2.066721in}}%
\pgfpathlineto{\pgfqpoint{4.870608in}{2.075177in}}%
\pgfpathlineto{\pgfqpoint{4.877709in}{2.085742in}}%
\pgfpathlineto{\pgfqpoint{4.888361in}{2.114394in}}%
\pgfpathlineto{\pgfqpoint{4.890136in}{2.114868in}}%
\pgfpathlineto{\pgfqpoint{4.891911in}{2.113690in}}%
\pgfpathlineto{\pgfqpoint{4.895462in}{2.107042in}}%
\pgfpathlineto{\pgfqpoint{4.900787in}{2.092125in}}%
\pgfpathlineto{\pgfqpoint{4.913214in}{2.047584in}}%
\pgfpathlineto{\pgfqpoint{4.916764in}{2.043330in}}%
\pgfpathlineto{\pgfqpoint{4.923865in}{2.039467in}}%
\pgfpathlineto{\pgfqpoint{4.927416in}{2.032425in}}%
\pgfpathlineto{\pgfqpoint{4.936292in}{2.010754in}}%
\pgfpathlineto{\pgfqpoint{4.938067in}{2.009019in}}%
\pgfpathlineto{\pgfqpoint{4.939843in}{2.008857in}}%
\pgfpathlineto{\pgfqpoint{4.943393in}{2.013014in}}%
\pgfpathlineto{\pgfqpoint{4.952269in}{2.029130in}}%
\pgfpathlineto{\pgfqpoint{4.954044in}{2.028182in}}%
\pgfpathlineto{\pgfqpoint{4.955820in}{2.025351in}}%
\pgfpathlineto{\pgfqpoint{4.959370in}{2.012997in}}%
\pgfpathlineto{\pgfqpoint{4.971797in}{1.950423in}}%
\pgfpathlineto{\pgfqpoint{4.978898in}{1.931583in}}%
\pgfpathlineto{\pgfqpoint{4.989549in}{1.907663in}}%
\pgfpathlineto{\pgfqpoint{4.991324in}{1.907224in}}%
\pgfpathlineto{\pgfqpoint{4.994875in}{1.911406in}}%
\pgfpathlineto{\pgfqpoint{5.000200in}{1.920738in}}%
\pgfpathlineto{\pgfqpoint{5.001976in}{1.921933in}}%
\pgfpathlineto{\pgfqpoint{5.003751in}{1.920853in}}%
\pgfpathlineto{\pgfqpoint{5.007301in}{1.913076in}}%
\pgfpathlineto{\pgfqpoint{5.014402in}{1.886681in}}%
\pgfpathlineto{\pgfqpoint{5.023278in}{1.856954in}}%
\pgfpathlineto{\pgfqpoint{5.025054in}{1.854169in}}%
\pgfpathlineto{\pgfqpoint{5.026829in}{1.853559in}}%
\pgfpathlineto{\pgfqpoint{5.028604in}{1.855163in}}%
\pgfpathlineto{\pgfqpoint{5.032155in}{1.864306in}}%
\pgfpathlineto{\pgfqpoint{5.044581in}{1.908112in}}%
\pgfpathlineto{\pgfqpoint{5.048132in}{1.914840in}}%
\pgfpathlineto{\pgfqpoint{5.049907in}{1.916050in}}%
\pgfpathlineto{\pgfqpoint{5.051682in}{1.914975in}}%
\pgfpathlineto{\pgfqpoint{5.053457in}{1.911325in}}%
\pgfpathlineto{\pgfqpoint{5.057008in}{1.894956in}}%
\pgfpathlineto{\pgfqpoint{5.069435in}{1.814960in}}%
\pgfpathlineto{\pgfqpoint{5.072985in}{1.806314in}}%
\pgfpathlineto{\pgfqpoint{5.076535in}{1.804605in}}%
\pgfpathlineto{\pgfqpoint{5.081861in}{1.805593in}}%
\pgfpathlineto{\pgfqpoint{5.085412in}{1.803640in}}%
\pgfpathlineto{\pgfqpoint{5.088962in}{1.798262in}}%
\pgfpathlineto{\pgfqpoint{5.101389in}{1.772762in}}%
\pgfpathlineto{\pgfqpoint{5.113815in}{1.733382in}}%
\pgfpathlineto{\pgfqpoint{5.119141in}{1.726108in}}%
\pgfpathlineto{\pgfqpoint{5.122691in}{1.713668in}}%
\pgfpathlineto{\pgfqpoint{5.128017in}{1.677349in}}%
\pgfpathlineto{\pgfqpoint{5.136893in}{1.608375in}}%
\pgfpathlineto{\pgfqpoint{5.142219in}{1.583603in}}%
\pgfpathlineto{\pgfqpoint{5.161747in}{1.517069in}}%
\pgfpathlineto{\pgfqpoint{5.168848in}{1.504907in}}%
\pgfpathlineto{\pgfqpoint{5.174173in}{1.488991in}}%
\pgfpathlineto{\pgfqpoint{5.183049in}{1.452404in}}%
\pgfpathlineto{\pgfqpoint{5.188375in}{1.420181in}}%
\pgfpathlineto{\pgfqpoint{5.195476in}{1.358283in}}%
\pgfpathlineto{\pgfqpoint{5.206127in}{1.264208in}}%
\pgfpathlineto{\pgfqpoint{5.225655in}{1.130593in}}%
\pgfpathlineto{\pgfqpoint{5.234531in}{1.039930in}}%
\pgfpathlineto{\pgfqpoint{5.245183in}{0.934526in}}%
\pgfpathlineto{\pgfqpoint{5.254059in}{0.871370in}}%
\pgfpathlineto{\pgfqpoint{5.266485in}{0.799159in}}%
\pgfpathlineto{\pgfqpoint{5.275361in}{0.758479in}}%
\pgfpathlineto{\pgfqpoint{5.284238in}{0.728710in}}%
\pgfpathlineto{\pgfqpoint{5.291339in}{0.712401in}}%
\pgfpathlineto{\pgfqpoint{5.296664in}{0.705157in}}%
\pgfpathlineto{\pgfqpoint{5.301990in}{0.702581in}}%
\pgfpathlineto{\pgfqpoint{5.307316in}{0.702993in}}%
\pgfpathlineto{\pgfqpoint{5.323293in}{0.706482in}}%
\pgfpathlineto{\pgfqpoint{5.335719in}{0.706362in}}%
\pgfpathlineto{\pgfqpoint{5.372999in}{0.701337in}}%
\pgfpathlineto{\pgfqpoint{5.397853in}{0.698907in}}%
\pgfpathlineto{\pgfqpoint{5.435132in}{0.697235in}}%
\pgfpathlineto{\pgfqpoint{5.465311in}{0.697336in}}%
\pgfpathlineto{\pgfqpoint{5.534545in}{0.696000in}}%
\pgfpathlineto{\pgfqpoint{5.534545in}{0.696000in}}%
\pgfusepath{stroke}%
\end{pgfscope}%
\begin{pgfscope}%
\pgfpathrectangle{\pgfqpoint{0.800000in}{0.528000in}}{\pgfqpoint{4.960000in}{3.696000in}} %
\pgfusepath{clip}%
\pgfsetrectcap%
\pgfsetroundjoin%
\pgfsetlinewidth{1.505625pt}%
\definecolor{currentstroke}{rgb}{0.839216,0.152941,0.156863}%
\pgfsetstrokecolor{currentstroke}%
\pgfsetdash{}{0pt}%
\pgfpathmoveto{\pgfqpoint{1.025455in}{0.696000in}}%
\pgfpathlineto{\pgfqpoint{1.121317in}{0.696000in}}%
\pgfpathlineto{\pgfqpoint{1.124868in}{0.697103in}}%
\pgfpathlineto{\pgfqpoint{1.140845in}{0.697168in}}%
\pgfpathlineto{\pgfqpoint{1.146170in}{0.699434in}}%
\pgfpathlineto{\pgfqpoint{1.147946in}{0.697961in}}%
\pgfpathlineto{\pgfqpoint{1.155047in}{0.700552in}}%
\pgfpathlineto{\pgfqpoint{1.163923in}{0.706248in}}%
\pgfpathlineto{\pgfqpoint{1.171024in}{0.707730in}}%
\pgfpathlineto{\pgfqpoint{1.172799in}{0.717842in}}%
\pgfpathlineto{\pgfqpoint{1.183450in}{0.725727in}}%
\pgfpathlineto{\pgfqpoint{1.185225in}{0.711419in}}%
\pgfpathlineto{\pgfqpoint{1.188776in}{0.706708in}}%
\pgfpathlineto{\pgfqpoint{1.194102in}{0.714231in}}%
\pgfpathlineto{\pgfqpoint{1.195877in}{0.890879in}}%
\pgfpathlineto{\pgfqpoint{1.201203in}{0.931515in}}%
\pgfpathlineto{\pgfqpoint{1.204753in}{0.942795in}}%
\pgfpathlineto{\pgfqpoint{1.208304in}{0.947176in}}%
\pgfpathlineto{\pgfqpoint{1.211854in}{0.949667in}}%
\pgfpathlineto{\pgfqpoint{1.213629in}{0.948475in}}%
\pgfpathlineto{\pgfqpoint{1.217180in}{0.950342in}}%
\pgfpathlineto{\pgfqpoint{1.220730in}{0.953537in}}%
\pgfpathlineto{\pgfqpoint{1.222505in}{0.957400in}}%
\pgfpathlineto{\pgfqpoint{1.227831in}{0.961904in}}%
\pgfpathlineto{\pgfqpoint{1.238482in}{0.976203in}}%
\pgfpathlineto{\pgfqpoint{1.243808in}{0.988140in}}%
\pgfpathlineto{\pgfqpoint{1.250909in}{1.010885in}}%
\pgfpathlineto{\pgfqpoint{1.272212in}{1.084445in}}%
\pgfpathlineto{\pgfqpoint{1.275762in}{1.089078in}}%
\pgfpathlineto{\pgfqpoint{1.281088in}{1.092745in}}%
\pgfpathlineto{\pgfqpoint{1.286414in}{1.094696in}}%
\pgfpathlineto{\pgfqpoint{1.291739in}{1.093131in}}%
\pgfpathlineto{\pgfqpoint{1.295290in}{1.091128in}}%
\pgfpathlineto{\pgfqpoint{1.297065in}{1.092024in}}%
\pgfpathlineto{\pgfqpoint{1.307717in}{1.088635in}}%
\pgfpathlineto{\pgfqpoint{1.314817in}{1.077595in}}%
\pgfpathlineto{\pgfqpoint{1.321918in}{1.054757in}}%
\pgfpathlineto{\pgfqpoint{1.327244in}{1.045048in}}%
\pgfpathlineto{\pgfqpoint{1.330795in}{1.043332in}}%
\pgfpathlineto{\pgfqpoint{1.334345in}{1.044967in}}%
\pgfpathlineto{\pgfqpoint{1.341446in}{1.051408in}}%
\pgfpathlineto{\pgfqpoint{1.348547in}{1.059518in}}%
\pgfpathlineto{\pgfqpoint{1.355648in}{1.068479in}}%
\pgfpathlineto{\pgfqpoint{1.359198in}{1.069198in}}%
\pgfpathlineto{\pgfqpoint{1.366299in}{1.065799in}}%
\pgfpathlineto{\pgfqpoint{1.369850in}{1.064903in}}%
\pgfpathlineto{\pgfqpoint{1.373400in}{1.061776in}}%
\pgfpathlineto{\pgfqpoint{1.378726in}{1.055165in}}%
\pgfpathlineto{\pgfqpoint{1.382276in}{1.053917in}}%
\pgfpathlineto{\pgfqpoint{1.385827in}{1.056454in}}%
\pgfpathlineto{\pgfqpoint{1.391152in}{1.064068in}}%
\pgfpathlineto{\pgfqpoint{1.396478in}{1.075527in}}%
\pgfpathlineto{\pgfqpoint{1.403579in}{1.100500in}}%
\pgfpathlineto{\pgfqpoint{1.414230in}{1.146980in}}%
\pgfpathlineto{\pgfqpoint{1.423107in}{1.184563in}}%
\pgfpathlineto{\pgfqpoint{1.428432in}{1.199375in}}%
\pgfpathlineto{\pgfqpoint{1.433758in}{1.207967in}}%
\pgfpathlineto{\pgfqpoint{1.447960in}{1.222249in}}%
\pgfpathlineto{\pgfqpoint{1.451510in}{1.223908in}}%
\pgfpathlineto{\pgfqpoint{1.455061in}{1.223581in}}%
\pgfpathlineto{\pgfqpoint{1.460387in}{1.217948in}}%
\pgfpathlineto{\pgfqpoint{1.469263in}{1.205923in}}%
\pgfpathlineto{\pgfqpoint{1.479914in}{1.195192in}}%
\pgfpathlineto{\pgfqpoint{1.490565in}{1.176782in}}%
\pgfpathlineto{\pgfqpoint{1.497666in}{1.165991in}}%
\pgfpathlineto{\pgfqpoint{1.504767in}{1.154763in}}%
\pgfpathlineto{\pgfqpoint{1.513644in}{1.138629in}}%
\pgfpathlineto{\pgfqpoint{1.517194in}{1.138032in}}%
\pgfpathlineto{\pgfqpoint{1.522520in}{1.142495in}}%
\pgfpathlineto{\pgfqpoint{1.531396in}{1.154038in}}%
\pgfpathlineto{\pgfqpoint{1.540272in}{1.167993in}}%
\pgfpathlineto{\pgfqpoint{1.543822in}{1.170282in}}%
\pgfpathlineto{\pgfqpoint{1.547373in}{1.169824in}}%
\pgfpathlineto{\pgfqpoint{1.558024in}{1.165664in}}%
\pgfpathlineto{\pgfqpoint{1.570451in}{1.165452in}}%
\pgfpathlineto{\pgfqpoint{1.579327in}{1.165371in}}%
\pgfpathlineto{\pgfqpoint{1.582878in}{1.162661in}}%
\pgfpathlineto{\pgfqpoint{1.586428in}{1.156128in}}%
\pgfpathlineto{\pgfqpoint{1.593529in}{1.140348in}}%
\pgfpathlineto{\pgfqpoint{1.597079in}{1.137379in}}%
\pgfpathlineto{\pgfqpoint{1.605956in}{1.135556in}}%
\pgfpathlineto{\pgfqpoint{1.609506in}{1.133805in}}%
\pgfpathlineto{\pgfqpoint{1.613057in}{1.134347in}}%
\pgfpathlineto{\pgfqpoint{1.620157in}{1.139704in}}%
\pgfpathlineto{\pgfqpoint{1.623708in}{1.139835in}}%
\pgfpathlineto{\pgfqpoint{1.627258in}{1.136018in}}%
\pgfpathlineto{\pgfqpoint{1.632584in}{1.123022in}}%
\pgfpathlineto{\pgfqpoint{1.650336in}{1.068436in}}%
\pgfpathlineto{\pgfqpoint{1.653887in}{1.062564in}}%
\pgfpathlineto{\pgfqpoint{1.657437in}{1.059894in}}%
\pgfpathlineto{\pgfqpoint{1.660988in}{1.061160in}}%
\pgfpathlineto{\pgfqpoint{1.664538in}{1.069625in}}%
\pgfpathlineto{\pgfqpoint{1.669864in}{1.087161in}}%
\pgfpathlineto{\pgfqpoint{1.678740in}{1.127970in}}%
\pgfpathlineto{\pgfqpoint{1.682291in}{1.136404in}}%
\pgfpathlineto{\pgfqpoint{1.684066in}{1.138136in}}%
\pgfpathlineto{\pgfqpoint{1.691167in}{1.137435in}}%
\pgfpathlineto{\pgfqpoint{1.694717in}{1.133990in}}%
\pgfpathlineto{\pgfqpoint{1.700043in}{1.126263in}}%
\pgfpathlineto{\pgfqpoint{1.707144in}{1.116876in}}%
\pgfpathlineto{\pgfqpoint{1.710694in}{1.115108in}}%
\pgfpathlineto{\pgfqpoint{1.712470in}{1.115283in}}%
\pgfpathlineto{\pgfqpoint{1.717795in}{1.120656in}}%
\pgfpathlineto{\pgfqpoint{1.723121in}{1.126257in}}%
\pgfpathlineto{\pgfqpoint{1.728447in}{1.127579in}}%
\pgfpathlineto{\pgfqpoint{1.731997in}{1.125814in}}%
\pgfpathlineto{\pgfqpoint{1.737323in}{1.117555in}}%
\pgfpathlineto{\pgfqpoint{1.742649in}{1.108126in}}%
\pgfpathlineto{\pgfqpoint{1.746199in}{1.104954in}}%
\pgfpathlineto{\pgfqpoint{1.749749in}{1.104729in}}%
\pgfpathlineto{\pgfqpoint{1.753300in}{1.106260in}}%
\pgfpathlineto{\pgfqpoint{1.756850in}{1.109026in}}%
\pgfpathlineto{\pgfqpoint{1.760401in}{1.114824in}}%
\pgfpathlineto{\pgfqpoint{1.767502in}{1.135560in}}%
\pgfpathlineto{\pgfqpoint{1.772827in}{1.149973in}}%
\pgfpathlineto{\pgfqpoint{1.776378in}{1.155836in}}%
\pgfpathlineto{\pgfqpoint{1.779928in}{1.157704in}}%
\pgfpathlineto{\pgfqpoint{1.783479in}{1.154468in}}%
\pgfpathlineto{\pgfqpoint{1.788805in}{1.140368in}}%
\pgfpathlineto{\pgfqpoint{1.794130in}{1.125437in}}%
\pgfpathlineto{\pgfqpoint{1.797681in}{1.121952in}}%
\pgfpathlineto{\pgfqpoint{1.799456in}{1.122658in}}%
\pgfpathlineto{\pgfqpoint{1.803006in}{1.127781in}}%
\pgfpathlineto{\pgfqpoint{1.818984in}{1.158435in}}%
\pgfpathlineto{\pgfqpoint{1.822534in}{1.161041in}}%
\pgfpathlineto{\pgfqpoint{1.826084in}{1.158937in}}%
\pgfpathlineto{\pgfqpoint{1.829635in}{1.151797in}}%
\pgfpathlineto{\pgfqpoint{1.833185in}{1.139210in}}%
\pgfpathlineto{\pgfqpoint{1.838511in}{1.110542in}}%
\pgfpathlineto{\pgfqpoint{1.849162in}{1.049621in}}%
\pgfpathlineto{\pgfqpoint{1.854488in}{1.030051in}}%
\pgfpathlineto{\pgfqpoint{1.858039in}{1.024573in}}%
\pgfpathlineto{\pgfqpoint{1.859814in}{1.024386in}}%
\pgfpathlineto{\pgfqpoint{1.863364in}{1.028487in}}%
\pgfpathlineto{\pgfqpoint{1.875791in}{1.050693in}}%
\pgfpathlineto{\pgfqpoint{1.879341in}{1.051937in}}%
\pgfpathlineto{\pgfqpoint{1.882892in}{1.047836in}}%
\pgfpathlineto{\pgfqpoint{1.886442in}{1.037737in}}%
\pgfpathlineto{\pgfqpoint{1.900644in}{0.986051in}}%
\pgfpathlineto{\pgfqpoint{1.911296in}{0.981144in}}%
\pgfpathlineto{\pgfqpoint{1.914846in}{0.985130in}}%
\pgfpathlineto{\pgfqpoint{1.920172in}{0.996263in}}%
\pgfpathlineto{\pgfqpoint{1.937924in}{1.041081in}}%
\pgfpathlineto{\pgfqpoint{1.941475in}{1.043544in}}%
\pgfpathlineto{\pgfqpoint{1.945025in}{1.042235in}}%
\pgfpathlineto{\pgfqpoint{1.955676in}{1.029208in}}%
\pgfpathlineto{\pgfqpoint{1.961002in}{1.021110in}}%
\pgfpathlineto{\pgfqpoint{1.962777in}{1.019746in}}%
\pgfpathlineto{\pgfqpoint{1.964553in}{1.020755in}}%
\pgfpathlineto{\pgfqpoint{1.973429in}{1.052501in}}%
\pgfpathlineto{\pgfqpoint{1.976979in}{1.059276in}}%
\pgfpathlineto{\pgfqpoint{1.994732in}{1.124478in}}%
\pgfpathlineto{\pgfqpoint{2.000057in}{1.138472in}}%
\pgfpathlineto{\pgfqpoint{2.003608in}{1.141806in}}%
\pgfpathlineto{\pgfqpoint{2.008933in}{1.143305in}}%
\pgfpathlineto{\pgfqpoint{2.010709in}{1.144706in}}%
\pgfpathlineto{\pgfqpoint{2.014259in}{1.151084in}}%
\pgfpathlineto{\pgfqpoint{2.023135in}{1.170666in}}%
\pgfpathlineto{\pgfqpoint{2.026686in}{1.172373in}}%
\pgfpathlineto{\pgfqpoint{2.030236in}{1.169007in}}%
\pgfpathlineto{\pgfqpoint{2.035562in}{1.155353in}}%
\pgfpathlineto{\pgfqpoint{2.042663in}{1.125426in}}%
\pgfpathlineto{\pgfqpoint{2.051539in}{1.087160in}}%
\pgfpathlineto{\pgfqpoint{2.056865in}{1.069131in}}%
\pgfpathlineto{\pgfqpoint{2.063966in}{1.052856in}}%
\pgfpathlineto{\pgfqpoint{2.067516in}{1.046984in}}%
\pgfpathlineto{\pgfqpoint{2.071067in}{1.044771in}}%
\pgfpathlineto{\pgfqpoint{2.074617in}{1.047450in}}%
\pgfpathlineto{\pgfqpoint{2.078168in}{1.054954in}}%
\pgfpathlineto{\pgfqpoint{2.088819in}{1.086061in}}%
\pgfpathlineto{\pgfqpoint{2.094145in}{1.100871in}}%
\pgfpathlineto{\pgfqpoint{2.097695in}{1.106369in}}%
\pgfpathlineto{\pgfqpoint{2.101246in}{1.107643in}}%
\pgfpathlineto{\pgfqpoint{2.104796in}{1.105462in}}%
\pgfpathlineto{\pgfqpoint{2.117223in}{1.093105in}}%
\pgfpathlineto{\pgfqpoint{2.120773in}{1.092420in}}%
\pgfpathlineto{\pgfqpoint{2.124324in}{1.093216in}}%
\pgfpathlineto{\pgfqpoint{2.127874in}{1.095595in}}%
\pgfpathlineto{\pgfqpoint{2.140301in}{1.109113in}}%
\pgfpathlineto{\pgfqpoint{2.143851in}{1.109603in}}%
\pgfpathlineto{\pgfqpoint{2.147402in}{1.110257in}}%
\pgfpathlineto{\pgfqpoint{2.150952in}{1.113012in}}%
\pgfpathlineto{\pgfqpoint{2.158053in}{1.117880in}}%
\pgfpathlineto{\pgfqpoint{2.166929in}{1.122912in}}%
\pgfpathlineto{\pgfqpoint{2.179356in}{1.138215in}}%
\pgfpathlineto{\pgfqpoint{2.182906in}{1.139260in}}%
\pgfpathlineto{\pgfqpoint{2.191782in}{1.140419in}}%
\pgfpathlineto{\pgfqpoint{2.195333in}{1.144921in}}%
\pgfpathlineto{\pgfqpoint{2.198883in}{1.153252in}}%
\pgfpathlineto{\pgfqpoint{2.211310in}{1.191786in}}%
\pgfpathlineto{\pgfqpoint{2.220186in}{1.217145in}}%
\pgfpathlineto{\pgfqpoint{2.223737in}{1.221146in}}%
\pgfpathlineto{\pgfqpoint{2.225512in}{1.221259in}}%
\pgfpathlineto{\pgfqpoint{2.229062in}{1.218259in}}%
\pgfpathlineto{\pgfqpoint{2.234388in}{1.206569in}}%
\pgfpathlineto{\pgfqpoint{2.241489in}{1.187708in}}%
\pgfpathlineto{\pgfqpoint{2.246815in}{1.179014in}}%
\pgfpathlineto{\pgfqpoint{2.257466in}{1.162533in}}%
\pgfpathlineto{\pgfqpoint{2.261016in}{1.162844in}}%
\pgfpathlineto{\pgfqpoint{2.266342in}{1.168198in}}%
\pgfpathlineto{\pgfqpoint{2.271668in}{1.174090in}}%
\pgfpathlineto{\pgfqpoint{2.278769in}{1.177746in}}%
\pgfpathlineto{\pgfqpoint{2.285870in}{1.190457in}}%
\pgfpathlineto{\pgfqpoint{2.289420in}{1.194325in}}%
\pgfpathlineto{\pgfqpoint{2.294746in}{1.199882in}}%
\pgfpathlineto{\pgfqpoint{2.298296in}{1.200552in}}%
\pgfpathlineto{\pgfqpoint{2.307173in}{1.197272in}}%
\pgfpathlineto{\pgfqpoint{2.312498in}{1.198335in}}%
\pgfpathlineto{\pgfqpoint{2.317824in}{1.200980in}}%
\pgfpathlineto{\pgfqpoint{2.321374in}{1.207040in}}%
\pgfpathlineto{\pgfqpoint{2.324925in}{1.223171in}}%
\pgfpathlineto{\pgfqpoint{2.335576in}{1.277018in}}%
\pgfpathlineto{\pgfqpoint{2.342677in}{1.302935in}}%
\pgfpathlineto{\pgfqpoint{2.346228in}{1.309186in}}%
\pgfpathlineto{\pgfqpoint{2.349778in}{1.310090in}}%
\pgfpathlineto{\pgfqpoint{2.355104in}{1.306553in}}%
\pgfpathlineto{\pgfqpoint{2.362205in}{1.298032in}}%
\pgfpathlineto{\pgfqpoint{2.372856in}{1.280628in}}%
\pgfpathlineto{\pgfqpoint{2.378182in}{1.277533in}}%
\pgfpathlineto{\pgfqpoint{2.383508in}{1.275053in}}%
\pgfpathlineto{\pgfqpoint{2.388833in}{1.269580in}}%
\pgfpathlineto{\pgfqpoint{2.404810in}{1.246989in}}%
\pgfpathlineto{\pgfqpoint{2.408361in}{1.245587in}}%
\pgfpathlineto{\pgfqpoint{2.417237in}{1.251212in}}%
\pgfpathlineto{\pgfqpoint{2.420787in}{1.254179in}}%
\pgfpathlineto{\pgfqpoint{2.427888in}{1.257798in}}%
\pgfpathlineto{\pgfqpoint{2.431439in}{1.263921in}}%
\pgfpathlineto{\pgfqpoint{2.449191in}{1.310905in}}%
\pgfpathlineto{\pgfqpoint{2.463393in}{1.351486in}}%
\pgfpathlineto{\pgfqpoint{2.466943in}{1.355950in}}%
\pgfpathlineto{\pgfqpoint{2.470494in}{1.356701in}}%
\pgfpathlineto{\pgfqpoint{2.479370in}{1.354878in}}%
\pgfpathlineto{\pgfqpoint{2.490021in}{1.355523in}}%
\pgfpathlineto{\pgfqpoint{2.498898in}{1.353359in}}%
\pgfpathlineto{\pgfqpoint{2.502448in}{1.356430in}}%
\pgfpathlineto{\pgfqpoint{2.505999in}{1.365210in}}%
\pgfpathlineto{\pgfqpoint{2.516650in}{1.395707in}}%
\pgfpathlineto{\pgfqpoint{2.518425in}{1.396981in}}%
\pgfpathlineto{\pgfqpoint{2.520200in}{1.395905in}}%
\pgfpathlineto{\pgfqpoint{2.523751in}{1.389315in}}%
\pgfpathlineto{\pgfqpoint{2.529077in}{1.368053in}}%
\pgfpathlineto{\pgfqpoint{2.545054in}{1.286492in}}%
\pgfpathlineto{\pgfqpoint{2.550379in}{1.273216in}}%
\pgfpathlineto{\pgfqpoint{2.557480in}{1.261747in}}%
\pgfpathlineto{\pgfqpoint{2.568132in}{1.249032in}}%
\pgfpathlineto{\pgfqpoint{2.571682in}{1.247366in}}%
\pgfpathlineto{\pgfqpoint{2.575233in}{1.249082in}}%
\pgfpathlineto{\pgfqpoint{2.578783in}{1.253722in}}%
\pgfpathlineto{\pgfqpoint{2.584109in}{1.264680in}}%
\pgfpathlineto{\pgfqpoint{2.589435in}{1.280212in}}%
\pgfpathlineto{\pgfqpoint{2.596535in}{1.303141in}}%
\pgfpathlineto{\pgfqpoint{2.601861in}{1.313894in}}%
\pgfpathlineto{\pgfqpoint{2.607187in}{1.319390in}}%
\pgfpathlineto{\pgfqpoint{2.610737in}{1.320856in}}%
\pgfpathlineto{\pgfqpoint{2.612513in}{1.320298in}}%
\pgfpathlineto{\pgfqpoint{2.616063in}{1.315272in}}%
\pgfpathlineto{\pgfqpoint{2.626714in}{1.293047in}}%
\pgfpathlineto{\pgfqpoint{2.630265in}{1.291330in}}%
\pgfpathlineto{\pgfqpoint{2.633815in}{1.292083in}}%
\pgfpathlineto{\pgfqpoint{2.639141in}{1.296304in}}%
\pgfpathlineto{\pgfqpoint{2.646242in}{1.304879in}}%
\pgfpathlineto{\pgfqpoint{2.656893in}{1.322519in}}%
\pgfpathlineto{\pgfqpoint{2.660444in}{1.323862in}}%
\pgfpathlineto{\pgfqpoint{2.663994in}{1.319355in}}%
\pgfpathlineto{\pgfqpoint{2.669320in}{1.305251in}}%
\pgfpathlineto{\pgfqpoint{2.674646in}{1.291298in}}%
\pgfpathlineto{\pgfqpoint{2.678196in}{1.286913in}}%
\pgfpathlineto{\pgfqpoint{2.679971in}{1.286460in}}%
\pgfpathlineto{\pgfqpoint{2.683522in}{1.288618in}}%
\pgfpathlineto{\pgfqpoint{2.688848in}{1.292285in}}%
\pgfpathlineto{\pgfqpoint{2.690623in}{1.292163in}}%
\pgfpathlineto{\pgfqpoint{2.694173in}{1.287985in}}%
\pgfpathlineto{\pgfqpoint{2.704825in}{1.269822in}}%
\pgfpathlineto{\pgfqpoint{2.708375in}{1.268793in}}%
\pgfpathlineto{\pgfqpoint{2.711926in}{1.269574in}}%
\pgfpathlineto{\pgfqpoint{2.719026in}{1.273343in}}%
\pgfpathlineto{\pgfqpoint{2.724352in}{1.279373in}}%
\pgfpathlineto{\pgfqpoint{2.731453in}{1.288759in}}%
\pgfpathlineto{\pgfqpoint{2.736779in}{1.293360in}}%
\pgfpathlineto{\pgfqpoint{2.740329in}{1.294319in}}%
\pgfpathlineto{\pgfqpoint{2.745655in}{1.294445in}}%
\pgfpathlineto{\pgfqpoint{2.749205in}{1.298243in}}%
\pgfpathlineto{\pgfqpoint{2.752756in}{1.306927in}}%
\pgfpathlineto{\pgfqpoint{2.759857in}{1.334110in}}%
\pgfpathlineto{\pgfqpoint{2.766958in}{1.358333in}}%
\pgfpathlineto{\pgfqpoint{2.772283in}{1.370215in}}%
\pgfpathlineto{\pgfqpoint{2.774059in}{1.372231in}}%
\pgfpathlineto{\pgfqpoint{2.775834in}{1.372410in}}%
\pgfpathlineto{\pgfqpoint{2.777609in}{1.371179in}}%
\pgfpathlineto{\pgfqpoint{2.781160in}{1.362259in}}%
\pgfpathlineto{\pgfqpoint{2.786485in}{1.337131in}}%
\pgfpathlineto{\pgfqpoint{2.793586in}{1.300901in}}%
\pgfpathlineto{\pgfqpoint{2.800687in}{1.277330in}}%
\pgfpathlineto{\pgfqpoint{2.816664in}{1.233995in}}%
\pgfpathlineto{\pgfqpoint{2.820215in}{1.230789in}}%
\pgfpathlineto{\pgfqpoint{2.823765in}{1.233020in}}%
\pgfpathlineto{\pgfqpoint{2.832641in}{1.244422in}}%
\pgfpathlineto{\pgfqpoint{2.839742in}{1.252401in}}%
\pgfpathlineto{\pgfqpoint{2.850394in}{1.268488in}}%
\pgfpathlineto{\pgfqpoint{2.853944in}{1.268894in}}%
\pgfpathlineto{\pgfqpoint{2.857495in}{1.267281in}}%
\pgfpathlineto{\pgfqpoint{2.871696in}{1.257748in}}%
\pgfpathlineto{\pgfqpoint{2.877022in}{1.249870in}}%
\pgfpathlineto{\pgfqpoint{2.885898in}{1.235253in}}%
\pgfpathlineto{\pgfqpoint{2.889449in}{1.233173in}}%
\pgfpathlineto{\pgfqpoint{2.892999in}{1.234576in}}%
\pgfpathlineto{\pgfqpoint{2.898325in}{1.241707in}}%
\pgfpathlineto{\pgfqpoint{2.908976in}{1.257988in}}%
\pgfpathlineto{\pgfqpoint{2.914302in}{1.263157in}}%
\pgfpathlineto{\pgfqpoint{2.917853in}{1.263498in}}%
\pgfpathlineto{\pgfqpoint{2.921403in}{1.261133in}}%
\pgfpathlineto{\pgfqpoint{2.937380in}{1.243068in}}%
\pgfpathlineto{\pgfqpoint{2.946256in}{1.227547in}}%
\pgfpathlineto{\pgfqpoint{2.949807in}{1.226179in}}%
\pgfpathlineto{\pgfqpoint{2.953357in}{1.228988in}}%
\pgfpathlineto{\pgfqpoint{2.969334in}{1.256104in}}%
\pgfpathlineto{\pgfqpoint{2.974660in}{1.279423in}}%
\pgfpathlineto{\pgfqpoint{2.981761in}{1.313006in}}%
\pgfpathlineto{\pgfqpoint{2.988862in}{1.333604in}}%
\pgfpathlineto{\pgfqpoint{2.992412in}{1.339576in}}%
\pgfpathlineto{\pgfqpoint{2.995963in}{1.341686in}}%
\pgfpathlineto{\pgfqpoint{2.997738in}{1.341365in}}%
\pgfpathlineto{\pgfqpoint{3.001288in}{1.337451in}}%
\pgfpathlineto{\pgfqpoint{3.020816in}{1.307887in}}%
\pgfpathlineto{\pgfqpoint{3.024366in}{1.306857in}}%
\pgfpathlineto{\pgfqpoint{3.026142in}{1.307191in}}%
\pgfpathlineto{\pgfqpoint{3.029692in}{1.310920in}}%
\pgfpathlineto{\pgfqpoint{3.035018in}{1.321833in}}%
\pgfpathlineto{\pgfqpoint{3.054545in}{1.377164in}}%
\pgfpathlineto{\pgfqpoint{3.058096in}{1.381051in}}%
\pgfpathlineto{\pgfqpoint{3.061646in}{1.381407in}}%
\pgfpathlineto{\pgfqpoint{3.065197in}{1.379347in}}%
\pgfpathlineto{\pgfqpoint{3.072298in}{1.376221in}}%
\pgfpathlineto{\pgfqpoint{3.074073in}{1.376492in}}%
\pgfpathlineto{\pgfqpoint{3.079399in}{1.382356in}}%
\pgfpathlineto{\pgfqpoint{3.097151in}{1.409132in}}%
\pgfpathlineto{\pgfqpoint{3.100702in}{1.409259in}}%
\pgfpathlineto{\pgfqpoint{3.104252in}{1.405837in}}%
\pgfpathlineto{\pgfqpoint{3.109578in}{1.397593in}}%
\pgfpathlineto{\pgfqpoint{3.118454in}{1.382783in}}%
\pgfpathlineto{\pgfqpoint{3.127330in}{1.361531in}}%
\pgfpathlineto{\pgfqpoint{3.129105in}{1.360021in}}%
\pgfpathlineto{\pgfqpoint{3.130880in}{1.360345in}}%
\pgfpathlineto{\pgfqpoint{3.134431in}{1.365194in}}%
\pgfpathlineto{\pgfqpoint{3.141532in}{1.380095in}}%
\pgfpathlineto{\pgfqpoint{3.145082in}{1.383734in}}%
\pgfpathlineto{\pgfqpoint{3.148633in}{1.382160in}}%
\pgfpathlineto{\pgfqpoint{3.152183in}{1.373928in}}%
\pgfpathlineto{\pgfqpoint{3.157509in}{1.352192in}}%
\pgfpathlineto{\pgfqpoint{3.169936in}{1.295465in}}%
\pgfpathlineto{\pgfqpoint{3.175261in}{1.277441in}}%
\pgfpathlineto{\pgfqpoint{3.178812in}{1.271832in}}%
\pgfpathlineto{\pgfqpoint{3.180587in}{1.271484in}}%
\pgfpathlineto{\pgfqpoint{3.182362in}{1.272389in}}%
\pgfpathlineto{\pgfqpoint{3.196564in}{1.288853in}}%
\pgfpathlineto{\pgfqpoint{3.200115in}{1.292534in}}%
\pgfpathlineto{\pgfqpoint{3.205440in}{1.307763in}}%
\pgfpathlineto{\pgfqpoint{3.216092in}{1.345933in}}%
\pgfpathlineto{\pgfqpoint{3.221417in}{1.366911in}}%
\pgfpathlineto{\pgfqpoint{3.226743in}{1.401883in}}%
\pgfpathlineto{\pgfqpoint{3.239170in}{1.495398in}}%
\pgfpathlineto{\pgfqpoint{3.242720in}{1.512981in}}%
\pgfpathlineto{\pgfqpoint{3.244495in}{1.517632in}}%
\pgfpathlineto{\pgfqpoint{3.246271in}{1.518749in}}%
\pgfpathlineto{\pgfqpoint{3.248046in}{1.517140in}}%
\pgfpathlineto{\pgfqpoint{3.251596in}{1.505048in}}%
\pgfpathlineto{\pgfqpoint{3.255147in}{1.483236in}}%
\pgfpathlineto{\pgfqpoint{3.260472in}{1.433029in}}%
\pgfpathlineto{\pgfqpoint{3.271124in}{1.292437in}}%
\pgfpathlineto{\pgfqpoint{3.274674in}{1.213707in}}%
\pgfpathlineto{\pgfqpoint{3.280000in}{1.030271in}}%
\pgfpathlineto{\pgfqpoint{3.283550in}{0.852400in}}%
\pgfpathlineto{\pgfqpoint{3.287101in}{0.775185in}}%
\pgfpathlineto{\pgfqpoint{3.290651in}{0.742451in}}%
\pgfpathlineto{\pgfqpoint{3.294202in}{0.725139in}}%
\pgfpathlineto{\pgfqpoint{3.295977in}{0.723271in}}%
\pgfpathlineto{\pgfqpoint{3.297752in}{0.729300in}}%
\pgfpathlineto{\pgfqpoint{3.304853in}{0.834278in}}%
\pgfpathlineto{\pgfqpoint{3.322606in}{1.156014in}}%
\pgfpathlineto{\pgfqpoint{3.326156in}{1.171762in}}%
\pgfpathlineto{\pgfqpoint{3.327931in}{1.171139in}}%
\pgfpathlineto{\pgfqpoint{3.331482in}{1.156537in}}%
\pgfpathlineto{\pgfqpoint{3.335032in}{1.138330in}}%
\pgfpathlineto{\pgfqpoint{3.340358in}{1.087414in}}%
\pgfpathlineto{\pgfqpoint{3.349234in}{0.996344in}}%
\pgfpathlineto{\pgfqpoint{3.354560in}{0.959246in}}%
\pgfpathlineto{\pgfqpoint{3.359885in}{0.938550in}}%
\pgfpathlineto{\pgfqpoint{3.365211in}{0.928726in}}%
\pgfpathlineto{\pgfqpoint{3.372312in}{0.918102in}}%
\pgfpathlineto{\pgfqpoint{3.377638in}{0.907793in}}%
\pgfpathlineto{\pgfqpoint{3.384739in}{0.886759in}}%
\pgfpathlineto{\pgfqpoint{3.390064in}{0.871625in}}%
\pgfpathlineto{\pgfqpoint{3.391840in}{0.869685in}}%
\pgfpathlineto{\pgfqpoint{3.393615in}{0.869837in}}%
\pgfpathlineto{\pgfqpoint{3.395390in}{0.872239in}}%
\pgfpathlineto{\pgfqpoint{3.398941in}{0.883247in}}%
\pgfpathlineto{\pgfqpoint{3.413142in}{0.940950in}}%
\pgfpathlineto{\pgfqpoint{3.423794in}{0.976134in}}%
\pgfpathlineto{\pgfqpoint{3.427344in}{0.981752in}}%
\pgfpathlineto{\pgfqpoint{3.429120in}{0.982386in}}%
\pgfpathlineto{\pgfqpoint{3.432670in}{0.980270in}}%
\pgfpathlineto{\pgfqpoint{3.437996in}{0.975853in}}%
\pgfpathlineto{\pgfqpoint{3.439771in}{0.975524in}}%
\pgfpathlineto{\pgfqpoint{3.441546in}{0.976366in}}%
\pgfpathlineto{\pgfqpoint{3.445097in}{0.982709in}}%
\pgfpathlineto{\pgfqpoint{3.450422in}{1.000222in}}%
\pgfpathlineto{\pgfqpoint{3.457523in}{1.023942in}}%
\pgfpathlineto{\pgfqpoint{3.469950in}{1.058563in}}%
\pgfpathlineto{\pgfqpoint{3.485927in}{1.125923in}}%
\pgfpathlineto{\pgfqpoint{3.496578in}{1.153951in}}%
\pgfpathlineto{\pgfqpoint{3.505455in}{1.177404in}}%
\pgfpathlineto{\pgfqpoint{3.510780in}{1.191226in}}%
\pgfpathlineto{\pgfqpoint{3.514331in}{1.196434in}}%
\pgfpathlineto{\pgfqpoint{3.526757in}{1.207411in}}%
\pgfpathlineto{\pgfqpoint{3.535634in}{1.220684in}}%
\pgfpathlineto{\pgfqpoint{3.539184in}{1.229196in}}%
\pgfpathlineto{\pgfqpoint{3.544510in}{1.250972in}}%
\pgfpathlineto{\pgfqpoint{3.551611in}{1.281290in}}%
\pgfpathlineto{\pgfqpoint{3.562262in}{1.312921in}}%
\pgfpathlineto{\pgfqpoint{3.565812in}{1.316317in}}%
\pgfpathlineto{\pgfqpoint{3.567588in}{1.314707in}}%
\pgfpathlineto{\pgfqpoint{3.571138in}{1.305724in}}%
\pgfpathlineto{\pgfqpoint{3.576464in}{1.290783in}}%
\pgfpathlineto{\pgfqpoint{3.580014in}{1.287811in}}%
\pgfpathlineto{\pgfqpoint{3.583565in}{1.290588in}}%
\pgfpathlineto{\pgfqpoint{3.587115in}{1.298227in}}%
\pgfpathlineto{\pgfqpoint{3.595991in}{1.322481in}}%
\pgfpathlineto{\pgfqpoint{3.603092in}{1.331525in}}%
\pgfpathlineto{\pgfqpoint{3.606643in}{1.340581in}}%
\pgfpathlineto{\pgfqpoint{3.624395in}{1.402436in}}%
\pgfpathlineto{\pgfqpoint{3.627946in}{1.407755in}}%
\pgfpathlineto{\pgfqpoint{3.635047in}{1.408817in}}%
\pgfpathlineto{\pgfqpoint{3.645698in}{1.402033in}}%
\pgfpathlineto{\pgfqpoint{3.649248in}{1.402460in}}%
\pgfpathlineto{\pgfqpoint{3.652799in}{1.406765in}}%
\pgfpathlineto{\pgfqpoint{3.656349in}{1.415567in}}%
\pgfpathlineto{\pgfqpoint{3.663450in}{1.444904in}}%
\pgfpathlineto{\pgfqpoint{3.674102in}{1.484804in}}%
\pgfpathlineto{\pgfqpoint{3.677652in}{1.489192in}}%
\pgfpathlineto{\pgfqpoint{3.679427in}{1.487859in}}%
\pgfpathlineto{\pgfqpoint{3.682978in}{1.479742in}}%
\pgfpathlineto{\pgfqpoint{3.691854in}{1.457684in}}%
\pgfpathlineto{\pgfqpoint{3.695404in}{1.454455in}}%
\pgfpathlineto{\pgfqpoint{3.698955in}{1.455408in}}%
\pgfpathlineto{\pgfqpoint{3.714932in}{1.471337in}}%
\pgfpathlineto{\pgfqpoint{3.718482in}{1.478852in}}%
\pgfpathlineto{\pgfqpoint{3.723808in}{1.498338in}}%
\pgfpathlineto{\pgfqpoint{3.730909in}{1.523965in}}%
\pgfpathlineto{\pgfqpoint{3.732684in}{1.525608in}}%
\pgfpathlineto{\pgfqpoint{3.734460in}{1.525124in}}%
\pgfpathlineto{\pgfqpoint{3.738010in}{1.519355in}}%
\pgfpathlineto{\pgfqpoint{3.743336in}{1.500351in}}%
\pgfpathlineto{\pgfqpoint{3.748661in}{1.482387in}}%
\pgfpathlineto{\pgfqpoint{3.752212in}{1.478279in}}%
\pgfpathlineto{\pgfqpoint{3.755762in}{1.481152in}}%
\pgfpathlineto{\pgfqpoint{3.759313in}{1.486867in}}%
\pgfpathlineto{\pgfqpoint{3.769964in}{1.508556in}}%
\pgfpathlineto{\pgfqpoint{3.775290in}{1.513652in}}%
\pgfpathlineto{\pgfqpoint{3.778840in}{1.515397in}}%
\pgfpathlineto{\pgfqpoint{3.780616in}{1.515020in}}%
\pgfpathlineto{\pgfqpoint{3.784166in}{1.509645in}}%
\pgfpathlineto{\pgfqpoint{3.794817in}{1.487753in}}%
\pgfpathlineto{\pgfqpoint{3.798368in}{1.485189in}}%
\pgfpathlineto{\pgfqpoint{3.801918in}{1.484669in}}%
\pgfpathlineto{\pgfqpoint{3.807244in}{1.485886in}}%
\pgfpathlineto{\pgfqpoint{3.814345in}{1.484887in}}%
\pgfpathlineto{\pgfqpoint{3.817895in}{1.486990in}}%
\pgfpathlineto{\pgfqpoint{3.821446in}{1.492938in}}%
\pgfpathlineto{\pgfqpoint{3.833873in}{1.518590in}}%
\pgfpathlineto{\pgfqpoint{3.840974in}{1.528912in}}%
\pgfpathlineto{\pgfqpoint{3.851625in}{1.541128in}}%
\pgfpathlineto{\pgfqpoint{3.856951in}{1.537685in}}%
\pgfpathlineto{\pgfqpoint{3.864052in}{1.532532in}}%
\pgfpathlineto{\pgfqpoint{3.869377in}{1.532391in}}%
\pgfpathlineto{\pgfqpoint{3.872928in}{1.534999in}}%
\pgfpathlineto{\pgfqpoint{3.876478in}{1.538011in}}%
\pgfpathlineto{\pgfqpoint{3.887130in}{1.540563in}}%
\pgfpathlineto{\pgfqpoint{3.890680in}{1.545700in}}%
\pgfpathlineto{\pgfqpoint{3.896006in}{1.558894in}}%
\pgfpathlineto{\pgfqpoint{3.903107in}{1.577349in}}%
\pgfpathlineto{\pgfqpoint{3.906657in}{1.579651in}}%
\pgfpathlineto{\pgfqpoint{3.913758in}{1.581545in}}%
\pgfpathlineto{\pgfqpoint{3.917309in}{1.589915in}}%
\pgfpathlineto{\pgfqpoint{3.933286in}{1.641669in}}%
\pgfpathlineto{\pgfqpoint{3.936836in}{1.644933in}}%
\pgfpathlineto{\pgfqpoint{3.940387in}{1.644344in}}%
\pgfpathlineto{\pgfqpoint{3.943937in}{1.640802in}}%
\pgfpathlineto{\pgfqpoint{3.959914in}{1.612324in}}%
\pgfpathlineto{\pgfqpoint{3.963465in}{1.611239in}}%
\pgfpathlineto{\pgfqpoint{3.965240in}{1.612252in}}%
\pgfpathlineto{\pgfqpoint{3.972341in}{1.619968in}}%
\pgfpathlineto{\pgfqpoint{3.975891in}{1.618908in}}%
\pgfpathlineto{\pgfqpoint{3.979442in}{1.615292in}}%
\pgfpathlineto{\pgfqpoint{3.982992in}{1.606427in}}%
\pgfpathlineto{\pgfqpoint{3.988318in}{1.589942in}}%
\pgfpathlineto{\pgfqpoint{3.990093in}{1.589411in}}%
\pgfpathlineto{\pgfqpoint{3.991868in}{1.591202in}}%
\pgfpathlineto{\pgfqpoint{3.995419in}{1.600512in}}%
\pgfpathlineto{\pgfqpoint{4.004295in}{1.627541in}}%
\pgfpathlineto{\pgfqpoint{4.006070in}{1.629910in}}%
\pgfpathlineto{\pgfqpoint{4.009621in}{1.630287in}}%
\pgfpathlineto{\pgfqpoint{4.013171in}{1.630969in}}%
\pgfpathlineto{\pgfqpoint{4.018497in}{1.630783in}}%
\pgfpathlineto{\pgfqpoint{4.022047in}{1.632561in}}%
\pgfpathlineto{\pgfqpoint{4.025598in}{1.631291in}}%
\pgfpathlineto{\pgfqpoint{4.029148in}{1.629437in}}%
\pgfpathlineto{\pgfqpoint{4.032699in}{1.629899in}}%
\pgfpathlineto{\pgfqpoint{4.039800in}{1.635197in}}%
\pgfpathlineto{\pgfqpoint{4.041575in}{1.636853in}}%
\pgfpathlineto{\pgfqpoint{4.048676in}{1.653406in}}%
\pgfpathlineto{\pgfqpoint{4.054001in}{1.665309in}}%
\pgfpathlineto{\pgfqpoint{4.057552in}{1.668000in}}%
\pgfpathlineto{\pgfqpoint{4.059327in}{1.667276in}}%
\pgfpathlineto{\pgfqpoint{4.062878in}{1.662136in}}%
\pgfpathlineto{\pgfqpoint{4.068203in}{1.652628in}}%
\pgfpathlineto{\pgfqpoint{4.071754in}{1.651548in}}%
\pgfpathlineto{\pgfqpoint{4.077079in}{1.654653in}}%
\pgfpathlineto{\pgfqpoint{4.078855in}{1.653827in}}%
\pgfpathlineto{\pgfqpoint{4.082405in}{1.645824in}}%
\pgfpathlineto{\pgfqpoint{4.085956in}{1.635275in}}%
\pgfpathlineto{\pgfqpoint{4.091281in}{1.627091in}}%
\pgfpathlineto{\pgfqpoint{4.093057in}{1.626218in}}%
\pgfpathlineto{\pgfqpoint{4.094832in}{1.627309in}}%
\pgfpathlineto{\pgfqpoint{4.098382in}{1.633689in}}%
\pgfpathlineto{\pgfqpoint{4.101933in}{1.646533in}}%
\pgfpathlineto{\pgfqpoint{4.109034in}{1.677497in}}%
\pgfpathlineto{\pgfqpoint{4.126786in}{1.726238in}}%
\pgfpathlineto{\pgfqpoint{4.128561in}{1.725350in}}%
\pgfpathlineto{\pgfqpoint{4.130336in}{1.721251in}}%
\pgfpathlineto{\pgfqpoint{4.133887in}{1.702926in}}%
\pgfpathlineto{\pgfqpoint{4.140988in}{1.658744in}}%
\pgfpathlineto{\pgfqpoint{4.146314in}{1.625276in}}%
\pgfpathlineto{\pgfqpoint{4.149864in}{1.615070in}}%
\pgfpathlineto{\pgfqpoint{4.153414in}{1.611818in}}%
\pgfpathlineto{\pgfqpoint{4.155190in}{1.611290in}}%
\pgfpathlineto{\pgfqpoint{4.156965in}{1.612028in}}%
\pgfpathlineto{\pgfqpoint{4.158740in}{1.614099in}}%
\pgfpathlineto{\pgfqpoint{4.165841in}{1.638307in}}%
\pgfpathlineto{\pgfqpoint{4.178268in}{1.681273in}}%
\pgfpathlineto{\pgfqpoint{4.181818in}{1.686560in}}%
\pgfpathlineto{\pgfqpoint{4.188919in}{1.692205in}}%
\pgfpathlineto{\pgfqpoint{4.194245in}{1.700893in}}%
\pgfpathlineto{\pgfqpoint{4.197795in}{1.702311in}}%
\pgfpathlineto{\pgfqpoint{4.201346in}{1.700994in}}%
\pgfpathlineto{\pgfqpoint{4.204896in}{1.695364in}}%
\pgfpathlineto{\pgfqpoint{4.208447in}{1.682726in}}%
\pgfpathlineto{\pgfqpoint{4.217323in}{1.639019in}}%
\pgfpathlineto{\pgfqpoint{4.220873in}{1.629839in}}%
\pgfpathlineto{\pgfqpoint{4.226199in}{1.623646in}}%
\pgfpathlineto{\pgfqpoint{4.236850in}{1.611142in}}%
\pgfpathlineto{\pgfqpoint{4.240401in}{1.611976in}}%
\pgfpathlineto{\pgfqpoint{4.243951in}{1.613338in}}%
\pgfpathlineto{\pgfqpoint{4.247502in}{1.611006in}}%
\pgfpathlineto{\pgfqpoint{4.251052in}{1.605925in}}%
\pgfpathlineto{\pgfqpoint{4.258153in}{1.592296in}}%
\pgfpathlineto{\pgfqpoint{4.261704in}{1.587705in}}%
\pgfpathlineto{\pgfqpoint{4.265254in}{1.582224in}}%
\pgfpathlineto{\pgfqpoint{4.270580in}{1.577530in}}%
\pgfpathlineto{\pgfqpoint{4.272355in}{1.577507in}}%
\pgfpathlineto{\pgfqpoint{4.274130in}{1.579363in}}%
\pgfpathlineto{\pgfqpoint{4.277681in}{1.588700in}}%
\pgfpathlineto{\pgfqpoint{4.288332in}{1.628099in}}%
\pgfpathlineto{\pgfqpoint{4.290107in}{1.629150in}}%
\pgfpathlineto{\pgfqpoint{4.291883in}{1.628129in}}%
\pgfpathlineto{\pgfqpoint{4.295433in}{1.621568in}}%
\pgfpathlineto{\pgfqpoint{4.302534in}{1.605225in}}%
\pgfpathlineto{\pgfqpoint{4.307860in}{1.598197in}}%
\pgfpathlineto{\pgfqpoint{4.313185in}{1.597146in}}%
\pgfpathlineto{\pgfqpoint{4.318511in}{1.596976in}}%
\pgfpathlineto{\pgfqpoint{4.325612in}{1.599619in}}%
\pgfpathlineto{\pgfqpoint{4.329162in}{1.605634in}}%
\pgfpathlineto{\pgfqpoint{4.332713in}{1.617629in}}%
\pgfpathlineto{\pgfqpoint{4.346915in}{1.687240in}}%
\pgfpathlineto{\pgfqpoint{4.348690in}{1.688722in}}%
\pgfpathlineto{\pgfqpoint{4.350465in}{1.687224in}}%
\pgfpathlineto{\pgfqpoint{4.352241in}{1.682872in}}%
\pgfpathlineto{\pgfqpoint{4.355791in}{1.664712in}}%
\pgfpathlineto{\pgfqpoint{4.362892in}{1.602397in}}%
\pgfpathlineto{\pgfqpoint{4.368218in}{1.566118in}}%
\pgfpathlineto{\pgfqpoint{4.369993in}{1.561657in}}%
\pgfpathlineto{\pgfqpoint{4.371768in}{1.560753in}}%
\pgfpathlineto{\pgfqpoint{4.373543in}{1.562619in}}%
\pgfpathlineto{\pgfqpoint{4.377094in}{1.573978in}}%
\pgfpathlineto{\pgfqpoint{4.385970in}{1.615013in}}%
\pgfpathlineto{\pgfqpoint{4.389520in}{1.622430in}}%
\pgfpathlineto{\pgfqpoint{4.393071in}{1.622557in}}%
\pgfpathlineto{\pgfqpoint{4.400172in}{1.619906in}}%
\pgfpathlineto{\pgfqpoint{4.403722in}{1.620894in}}%
\pgfpathlineto{\pgfqpoint{4.409048in}{1.623541in}}%
\pgfpathlineto{\pgfqpoint{4.412598in}{1.622614in}}%
\pgfpathlineto{\pgfqpoint{4.416149in}{1.616992in}}%
\pgfpathlineto{\pgfqpoint{4.426800in}{1.590654in}}%
\pgfpathlineto{\pgfqpoint{4.428576in}{1.591572in}}%
\pgfpathlineto{\pgfqpoint{4.432126in}{1.599735in}}%
\pgfpathlineto{\pgfqpoint{4.437452in}{1.611253in}}%
\pgfpathlineto{\pgfqpoint{4.439227in}{1.612333in}}%
\pgfpathlineto{\pgfqpoint{4.441002in}{1.611558in}}%
\pgfpathlineto{\pgfqpoint{4.446328in}{1.606520in}}%
\pgfpathlineto{\pgfqpoint{4.449878in}{1.607182in}}%
\pgfpathlineto{\pgfqpoint{4.455204in}{1.608862in}}%
\pgfpathlineto{\pgfqpoint{4.456979in}{1.607722in}}%
\pgfpathlineto{\pgfqpoint{4.460530in}{1.599824in}}%
\pgfpathlineto{\pgfqpoint{4.471181in}{1.567854in}}%
\pgfpathlineto{\pgfqpoint{4.474732in}{1.565579in}}%
\pgfpathlineto{\pgfqpoint{4.485383in}{1.563208in}}%
\pgfpathlineto{\pgfqpoint{4.487158in}{1.564245in}}%
\pgfpathlineto{\pgfqpoint{4.488933in}{1.567138in}}%
\pgfpathlineto{\pgfqpoint{4.492484in}{1.579582in}}%
\pgfpathlineto{\pgfqpoint{4.504911in}{1.639390in}}%
\pgfpathlineto{\pgfqpoint{4.508461in}{1.643838in}}%
\pgfpathlineto{\pgfqpoint{4.517337in}{1.646978in}}%
\pgfpathlineto{\pgfqpoint{4.526213in}{1.657452in}}%
\pgfpathlineto{\pgfqpoint{4.529764in}{1.655951in}}%
\pgfpathlineto{\pgfqpoint{4.535089in}{1.650412in}}%
\pgfpathlineto{\pgfqpoint{4.543966in}{1.637212in}}%
\pgfpathlineto{\pgfqpoint{4.549291in}{1.618835in}}%
\pgfpathlineto{\pgfqpoint{4.556392in}{1.590293in}}%
\pgfpathlineto{\pgfqpoint{4.561718in}{1.580059in}}%
\pgfpathlineto{\pgfqpoint{4.565268in}{1.577445in}}%
\pgfpathlineto{\pgfqpoint{4.568819in}{1.579193in}}%
\pgfpathlineto{\pgfqpoint{4.577695in}{1.586003in}}%
\pgfpathlineto{\pgfqpoint{4.581246in}{1.583034in}}%
\pgfpathlineto{\pgfqpoint{4.584796in}{1.577288in}}%
\pgfpathlineto{\pgfqpoint{4.597223in}{1.549175in}}%
\pgfpathlineto{\pgfqpoint{4.598998in}{1.549549in}}%
\pgfpathlineto{\pgfqpoint{4.602548in}{1.556011in}}%
\pgfpathlineto{\pgfqpoint{4.632727in}{1.623717in}}%
\pgfpathlineto{\pgfqpoint{4.636278in}{1.625386in}}%
\pgfpathlineto{\pgfqpoint{4.639828in}{1.622622in}}%
\pgfpathlineto{\pgfqpoint{4.645154in}{1.613875in}}%
\pgfpathlineto{\pgfqpoint{4.648704in}{1.602988in}}%
\pgfpathlineto{\pgfqpoint{4.654030in}{1.579140in}}%
\pgfpathlineto{\pgfqpoint{4.661131in}{1.546690in}}%
\pgfpathlineto{\pgfqpoint{4.664681in}{1.538272in}}%
\pgfpathlineto{\pgfqpoint{4.666457in}{1.537129in}}%
\pgfpathlineto{\pgfqpoint{4.668232in}{1.538533in}}%
\pgfpathlineto{\pgfqpoint{4.673558in}{1.551191in}}%
\pgfpathlineto{\pgfqpoint{4.678883in}{1.563974in}}%
\pgfpathlineto{\pgfqpoint{4.682434in}{1.567025in}}%
\pgfpathlineto{\pgfqpoint{4.687759in}{1.566829in}}%
\pgfpathlineto{\pgfqpoint{4.693085in}{1.563134in}}%
\pgfpathlineto{\pgfqpoint{4.696636in}{1.558368in}}%
\pgfpathlineto{\pgfqpoint{4.703737in}{1.539291in}}%
\pgfpathlineto{\pgfqpoint{4.710838in}{1.523228in}}%
\pgfpathlineto{\pgfqpoint{4.712613in}{1.521898in}}%
\pgfpathlineto{\pgfqpoint{4.714388in}{1.522023in}}%
\pgfpathlineto{\pgfqpoint{4.716163in}{1.523982in}}%
\pgfpathlineto{\pgfqpoint{4.719714in}{1.534938in}}%
\pgfpathlineto{\pgfqpoint{4.726815in}{1.574710in}}%
\pgfpathlineto{\pgfqpoint{4.732140in}{1.600186in}}%
\pgfpathlineto{\pgfqpoint{4.735691in}{1.605094in}}%
\pgfpathlineto{\pgfqpoint{4.737466in}{1.605005in}}%
\pgfpathlineto{\pgfqpoint{4.744567in}{1.599340in}}%
\pgfpathlineto{\pgfqpoint{4.748117in}{1.602217in}}%
\pgfpathlineto{\pgfqpoint{4.753443in}{1.611243in}}%
\pgfpathlineto{\pgfqpoint{4.762319in}{1.635898in}}%
\pgfpathlineto{\pgfqpoint{4.769420in}{1.654591in}}%
\pgfpathlineto{\pgfqpoint{4.774746in}{1.665581in}}%
\pgfpathlineto{\pgfqpoint{4.776521in}{1.666439in}}%
\pgfpathlineto{\pgfqpoint{4.780072in}{1.664344in}}%
\pgfpathlineto{\pgfqpoint{4.787173in}{1.654903in}}%
\pgfpathlineto{\pgfqpoint{4.790723in}{1.655027in}}%
\pgfpathlineto{\pgfqpoint{4.796049in}{1.656268in}}%
\pgfpathlineto{\pgfqpoint{4.797824in}{1.655821in}}%
\pgfpathlineto{\pgfqpoint{4.799599in}{1.653739in}}%
\pgfpathlineto{\pgfqpoint{4.803150in}{1.644238in}}%
\pgfpathlineto{\pgfqpoint{4.813801in}{1.610031in}}%
\pgfpathlineto{\pgfqpoint{4.819127in}{1.603126in}}%
\pgfpathlineto{\pgfqpoint{4.820902in}{1.602584in}}%
\pgfpathlineto{\pgfqpoint{4.824452in}{1.606651in}}%
\pgfpathlineto{\pgfqpoint{4.829778in}{1.620714in}}%
\pgfpathlineto{\pgfqpoint{4.849306in}{1.686334in}}%
\pgfpathlineto{\pgfqpoint{4.851081in}{1.685140in}}%
\pgfpathlineto{\pgfqpoint{4.854631in}{1.671989in}}%
\pgfpathlineto{\pgfqpoint{4.867058in}{1.604786in}}%
\pgfpathlineto{\pgfqpoint{4.868833in}{1.603510in}}%
\pgfpathlineto{\pgfqpoint{4.870608in}{1.604228in}}%
\pgfpathlineto{\pgfqpoint{4.874159in}{1.610011in}}%
\pgfpathlineto{\pgfqpoint{4.879485in}{1.618007in}}%
\pgfpathlineto{\pgfqpoint{4.886586in}{1.624180in}}%
\pgfpathlineto{\pgfqpoint{4.890136in}{1.627514in}}%
\pgfpathlineto{\pgfqpoint{4.895462in}{1.637427in}}%
\pgfpathlineto{\pgfqpoint{4.904338in}{1.654874in}}%
\pgfpathlineto{\pgfqpoint{4.909664in}{1.659804in}}%
\pgfpathlineto{\pgfqpoint{4.913214in}{1.659557in}}%
\pgfpathlineto{\pgfqpoint{4.916764in}{1.657266in}}%
\pgfpathlineto{\pgfqpoint{4.920315in}{1.651516in}}%
\pgfpathlineto{\pgfqpoint{4.927416in}{1.637064in}}%
\pgfpathlineto{\pgfqpoint{4.930966in}{1.634224in}}%
\pgfpathlineto{\pgfqpoint{4.941618in}{1.634317in}}%
\pgfpathlineto{\pgfqpoint{4.945168in}{1.629800in}}%
\pgfpathlineto{\pgfqpoint{4.952269in}{1.616293in}}%
\pgfpathlineto{\pgfqpoint{4.954044in}{1.614937in}}%
\pgfpathlineto{\pgfqpoint{4.955820in}{1.615021in}}%
\pgfpathlineto{\pgfqpoint{4.959370in}{1.619683in}}%
\pgfpathlineto{\pgfqpoint{4.964696in}{1.628412in}}%
\pgfpathlineto{\pgfqpoint{4.968246in}{1.631052in}}%
\pgfpathlineto{\pgfqpoint{4.970021in}{1.630404in}}%
\pgfpathlineto{\pgfqpoint{4.975347in}{1.622302in}}%
\pgfpathlineto{\pgfqpoint{4.993099in}{1.593396in}}%
\pgfpathlineto{\pgfqpoint{4.996650in}{1.580117in}}%
\pgfpathlineto{\pgfqpoint{5.012627in}{1.508174in}}%
\pgfpathlineto{\pgfqpoint{5.021503in}{1.473251in}}%
\pgfpathlineto{\pgfqpoint{5.026829in}{1.458667in}}%
\pgfpathlineto{\pgfqpoint{5.030379in}{1.453949in}}%
\pgfpathlineto{\pgfqpoint{5.032155in}{1.453034in}}%
\pgfpathlineto{\pgfqpoint{5.033930in}{1.453601in}}%
\pgfpathlineto{\pgfqpoint{5.037480in}{1.458715in}}%
\pgfpathlineto{\pgfqpoint{5.049907in}{1.483211in}}%
\pgfpathlineto{\pgfqpoint{5.062334in}{1.515550in}}%
\pgfpathlineto{\pgfqpoint{5.064109in}{1.516932in}}%
\pgfpathlineto{\pgfqpoint{5.065884in}{1.516328in}}%
\pgfpathlineto{\pgfqpoint{5.069435in}{1.510068in}}%
\pgfpathlineto{\pgfqpoint{5.081861in}{1.477354in}}%
\pgfpathlineto{\pgfqpoint{5.083636in}{1.477048in}}%
\pgfpathlineto{\pgfqpoint{5.087187in}{1.481187in}}%
\pgfpathlineto{\pgfqpoint{5.092513in}{1.488146in}}%
\pgfpathlineto{\pgfqpoint{5.096063in}{1.489781in}}%
\pgfpathlineto{\pgfqpoint{5.103164in}{1.491075in}}%
\pgfpathlineto{\pgfqpoint{5.106714in}{1.492281in}}%
\pgfpathlineto{\pgfqpoint{5.108490in}{1.490990in}}%
\pgfpathlineto{\pgfqpoint{5.112040in}{1.483425in}}%
\pgfpathlineto{\pgfqpoint{5.117366in}{1.462220in}}%
\pgfpathlineto{\pgfqpoint{5.122691in}{1.442866in}}%
\pgfpathlineto{\pgfqpoint{5.135118in}{1.408252in}}%
\pgfpathlineto{\pgfqpoint{5.140444in}{1.380375in}}%
\pgfpathlineto{\pgfqpoint{5.147545in}{1.341380in}}%
\pgfpathlineto{\pgfqpoint{5.154646in}{1.315372in}}%
\pgfpathlineto{\pgfqpoint{5.165297in}{1.279680in}}%
\pgfpathlineto{\pgfqpoint{5.184825in}{1.206349in}}%
\pgfpathlineto{\pgfqpoint{5.191926in}{1.176797in}}%
\pgfpathlineto{\pgfqpoint{5.197251in}{1.146373in}}%
\pgfpathlineto{\pgfqpoint{5.204352in}{1.090756in}}%
\pgfpathlineto{\pgfqpoint{5.216779in}{0.977228in}}%
\pgfpathlineto{\pgfqpoint{5.232756in}{0.817724in}}%
\pgfpathlineto{\pgfqpoint{5.241632in}{0.751807in}}%
\pgfpathlineto{\pgfqpoint{5.246958in}{0.726166in}}%
\pgfpathlineto{\pgfqpoint{5.252283in}{0.710966in}}%
\pgfpathlineto{\pgfqpoint{5.255834in}{0.706652in}}%
\pgfpathlineto{\pgfqpoint{5.259384in}{0.705603in}}%
\pgfpathlineto{\pgfqpoint{5.264710in}{0.707352in}}%
\pgfpathlineto{\pgfqpoint{5.271811in}{0.709685in}}%
\pgfpathlineto{\pgfqpoint{5.278912in}{0.709432in}}%
\pgfpathlineto{\pgfqpoint{5.287788in}{0.706579in}}%
\pgfpathlineto{\pgfqpoint{5.309091in}{0.698989in}}%
\pgfpathlineto{\pgfqpoint{5.321518in}{0.697482in}}%
\pgfpathlineto{\pgfqpoint{5.353472in}{0.696508in}}%
\pgfpathlineto{\pgfqpoint{5.440458in}{0.696472in}}%
\pgfpathlineto{\pgfqpoint{5.491940in}{0.696251in}}%
\pgfpathlineto{\pgfqpoint{5.534545in}{0.696000in}}%
\pgfpathlineto{\pgfqpoint{5.534545in}{0.696000in}}%
\pgfusepath{stroke}%
\end{pgfscope}%
\begin{pgfscope}%
\pgfpathrectangle{\pgfqpoint{0.800000in}{0.528000in}}{\pgfqpoint{4.960000in}{3.696000in}} %
\pgfusepath{clip}%
\pgfsetrectcap%
\pgfsetroundjoin%
\pgfsetlinewidth{1.505625pt}%
\definecolor{currentstroke}{rgb}{0.580392,0.403922,0.741176}%
\pgfsetstrokecolor{currentstroke}%
\pgfsetdash{}{0pt}%
\pgfpathmoveto{\pgfqpoint{1.025455in}{0.696000in}}%
\pgfpathlineto{\pgfqpoint{1.139069in}{0.696702in}}%
\pgfpathlineto{\pgfqpoint{1.140845in}{0.697473in}}%
\pgfpathlineto{\pgfqpoint{1.146170in}{0.705520in}}%
\pgfpathlineto{\pgfqpoint{1.149721in}{0.715917in}}%
\pgfpathlineto{\pgfqpoint{1.151496in}{0.698705in}}%
\pgfpathlineto{\pgfqpoint{1.155047in}{0.702218in}}%
\pgfpathlineto{\pgfqpoint{1.165698in}{0.715670in}}%
\pgfpathlineto{\pgfqpoint{1.169248in}{0.717735in}}%
\pgfpathlineto{\pgfqpoint{1.171024in}{0.716378in}}%
\pgfpathlineto{\pgfqpoint{1.172799in}{0.726837in}}%
\pgfpathlineto{\pgfqpoint{1.176349in}{0.727315in}}%
\pgfpathlineto{\pgfqpoint{1.178125in}{0.725460in}}%
\pgfpathlineto{\pgfqpoint{1.181675in}{0.727487in}}%
\pgfpathlineto{\pgfqpoint{1.183450in}{0.726390in}}%
\pgfpathlineto{\pgfqpoint{1.185225in}{0.749228in}}%
\pgfpathlineto{\pgfqpoint{1.192326in}{0.741854in}}%
\pgfpathlineto{\pgfqpoint{1.194102in}{0.740755in}}%
\pgfpathlineto{\pgfqpoint{1.195877in}{0.777040in}}%
\pgfpathlineto{\pgfqpoint{1.201203in}{0.795715in}}%
\pgfpathlineto{\pgfqpoint{1.204753in}{0.805866in}}%
\pgfpathlineto{\pgfqpoint{1.210079in}{0.812442in}}%
\pgfpathlineto{\pgfqpoint{1.242033in}{0.844528in}}%
\pgfpathlineto{\pgfqpoint{1.247359in}{0.853720in}}%
\pgfpathlineto{\pgfqpoint{1.252684in}{0.865976in}}%
\pgfpathlineto{\pgfqpoint{1.273987in}{0.920803in}}%
\pgfpathlineto{\pgfqpoint{1.288189in}{0.938145in}}%
\pgfpathlineto{\pgfqpoint{1.291739in}{0.939253in}}%
\pgfpathlineto{\pgfqpoint{1.300616in}{0.937009in}}%
\pgfpathlineto{\pgfqpoint{1.304166in}{0.939128in}}%
\pgfpathlineto{\pgfqpoint{1.307717in}{0.941364in}}%
\pgfpathlineto{\pgfqpoint{1.334345in}{0.963795in}}%
\pgfpathlineto{\pgfqpoint{1.344996in}{0.979317in}}%
\pgfpathlineto{\pgfqpoint{1.360974in}{1.011526in}}%
\pgfpathlineto{\pgfqpoint{1.373400in}{1.030881in}}%
\pgfpathlineto{\pgfqpoint{1.380501in}{1.043127in}}%
\pgfpathlineto{\pgfqpoint{1.394703in}{1.073685in}}%
\pgfpathlineto{\pgfqpoint{1.401804in}{1.082130in}}%
\pgfpathlineto{\pgfqpoint{1.414230in}{1.100216in}}%
\pgfpathlineto{\pgfqpoint{1.430208in}{1.115034in}}%
\pgfpathlineto{\pgfqpoint{1.433758in}{1.115592in}}%
\pgfpathlineto{\pgfqpoint{1.437309in}{1.114397in}}%
\pgfpathlineto{\pgfqpoint{1.447960in}{1.104721in}}%
\pgfpathlineto{\pgfqpoint{1.462162in}{1.102313in}}%
\pgfpathlineto{\pgfqpoint{1.472813in}{1.095575in}}%
\pgfpathlineto{\pgfqpoint{1.490565in}{1.089598in}}%
\pgfpathlineto{\pgfqpoint{1.499442in}{1.085290in}}%
\pgfpathlineto{\pgfqpoint{1.506543in}{1.082783in}}%
\pgfpathlineto{\pgfqpoint{1.511868in}{1.080405in}}%
\pgfpathlineto{\pgfqpoint{1.515419in}{1.080568in}}%
\pgfpathlineto{\pgfqpoint{1.518969in}{1.082242in}}%
\pgfpathlineto{\pgfqpoint{1.526070in}{1.087352in}}%
\pgfpathlineto{\pgfqpoint{1.531396in}{1.090925in}}%
\pgfpathlineto{\pgfqpoint{1.536722in}{1.091799in}}%
\pgfpathlineto{\pgfqpoint{1.543822in}{1.087216in}}%
\pgfpathlineto{\pgfqpoint{1.550923in}{1.083231in}}%
\pgfpathlineto{\pgfqpoint{1.554474in}{1.084510in}}%
\pgfpathlineto{\pgfqpoint{1.559800in}{1.089591in}}%
\pgfpathlineto{\pgfqpoint{1.566901in}{1.095691in}}%
\pgfpathlineto{\pgfqpoint{1.584653in}{1.103669in}}%
\pgfpathlineto{\pgfqpoint{1.591754in}{1.099185in}}%
\pgfpathlineto{\pgfqpoint{1.598855in}{1.095463in}}%
\pgfpathlineto{\pgfqpoint{1.616607in}{1.093122in}}%
\pgfpathlineto{\pgfqpoint{1.627258in}{1.098950in}}%
\pgfpathlineto{\pgfqpoint{1.646786in}{1.099880in}}%
\pgfpathlineto{\pgfqpoint{1.659213in}{1.103816in}}%
\pgfpathlineto{\pgfqpoint{1.671639in}{1.115955in}}%
\pgfpathlineto{\pgfqpoint{1.685841in}{1.123400in}}%
\pgfpathlineto{\pgfqpoint{1.689392in}{1.123171in}}%
\pgfpathlineto{\pgfqpoint{1.692942in}{1.120678in}}%
\pgfpathlineto{\pgfqpoint{1.698268in}{1.111311in}}%
\pgfpathlineto{\pgfqpoint{1.703593in}{1.096981in}}%
\pgfpathlineto{\pgfqpoint{1.710694in}{1.077969in}}%
\pgfpathlineto{\pgfqpoint{1.714245in}{1.074246in}}%
\pgfpathlineto{\pgfqpoint{1.716020in}{1.074150in}}%
\pgfpathlineto{\pgfqpoint{1.719571in}{1.077938in}}%
\pgfpathlineto{\pgfqpoint{1.728447in}{1.091259in}}%
\pgfpathlineto{\pgfqpoint{1.731997in}{1.093113in}}%
\pgfpathlineto{\pgfqpoint{1.735548in}{1.092249in}}%
\pgfpathlineto{\pgfqpoint{1.749749in}{1.082215in}}%
\pgfpathlineto{\pgfqpoint{1.755075in}{1.082912in}}%
\pgfpathlineto{\pgfqpoint{1.760401in}{1.087204in}}%
\pgfpathlineto{\pgfqpoint{1.771052in}{1.098724in}}%
\pgfpathlineto{\pgfqpoint{1.776378in}{1.100551in}}%
\pgfpathlineto{\pgfqpoint{1.779928in}{1.100138in}}%
\pgfpathlineto{\pgfqpoint{1.783479in}{1.098376in}}%
\pgfpathlineto{\pgfqpoint{1.788805in}{1.091834in}}%
\pgfpathlineto{\pgfqpoint{1.799456in}{1.077540in}}%
\pgfpathlineto{\pgfqpoint{1.803006in}{1.076571in}}%
\pgfpathlineto{\pgfqpoint{1.815433in}{1.081835in}}%
\pgfpathlineto{\pgfqpoint{1.822534in}{1.085492in}}%
\pgfpathlineto{\pgfqpoint{1.831410in}{1.087209in}}%
\pgfpathlineto{\pgfqpoint{1.843837in}{1.084860in}}%
\pgfpathlineto{\pgfqpoint{1.859814in}{1.075664in}}%
\pgfpathlineto{\pgfqpoint{1.865140in}{1.068396in}}%
\pgfpathlineto{\pgfqpoint{1.872241in}{1.055270in}}%
\pgfpathlineto{\pgfqpoint{1.875791in}{1.052302in}}%
\pgfpathlineto{\pgfqpoint{1.879341in}{1.053218in}}%
\pgfpathlineto{\pgfqpoint{1.888218in}{1.059182in}}%
\pgfpathlineto{\pgfqpoint{1.891768in}{1.056001in}}%
\pgfpathlineto{\pgfqpoint{1.898869in}{1.044732in}}%
\pgfpathlineto{\pgfqpoint{1.909520in}{1.027625in}}%
\pgfpathlineto{\pgfqpoint{1.914846in}{1.023117in}}%
\pgfpathlineto{\pgfqpoint{1.916621in}{1.022718in}}%
\pgfpathlineto{\pgfqpoint{1.918397in}{1.020303in}}%
\pgfpathlineto{\pgfqpoint{1.921947in}{1.021732in}}%
\pgfpathlineto{\pgfqpoint{1.925497in}{1.025728in}}%
\pgfpathlineto{\pgfqpoint{1.945025in}{1.050997in}}%
\pgfpathlineto{\pgfqpoint{1.950351in}{1.054594in}}%
\pgfpathlineto{\pgfqpoint{1.953901in}{1.054591in}}%
\pgfpathlineto{\pgfqpoint{1.959227in}{1.052099in}}%
\pgfpathlineto{\pgfqpoint{1.962777in}{1.050719in}}%
\pgfpathlineto{\pgfqpoint{1.966328in}{1.051619in}}%
\pgfpathlineto{\pgfqpoint{1.971654in}{1.057446in}}%
\pgfpathlineto{\pgfqpoint{1.998282in}{1.096396in}}%
\pgfpathlineto{\pgfqpoint{2.003608in}{1.100813in}}%
\pgfpathlineto{\pgfqpoint{2.026686in}{1.107536in}}%
\pgfpathlineto{\pgfqpoint{2.032011in}{1.107505in}}%
\pgfpathlineto{\pgfqpoint{2.037337in}{1.106596in}}%
\pgfpathlineto{\pgfqpoint{2.044438in}{1.105195in}}%
\pgfpathlineto{\pgfqpoint{2.047989in}{1.104578in}}%
\pgfpathlineto{\pgfqpoint{2.051539in}{1.102911in}}%
\pgfpathlineto{\pgfqpoint{2.058640in}{1.095504in}}%
\pgfpathlineto{\pgfqpoint{2.063966in}{1.090712in}}%
\pgfpathlineto{\pgfqpoint{2.067516in}{1.089121in}}%
\pgfpathlineto{\pgfqpoint{2.071067in}{1.089099in}}%
\pgfpathlineto{\pgfqpoint{2.076392in}{1.092382in}}%
\pgfpathlineto{\pgfqpoint{2.081718in}{1.098675in}}%
\pgfpathlineto{\pgfqpoint{2.097695in}{1.119728in}}%
\pgfpathlineto{\pgfqpoint{2.101246in}{1.119581in}}%
\pgfpathlineto{\pgfqpoint{2.104796in}{1.117804in}}%
\pgfpathlineto{\pgfqpoint{2.111897in}{1.112912in}}%
\pgfpathlineto{\pgfqpoint{2.118998in}{1.108620in}}%
\pgfpathlineto{\pgfqpoint{2.126099in}{1.106940in}}%
\pgfpathlineto{\pgfqpoint{2.140301in}{1.107976in}}%
\pgfpathlineto{\pgfqpoint{2.147402in}{1.109289in}}%
\pgfpathlineto{\pgfqpoint{2.154503in}{1.109350in}}%
\pgfpathlineto{\pgfqpoint{2.166929in}{1.106643in}}%
\pgfpathlineto{\pgfqpoint{2.175805in}{1.105676in}}%
\pgfpathlineto{\pgfqpoint{2.181131in}{1.101421in}}%
\pgfpathlineto{\pgfqpoint{2.191782in}{1.087424in}}%
\pgfpathlineto{\pgfqpoint{2.195333in}{1.087314in}}%
\pgfpathlineto{\pgfqpoint{2.198883in}{1.091988in}}%
\pgfpathlineto{\pgfqpoint{2.204209in}{1.104307in}}%
\pgfpathlineto{\pgfqpoint{2.211310in}{1.121239in}}%
\pgfpathlineto{\pgfqpoint{2.236163in}{1.165287in}}%
\pgfpathlineto{\pgfqpoint{2.243264in}{1.171820in}}%
\pgfpathlineto{\pgfqpoint{2.246815in}{1.173158in}}%
\pgfpathlineto{\pgfqpoint{2.250365in}{1.172596in}}%
\pgfpathlineto{\pgfqpoint{2.257466in}{1.169885in}}%
\pgfpathlineto{\pgfqpoint{2.262792in}{1.170204in}}%
\pgfpathlineto{\pgfqpoint{2.271668in}{1.170637in}}%
\pgfpathlineto{\pgfqpoint{2.282319in}{1.166237in}}%
\pgfpathlineto{\pgfqpoint{2.285870in}{1.164641in}}%
\pgfpathlineto{\pgfqpoint{2.292971in}{1.160476in}}%
\pgfpathlineto{\pgfqpoint{2.301847in}{1.152381in}}%
\pgfpathlineto{\pgfqpoint{2.314273in}{1.129114in}}%
\pgfpathlineto{\pgfqpoint{2.319599in}{1.118166in}}%
\pgfpathlineto{\pgfqpoint{2.323150in}{1.114026in}}%
\pgfpathlineto{\pgfqpoint{2.324925in}{1.113190in}}%
\pgfpathlineto{\pgfqpoint{2.328475in}{1.116462in}}%
\pgfpathlineto{\pgfqpoint{2.333801in}{1.127493in}}%
\pgfpathlineto{\pgfqpoint{2.340902in}{1.143788in}}%
\pgfpathlineto{\pgfqpoint{2.348003in}{1.154266in}}%
\pgfpathlineto{\pgfqpoint{2.353329in}{1.158398in}}%
\pgfpathlineto{\pgfqpoint{2.358654in}{1.158862in}}%
\pgfpathlineto{\pgfqpoint{2.365755in}{1.154581in}}%
\pgfpathlineto{\pgfqpoint{2.372856in}{1.149465in}}%
\pgfpathlineto{\pgfqpoint{2.378182in}{1.149053in}}%
\pgfpathlineto{\pgfqpoint{2.385283in}{1.153480in}}%
\pgfpathlineto{\pgfqpoint{2.410136in}{1.176137in}}%
\pgfpathlineto{\pgfqpoint{2.417237in}{1.178059in}}%
\pgfpathlineto{\pgfqpoint{2.431439in}{1.184665in}}%
\pgfpathlineto{\pgfqpoint{2.440315in}{1.182852in}}%
\pgfpathlineto{\pgfqpoint{2.443865in}{1.183104in}}%
\pgfpathlineto{\pgfqpoint{2.447416in}{1.183829in}}%
\pgfpathlineto{\pgfqpoint{2.454517in}{1.188190in}}%
\pgfpathlineto{\pgfqpoint{2.459843in}{1.195927in}}%
\pgfpathlineto{\pgfqpoint{2.465168in}{1.204642in}}%
\pgfpathlineto{\pgfqpoint{2.470494in}{1.208608in}}%
\pgfpathlineto{\pgfqpoint{2.475820in}{1.209873in}}%
\pgfpathlineto{\pgfqpoint{2.482921in}{1.212265in}}%
\pgfpathlineto{\pgfqpoint{2.488246in}{1.209629in}}%
\pgfpathlineto{\pgfqpoint{2.498898in}{1.199612in}}%
\pgfpathlineto{\pgfqpoint{2.505999in}{1.197842in}}%
\pgfpathlineto{\pgfqpoint{2.511324in}{1.198817in}}%
\pgfpathlineto{\pgfqpoint{2.514875in}{1.199834in}}%
\pgfpathlineto{\pgfqpoint{2.520200in}{1.199295in}}%
\pgfpathlineto{\pgfqpoint{2.523751in}{1.197833in}}%
\pgfpathlineto{\pgfqpoint{2.529077in}{1.193634in}}%
\pgfpathlineto{\pgfqpoint{2.534402in}{1.187976in}}%
\pgfpathlineto{\pgfqpoint{2.541503in}{1.178234in}}%
\pgfpathlineto{\pgfqpoint{2.546829in}{1.167827in}}%
\pgfpathlineto{\pgfqpoint{2.553930in}{1.154566in}}%
\pgfpathlineto{\pgfqpoint{2.557480in}{1.151931in}}%
\pgfpathlineto{\pgfqpoint{2.559256in}{1.151721in}}%
\pgfpathlineto{\pgfqpoint{2.566356in}{1.157184in}}%
\pgfpathlineto{\pgfqpoint{2.584109in}{1.170286in}}%
\pgfpathlineto{\pgfqpoint{2.589435in}{1.170840in}}%
\pgfpathlineto{\pgfqpoint{2.592985in}{1.170395in}}%
\pgfpathlineto{\pgfqpoint{2.596535in}{1.168979in}}%
\pgfpathlineto{\pgfqpoint{2.616063in}{1.151252in}}%
\pgfpathlineto{\pgfqpoint{2.621389in}{1.142748in}}%
\pgfpathlineto{\pgfqpoint{2.626714in}{1.133861in}}%
\pgfpathlineto{\pgfqpoint{2.632040in}{1.127195in}}%
\pgfpathlineto{\pgfqpoint{2.640916in}{1.121945in}}%
\pgfpathlineto{\pgfqpoint{2.648017in}{1.123790in}}%
\pgfpathlineto{\pgfqpoint{2.653343in}{1.127034in}}%
\pgfpathlineto{\pgfqpoint{2.660444in}{1.131688in}}%
\pgfpathlineto{\pgfqpoint{2.669320in}{1.131356in}}%
\pgfpathlineto{\pgfqpoint{2.676421in}{1.128143in}}%
\pgfpathlineto{\pgfqpoint{2.681747in}{1.126227in}}%
\pgfpathlineto{\pgfqpoint{2.690623in}{1.118919in}}%
\pgfpathlineto{\pgfqpoint{2.697724in}{1.110140in}}%
\pgfpathlineto{\pgfqpoint{2.708375in}{1.103406in}}%
\pgfpathlineto{\pgfqpoint{2.719026in}{1.091816in}}%
\pgfpathlineto{\pgfqpoint{2.722577in}{1.087897in}}%
\pgfpathlineto{\pgfqpoint{2.729678in}{1.078338in}}%
\pgfpathlineto{\pgfqpoint{2.736779in}{1.067078in}}%
\pgfpathlineto{\pgfqpoint{2.749205in}{1.050671in}}%
\pgfpathlineto{\pgfqpoint{2.754531in}{1.051259in}}%
\pgfpathlineto{\pgfqpoint{2.766958in}{1.057791in}}%
\pgfpathlineto{\pgfqpoint{2.772283in}{1.066762in}}%
\pgfpathlineto{\pgfqpoint{2.782935in}{1.087907in}}%
\pgfpathlineto{\pgfqpoint{2.784710in}{1.087297in}}%
\pgfpathlineto{\pgfqpoint{2.788261in}{1.087794in}}%
\pgfpathlineto{\pgfqpoint{2.791811in}{1.086950in}}%
\pgfpathlineto{\pgfqpoint{2.795361in}{1.085732in}}%
\pgfpathlineto{\pgfqpoint{2.800687in}{1.081546in}}%
\pgfpathlineto{\pgfqpoint{2.804238in}{1.076895in}}%
\pgfpathlineto{\pgfqpoint{2.816664in}{1.057819in}}%
\pgfpathlineto{\pgfqpoint{2.818440in}{1.056589in}}%
\pgfpathlineto{\pgfqpoint{2.827316in}{1.058839in}}%
\pgfpathlineto{\pgfqpoint{2.829091in}{1.059590in}}%
\pgfpathlineto{\pgfqpoint{2.843293in}{1.056695in}}%
\pgfpathlineto{\pgfqpoint{2.848618in}{1.055848in}}%
\pgfpathlineto{\pgfqpoint{2.852169in}{1.053538in}}%
\pgfpathlineto{\pgfqpoint{2.862820in}{1.045883in}}%
\pgfpathlineto{\pgfqpoint{2.864596in}{1.046616in}}%
\pgfpathlineto{\pgfqpoint{2.873472in}{1.040572in}}%
\pgfpathlineto{\pgfqpoint{2.877022in}{1.038818in}}%
\pgfpathlineto{\pgfqpoint{2.878797in}{1.035610in}}%
\pgfpathlineto{\pgfqpoint{2.884123in}{1.036372in}}%
\pgfpathlineto{\pgfqpoint{2.887674in}{1.040763in}}%
\pgfpathlineto{\pgfqpoint{2.892999in}{1.054776in}}%
\pgfpathlineto{\pgfqpoint{2.907201in}{1.098895in}}%
\pgfpathlineto{\pgfqpoint{2.914302in}{1.109983in}}%
\pgfpathlineto{\pgfqpoint{2.919628in}{1.113922in}}%
\pgfpathlineto{\pgfqpoint{2.926729in}{1.112992in}}%
\pgfpathlineto{\pgfqpoint{2.930279in}{1.111030in}}%
\pgfpathlineto{\pgfqpoint{2.951582in}{1.104738in}}%
\pgfpathlineto{\pgfqpoint{2.956908in}{1.109496in}}%
\pgfpathlineto{\pgfqpoint{2.974660in}{1.138888in}}%
\pgfpathlineto{\pgfqpoint{2.988862in}{1.174602in}}%
\pgfpathlineto{\pgfqpoint{2.997738in}{1.188412in}}%
\pgfpathlineto{\pgfqpoint{3.004839in}{1.191642in}}%
\pgfpathlineto{\pgfqpoint{3.008389in}{1.190825in}}%
\pgfpathlineto{\pgfqpoint{3.010165in}{1.190743in}}%
\pgfpathlineto{\pgfqpoint{3.017266in}{1.186687in}}%
\pgfpathlineto{\pgfqpoint{3.022591in}{1.187052in}}%
\pgfpathlineto{\pgfqpoint{3.026142in}{1.189642in}}%
\pgfpathlineto{\pgfqpoint{3.036793in}{1.203841in}}%
\pgfpathlineto{\pgfqpoint{3.042119in}{1.209617in}}%
\pgfpathlineto{\pgfqpoint{3.050995in}{1.223917in}}%
\pgfpathlineto{\pgfqpoint{3.058096in}{1.236584in}}%
\pgfpathlineto{\pgfqpoint{3.066972in}{1.244739in}}%
\pgfpathlineto{\pgfqpoint{3.068747in}{1.244962in}}%
\pgfpathlineto{\pgfqpoint{3.079399in}{1.254922in}}%
\pgfpathlineto{\pgfqpoint{3.095376in}{1.274263in}}%
\pgfpathlineto{\pgfqpoint{3.109578in}{1.299462in}}%
\pgfpathlineto{\pgfqpoint{3.114903in}{1.302421in}}%
\pgfpathlineto{\pgfqpoint{3.122004in}{1.300353in}}%
\pgfpathlineto{\pgfqpoint{3.127330in}{1.297472in}}%
\pgfpathlineto{\pgfqpoint{3.129105in}{1.296802in}}%
\pgfpathlineto{\pgfqpoint{3.134431in}{1.300544in}}%
\pgfpathlineto{\pgfqpoint{3.141532in}{1.313289in}}%
\pgfpathlineto{\pgfqpoint{3.148633in}{1.325649in}}%
\pgfpathlineto{\pgfqpoint{3.152183in}{1.327805in}}%
\pgfpathlineto{\pgfqpoint{3.155734in}{1.326394in}}%
\pgfpathlineto{\pgfqpoint{3.157509in}{1.324957in}}%
\pgfpathlineto{\pgfqpoint{3.164610in}{1.310548in}}%
\pgfpathlineto{\pgfqpoint{3.171711in}{1.290593in}}%
\pgfpathlineto{\pgfqpoint{3.180587in}{1.267406in}}%
\pgfpathlineto{\pgfqpoint{3.184137in}{1.263372in}}%
\pgfpathlineto{\pgfqpoint{3.185913in}{1.262643in}}%
\pgfpathlineto{\pgfqpoint{3.189463in}{1.264560in}}%
\pgfpathlineto{\pgfqpoint{3.194789in}{1.274645in}}%
\pgfpathlineto{\pgfqpoint{3.207215in}{1.309418in}}%
\pgfpathlineto{\pgfqpoint{3.210766in}{1.313995in}}%
\pgfpathlineto{\pgfqpoint{3.214316in}{1.315663in}}%
\pgfpathlineto{\pgfqpoint{3.219642in}{1.314969in}}%
\pgfpathlineto{\pgfqpoint{3.224968in}{1.318015in}}%
\pgfpathlineto{\pgfqpoint{3.235619in}{1.332489in}}%
\pgfpathlineto{\pgfqpoint{3.242720in}{1.333525in}}%
\pgfpathlineto{\pgfqpoint{3.246271in}{1.331797in}}%
\pgfpathlineto{\pgfqpoint{3.248046in}{1.330934in}}%
\pgfpathlineto{\pgfqpoint{3.253372in}{1.322914in}}%
\pgfpathlineto{\pgfqpoint{3.258697in}{1.306923in}}%
\pgfpathlineto{\pgfqpoint{3.265798in}{1.276242in}}%
\pgfpathlineto{\pgfqpoint{3.269349in}{1.247499in}}%
\pgfpathlineto{\pgfqpoint{3.272899in}{1.191481in}}%
\pgfpathlineto{\pgfqpoint{3.276450in}{1.098373in}}%
\pgfpathlineto{\pgfqpoint{3.283550in}{0.897896in}}%
\pgfpathlineto{\pgfqpoint{3.290651in}{0.817716in}}%
\pgfpathlineto{\pgfqpoint{3.294202in}{0.789018in}}%
\pgfpathlineto{\pgfqpoint{3.295977in}{0.784648in}}%
\pgfpathlineto{\pgfqpoint{3.297752in}{0.785029in}}%
\pgfpathlineto{\pgfqpoint{3.301303in}{0.779535in}}%
\pgfpathlineto{\pgfqpoint{3.303078in}{0.774729in}}%
\pgfpathlineto{\pgfqpoint{3.304853in}{0.756386in}}%
\pgfpathlineto{\pgfqpoint{3.313729in}{0.825643in}}%
\pgfpathlineto{\pgfqpoint{3.317280in}{0.840579in}}%
\pgfpathlineto{\pgfqpoint{3.320830in}{0.861017in}}%
\pgfpathlineto{\pgfqpoint{3.331482in}{0.903685in}}%
\pgfpathlineto{\pgfqpoint{3.336807in}{0.918628in}}%
\pgfpathlineto{\pgfqpoint{3.340358in}{0.923244in}}%
\pgfpathlineto{\pgfqpoint{3.343908in}{0.923817in}}%
\pgfpathlineto{\pgfqpoint{3.352785in}{0.916808in}}%
\pgfpathlineto{\pgfqpoint{3.361661in}{0.910920in}}%
\pgfpathlineto{\pgfqpoint{3.379413in}{0.901103in}}%
\pgfpathlineto{\pgfqpoint{3.386514in}{0.897481in}}%
\pgfpathlineto{\pgfqpoint{3.395390in}{0.891965in}}%
\pgfpathlineto{\pgfqpoint{3.398941in}{0.890371in}}%
\pgfpathlineto{\pgfqpoint{3.404266in}{0.884526in}}%
\pgfpathlineto{\pgfqpoint{3.416693in}{0.865914in}}%
\pgfpathlineto{\pgfqpoint{3.420243in}{0.865222in}}%
\pgfpathlineto{\pgfqpoint{3.432670in}{0.866827in}}%
\pgfpathlineto{\pgfqpoint{3.437996in}{0.865732in}}%
\pgfpathlineto{\pgfqpoint{3.439771in}{0.867666in}}%
\pgfpathlineto{\pgfqpoint{3.446872in}{0.868105in}}%
\pgfpathlineto{\pgfqpoint{3.455748in}{0.871526in}}%
\pgfpathlineto{\pgfqpoint{3.462849in}{0.876848in}}%
\pgfpathlineto{\pgfqpoint{3.475276in}{0.887663in}}%
\pgfpathlineto{\pgfqpoint{3.489477in}{0.899892in}}%
\pgfpathlineto{\pgfqpoint{3.517881in}{0.931720in}}%
\pgfpathlineto{\pgfqpoint{3.526757in}{0.939978in}}%
\pgfpathlineto{\pgfqpoint{3.537409in}{0.945324in}}%
\pgfpathlineto{\pgfqpoint{3.548060in}{0.947986in}}%
\pgfpathlineto{\pgfqpoint{3.556936in}{0.948528in}}%
\pgfpathlineto{\pgfqpoint{3.562262in}{0.946438in}}%
\pgfpathlineto{\pgfqpoint{3.571138in}{0.942351in}}%
\pgfpathlineto{\pgfqpoint{3.574689in}{0.942179in}}%
\pgfpathlineto{\pgfqpoint{3.580014in}{0.944881in}}%
\pgfpathlineto{\pgfqpoint{3.583565in}{0.947877in}}%
\pgfpathlineto{\pgfqpoint{3.590666in}{0.959461in}}%
\pgfpathlineto{\pgfqpoint{3.599542in}{0.974751in}}%
\pgfpathlineto{\pgfqpoint{3.604868in}{0.979999in}}%
\pgfpathlineto{\pgfqpoint{3.611969in}{0.982925in}}%
\pgfpathlineto{\pgfqpoint{3.627946in}{0.985072in}}%
\pgfpathlineto{\pgfqpoint{3.633271in}{0.984520in}}%
\pgfpathlineto{\pgfqpoint{3.638597in}{0.984652in}}%
\pgfpathlineto{\pgfqpoint{3.643923in}{0.986051in}}%
\pgfpathlineto{\pgfqpoint{3.651024in}{0.989110in}}%
\pgfpathlineto{\pgfqpoint{3.656349in}{0.989993in}}%
\pgfpathlineto{\pgfqpoint{3.659900in}{0.986718in}}%
\pgfpathlineto{\pgfqpoint{3.670551in}{0.982604in}}%
\pgfpathlineto{\pgfqpoint{3.675877in}{0.983705in}}%
\pgfpathlineto{\pgfqpoint{3.679427in}{0.985366in}}%
\pgfpathlineto{\pgfqpoint{3.681203in}{0.987720in}}%
\pgfpathlineto{\pgfqpoint{3.690079in}{0.989732in}}%
\pgfpathlineto{\pgfqpoint{3.695404in}{0.991169in}}%
\pgfpathlineto{\pgfqpoint{3.697180in}{0.989513in}}%
\pgfpathlineto{\pgfqpoint{3.706056in}{0.990222in}}%
\pgfpathlineto{\pgfqpoint{3.707831in}{0.992379in}}%
\pgfpathlineto{\pgfqpoint{3.709606in}{0.992531in}}%
\pgfpathlineto{\pgfqpoint{3.711382in}{0.990694in}}%
\pgfpathlineto{\pgfqpoint{3.716707in}{0.991218in}}%
\pgfpathlineto{\pgfqpoint{3.730909in}{0.990411in}}%
\pgfpathlineto{\pgfqpoint{3.732684in}{0.990873in}}%
\pgfpathlineto{\pgfqpoint{3.734460in}{0.993045in}}%
\pgfpathlineto{\pgfqpoint{3.739785in}{0.994285in}}%
\pgfpathlineto{\pgfqpoint{3.745111in}{0.996855in}}%
\pgfpathlineto{\pgfqpoint{3.755762in}{1.007183in}}%
\pgfpathlineto{\pgfqpoint{3.759313in}{1.008572in}}%
\pgfpathlineto{\pgfqpoint{3.762863in}{1.007221in}}%
\pgfpathlineto{\pgfqpoint{3.775290in}{1.000010in}}%
\pgfpathlineto{\pgfqpoint{3.780616in}{0.999471in}}%
\pgfpathlineto{\pgfqpoint{3.784166in}{1.000626in}}%
\pgfpathlineto{\pgfqpoint{3.785941in}{1.003216in}}%
\pgfpathlineto{\pgfqpoint{3.794817in}{1.006688in}}%
\pgfpathlineto{\pgfqpoint{3.796593in}{1.004885in}}%
\pgfpathlineto{\pgfqpoint{3.803694in}{1.006378in}}%
\pgfpathlineto{\pgfqpoint{3.807244in}{1.009719in}}%
\pgfpathlineto{\pgfqpoint{3.821446in}{1.017720in}}%
\pgfpathlineto{\pgfqpoint{3.824996in}{1.022000in}}%
\pgfpathlineto{\pgfqpoint{3.835648in}{1.026263in}}%
\pgfpathlineto{\pgfqpoint{3.837423in}{1.025242in}}%
\pgfpathlineto{\pgfqpoint{3.844524in}{1.030409in}}%
\pgfpathlineto{\pgfqpoint{3.849850in}{1.034335in}}%
\pgfpathlineto{\pgfqpoint{3.851625in}{1.037293in}}%
\pgfpathlineto{\pgfqpoint{3.860501in}{1.038690in}}%
\pgfpathlineto{\pgfqpoint{3.874703in}{1.037872in}}%
\pgfpathlineto{\pgfqpoint{3.883579in}{1.045463in}}%
\pgfpathlineto{\pgfqpoint{3.888905in}{1.050804in}}%
\pgfpathlineto{\pgfqpoint{3.890680in}{1.050635in}}%
\pgfpathlineto{\pgfqpoint{3.906657in}{1.067628in}}%
\pgfpathlineto{\pgfqpoint{3.926185in}{1.079392in}}%
\pgfpathlineto{\pgfqpoint{3.936836in}{1.092159in}}%
\pgfpathlineto{\pgfqpoint{3.940387in}{1.093509in}}%
\pgfpathlineto{\pgfqpoint{3.945712in}{1.092912in}}%
\pgfpathlineto{\pgfqpoint{3.954588in}{1.089563in}}%
\pgfpathlineto{\pgfqpoint{3.958139in}{1.090005in}}%
\pgfpathlineto{\pgfqpoint{3.963465in}{1.094949in}}%
\pgfpathlineto{\pgfqpoint{3.974116in}{1.103288in}}%
\pgfpathlineto{\pgfqpoint{3.982992in}{1.110403in}}%
\pgfpathlineto{\pgfqpoint{3.990093in}{1.115137in}}%
\pgfpathlineto{\pgfqpoint{3.993644in}{1.114849in}}%
\pgfpathlineto{\pgfqpoint{3.998969in}{1.111359in}}%
\pgfpathlineto{\pgfqpoint{4.004295in}{1.108627in}}%
\pgfpathlineto{\pgfqpoint{4.007845in}{1.108774in}}%
\pgfpathlineto{\pgfqpoint{4.011396in}{1.110822in}}%
\pgfpathlineto{\pgfqpoint{4.020272in}{1.119147in}}%
\pgfpathlineto{\pgfqpoint{4.023822in}{1.121402in}}%
\pgfpathlineto{\pgfqpoint{4.029148in}{1.121722in}}%
\pgfpathlineto{\pgfqpoint{4.032699in}{1.120280in}}%
\pgfpathlineto{\pgfqpoint{4.043350in}{1.112234in}}%
\pgfpathlineto{\pgfqpoint{4.046901in}{1.112992in}}%
\pgfpathlineto{\pgfqpoint{4.052226in}{1.117096in}}%
\pgfpathlineto{\pgfqpoint{4.057552in}{1.121042in}}%
\pgfpathlineto{\pgfqpoint{4.061102in}{1.120398in}}%
\pgfpathlineto{\pgfqpoint{4.066428in}{1.116115in}}%
\pgfpathlineto{\pgfqpoint{4.073529in}{1.108732in}}%
\pgfpathlineto{\pgfqpoint{4.077079in}{1.109131in}}%
\pgfpathlineto{\pgfqpoint{4.080630in}{1.113412in}}%
\pgfpathlineto{\pgfqpoint{4.089506in}{1.125654in}}%
\pgfpathlineto{\pgfqpoint{4.093057in}{1.125885in}}%
\pgfpathlineto{\pgfqpoint{4.103708in}{1.119904in}}%
\pgfpathlineto{\pgfqpoint{4.109034in}{1.117816in}}%
\pgfpathlineto{\pgfqpoint{4.112584in}{1.113321in}}%
\pgfpathlineto{\pgfqpoint{4.116135in}{1.104945in}}%
\pgfpathlineto{\pgfqpoint{4.123236in}{1.086583in}}%
\pgfpathlineto{\pgfqpoint{4.126786in}{1.085469in}}%
\pgfpathlineto{\pgfqpoint{4.128561in}{1.087468in}}%
\pgfpathlineto{\pgfqpoint{4.140988in}{1.088722in}}%
\pgfpathlineto{\pgfqpoint{4.144538in}{1.091454in}}%
\pgfpathlineto{\pgfqpoint{4.155190in}{1.103612in}}%
\pgfpathlineto{\pgfqpoint{4.160515in}{1.114060in}}%
\pgfpathlineto{\pgfqpoint{4.167616in}{1.129021in}}%
\pgfpathlineto{\pgfqpoint{4.171167in}{1.131173in}}%
\pgfpathlineto{\pgfqpoint{4.174717in}{1.128651in}}%
\pgfpathlineto{\pgfqpoint{4.178268in}{1.122880in}}%
\pgfpathlineto{\pgfqpoint{4.187144in}{1.101356in}}%
\pgfpathlineto{\pgfqpoint{4.194245in}{1.082106in}}%
\pgfpathlineto{\pgfqpoint{4.197795in}{1.076451in}}%
\pgfpathlineto{\pgfqpoint{4.201346in}{1.074120in}}%
\pgfpathlineto{\pgfqpoint{4.204896in}{1.073911in}}%
\pgfpathlineto{\pgfqpoint{4.206671in}{1.076021in}}%
\pgfpathlineto{\pgfqpoint{4.213772in}{1.075591in}}%
\pgfpathlineto{\pgfqpoint{4.220873in}{1.072228in}}%
\pgfpathlineto{\pgfqpoint{4.231525in}{1.061527in}}%
\pgfpathlineto{\pgfqpoint{4.235075in}{1.061897in}}%
\pgfpathlineto{\pgfqpoint{4.261704in}{1.073705in}}%
\pgfpathlineto{\pgfqpoint{4.267029in}{1.079042in}}%
\pgfpathlineto{\pgfqpoint{4.270580in}{1.082543in}}%
\pgfpathlineto{\pgfqpoint{4.274130in}{1.081413in}}%
\pgfpathlineto{\pgfqpoint{4.277681in}{1.076252in}}%
\pgfpathlineto{\pgfqpoint{4.284782in}{1.055729in}}%
\pgfpathlineto{\pgfqpoint{4.288332in}{1.049009in}}%
\pgfpathlineto{\pgfqpoint{4.291883in}{1.045844in}}%
\pgfpathlineto{\pgfqpoint{4.293658in}{1.045136in}}%
\pgfpathlineto{\pgfqpoint{4.295433in}{1.046624in}}%
\pgfpathlineto{\pgfqpoint{4.304309in}{1.043352in}}%
\pgfpathlineto{\pgfqpoint{4.307860in}{1.044401in}}%
\pgfpathlineto{\pgfqpoint{4.311410in}{1.048486in}}%
\pgfpathlineto{\pgfqpoint{4.327387in}{1.078051in}}%
\pgfpathlineto{\pgfqpoint{4.330938in}{1.079576in}}%
\pgfpathlineto{\pgfqpoint{4.336263in}{1.077409in}}%
\pgfpathlineto{\pgfqpoint{4.348690in}{1.072017in}}%
\pgfpathlineto{\pgfqpoint{4.350465in}{1.072245in}}%
\pgfpathlineto{\pgfqpoint{4.352241in}{1.074451in}}%
\pgfpathlineto{\pgfqpoint{4.355791in}{1.075730in}}%
\pgfpathlineto{\pgfqpoint{4.373543in}{1.089796in}}%
\pgfpathlineto{\pgfqpoint{4.375319in}{1.088798in}}%
\pgfpathlineto{\pgfqpoint{4.389520in}{1.097012in}}%
\pgfpathlineto{\pgfqpoint{4.391296in}{1.097487in}}%
\pgfpathlineto{\pgfqpoint{4.393071in}{1.099751in}}%
\pgfpathlineto{\pgfqpoint{4.398397in}{1.100341in}}%
\pgfpathlineto{\pgfqpoint{4.401947in}{1.097651in}}%
\pgfpathlineto{\pgfqpoint{4.412598in}{1.094066in}}%
\pgfpathlineto{\pgfqpoint{4.416149in}{1.098018in}}%
\pgfpathlineto{\pgfqpoint{4.426800in}{1.105716in}}%
\pgfpathlineto{\pgfqpoint{4.432126in}{1.106734in}}%
\pgfpathlineto{\pgfqpoint{4.435676in}{1.106973in}}%
\pgfpathlineto{\pgfqpoint{4.437452in}{1.108778in}}%
\pgfpathlineto{\pgfqpoint{4.442777in}{1.106920in}}%
\pgfpathlineto{\pgfqpoint{4.448103in}{1.102129in}}%
\pgfpathlineto{\pgfqpoint{4.453429in}{1.094886in}}%
\pgfpathlineto{\pgfqpoint{4.462305in}{1.081848in}}%
\pgfpathlineto{\pgfqpoint{4.467631in}{1.077556in}}%
\pgfpathlineto{\pgfqpoint{4.469406in}{1.076883in}}%
\pgfpathlineto{\pgfqpoint{4.471181in}{1.074534in}}%
\pgfpathlineto{\pgfqpoint{4.474732in}{1.074959in}}%
\pgfpathlineto{\pgfqpoint{4.478282in}{1.077768in}}%
\pgfpathlineto{\pgfqpoint{4.481832in}{1.083314in}}%
\pgfpathlineto{\pgfqpoint{4.492484in}{1.096752in}}%
\pgfpathlineto{\pgfqpoint{4.497810in}{1.101043in}}%
\pgfpathlineto{\pgfqpoint{4.501360in}{1.102719in}}%
\pgfpathlineto{\pgfqpoint{4.504911in}{1.102106in}}%
\pgfpathlineto{\pgfqpoint{4.513787in}{1.098041in}}%
\pgfpathlineto{\pgfqpoint{4.524438in}{1.096341in}}%
\pgfpathlineto{\pgfqpoint{4.533314in}{1.091605in}}%
\pgfpathlineto{\pgfqpoint{4.540415in}{1.094204in}}%
\pgfpathlineto{\pgfqpoint{4.549291in}{1.097368in}}%
\pgfpathlineto{\pgfqpoint{4.552842in}{1.098362in}}%
\pgfpathlineto{\pgfqpoint{4.554617in}{1.100321in}}%
\pgfpathlineto{\pgfqpoint{4.558168in}{1.099237in}}%
\pgfpathlineto{\pgfqpoint{4.561718in}{1.096142in}}%
\pgfpathlineto{\pgfqpoint{4.565268in}{1.093651in}}%
\pgfpathlineto{\pgfqpoint{4.570594in}{1.086716in}}%
\pgfpathlineto{\pgfqpoint{4.577695in}{1.076935in}}%
\pgfpathlineto{\pgfqpoint{4.581246in}{1.076313in}}%
\pgfpathlineto{\pgfqpoint{4.584796in}{1.079301in}}%
\pgfpathlineto{\pgfqpoint{4.590122in}{1.085569in}}%
\pgfpathlineto{\pgfqpoint{4.591897in}{1.087020in}}%
\pgfpathlineto{\pgfqpoint{4.593672in}{1.090122in}}%
\pgfpathlineto{\pgfqpoint{4.597223in}{1.091317in}}%
\pgfpathlineto{\pgfqpoint{4.598998in}{1.089333in}}%
\pgfpathlineto{\pgfqpoint{4.602548in}{1.089824in}}%
\pgfpathlineto{\pgfqpoint{4.607874in}{1.093293in}}%
\pgfpathlineto{\pgfqpoint{4.611424in}{1.097996in}}%
\pgfpathlineto{\pgfqpoint{4.618525in}{1.109760in}}%
\pgfpathlineto{\pgfqpoint{4.622076in}{1.111832in}}%
\pgfpathlineto{\pgfqpoint{4.623851in}{1.110997in}}%
\pgfpathlineto{\pgfqpoint{4.627402in}{1.105457in}}%
\pgfpathlineto{\pgfqpoint{4.636278in}{1.087288in}}%
\pgfpathlineto{\pgfqpoint{4.639828in}{1.084711in}}%
\pgfpathlineto{\pgfqpoint{4.643379in}{1.084250in}}%
\pgfpathlineto{\pgfqpoint{4.648704in}{1.086847in}}%
\pgfpathlineto{\pgfqpoint{4.652255in}{1.090821in}}%
\pgfpathlineto{\pgfqpoint{4.655805in}{1.098071in}}%
\pgfpathlineto{\pgfqpoint{4.662906in}{1.107397in}}%
\pgfpathlineto{\pgfqpoint{4.666457in}{1.109081in}}%
\pgfpathlineto{\pgfqpoint{4.671782in}{1.107076in}}%
\pgfpathlineto{\pgfqpoint{4.687759in}{1.096300in}}%
\pgfpathlineto{\pgfqpoint{4.693085in}{1.095032in}}%
\pgfpathlineto{\pgfqpoint{4.696636in}{1.096771in}}%
\pgfpathlineto{\pgfqpoint{4.698411in}{1.098342in}}%
\pgfpathlineto{\pgfqpoint{4.701961in}{1.103469in}}%
\pgfpathlineto{\pgfqpoint{4.705512in}{1.107555in}}%
\pgfpathlineto{\pgfqpoint{4.712613in}{1.119864in}}%
\pgfpathlineto{\pgfqpoint{4.714388in}{1.123359in}}%
\pgfpathlineto{\pgfqpoint{4.716163in}{1.124150in}}%
\pgfpathlineto{\pgfqpoint{4.719714in}{1.127717in}}%
\pgfpathlineto{\pgfqpoint{4.723264in}{1.128238in}}%
\pgfpathlineto{\pgfqpoint{4.728590in}{1.124435in}}%
\pgfpathlineto{\pgfqpoint{4.749893in}{1.109181in}}%
\pgfpathlineto{\pgfqpoint{4.755218in}{1.107363in}}%
\pgfpathlineto{\pgfqpoint{4.762319in}{1.107867in}}%
\pgfpathlineto{\pgfqpoint{4.765870in}{1.110470in}}%
\pgfpathlineto{\pgfqpoint{4.780072in}{1.123789in}}%
\pgfpathlineto{\pgfqpoint{4.783622in}{1.123966in}}%
\pgfpathlineto{\pgfqpoint{4.788948in}{1.120118in}}%
\pgfpathlineto{\pgfqpoint{4.797824in}{1.114181in}}%
\pgfpathlineto{\pgfqpoint{4.803150in}{1.111598in}}%
\pgfpathlineto{\pgfqpoint{4.810251in}{1.106759in}}%
\pgfpathlineto{\pgfqpoint{4.815576in}{1.105619in}}%
\pgfpathlineto{\pgfqpoint{4.819127in}{1.105166in}}%
\pgfpathlineto{\pgfqpoint{4.824452in}{1.101425in}}%
\pgfpathlineto{\pgfqpoint{4.838654in}{1.087488in}}%
\pgfpathlineto{\pgfqpoint{4.843980in}{1.079229in}}%
\pgfpathlineto{\pgfqpoint{4.847530in}{1.076662in}}%
\pgfpathlineto{\pgfqpoint{4.851081in}{1.078162in}}%
\pgfpathlineto{\pgfqpoint{4.861732in}{1.086696in}}%
\pgfpathlineto{\pgfqpoint{4.867058in}{1.086788in}}%
\pgfpathlineto{\pgfqpoint{4.877709in}{1.085503in}}%
\pgfpathlineto{\pgfqpoint{4.893686in}{1.091791in}}%
\pgfpathlineto{\pgfqpoint{4.897237in}{1.091466in}}%
\pgfpathlineto{\pgfqpoint{4.902563in}{1.087849in}}%
\pgfpathlineto{\pgfqpoint{4.913214in}{1.077939in}}%
\pgfpathlineto{\pgfqpoint{4.918540in}{1.076471in}}%
\pgfpathlineto{\pgfqpoint{4.922090in}{1.075803in}}%
\pgfpathlineto{\pgfqpoint{4.925641in}{1.073291in}}%
\pgfpathlineto{\pgfqpoint{4.936292in}{1.064238in}}%
\pgfpathlineto{\pgfqpoint{4.939843in}{1.064669in}}%
\pgfpathlineto{\pgfqpoint{4.950494in}{1.069680in}}%
\pgfpathlineto{\pgfqpoint{4.954044in}{1.067990in}}%
\pgfpathlineto{\pgfqpoint{4.968246in}{1.054058in}}%
\pgfpathlineto{\pgfqpoint{4.975347in}{1.050259in}}%
\pgfpathlineto{\pgfqpoint{4.982448in}{1.043127in}}%
\pgfpathlineto{\pgfqpoint{4.987774in}{1.038481in}}%
\pgfpathlineto{\pgfqpoint{4.991324in}{1.038847in}}%
\pgfpathlineto{\pgfqpoint{5.000200in}{1.044703in}}%
\pgfpathlineto{\pgfqpoint{5.003751in}{1.042173in}}%
\pgfpathlineto{\pgfqpoint{5.010852in}{1.032119in}}%
\pgfpathlineto{\pgfqpoint{5.019728in}{1.018296in}}%
\pgfpathlineto{\pgfqpoint{5.026829in}{1.008386in}}%
\pgfpathlineto{\pgfqpoint{5.030379in}{1.007656in}}%
\pgfpathlineto{\pgfqpoint{5.033930in}{1.009875in}}%
\pgfpathlineto{\pgfqpoint{5.046356in}{1.023739in}}%
\pgfpathlineto{\pgfqpoint{5.049907in}{1.023467in}}%
\pgfpathlineto{\pgfqpoint{5.055233in}{1.019273in}}%
\pgfpathlineto{\pgfqpoint{5.071210in}{1.004386in}}%
\pgfpathlineto{\pgfqpoint{5.076535in}{1.002488in}}%
\pgfpathlineto{\pgfqpoint{5.096063in}{0.989731in}}%
\pgfpathlineto{\pgfqpoint{5.101389in}{0.982646in}}%
\pgfpathlineto{\pgfqpoint{5.106714in}{0.976673in}}%
\pgfpathlineto{\pgfqpoint{5.110265in}{0.976267in}}%
\pgfpathlineto{\pgfqpoint{5.115591in}{0.977064in}}%
\pgfpathlineto{\pgfqpoint{5.119141in}{0.974397in}}%
\pgfpathlineto{\pgfqpoint{5.124467in}{0.965223in}}%
\pgfpathlineto{\pgfqpoint{5.136893in}{0.941380in}}%
\pgfpathlineto{\pgfqpoint{5.145770in}{0.925657in}}%
\pgfpathlineto{\pgfqpoint{5.156421in}{0.902526in}}%
\pgfpathlineto{\pgfqpoint{5.172398in}{0.879717in}}%
\pgfpathlineto{\pgfqpoint{5.184825in}{0.859368in}}%
\pgfpathlineto{\pgfqpoint{5.191926in}{0.846368in}}%
\pgfpathlineto{\pgfqpoint{5.200802in}{0.828987in}}%
\pgfpathlineto{\pgfqpoint{5.215004in}{0.800222in}}%
\pgfpathlineto{\pgfqpoint{5.223880in}{0.767980in}}%
\pgfpathlineto{\pgfqpoint{5.232756in}{0.737371in}}%
\pgfpathlineto{\pgfqpoint{5.239857in}{0.722009in}}%
\pgfpathlineto{\pgfqpoint{5.245183in}{0.715189in}}%
\pgfpathlineto{\pgfqpoint{5.252283in}{0.709335in}}%
\pgfpathlineto{\pgfqpoint{5.262935in}{0.703932in}}%
\pgfpathlineto{\pgfqpoint{5.275361in}{0.700049in}}%
\pgfpathlineto{\pgfqpoint{5.293114in}{0.697118in}}%
\pgfpathlineto{\pgfqpoint{5.309091in}{0.696432in}}%
\pgfpathlineto{\pgfqpoint{5.483064in}{0.696353in}}%
\pgfpathlineto{\pgfqpoint{5.484839in}{0.706597in}}%
\pgfpathlineto{\pgfqpoint{5.488389in}{0.705818in}}%
\pgfpathlineto{\pgfqpoint{5.490165in}{0.696196in}}%
\pgfpathlineto{\pgfqpoint{5.534545in}{0.696000in}}%
\pgfpathlineto{\pgfqpoint{5.534545in}{0.696000in}}%
\pgfusepath{stroke}%
\end{pgfscope}%
\begin{pgfscope}%
\pgfsetrectcap%
\pgfsetmiterjoin%
\pgfsetlinewidth{0.803000pt}%
\definecolor{currentstroke}{rgb}{0.000000,0.000000,0.000000}%
\pgfsetstrokecolor{currentstroke}%
\pgfsetdash{}{0pt}%
\pgfpathmoveto{\pgfqpoint{0.800000in}{0.528000in}}%
\pgfpathlineto{\pgfqpoint{0.800000in}{4.224000in}}%
\pgfusepath{stroke}%
\end{pgfscope}%
\begin{pgfscope}%
\pgfsetrectcap%
\pgfsetmiterjoin%
\pgfsetlinewidth{0.803000pt}%
\definecolor{currentstroke}{rgb}{0.000000,0.000000,0.000000}%
\pgfsetstrokecolor{currentstroke}%
\pgfsetdash{}{0pt}%
\pgfpathmoveto{\pgfqpoint{5.760000in}{0.528000in}}%
\pgfpathlineto{\pgfqpoint{5.760000in}{4.224000in}}%
\pgfusepath{stroke}%
\end{pgfscope}%
\begin{pgfscope}%
\pgfsetrectcap%
\pgfsetmiterjoin%
\pgfsetlinewidth{0.803000pt}%
\definecolor{currentstroke}{rgb}{0.000000,0.000000,0.000000}%
\pgfsetstrokecolor{currentstroke}%
\pgfsetdash{}{0pt}%
\pgfpathmoveto{\pgfqpoint{0.800000in}{0.528000in}}%
\pgfpathlineto{\pgfqpoint{5.760000in}{0.528000in}}%
\pgfusepath{stroke}%
\end{pgfscope}%
\begin{pgfscope}%
\pgfsetrectcap%
\pgfsetmiterjoin%
\pgfsetlinewidth{0.803000pt}%
\definecolor{currentstroke}{rgb}{0.000000,0.000000,0.000000}%
\pgfsetstrokecolor{currentstroke}%
\pgfsetdash{}{0pt}%
\pgfpathmoveto{\pgfqpoint{0.800000in}{4.224000in}}%
\pgfpathlineto{\pgfqpoint{5.760000in}{4.224000in}}%
\pgfusepath{stroke}%
\end{pgfscope}%
\begin{pgfscope}%
\pgfsetbuttcap%
\pgfsetmiterjoin%
\definecolor{currentfill}{rgb}{1.000000,1.000000,1.000000}%
\pgfsetfillcolor{currentfill}%
\pgfsetfillopacity{0.800000}%
\pgfsetlinewidth{1.003750pt}%
\definecolor{currentstroke}{rgb}{0.800000,0.800000,0.800000}%
\pgfsetstrokecolor{currentstroke}%
\pgfsetstrokeopacity{0.800000}%
\pgfsetdash{}{0pt}%
\pgfpathmoveto{\pgfqpoint{0.897222in}{3.093603in}}%
\pgfpathlineto{\pgfqpoint{1.430032in}{3.093603in}}%
\pgfpathquadraticcurveto{\pgfqpoint{1.457810in}{3.093603in}}{\pgfqpoint{1.457810in}{3.121381in}}%
\pgfpathlineto{\pgfqpoint{1.457810in}{4.126778in}}%
\pgfpathquadraticcurveto{\pgfqpoint{1.457810in}{4.154556in}}{\pgfqpoint{1.430032in}{4.154556in}}%
\pgfpathlineto{\pgfqpoint{0.897222in}{4.154556in}}%
\pgfpathquadraticcurveto{\pgfqpoint{0.869444in}{4.154556in}}{\pgfqpoint{0.869444in}{4.126778in}}%
\pgfpathlineto{\pgfqpoint{0.869444in}{3.121381in}}%
\pgfpathquadraticcurveto{\pgfqpoint{0.869444in}{3.093603in}}{\pgfqpoint{0.897222in}{3.093603in}}%
\pgfpathclose%
\pgfusepath{stroke,fill}%
\end{pgfscope}%
\begin{pgfscope}%
\pgfsetrectcap%
\pgfsetroundjoin%
\pgfsetlinewidth{1.505625pt}%
\definecolor{currentstroke}{rgb}{0.121569,0.466667,0.705882}%
\pgfsetstrokecolor{currentstroke}%
\pgfsetdash{}{0pt}%
\pgfpathmoveto{\pgfqpoint{0.925000in}{4.042088in}}%
\pgfpathlineto{\pgfqpoint{1.202778in}{4.042088in}}%
\pgfusepath{stroke}%
\end{pgfscope}%
\begin{pgfscope}%
\pgftext[x=1.313889in,y=3.993477in,left,base]{\sffamily\fontsize{10.000000}{12.000000}\selectfont 1}%
\end{pgfscope}%
\begin{pgfscope}%
\pgfsetrectcap%
\pgfsetroundjoin%
\pgfsetlinewidth{1.505625pt}%
\definecolor{currentstroke}{rgb}{1.000000,0.498039,0.054902}%
\pgfsetstrokecolor{currentstroke}%
\pgfsetdash{}{0pt}%
\pgfpathmoveto{\pgfqpoint{0.925000in}{3.838231in}}%
\pgfpathlineto{\pgfqpoint{1.202778in}{3.838231in}}%
\pgfusepath{stroke}%
\end{pgfscope}%
\begin{pgfscope}%
\pgftext[x=1.313889in,y=3.789620in,left,base]{\sffamily\fontsize{10.000000}{12.000000}\selectfont 2}%
\end{pgfscope}%
\begin{pgfscope}%
\pgfsetrectcap%
\pgfsetroundjoin%
\pgfsetlinewidth{1.505625pt}%
\definecolor{currentstroke}{rgb}{0.172549,0.627451,0.172549}%
\pgfsetstrokecolor{currentstroke}%
\pgfsetdash{}{0pt}%
\pgfpathmoveto{\pgfqpoint{0.925000in}{3.634374in}}%
\pgfpathlineto{\pgfqpoint{1.202778in}{3.634374in}}%
\pgfusepath{stroke}%
\end{pgfscope}%
\begin{pgfscope}%
\pgftext[x=1.313889in,y=3.585763in,left,base]{\sffamily\fontsize{10.000000}{12.000000}\selectfont 3}%
\end{pgfscope}%
\begin{pgfscope}%
\pgfsetrectcap%
\pgfsetroundjoin%
\pgfsetlinewidth{1.505625pt}%
\definecolor{currentstroke}{rgb}{0.839216,0.152941,0.156863}%
\pgfsetstrokecolor{currentstroke}%
\pgfsetdash{}{0pt}%
\pgfpathmoveto{\pgfqpoint{0.925000in}{3.430516in}}%
\pgfpathlineto{\pgfqpoint{1.202778in}{3.430516in}}%
\pgfusepath{stroke}%
\end{pgfscope}%
\begin{pgfscope}%
\pgftext[x=1.313889in,y=3.381905in,left,base]{\sffamily\fontsize{10.000000}{12.000000}\selectfont 4}%
\end{pgfscope}%
\begin{pgfscope}%
\pgfsetrectcap%
\pgfsetroundjoin%
\pgfsetlinewidth{1.505625pt}%
\definecolor{currentstroke}{rgb}{0.580392,0.403922,0.741176}%
\pgfsetstrokecolor{currentstroke}%
\pgfsetdash{}{0pt}%
\pgfpathmoveto{\pgfqpoint{0.925000in}{3.226659in}}%
\pgfpathlineto{\pgfqpoint{1.202778in}{3.226659in}}%
\pgfusepath{stroke}%
\end{pgfscope}%
\begin{pgfscope}%
\pgftext[x=1.313889in,y=3.178048in,left,base]{\sffamily\fontsize{10.000000}{12.000000}\selectfont 5}%
\end{pgfscope}%
\end{pgfpicture}%
\makeatother%
\endgroup%
}
        \label{fig:sub1}
    \end{subfigure}%
    \begin{subfigure}{.5\textwidth}
        \centering
        \caption{BuK\_23}
        \scalebox{0.5}{%% Creator: Matplotlib, PGF backend
%%
%% To include the figure in your LaTeX document, write
%%   \input{<filename>.pgf}
%%
%% Make sure the required packages are loaded in your preamble
%%   \usepackage{pgf}
%%
%% Figures using additional raster images can only be included by \input if
%% they are in the same directory as the main LaTeX file. For loading figures
%% from other directories you can use the `import` package
%%   \usepackage{import}
%% and then include the figures with
%%   \import{<path to file>}{<filename>.pgf}
%%
%% Matplotlib used the following preamble
%%   \usepackage{fontspec}
%%   \setmainfont{DejaVu Serif}
%%   \setsansfont{DejaVu Sans}
%%   \setmonofont{DejaVu Sans Mono}
%%
\begingroup%
\makeatletter%
\begin{pgfpicture}%
\pgfpathrectangle{\pgfpointorigin}{\pgfqpoint{6.400000in}{4.800000in}}%
\pgfusepath{use as bounding box, clip}%
\begin{pgfscope}%
\pgfsetbuttcap%
\pgfsetmiterjoin%
\definecolor{currentfill}{rgb}{1.000000,1.000000,1.000000}%
\pgfsetfillcolor{currentfill}%
\pgfsetlinewidth{0.000000pt}%
\definecolor{currentstroke}{rgb}{1.000000,1.000000,1.000000}%
\pgfsetstrokecolor{currentstroke}%
\pgfsetdash{}{0pt}%
\pgfpathmoveto{\pgfqpoint{0.000000in}{0.000000in}}%
\pgfpathlineto{\pgfqpoint{6.400000in}{0.000000in}}%
\pgfpathlineto{\pgfqpoint{6.400000in}{4.800000in}}%
\pgfpathlineto{\pgfqpoint{0.000000in}{4.800000in}}%
\pgfpathclose%
\pgfusepath{fill}%
\end{pgfscope}%
\begin{pgfscope}%
\pgfsetbuttcap%
\pgfsetmiterjoin%
\definecolor{currentfill}{rgb}{1.000000,1.000000,1.000000}%
\pgfsetfillcolor{currentfill}%
\pgfsetlinewidth{0.000000pt}%
\definecolor{currentstroke}{rgb}{0.000000,0.000000,0.000000}%
\pgfsetstrokecolor{currentstroke}%
\pgfsetstrokeopacity{0.000000}%
\pgfsetdash{}{0pt}%
\pgfpathmoveto{\pgfqpoint{0.800000in}{0.528000in}}%
\pgfpathlineto{\pgfqpoint{5.760000in}{0.528000in}}%
\pgfpathlineto{\pgfqpoint{5.760000in}{4.224000in}}%
\pgfpathlineto{\pgfqpoint{0.800000in}{4.224000in}}%
\pgfpathclose%
\pgfusepath{fill}%
\end{pgfscope}%
\begin{pgfscope}%
\pgfsetbuttcap%
\pgfsetroundjoin%
\definecolor{currentfill}{rgb}{0.000000,0.000000,0.000000}%
\pgfsetfillcolor{currentfill}%
\pgfsetlinewidth{0.803000pt}%
\definecolor{currentstroke}{rgb}{0.000000,0.000000,0.000000}%
\pgfsetstrokecolor{currentstroke}%
\pgfsetdash{}{0pt}%
\pgfsys@defobject{currentmarker}{\pgfqpoint{0.000000in}{-0.048611in}}{\pgfqpoint{0.000000in}{0.000000in}}{%
\pgfpathmoveto{\pgfqpoint{0.000000in}{0.000000in}}%
\pgfpathlineto{\pgfqpoint{0.000000in}{-0.048611in}}%
\pgfusepath{stroke,fill}%
}%
\begin{pgfscope}%
\pgfsys@transformshift{1.025455in}{0.528000in}%
\pgfsys@useobject{currentmarker}{}%
\end{pgfscope}%
\end{pgfscope}%
\begin{pgfscope}%
\pgftext[x=1.025455in,y=0.430778in,,top]{\sffamily\fontsize{10.000000}{12.000000}\selectfont 0}%
\end{pgfscope}%
\begin{pgfscope}%
\pgfsetbuttcap%
\pgfsetroundjoin%
\definecolor{currentfill}{rgb}{0.000000,0.000000,0.000000}%
\pgfsetfillcolor{currentfill}%
\pgfsetlinewidth{0.803000pt}%
\definecolor{currentstroke}{rgb}{0.000000,0.000000,0.000000}%
\pgfsetstrokecolor{currentstroke}%
\pgfsetdash{}{0pt}%
\pgfsys@defobject{currentmarker}{\pgfqpoint{0.000000in}{-0.048611in}}{\pgfqpoint{0.000000in}{0.000000in}}{%
\pgfpathmoveto{\pgfqpoint{0.000000in}{0.000000in}}%
\pgfpathlineto{\pgfqpoint{0.000000in}{-0.048611in}}%
\pgfusepath{stroke,fill}%
}%
\begin{pgfscope}%
\pgfsys@transformshift{1.731324in}{0.528000in}%
\pgfsys@useobject{currentmarker}{}%
\end{pgfscope}%
\end{pgfscope}%
\begin{pgfscope}%
\pgftext[x=1.731324in,y=0.430778in,,top]{\sffamily\fontsize{10.000000}{12.000000}\selectfont 500}%
\end{pgfscope}%
\begin{pgfscope}%
\pgfsetbuttcap%
\pgfsetroundjoin%
\definecolor{currentfill}{rgb}{0.000000,0.000000,0.000000}%
\pgfsetfillcolor{currentfill}%
\pgfsetlinewidth{0.803000pt}%
\definecolor{currentstroke}{rgb}{0.000000,0.000000,0.000000}%
\pgfsetstrokecolor{currentstroke}%
\pgfsetdash{}{0pt}%
\pgfsys@defobject{currentmarker}{\pgfqpoint{0.000000in}{-0.048611in}}{\pgfqpoint{0.000000in}{0.000000in}}{%
\pgfpathmoveto{\pgfqpoint{0.000000in}{0.000000in}}%
\pgfpathlineto{\pgfqpoint{0.000000in}{-0.048611in}}%
\pgfusepath{stroke,fill}%
}%
\begin{pgfscope}%
\pgfsys@transformshift{2.437192in}{0.528000in}%
\pgfsys@useobject{currentmarker}{}%
\end{pgfscope}%
\end{pgfscope}%
\begin{pgfscope}%
\pgftext[x=2.437192in,y=0.430778in,,top]{\sffamily\fontsize{10.000000}{12.000000}\selectfont 1000}%
\end{pgfscope}%
\begin{pgfscope}%
\pgfsetbuttcap%
\pgfsetroundjoin%
\definecolor{currentfill}{rgb}{0.000000,0.000000,0.000000}%
\pgfsetfillcolor{currentfill}%
\pgfsetlinewidth{0.803000pt}%
\definecolor{currentstroke}{rgb}{0.000000,0.000000,0.000000}%
\pgfsetstrokecolor{currentstroke}%
\pgfsetdash{}{0pt}%
\pgfsys@defobject{currentmarker}{\pgfqpoint{0.000000in}{-0.048611in}}{\pgfqpoint{0.000000in}{0.000000in}}{%
\pgfpathmoveto{\pgfqpoint{0.000000in}{0.000000in}}%
\pgfpathlineto{\pgfqpoint{0.000000in}{-0.048611in}}%
\pgfusepath{stroke,fill}%
}%
\begin{pgfscope}%
\pgfsys@transformshift{3.143061in}{0.528000in}%
\pgfsys@useobject{currentmarker}{}%
\end{pgfscope}%
\end{pgfscope}%
\begin{pgfscope}%
\pgftext[x=3.143061in,y=0.430778in,,top]{\sffamily\fontsize{10.000000}{12.000000}\selectfont 1500}%
\end{pgfscope}%
\begin{pgfscope}%
\pgfsetbuttcap%
\pgfsetroundjoin%
\definecolor{currentfill}{rgb}{0.000000,0.000000,0.000000}%
\pgfsetfillcolor{currentfill}%
\pgfsetlinewidth{0.803000pt}%
\definecolor{currentstroke}{rgb}{0.000000,0.000000,0.000000}%
\pgfsetstrokecolor{currentstroke}%
\pgfsetdash{}{0pt}%
\pgfsys@defobject{currentmarker}{\pgfqpoint{0.000000in}{-0.048611in}}{\pgfqpoint{0.000000in}{0.000000in}}{%
\pgfpathmoveto{\pgfqpoint{0.000000in}{0.000000in}}%
\pgfpathlineto{\pgfqpoint{0.000000in}{-0.048611in}}%
\pgfusepath{stroke,fill}%
}%
\begin{pgfscope}%
\pgfsys@transformshift{3.848930in}{0.528000in}%
\pgfsys@useobject{currentmarker}{}%
\end{pgfscope}%
\end{pgfscope}%
\begin{pgfscope}%
\pgftext[x=3.848930in,y=0.430778in,,top]{\sffamily\fontsize{10.000000}{12.000000}\selectfont 2000}%
\end{pgfscope}%
\begin{pgfscope}%
\pgfsetbuttcap%
\pgfsetroundjoin%
\definecolor{currentfill}{rgb}{0.000000,0.000000,0.000000}%
\pgfsetfillcolor{currentfill}%
\pgfsetlinewidth{0.803000pt}%
\definecolor{currentstroke}{rgb}{0.000000,0.000000,0.000000}%
\pgfsetstrokecolor{currentstroke}%
\pgfsetdash{}{0pt}%
\pgfsys@defobject{currentmarker}{\pgfqpoint{0.000000in}{-0.048611in}}{\pgfqpoint{0.000000in}{0.000000in}}{%
\pgfpathmoveto{\pgfqpoint{0.000000in}{0.000000in}}%
\pgfpathlineto{\pgfqpoint{0.000000in}{-0.048611in}}%
\pgfusepath{stroke,fill}%
}%
\begin{pgfscope}%
\pgfsys@transformshift{4.554799in}{0.528000in}%
\pgfsys@useobject{currentmarker}{}%
\end{pgfscope}%
\end{pgfscope}%
\begin{pgfscope}%
\pgftext[x=4.554799in,y=0.430778in,,top]{\sffamily\fontsize{10.000000}{12.000000}\selectfont 2500}%
\end{pgfscope}%
\begin{pgfscope}%
\pgfsetbuttcap%
\pgfsetroundjoin%
\definecolor{currentfill}{rgb}{0.000000,0.000000,0.000000}%
\pgfsetfillcolor{currentfill}%
\pgfsetlinewidth{0.803000pt}%
\definecolor{currentstroke}{rgb}{0.000000,0.000000,0.000000}%
\pgfsetstrokecolor{currentstroke}%
\pgfsetdash{}{0pt}%
\pgfsys@defobject{currentmarker}{\pgfqpoint{0.000000in}{-0.048611in}}{\pgfqpoint{0.000000in}{0.000000in}}{%
\pgfpathmoveto{\pgfqpoint{0.000000in}{0.000000in}}%
\pgfpathlineto{\pgfqpoint{0.000000in}{-0.048611in}}%
\pgfusepath{stroke,fill}%
}%
\begin{pgfscope}%
\pgfsys@transformshift{5.260668in}{0.528000in}%
\pgfsys@useobject{currentmarker}{}%
\end{pgfscope}%
\end{pgfscope}%
\begin{pgfscope}%
\pgftext[x=5.260668in,y=0.430778in,,top]{\sffamily\fontsize{10.000000}{12.000000}\selectfont 3000}%
\end{pgfscope}%
\begin{pgfscope}%
\pgftext[x=3.280000in,y=0.240809in,,top]{\sffamily\fontsize{10.000000}{12.000000}\selectfont Frame}%
\end{pgfscope}%
\begin{pgfscope}%
\pgfsetbuttcap%
\pgfsetroundjoin%
\definecolor{currentfill}{rgb}{0.000000,0.000000,0.000000}%
\pgfsetfillcolor{currentfill}%
\pgfsetlinewidth{0.803000pt}%
\definecolor{currentstroke}{rgb}{0.000000,0.000000,0.000000}%
\pgfsetstrokecolor{currentstroke}%
\pgfsetdash{}{0pt}%
\pgfsys@defobject{currentmarker}{\pgfqpoint{-0.048611in}{0.000000in}}{\pgfqpoint{0.000000in}{0.000000in}}{%
\pgfpathmoveto{\pgfqpoint{0.000000in}{0.000000in}}%
\pgfpathlineto{\pgfqpoint{-0.048611in}{0.000000in}}%
\pgfusepath{stroke,fill}%
}%
\begin{pgfscope}%
\pgfsys@transformshift{0.800000in}{0.696000in}%
\pgfsys@useobject{currentmarker}{}%
\end{pgfscope}%
\end{pgfscope}%
\begin{pgfscope}%
\pgftext[x=0.393533in,y=0.643238in,left,base]{\sffamily\fontsize{10.000000}{12.000000}\selectfont 0.00}%
\end{pgfscope}%
\begin{pgfscope}%
\pgfsetbuttcap%
\pgfsetroundjoin%
\definecolor{currentfill}{rgb}{0.000000,0.000000,0.000000}%
\pgfsetfillcolor{currentfill}%
\pgfsetlinewidth{0.803000pt}%
\definecolor{currentstroke}{rgb}{0.000000,0.000000,0.000000}%
\pgfsetstrokecolor{currentstroke}%
\pgfsetdash{}{0pt}%
\pgfsys@defobject{currentmarker}{\pgfqpoint{-0.048611in}{0.000000in}}{\pgfqpoint{0.000000in}{0.000000in}}{%
\pgfpathmoveto{\pgfqpoint{0.000000in}{0.000000in}}%
\pgfpathlineto{\pgfqpoint{-0.048611in}{0.000000in}}%
\pgfusepath{stroke,fill}%
}%
\begin{pgfscope}%
\pgfsys@transformshift{0.800000in}{1.164515in}%
\pgfsys@useobject{currentmarker}{}%
\end{pgfscope}%
\end{pgfscope}%
\begin{pgfscope}%
\pgftext[x=0.393533in,y=1.111753in,left,base]{\sffamily\fontsize{10.000000}{12.000000}\selectfont 0.02}%
\end{pgfscope}%
\begin{pgfscope}%
\pgfsetbuttcap%
\pgfsetroundjoin%
\definecolor{currentfill}{rgb}{0.000000,0.000000,0.000000}%
\pgfsetfillcolor{currentfill}%
\pgfsetlinewidth{0.803000pt}%
\definecolor{currentstroke}{rgb}{0.000000,0.000000,0.000000}%
\pgfsetstrokecolor{currentstroke}%
\pgfsetdash{}{0pt}%
\pgfsys@defobject{currentmarker}{\pgfqpoint{-0.048611in}{0.000000in}}{\pgfqpoint{0.000000in}{0.000000in}}{%
\pgfpathmoveto{\pgfqpoint{0.000000in}{0.000000in}}%
\pgfpathlineto{\pgfqpoint{-0.048611in}{0.000000in}}%
\pgfusepath{stroke,fill}%
}%
\begin{pgfscope}%
\pgfsys@transformshift{0.800000in}{1.633029in}%
\pgfsys@useobject{currentmarker}{}%
\end{pgfscope}%
\end{pgfscope}%
\begin{pgfscope}%
\pgftext[x=0.393533in,y=1.580268in,left,base]{\sffamily\fontsize{10.000000}{12.000000}\selectfont 0.04}%
\end{pgfscope}%
\begin{pgfscope}%
\pgfsetbuttcap%
\pgfsetroundjoin%
\definecolor{currentfill}{rgb}{0.000000,0.000000,0.000000}%
\pgfsetfillcolor{currentfill}%
\pgfsetlinewidth{0.803000pt}%
\definecolor{currentstroke}{rgb}{0.000000,0.000000,0.000000}%
\pgfsetstrokecolor{currentstroke}%
\pgfsetdash{}{0pt}%
\pgfsys@defobject{currentmarker}{\pgfqpoint{-0.048611in}{0.000000in}}{\pgfqpoint{0.000000in}{0.000000in}}{%
\pgfpathmoveto{\pgfqpoint{0.000000in}{0.000000in}}%
\pgfpathlineto{\pgfqpoint{-0.048611in}{0.000000in}}%
\pgfusepath{stroke,fill}%
}%
\begin{pgfscope}%
\pgfsys@transformshift{0.800000in}{2.101544in}%
\pgfsys@useobject{currentmarker}{}%
\end{pgfscope}%
\end{pgfscope}%
\begin{pgfscope}%
\pgftext[x=0.393533in,y=2.048783in,left,base]{\sffamily\fontsize{10.000000}{12.000000}\selectfont 0.06}%
\end{pgfscope}%
\begin{pgfscope}%
\pgfsetbuttcap%
\pgfsetroundjoin%
\definecolor{currentfill}{rgb}{0.000000,0.000000,0.000000}%
\pgfsetfillcolor{currentfill}%
\pgfsetlinewidth{0.803000pt}%
\definecolor{currentstroke}{rgb}{0.000000,0.000000,0.000000}%
\pgfsetstrokecolor{currentstroke}%
\pgfsetdash{}{0pt}%
\pgfsys@defobject{currentmarker}{\pgfqpoint{-0.048611in}{0.000000in}}{\pgfqpoint{0.000000in}{0.000000in}}{%
\pgfpathmoveto{\pgfqpoint{0.000000in}{0.000000in}}%
\pgfpathlineto{\pgfqpoint{-0.048611in}{0.000000in}}%
\pgfusepath{stroke,fill}%
}%
\begin{pgfscope}%
\pgfsys@transformshift{0.800000in}{2.570059in}%
\pgfsys@useobject{currentmarker}{}%
\end{pgfscope}%
\end{pgfscope}%
\begin{pgfscope}%
\pgftext[x=0.393533in,y=2.517297in,left,base]{\sffamily\fontsize{10.000000}{12.000000}\selectfont 0.08}%
\end{pgfscope}%
\begin{pgfscope}%
\pgfsetbuttcap%
\pgfsetroundjoin%
\definecolor{currentfill}{rgb}{0.000000,0.000000,0.000000}%
\pgfsetfillcolor{currentfill}%
\pgfsetlinewidth{0.803000pt}%
\definecolor{currentstroke}{rgb}{0.000000,0.000000,0.000000}%
\pgfsetstrokecolor{currentstroke}%
\pgfsetdash{}{0pt}%
\pgfsys@defobject{currentmarker}{\pgfqpoint{-0.048611in}{0.000000in}}{\pgfqpoint{0.000000in}{0.000000in}}{%
\pgfpathmoveto{\pgfqpoint{0.000000in}{0.000000in}}%
\pgfpathlineto{\pgfqpoint{-0.048611in}{0.000000in}}%
\pgfusepath{stroke,fill}%
}%
\begin{pgfscope}%
\pgfsys@transformshift{0.800000in}{3.038573in}%
\pgfsys@useobject{currentmarker}{}%
\end{pgfscope}%
\end{pgfscope}%
\begin{pgfscope}%
\pgftext[x=0.393533in,y=2.985812in,left,base]{\sffamily\fontsize{10.000000}{12.000000}\selectfont 0.10}%
\end{pgfscope}%
\begin{pgfscope}%
\pgfsetbuttcap%
\pgfsetroundjoin%
\definecolor{currentfill}{rgb}{0.000000,0.000000,0.000000}%
\pgfsetfillcolor{currentfill}%
\pgfsetlinewidth{0.803000pt}%
\definecolor{currentstroke}{rgb}{0.000000,0.000000,0.000000}%
\pgfsetstrokecolor{currentstroke}%
\pgfsetdash{}{0pt}%
\pgfsys@defobject{currentmarker}{\pgfqpoint{-0.048611in}{0.000000in}}{\pgfqpoint{0.000000in}{0.000000in}}{%
\pgfpathmoveto{\pgfqpoint{0.000000in}{0.000000in}}%
\pgfpathlineto{\pgfqpoint{-0.048611in}{0.000000in}}%
\pgfusepath{stroke,fill}%
}%
\begin{pgfscope}%
\pgfsys@transformshift{0.800000in}{3.507088in}%
\pgfsys@useobject{currentmarker}{}%
\end{pgfscope}%
\end{pgfscope}%
\begin{pgfscope}%
\pgftext[x=0.393533in,y=3.454327in,left,base]{\sffamily\fontsize{10.000000}{12.000000}\selectfont 0.12}%
\end{pgfscope}%
\begin{pgfscope}%
\pgfsetbuttcap%
\pgfsetroundjoin%
\definecolor{currentfill}{rgb}{0.000000,0.000000,0.000000}%
\pgfsetfillcolor{currentfill}%
\pgfsetlinewidth{0.803000pt}%
\definecolor{currentstroke}{rgb}{0.000000,0.000000,0.000000}%
\pgfsetstrokecolor{currentstroke}%
\pgfsetdash{}{0pt}%
\pgfsys@defobject{currentmarker}{\pgfqpoint{-0.048611in}{0.000000in}}{\pgfqpoint{0.000000in}{0.000000in}}{%
\pgfpathmoveto{\pgfqpoint{0.000000in}{0.000000in}}%
\pgfpathlineto{\pgfqpoint{-0.048611in}{0.000000in}}%
\pgfusepath{stroke,fill}%
}%
\begin{pgfscope}%
\pgfsys@transformshift{0.800000in}{3.975603in}%
\pgfsys@useobject{currentmarker}{}%
\end{pgfscope}%
\end{pgfscope}%
\begin{pgfscope}%
\pgftext[x=0.393533in,y=3.922841in,left,base]{\sffamily\fontsize{10.000000}{12.000000}\selectfont 0.14}%
\end{pgfscope}%
\begin{pgfscope}%
\pgftext[x=0.337977in,y=2.376000in,,bottom,rotate=90.000000]{\sffamily\fontsize{10.000000}{12.000000}\selectfont Amplitude}%
\end{pgfscope}%
\begin{pgfscope}%
\pgfpathrectangle{\pgfqpoint{0.800000in}{0.528000in}}{\pgfqpoint{4.960000in}{3.696000in}} %
\pgfusepath{clip}%
\pgfsetrectcap%
\pgfsetroundjoin%
\pgfsetlinewidth{1.505625pt}%
\definecolor{currentstroke}{rgb}{0.121569,0.466667,0.705882}%
\pgfsetstrokecolor{currentstroke}%
\pgfsetdash{}{0pt}%
\pgfpathmoveto{\pgfqpoint{1.025455in}{0.696000in}}%
\pgfpathlineto{\pgfqpoint{1.122864in}{0.696000in}}%
\pgfpathlineto{\pgfqpoint{1.124276in}{0.726915in}}%
\pgfpathlineto{\pgfqpoint{1.125688in}{0.719481in}}%
\pgfpathlineto{\pgfqpoint{1.128511in}{0.719094in}}%
\pgfpathlineto{\pgfqpoint{1.131335in}{0.722554in}}%
\pgfpathlineto{\pgfqpoint{1.138394in}{0.742600in}}%
\pgfpathlineto{\pgfqpoint{1.139805in}{0.738570in}}%
\pgfpathlineto{\pgfqpoint{1.142629in}{0.723043in}}%
\pgfpathlineto{\pgfqpoint{1.144041in}{0.697046in}}%
\pgfpathlineto{\pgfqpoint{1.145452in}{0.717534in}}%
\pgfpathlineto{\pgfqpoint{1.146864in}{0.715895in}}%
\pgfpathlineto{\pgfqpoint{1.148276in}{0.716231in}}%
\pgfpathlineto{\pgfqpoint{1.151099in}{0.722117in}}%
\pgfpathlineto{\pgfqpoint{1.152511in}{0.721770in}}%
\pgfpathlineto{\pgfqpoint{1.155334in}{0.714117in}}%
\pgfpathlineto{\pgfqpoint{1.158158in}{0.716080in}}%
\pgfpathlineto{\pgfqpoint{1.159570in}{0.717729in}}%
\pgfpathlineto{\pgfqpoint{1.160981in}{0.721099in}}%
\pgfpathlineto{\pgfqpoint{1.163805in}{0.739412in}}%
\pgfpathlineto{\pgfqpoint{1.165217in}{0.697299in}}%
\pgfpathlineto{\pgfqpoint{1.166628in}{0.743835in}}%
\pgfpathlineto{\pgfqpoint{1.169452in}{0.734302in}}%
\pgfpathlineto{\pgfqpoint{1.170864in}{0.727277in}}%
\pgfpathlineto{\pgfqpoint{1.173687in}{0.768602in}}%
\pgfpathlineto{\pgfqpoint{1.177922in}{0.878665in}}%
\pgfpathlineto{\pgfqpoint{1.183569in}{1.110529in}}%
\pgfpathlineto{\pgfqpoint{1.199098in}{1.818959in}}%
\pgfpathlineto{\pgfqpoint{1.204745in}{2.003323in}}%
\pgfpathlineto{\pgfqpoint{1.210392in}{2.114744in}}%
\pgfpathlineto{\pgfqpoint{1.214627in}{2.159891in}}%
\pgfpathlineto{\pgfqpoint{1.218863in}{2.180180in}}%
\pgfpathlineto{\pgfqpoint{1.224510in}{2.205432in}}%
\pgfpathlineto{\pgfqpoint{1.228745in}{2.240307in}}%
\pgfpathlineto{\pgfqpoint{1.234392in}{2.314974in}}%
\pgfpathlineto{\pgfqpoint{1.245686in}{2.474875in}}%
\pgfpathlineto{\pgfqpoint{1.251333in}{2.514662in}}%
\pgfpathlineto{\pgfqpoint{1.261215in}{2.560589in}}%
\pgfpathlineto{\pgfqpoint{1.265450in}{2.571168in}}%
\pgfpathlineto{\pgfqpoint{1.268273in}{2.572162in}}%
\pgfpathlineto{\pgfqpoint{1.271097in}{2.569434in}}%
\pgfpathlineto{\pgfqpoint{1.275332in}{2.560283in}}%
\pgfpathlineto{\pgfqpoint{1.286626in}{2.532667in}}%
\pgfpathlineto{\pgfqpoint{1.289450in}{2.530822in}}%
\pgfpathlineto{\pgfqpoint{1.292273in}{2.532876in}}%
\pgfpathlineto{\pgfqpoint{1.300743in}{2.544933in}}%
\pgfpathlineto{\pgfqpoint{1.302155in}{2.544601in}}%
\pgfpathlineto{\pgfqpoint{1.304979in}{2.539684in}}%
\pgfpathlineto{\pgfqpoint{1.307802in}{2.527220in}}%
\pgfpathlineto{\pgfqpoint{1.312037in}{2.493925in}}%
\pgfpathlineto{\pgfqpoint{1.317684in}{2.426472in}}%
\pgfpathlineto{\pgfqpoint{1.326155in}{2.295829in}}%
\pgfpathlineto{\pgfqpoint{1.338860in}{2.091543in}}%
\pgfpathlineto{\pgfqpoint{1.343096in}{2.056709in}}%
\pgfpathlineto{\pgfqpoint{1.344507in}{2.052296in}}%
\pgfpathlineto{\pgfqpoint{1.345919in}{2.052215in}}%
\pgfpathlineto{\pgfqpoint{1.347331in}{2.056680in}}%
\pgfpathlineto{\pgfqpoint{1.350154in}{2.079015in}}%
\pgfpathlineto{\pgfqpoint{1.355801in}{2.157901in}}%
\pgfpathlineto{\pgfqpoint{1.369919in}{2.373251in}}%
\pgfpathlineto{\pgfqpoint{1.375566in}{2.429000in}}%
\pgfpathlineto{\pgfqpoint{1.388271in}{2.523692in}}%
\pgfpathlineto{\pgfqpoint{1.391095in}{2.536034in}}%
\pgfpathlineto{\pgfqpoint{1.392506in}{2.538710in}}%
\pgfpathlineto{\pgfqpoint{1.393918in}{2.538982in}}%
\pgfpathlineto{\pgfqpoint{1.395330in}{2.536800in}}%
\pgfpathlineto{\pgfqpoint{1.396742in}{2.532057in}}%
\pgfpathlineto{\pgfqpoint{1.400977in}{2.495001in}}%
\pgfpathlineto{\pgfqpoint{1.423565in}{2.217411in}}%
\pgfpathlineto{\pgfqpoint{1.427800in}{2.106515in}}%
\pgfpathlineto{\pgfqpoint{1.441917in}{1.673604in}}%
\pgfpathlineto{\pgfqpoint{1.444741in}{1.649414in}}%
\pgfpathlineto{\pgfqpoint{1.446152in}{1.649344in}}%
\pgfpathlineto{\pgfqpoint{1.447564in}{1.656122in}}%
\pgfpathlineto{\pgfqpoint{1.450388in}{1.685729in}}%
\pgfpathlineto{\pgfqpoint{1.457446in}{1.804878in}}%
\pgfpathlineto{\pgfqpoint{1.475799in}{2.150346in}}%
\pgfpathlineto{\pgfqpoint{1.481446in}{2.233346in}}%
\pgfpathlineto{\pgfqpoint{1.487093in}{2.283027in}}%
\pgfpathlineto{\pgfqpoint{1.491328in}{2.303616in}}%
\pgfpathlineto{\pgfqpoint{1.494152in}{2.304984in}}%
\pgfpathlineto{\pgfqpoint{1.496975in}{2.299471in}}%
\pgfpathlineto{\pgfqpoint{1.499798in}{2.285568in}}%
\pgfpathlineto{\pgfqpoint{1.502622in}{2.260096in}}%
\pgfpathlineto{\pgfqpoint{1.505445in}{2.218620in}}%
\pgfpathlineto{\pgfqpoint{1.509681in}{2.116236in}}%
\pgfpathlineto{\pgfqpoint{1.515328in}{1.905960in}}%
\pgfpathlineto{\pgfqpoint{1.526622in}{1.443914in}}%
\pgfpathlineto{\pgfqpoint{1.530857in}{1.350088in}}%
\pgfpathlineto{\pgfqpoint{1.533680in}{1.315880in}}%
\pgfpathlineto{\pgfqpoint{1.535092in}{1.308721in}}%
\pgfpathlineto{\pgfqpoint{1.536504in}{1.308516in}}%
\pgfpathlineto{\pgfqpoint{1.537915in}{1.313519in}}%
\pgfpathlineto{\pgfqpoint{1.540739in}{1.338159in}}%
\pgfpathlineto{\pgfqpoint{1.542151in}{1.355583in}}%
\pgfpathlineto{\pgfqpoint{1.557680in}{1.707104in}}%
\pgfpathlineto{\pgfqpoint{1.566150in}{1.890890in}}%
\pgfpathlineto{\pgfqpoint{1.571797in}{2.057349in}}%
\pgfpathlineto{\pgfqpoint{1.580268in}{2.383661in}}%
\pgfpathlineto{\pgfqpoint{1.588738in}{2.677939in}}%
\pgfpathlineto{\pgfqpoint{1.592973in}{2.763043in}}%
\pgfpathlineto{\pgfqpoint{1.595797in}{2.793066in}}%
\pgfpathlineto{\pgfqpoint{1.597208in}{2.798994in}}%
\pgfpathlineto{\pgfqpoint{1.598620in}{2.799301in}}%
\pgfpathlineto{\pgfqpoint{1.600032in}{2.794861in}}%
\pgfpathlineto{\pgfqpoint{1.602855in}{2.766395in}}%
\pgfpathlineto{\pgfqpoint{1.607091in}{2.672673in}}%
\pgfpathlineto{\pgfqpoint{1.612738in}{2.461635in}}%
\pgfpathlineto{\pgfqpoint{1.629678in}{1.756357in}}%
\pgfpathlineto{\pgfqpoint{1.636737in}{1.566147in}}%
\pgfpathlineto{\pgfqpoint{1.640972in}{1.493301in}}%
\pgfpathlineto{\pgfqpoint{1.643796in}{1.474949in}}%
\pgfpathlineto{\pgfqpoint{1.645207in}{1.476843in}}%
\pgfpathlineto{\pgfqpoint{1.646619in}{1.486738in}}%
\pgfpathlineto{\pgfqpoint{1.649443in}{1.529363in}}%
\pgfpathlineto{\pgfqpoint{1.653678in}{1.637978in}}%
\pgfpathlineto{\pgfqpoint{1.659325in}{1.857457in}}%
\pgfpathlineto{\pgfqpoint{1.669207in}{2.373331in}}%
\pgfpathlineto{\pgfqpoint{1.684736in}{3.193914in}}%
\pgfpathlineto{\pgfqpoint{1.690383in}{3.381346in}}%
\pgfpathlineto{\pgfqpoint{1.694618in}{3.462474in}}%
\pgfpathlineto{\pgfqpoint{1.697442in}{3.488327in}}%
\pgfpathlineto{\pgfqpoint{1.698854in}{3.492063in}}%
\pgfpathlineto{\pgfqpoint{1.700265in}{3.489711in}}%
\pgfpathlineto{\pgfqpoint{1.701677in}{3.480514in}}%
\pgfpathlineto{\pgfqpoint{1.704500in}{3.439107in}}%
\pgfpathlineto{\pgfqpoint{1.708736in}{3.320502in}}%
\pgfpathlineto{\pgfqpoint{1.715794in}{3.027716in}}%
\pgfpathlineto{\pgfqpoint{1.722853in}{2.745742in}}%
\pgfpathlineto{\pgfqpoint{1.728500in}{2.602731in}}%
\pgfpathlineto{\pgfqpoint{1.734147in}{2.512943in}}%
\pgfpathlineto{\pgfqpoint{1.736970in}{2.487980in}}%
\pgfpathlineto{\pgfqpoint{1.738382in}{2.483679in}}%
\pgfpathlineto{\pgfqpoint{1.739794in}{2.485800in}}%
\pgfpathlineto{\pgfqpoint{1.742617in}{2.509694in}}%
\pgfpathlineto{\pgfqpoint{1.745441in}{2.561604in}}%
\pgfpathlineto{\pgfqpoint{1.751088in}{2.735329in}}%
\pgfpathlineto{\pgfqpoint{1.758147in}{3.058766in}}%
\pgfpathlineto{\pgfqpoint{1.773676in}{3.820054in}}%
\pgfpathlineto{\pgfqpoint{1.779323in}{3.983216in}}%
\pgfpathlineto{\pgfqpoint{1.783558in}{4.045882in}}%
\pgfpathlineto{\pgfqpoint{1.784970in}{4.054416in}}%
\pgfpathlineto{\pgfqpoint{1.786381in}{4.056000in}}%
\pgfpathlineto{\pgfqpoint{1.787793in}{4.050045in}}%
\pgfpathlineto{\pgfqpoint{1.790616in}{4.014525in}}%
\pgfpathlineto{\pgfqpoint{1.794852in}{3.905937in}}%
\pgfpathlineto{\pgfqpoint{1.807557in}{3.495174in}}%
\pgfpathlineto{\pgfqpoint{1.820263in}{3.228055in}}%
\pgfpathlineto{\pgfqpoint{1.830145in}{3.044608in}}%
\pgfpathlineto{\pgfqpoint{1.837204in}{2.947504in}}%
\pgfpathlineto{\pgfqpoint{1.840027in}{2.925884in}}%
\pgfpathlineto{\pgfqpoint{1.841439in}{2.920947in}}%
\pgfpathlineto{\pgfqpoint{1.842851in}{2.920814in}}%
\pgfpathlineto{\pgfqpoint{1.844263in}{2.925849in}}%
\pgfpathlineto{\pgfqpoint{1.847086in}{2.952851in}}%
\pgfpathlineto{\pgfqpoint{1.851321in}{3.026706in}}%
\pgfpathlineto{\pgfqpoint{1.856968in}{3.188838in}}%
\pgfpathlineto{\pgfqpoint{1.866850in}{3.573053in}}%
\pgfpathlineto{\pgfqpoint{1.873909in}{3.823736in}}%
\pgfpathlineto{\pgfqpoint{1.878144in}{3.922291in}}%
\pgfpathlineto{\pgfqpoint{1.880968in}{3.955468in}}%
\pgfpathlineto{\pgfqpoint{1.882379in}{3.960892in}}%
\pgfpathlineto{\pgfqpoint{1.883791in}{3.958380in}}%
\pgfpathlineto{\pgfqpoint{1.886615in}{3.928965in}}%
\pgfpathlineto{\pgfqpoint{1.890850in}{3.829801in}}%
\pgfpathlineto{\pgfqpoint{1.897909in}{3.560684in}}%
\pgfpathlineto{\pgfqpoint{1.907791in}{3.182764in}}%
\pgfpathlineto{\pgfqpoint{1.919085in}{2.860184in}}%
\pgfpathlineto{\pgfqpoint{1.930379in}{2.573028in}}%
\pgfpathlineto{\pgfqpoint{1.934614in}{2.504254in}}%
\pgfpathlineto{\pgfqpoint{1.938849in}{2.467218in}}%
\pgfpathlineto{\pgfqpoint{1.941672in}{2.457305in}}%
\pgfpathlineto{\pgfqpoint{1.943084in}{2.456666in}}%
\pgfpathlineto{\pgfqpoint{1.947319in}{2.468855in}}%
\pgfpathlineto{\pgfqpoint{1.950143in}{2.489553in}}%
\pgfpathlineto{\pgfqpoint{1.954378in}{2.541962in}}%
\pgfpathlineto{\pgfqpoint{1.960025in}{2.654968in}}%
\pgfpathlineto{\pgfqpoint{1.974142in}{2.975703in}}%
\pgfpathlineto{\pgfqpoint{1.978378in}{3.017490in}}%
\pgfpathlineto{\pgfqpoint{1.979789in}{3.021025in}}%
\pgfpathlineto{\pgfqpoint{1.981201in}{3.019087in}}%
\pgfpathlineto{\pgfqpoint{1.984025in}{2.997367in}}%
\pgfpathlineto{\pgfqpoint{1.986848in}{2.951808in}}%
\pgfpathlineto{\pgfqpoint{1.991083in}{2.844209in}}%
\pgfpathlineto{\pgfqpoint{1.998142in}{2.566748in}}%
\pgfpathlineto{\pgfqpoint{2.022142in}{1.485860in}}%
\pgfpathlineto{\pgfqpoint{2.027788in}{1.364090in}}%
\pgfpathlineto{\pgfqpoint{2.030612in}{1.338002in}}%
\pgfpathlineto{\pgfqpoint{2.032024in}{1.341289in}}%
\pgfpathlineto{\pgfqpoint{2.034847in}{1.368027in}}%
\pgfpathlineto{\pgfqpoint{2.039082in}{1.446137in}}%
\pgfpathlineto{\pgfqpoint{2.044729in}{1.600794in}}%
\pgfpathlineto{\pgfqpoint{2.053200in}{1.908338in}}%
\pgfpathlineto{\pgfqpoint{2.071552in}{2.653559in}}%
\pgfpathlineto{\pgfqpoint{2.075788in}{2.731991in}}%
\pgfpathlineto{\pgfqpoint{2.078611in}{2.745013in}}%
\pgfpathlineto{\pgfqpoint{2.080023in}{2.740215in}}%
\pgfpathlineto{\pgfqpoint{2.082846in}{2.709888in}}%
\pgfpathlineto{\pgfqpoint{2.087081in}{2.620017in}}%
\pgfpathlineto{\pgfqpoint{2.091317in}{2.482324in}}%
\pgfpathlineto{\pgfqpoint{2.098375in}{2.165845in}}%
\pgfpathlineto{\pgfqpoint{2.112493in}{1.492856in}}%
\pgfpathlineto{\pgfqpoint{2.116728in}{1.348165in}}%
\pgfpathlineto{\pgfqpoint{2.120963in}{1.258950in}}%
\pgfpathlineto{\pgfqpoint{2.123787in}{1.236232in}}%
\pgfpathlineto{\pgfqpoint{2.125198in}{1.235225in}}%
\pgfpathlineto{\pgfqpoint{2.126610in}{1.239509in}}%
\pgfpathlineto{\pgfqpoint{2.129434in}{1.285323in}}%
\pgfpathlineto{\pgfqpoint{2.139316in}{1.504499in}}%
\pgfpathlineto{\pgfqpoint{2.144963in}{1.688201in}}%
\pgfpathlineto{\pgfqpoint{2.152021in}{2.001293in}}%
\pgfpathlineto{\pgfqpoint{2.170374in}{2.888659in}}%
\pgfpathlineto{\pgfqpoint{2.174609in}{2.971170in}}%
\pgfpathlineto{\pgfqpoint{2.177433in}{2.987107in}}%
\pgfpathlineto{\pgfqpoint{2.178844in}{2.984034in}}%
\pgfpathlineto{\pgfqpoint{2.181668in}{2.957051in}}%
\pgfpathlineto{\pgfqpoint{2.185903in}{2.867770in}}%
\pgfpathlineto{\pgfqpoint{2.191550in}{2.674351in}}%
\pgfpathlineto{\pgfqpoint{2.201432in}{2.236529in}}%
\pgfpathlineto{\pgfqpoint{2.216961in}{1.509609in}}%
\pgfpathlineto{\pgfqpoint{2.222608in}{1.372390in}}%
\pgfpathlineto{\pgfqpoint{2.226844in}{1.308414in}}%
\pgfpathlineto{\pgfqpoint{2.229667in}{1.291094in}}%
\pgfpathlineto{\pgfqpoint{2.231079in}{1.288015in}}%
\pgfpathlineto{\pgfqpoint{2.232490in}{1.288442in}}%
\pgfpathlineto{\pgfqpoint{2.235314in}{1.300269in}}%
\pgfpathlineto{\pgfqpoint{2.238137in}{1.325298in}}%
\pgfpathlineto{\pgfqpoint{2.243784in}{1.416498in}}%
\pgfpathlineto{\pgfqpoint{2.249431in}{1.561276in}}%
\pgfpathlineto{\pgfqpoint{2.257902in}{1.856559in}}%
\pgfpathlineto{\pgfqpoint{2.267784in}{2.196050in}}%
\pgfpathlineto{\pgfqpoint{2.272019in}{2.287155in}}%
\pgfpathlineto{\pgfqpoint{2.274843in}{2.318932in}}%
\pgfpathlineto{\pgfqpoint{2.276254in}{2.325243in}}%
\pgfpathlineto{\pgfqpoint{2.277666in}{2.325337in}}%
\pgfpathlineto{\pgfqpoint{2.279078in}{2.319440in}}%
\pgfpathlineto{\pgfqpoint{2.281901in}{2.288960in}}%
\pgfpathlineto{\pgfqpoint{2.286137in}{2.188708in}}%
\pgfpathlineto{\pgfqpoint{2.290372in}{2.011759in}}%
\pgfpathlineto{\pgfqpoint{2.303077in}{1.320465in}}%
\pgfpathlineto{\pgfqpoint{2.305901in}{1.286063in}}%
\pgfpathlineto{\pgfqpoint{2.307313in}{1.292574in}}%
\pgfpathlineto{\pgfqpoint{2.310136in}{1.344175in}}%
\pgfpathlineto{\pgfqpoint{2.317195in}{1.534819in}}%
\pgfpathlineto{\pgfqpoint{2.324253in}{1.733042in}}%
\pgfpathlineto{\pgfqpoint{2.329900in}{1.836901in}}%
\pgfpathlineto{\pgfqpoint{2.334136in}{1.881430in}}%
\pgfpathlineto{\pgfqpoint{2.335547in}{1.886062in}}%
\pgfpathlineto{\pgfqpoint{2.336959in}{1.886652in}}%
\pgfpathlineto{\pgfqpoint{2.338371in}{1.883460in}}%
\pgfpathlineto{\pgfqpoint{2.341194in}{1.867599in}}%
\pgfpathlineto{\pgfqpoint{2.345429in}{1.825184in}}%
\pgfpathlineto{\pgfqpoint{2.349665in}{1.786202in}}%
\pgfpathlineto{\pgfqpoint{2.352488in}{1.776952in}}%
\pgfpathlineto{\pgfqpoint{2.353900in}{1.779387in}}%
\pgfpathlineto{\pgfqpoint{2.356723in}{1.799519in}}%
\pgfpathlineto{\pgfqpoint{2.360959in}{1.868376in}}%
\pgfpathlineto{\pgfqpoint{2.373664in}{2.118294in}}%
\pgfpathlineto{\pgfqpoint{2.375076in}{2.120802in}}%
\pgfpathlineto{\pgfqpoint{2.376488in}{2.114222in}}%
\pgfpathlineto{\pgfqpoint{2.379311in}{2.073060in}}%
\pgfpathlineto{\pgfqpoint{2.383546in}{1.946513in}}%
\pgfpathlineto{\pgfqpoint{2.392017in}{1.557548in}}%
\pgfpathlineto{\pgfqpoint{2.396252in}{1.382538in}}%
\pgfpathlineto{\pgfqpoint{2.400487in}{1.277698in}}%
\pgfpathlineto{\pgfqpoint{2.401899in}{1.261828in}}%
\pgfpathlineto{\pgfqpoint{2.403311in}{1.258843in}}%
\pgfpathlineto{\pgfqpoint{2.404722in}{1.269315in}}%
\pgfpathlineto{\pgfqpoint{2.407546in}{1.328382in}}%
\pgfpathlineto{\pgfqpoint{2.413193in}{1.535368in}}%
\pgfpathlineto{\pgfqpoint{2.427310in}{2.084939in}}%
\pgfpathlineto{\pgfqpoint{2.431546in}{2.140826in}}%
\pgfpathlineto{\pgfqpoint{2.434369in}{2.152938in}}%
\pgfpathlineto{\pgfqpoint{2.435781in}{2.154030in}}%
\pgfpathlineto{\pgfqpoint{2.438604in}{2.141679in}}%
\pgfpathlineto{\pgfqpoint{2.442839in}{2.107572in}}%
\pgfpathlineto{\pgfqpoint{2.451310in}{2.038000in}}%
\pgfpathlineto{\pgfqpoint{2.454133in}{2.028056in}}%
\pgfpathlineto{\pgfqpoint{2.455545in}{2.028758in}}%
\pgfpathlineto{\pgfqpoint{2.458369in}{2.040705in}}%
\pgfpathlineto{\pgfqpoint{2.462604in}{2.078676in}}%
\pgfpathlineto{\pgfqpoint{2.468251in}{2.131100in}}%
\pgfpathlineto{\pgfqpoint{2.469662in}{2.138830in}}%
\pgfpathlineto{\pgfqpoint{2.471074in}{2.139226in}}%
\pgfpathlineto{\pgfqpoint{2.472486in}{2.134374in}}%
\pgfpathlineto{\pgfqpoint{2.475309in}{2.107980in}}%
\pgfpathlineto{\pgfqpoint{2.479545in}{2.030497in}}%
\pgfpathlineto{\pgfqpoint{2.486603in}{1.824697in}}%
\pgfpathlineto{\pgfqpoint{2.493662in}{1.627031in}}%
\pgfpathlineto{\pgfqpoint{2.496485in}{1.604187in}}%
\pgfpathlineto{\pgfqpoint{2.497897in}{1.600793in}}%
\pgfpathlineto{\pgfqpoint{2.499309in}{1.603477in}}%
\pgfpathlineto{\pgfqpoint{2.502132in}{1.626004in}}%
\pgfpathlineto{\pgfqpoint{2.506368in}{1.695849in}}%
\pgfpathlineto{\pgfqpoint{2.513426in}{1.857636in}}%
\pgfpathlineto{\pgfqpoint{2.521897in}{2.078318in}}%
\pgfpathlineto{\pgfqpoint{2.527544in}{2.176005in}}%
\pgfpathlineto{\pgfqpoint{2.533191in}{2.230221in}}%
\pgfpathlineto{\pgfqpoint{2.541661in}{2.282238in}}%
\pgfpathlineto{\pgfqpoint{2.548720in}{2.330912in}}%
\pgfpathlineto{\pgfqpoint{2.554367in}{2.366877in}}%
\pgfpathlineto{\pgfqpoint{2.557190in}{2.374506in}}%
\pgfpathlineto{\pgfqpoint{2.558602in}{2.374295in}}%
\pgfpathlineto{\pgfqpoint{2.560014in}{2.371029in}}%
\pgfpathlineto{\pgfqpoint{2.562837in}{2.353711in}}%
\pgfpathlineto{\pgfqpoint{2.565661in}{2.321444in}}%
\pgfpathlineto{\pgfqpoint{2.569896in}{2.246040in}}%
\pgfpathlineto{\pgfqpoint{2.584013in}{1.952900in}}%
\pgfpathlineto{\pgfqpoint{2.588248in}{1.910870in}}%
\pgfpathlineto{\pgfqpoint{2.598131in}{1.853293in}}%
\pgfpathlineto{\pgfqpoint{2.610836in}{1.743795in}}%
\pgfpathlineto{\pgfqpoint{2.630601in}{1.605572in}}%
\pgfpathlineto{\pgfqpoint{2.632012in}{1.606226in}}%
\pgfpathlineto{\pgfqpoint{2.633424in}{1.611373in}}%
\pgfpathlineto{\pgfqpoint{2.636248in}{1.634540in}}%
\pgfpathlineto{\pgfqpoint{2.641894in}{1.716782in}}%
\pgfpathlineto{\pgfqpoint{2.650365in}{1.844961in}}%
\pgfpathlineto{\pgfqpoint{2.654600in}{1.876488in}}%
\pgfpathlineto{\pgfqpoint{2.656012in}{1.880184in}}%
\pgfpathlineto{\pgfqpoint{2.657424in}{1.880125in}}%
\pgfpathlineto{\pgfqpoint{2.658835in}{1.876291in}}%
\pgfpathlineto{\pgfqpoint{2.661659in}{1.858547in}}%
\pgfpathlineto{\pgfqpoint{2.670129in}{1.788775in}}%
\pgfpathlineto{\pgfqpoint{2.671541in}{1.786508in}}%
\pgfpathlineto{\pgfqpoint{2.672953in}{1.789888in}}%
\pgfpathlineto{\pgfqpoint{2.675776in}{1.817419in}}%
\pgfpathlineto{\pgfqpoint{2.678600in}{1.874707in}}%
\pgfpathlineto{\pgfqpoint{2.682835in}{2.018401in}}%
\pgfpathlineto{\pgfqpoint{2.688482in}{2.301239in}}%
\pgfpathlineto{\pgfqpoint{2.701187in}{2.990120in}}%
\pgfpathlineto{\pgfqpoint{2.704011in}{3.041706in}}%
\pgfpathlineto{\pgfqpoint{2.705423in}{3.037846in}}%
\pgfpathlineto{\pgfqpoint{2.706834in}{3.008815in}}%
\pgfpathlineto{\pgfqpoint{2.709658in}{2.871767in}}%
\pgfpathlineto{\pgfqpoint{2.713893in}{2.465132in}}%
\pgfpathlineto{\pgfqpoint{2.720952in}{1.637000in}}%
\pgfpathlineto{\pgfqpoint{2.723775in}{1.508661in}}%
\pgfpathlineto{\pgfqpoint{2.725187in}{1.473303in}}%
\pgfpathlineto{\pgfqpoint{2.726599in}{1.468693in}}%
\pgfpathlineto{\pgfqpoint{2.732246in}{1.486957in}}%
\pgfpathlineto{\pgfqpoint{2.735069in}{1.516148in}}%
\pgfpathlineto{\pgfqpoint{2.739304in}{1.575638in}}%
\pgfpathlineto{\pgfqpoint{2.740716in}{1.579262in}}%
\pgfpathlineto{\pgfqpoint{2.742128in}{1.570042in}}%
\pgfpathlineto{\pgfqpoint{2.744951in}{1.528502in}}%
\pgfpathlineto{\pgfqpoint{2.750598in}{1.405785in}}%
\pgfpathlineto{\pgfqpoint{2.754833in}{1.231806in}}%
\pgfpathlineto{\pgfqpoint{2.761892in}{0.890771in}}%
\pgfpathlineto{\pgfqpoint{2.763304in}{0.885625in}}%
\pgfpathlineto{\pgfqpoint{2.764716in}{0.889194in}}%
\pgfpathlineto{\pgfqpoint{2.768951in}{0.916764in}}%
\pgfpathlineto{\pgfqpoint{2.770363in}{0.918559in}}%
\pgfpathlineto{\pgfqpoint{2.771774in}{0.914584in}}%
\pgfpathlineto{\pgfqpoint{2.773186in}{0.903957in}}%
\pgfpathlineto{\pgfqpoint{2.776010in}{1.224960in}}%
\pgfpathlineto{\pgfqpoint{2.781657in}{1.515065in}}%
\pgfpathlineto{\pgfqpoint{2.784480in}{1.601628in}}%
\pgfpathlineto{\pgfqpoint{2.785892in}{0.807174in}}%
\pgfpathlineto{\pgfqpoint{2.790127in}{0.764928in}}%
\pgfpathlineto{\pgfqpoint{2.791539in}{0.754094in}}%
\pgfpathlineto{\pgfqpoint{2.792950in}{0.752218in}}%
\pgfpathlineto{\pgfqpoint{2.802833in}{0.783547in}}%
\pgfpathlineto{\pgfqpoint{2.807068in}{0.798761in}}%
\pgfpathlineto{\pgfqpoint{2.812715in}{0.805467in}}%
\pgfpathlineto{\pgfqpoint{2.815538in}{0.807682in}}%
\pgfpathlineto{\pgfqpoint{2.816950in}{0.806766in}}%
\pgfpathlineto{\pgfqpoint{2.818362in}{0.809052in}}%
\pgfpathlineto{\pgfqpoint{2.819773in}{0.807692in}}%
\pgfpathlineto{\pgfqpoint{2.821185in}{0.809735in}}%
\pgfpathlineto{\pgfqpoint{2.822597in}{0.809644in}}%
\pgfpathlineto{\pgfqpoint{2.824009in}{0.807209in}}%
\pgfpathlineto{\pgfqpoint{2.825420in}{0.808575in}}%
\pgfpathlineto{\pgfqpoint{2.828244in}{0.804955in}}%
\pgfpathlineto{\pgfqpoint{2.831067in}{0.803737in}}%
\pgfpathlineto{\pgfqpoint{2.835303in}{0.801730in}}%
\pgfpathlineto{\pgfqpoint{2.843773in}{0.803489in}}%
\pgfpathlineto{\pgfqpoint{2.845185in}{0.805809in}}%
\pgfpathlineto{\pgfqpoint{2.848008in}{0.803253in}}%
\pgfpathlineto{\pgfqpoint{2.852243in}{0.801436in}}%
\pgfpathlineto{\pgfqpoint{2.873419in}{0.803871in}}%
\pgfpathlineto{\pgfqpoint{2.876243in}{0.807877in}}%
\pgfpathlineto{\pgfqpoint{2.880478in}{0.811090in}}%
\pgfpathlineto{\pgfqpoint{2.890360in}{0.813587in}}%
\pgfpathlineto{\pgfqpoint{2.898831in}{0.816998in}}%
\pgfpathlineto{\pgfqpoint{2.908713in}{0.822261in}}%
\pgfpathlineto{\pgfqpoint{2.910125in}{0.822710in}}%
\pgfpathlineto{\pgfqpoint{2.914360in}{0.826546in}}%
\pgfpathlineto{\pgfqpoint{2.939771in}{0.845813in}}%
\pgfpathlineto{\pgfqpoint{2.944006in}{0.846392in}}%
\pgfpathlineto{\pgfqpoint{2.946830in}{0.844924in}}%
\pgfpathlineto{\pgfqpoint{2.955300in}{0.839796in}}%
\pgfpathlineto{\pgfqpoint{2.959535in}{0.834242in}}%
\pgfpathlineto{\pgfqpoint{2.968006in}{0.818337in}}%
\pgfpathlineto{\pgfqpoint{2.973653in}{0.815113in}}%
\pgfpathlineto{\pgfqpoint{2.975065in}{0.815027in}}%
\pgfpathlineto{\pgfqpoint{2.983535in}{0.821495in}}%
\pgfpathlineto{\pgfqpoint{2.984947in}{0.822090in}}%
\pgfpathlineto{\pgfqpoint{2.987770in}{0.824264in}}%
\pgfpathlineto{\pgfqpoint{2.993417in}{0.828199in}}%
\pgfpathlineto{\pgfqpoint{2.994829in}{0.827891in}}%
\pgfpathlineto{\pgfqpoint{2.997652in}{0.829256in}}%
\pgfpathlineto{\pgfqpoint{3.016005in}{0.832306in}}%
\pgfpathlineto{\pgfqpoint{3.020240in}{0.835187in}}%
\pgfpathlineto{\pgfqpoint{3.025887in}{0.839523in}}%
\pgfpathlineto{\pgfqpoint{3.035769in}{0.850627in}}%
\pgfpathlineto{\pgfqpoint{3.041416in}{0.854036in}}%
\pgfpathlineto{\pgfqpoint{3.044240in}{0.853469in}}%
\pgfpathlineto{\pgfqpoint{3.051298in}{0.847161in}}%
\pgfpathlineto{\pgfqpoint{3.062592in}{0.833532in}}%
\pgfpathlineto{\pgfqpoint{3.071063in}{0.827129in}}%
\pgfpathlineto{\pgfqpoint{3.072475in}{0.826483in}}%
\pgfpathlineto{\pgfqpoint{3.073886in}{0.823192in}}%
\pgfpathlineto{\pgfqpoint{3.078121in}{0.823428in}}%
\pgfpathlineto{\pgfqpoint{3.079533in}{0.826317in}}%
\pgfpathlineto{\pgfqpoint{3.080945in}{0.826046in}}%
\pgfpathlineto{\pgfqpoint{3.082357in}{0.827124in}}%
\pgfpathlineto{\pgfqpoint{3.088004in}{0.824628in}}%
\pgfpathlineto{\pgfqpoint{3.089415in}{0.824478in}}%
\pgfpathlineto{\pgfqpoint{3.090827in}{0.822686in}}%
\pgfpathlineto{\pgfqpoint{3.093651in}{0.821935in}}%
\pgfpathlineto{\pgfqpoint{3.102121in}{0.816523in}}%
\pgfpathlineto{\pgfqpoint{3.112003in}{0.811359in}}%
\pgfpathlineto{\pgfqpoint{3.124709in}{0.826677in}}%
\pgfpathlineto{\pgfqpoint{3.130356in}{0.835165in}}%
\pgfpathlineto{\pgfqpoint{3.137414in}{0.839582in}}%
\pgfpathlineto{\pgfqpoint{3.147297in}{0.832305in}}%
\pgfpathlineto{\pgfqpoint{3.160002in}{0.805548in}}%
\pgfpathlineto{\pgfqpoint{3.168473in}{0.799118in}}%
\pgfpathlineto{\pgfqpoint{3.169884in}{0.798614in}}%
\pgfpathlineto{\pgfqpoint{3.171296in}{0.799344in}}%
\pgfpathlineto{\pgfqpoint{3.174120in}{0.799357in}}%
\pgfpathlineto{\pgfqpoint{3.178355in}{0.800153in}}%
\pgfpathlineto{\pgfqpoint{3.188237in}{0.788473in}}%
\pgfpathlineto{\pgfqpoint{3.193884in}{0.778450in}}%
\pgfpathlineto{\pgfqpoint{3.195296in}{0.777838in}}%
\pgfpathlineto{\pgfqpoint{3.198119in}{0.774165in}}%
\pgfpathlineto{\pgfqpoint{3.200943in}{0.771929in}}%
\pgfpathlineto{\pgfqpoint{3.202354in}{0.772616in}}%
\pgfpathlineto{\pgfqpoint{3.203766in}{0.771452in}}%
\pgfpathlineto{\pgfqpoint{3.210825in}{0.775520in}}%
\pgfpathlineto{\pgfqpoint{3.229177in}{0.816091in}}%
\pgfpathlineto{\pgfqpoint{3.237648in}{0.828320in}}%
\pgfpathlineto{\pgfqpoint{3.244707in}{0.832487in}}%
\pgfpathlineto{\pgfqpoint{3.247530in}{0.832930in}}%
\pgfpathlineto{\pgfqpoint{3.248942in}{0.832540in}}%
\pgfpathlineto{\pgfqpoint{3.250354in}{0.833359in}}%
\pgfpathlineto{\pgfqpoint{3.256000in}{0.831411in}}%
\pgfpathlineto{\pgfqpoint{3.272941in}{0.828197in}}%
\pgfpathlineto{\pgfqpoint{3.274353in}{0.826507in}}%
\pgfpathlineto{\pgfqpoint{3.277177in}{0.828522in}}%
\pgfpathlineto{\pgfqpoint{3.284235in}{0.822981in}}%
\pgfpathlineto{\pgfqpoint{3.288470in}{0.817202in}}%
\pgfpathlineto{\pgfqpoint{3.292706in}{0.812230in}}%
\pgfpathlineto{\pgfqpoint{3.294117in}{0.812198in}}%
\pgfpathlineto{\pgfqpoint{3.296941in}{0.809798in}}%
\pgfpathlineto{\pgfqpoint{3.298353in}{0.810276in}}%
\pgfpathlineto{\pgfqpoint{3.299764in}{0.809469in}}%
\pgfpathlineto{\pgfqpoint{3.302588in}{0.811609in}}%
\pgfpathlineto{\pgfqpoint{3.315293in}{0.824160in}}%
\pgfpathlineto{\pgfqpoint{3.316705in}{0.823724in}}%
\pgfpathlineto{\pgfqpoint{3.319529in}{0.825252in}}%
\pgfpathlineto{\pgfqpoint{3.325176in}{0.826954in}}%
\pgfpathlineto{\pgfqpoint{3.327999in}{0.826393in}}%
\pgfpathlineto{\pgfqpoint{3.337881in}{0.828480in}}%
\pgfpathlineto{\pgfqpoint{3.342116in}{0.831251in}}%
\pgfpathlineto{\pgfqpoint{3.353410in}{0.834404in}}%
\pgfpathlineto{\pgfqpoint{3.356234in}{0.832222in}}%
\pgfpathlineto{\pgfqpoint{3.363293in}{0.821618in}}%
\pgfpathlineto{\pgfqpoint{3.366116in}{0.820058in}}%
\pgfpathlineto{\pgfqpoint{3.368939in}{0.818976in}}%
\pgfpathlineto{\pgfqpoint{3.371763in}{0.818363in}}%
\pgfpathlineto{\pgfqpoint{3.374586in}{0.815168in}}%
\pgfpathlineto{\pgfqpoint{3.377410in}{0.814062in}}%
\pgfpathlineto{\pgfqpoint{3.380233in}{0.815842in}}%
\pgfpathlineto{\pgfqpoint{3.381645in}{0.814522in}}%
\pgfpathlineto{\pgfqpoint{3.383057in}{0.814766in}}%
\pgfpathlineto{\pgfqpoint{3.395763in}{0.802799in}}%
\pgfpathlineto{\pgfqpoint{3.397174in}{0.803939in}}%
\pgfpathlineto{\pgfqpoint{3.398586in}{0.803513in}}%
\pgfpathlineto{\pgfqpoint{3.404233in}{0.809961in}}%
\pgfpathlineto{\pgfqpoint{3.405645in}{0.809853in}}%
\pgfpathlineto{\pgfqpoint{3.407056in}{0.811498in}}%
\pgfpathlineto{\pgfqpoint{3.412703in}{0.809914in}}%
\pgfpathlineto{\pgfqpoint{3.422586in}{0.822030in}}%
\pgfpathlineto{\pgfqpoint{3.425409in}{0.827473in}}%
\pgfpathlineto{\pgfqpoint{3.429644in}{0.833366in}}%
\pgfpathlineto{\pgfqpoint{3.432468in}{0.838121in}}%
\pgfpathlineto{\pgfqpoint{3.435291in}{0.840609in}}%
\pgfpathlineto{\pgfqpoint{3.440938in}{0.837949in}}%
\pgfpathlineto{\pgfqpoint{3.447997in}{0.827254in}}%
\pgfpathlineto{\pgfqpoint{3.453644in}{0.824586in}}%
\pgfpathlineto{\pgfqpoint{3.457879in}{0.823592in}}%
\pgfpathlineto{\pgfqpoint{3.467761in}{0.819953in}}%
\pgfpathlineto{\pgfqpoint{3.480467in}{0.821423in}}%
\pgfpathlineto{\pgfqpoint{3.483290in}{0.817460in}}%
\pgfpathlineto{\pgfqpoint{3.487525in}{0.811484in}}%
\pgfpathlineto{\pgfqpoint{3.491761in}{0.808556in}}%
\pgfpathlineto{\pgfqpoint{3.498819in}{0.816328in}}%
\pgfpathlineto{\pgfqpoint{3.500231in}{0.816616in}}%
\pgfpathlineto{\pgfqpoint{3.507290in}{0.803957in}}%
\pgfpathlineto{\pgfqpoint{3.510113in}{0.798923in}}%
\pgfpathlineto{\pgfqpoint{3.511525in}{0.796591in}}%
\pgfpathlineto{\pgfqpoint{3.515760in}{0.800563in}}%
\pgfpathlineto{\pgfqpoint{3.527054in}{0.824294in}}%
\pgfpathlineto{\pgfqpoint{3.531289in}{0.828116in}}%
\pgfpathlineto{\pgfqpoint{3.534113in}{0.826222in}}%
\pgfpathlineto{\pgfqpoint{3.539760in}{0.815885in}}%
\pgfpathlineto{\pgfqpoint{3.542583in}{0.814126in}}%
\pgfpathlineto{\pgfqpoint{3.548230in}{0.815630in}}%
\pgfpathlineto{\pgfqpoint{3.552465in}{0.815687in}}%
\pgfpathlineto{\pgfqpoint{3.555289in}{0.818342in}}%
\pgfpathlineto{\pgfqpoint{3.562348in}{0.830964in}}%
\pgfpathlineto{\pgfqpoint{3.563759in}{0.830513in}}%
\pgfpathlineto{\pgfqpoint{3.566583in}{0.823408in}}%
\pgfpathlineto{\pgfqpoint{3.577877in}{0.784943in}}%
\pgfpathlineto{\pgfqpoint{3.579288in}{0.784944in}}%
\pgfpathlineto{\pgfqpoint{3.589171in}{0.796342in}}%
\pgfpathlineto{\pgfqpoint{3.597641in}{0.783179in}}%
\pgfpathlineto{\pgfqpoint{3.603288in}{0.788105in}}%
\pgfpathlineto{\pgfqpoint{3.607523in}{0.795386in}}%
\pgfpathlineto{\pgfqpoint{3.610347in}{0.801092in}}%
\pgfpathlineto{\pgfqpoint{3.613170in}{0.804661in}}%
\pgfpathlineto{\pgfqpoint{3.614582in}{0.804801in}}%
\pgfpathlineto{\pgfqpoint{3.615994in}{0.806315in}}%
\pgfpathlineto{\pgfqpoint{3.617405in}{0.805908in}}%
\pgfpathlineto{\pgfqpoint{3.620229in}{0.806803in}}%
\pgfpathlineto{\pgfqpoint{3.623052in}{0.802501in}}%
\pgfpathlineto{\pgfqpoint{3.624464in}{0.801974in}}%
\pgfpathlineto{\pgfqpoint{3.630111in}{0.794542in}}%
\pgfpathlineto{\pgfqpoint{3.631523in}{0.793564in}}%
\pgfpathlineto{\pgfqpoint{3.634346in}{0.796364in}}%
\pgfpathlineto{\pgfqpoint{3.649875in}{0.830560in}}%
\pgfpathlineto{\pgfqpoint{3.654111in}{0.831285in}}%
\pgfpathlineto{\pgfqpoint{3.663993in}{0.824611in}}%
\pgfpathlineto{\pgfqpoint{3.668228in}{0.814108in}}%
\pgfpathlineto{\pgfqpoint{3.675287in}{0.794584in}}%
\pgfpathlineto{\pgfqpoint{3.678110in}{0.791924in}}%
\pgfpathlineto{\pgfqpoint{3.680934in}{0.789681in}}%
\pgfpathlineto{\pgfqpoint{3.682345in}{0.791584in}}%
\pgfpathlineto{\pgfqpoint{3.685169in}{0.790610in}}%
\pgfpathlineto{\pgfqpoint{3.693639in}{0.781298in}}%
\pgfpathlineto{\pgfqpoint{3.699286in}{0.782454in}}%
\pgfpathlineto{\pgfqpoint{3.700698in}{0.784203in}}%
\pgfpathlineto{\pgfqpoint{3.702110in}{0.783718in}}%
\pgfpathlineto{\pgfqpoint{3.704933in}{0.786289in}}%
\pgfpathlineto{\pgfqpoint{3.707757in}{0.788107in}}%
\pgfpathlineto{\pgfqpoint{3.711992in}{0.785233in}}%
\pgfpathlineto{\pgfqpoint{3.714815in}{0.783059in}}%
\pgfpathlineto{\pgfqpoint{3.717639in}{0.781999in}}%
\pgfpathlineto{\pgfqpoint{3.726109in}{0.789460in}}%
\pgfpathlineto{\pgfqpoint{3.730344in}{0.795762in}}%
\pgfpathlineto{\pgfqpoint{3.734580in}{0.800588in}}%
\pgfpathlineto{\pgfqpoint{3.745874in}{0.815240in}}%
\pgfpathlineto{\pgfqpoint{3.748697in}{0.814851in}}%
\pgfpathlineto{\pgfqpoint{3.759991in}{0.808022in}}%
\pgfpathlineto{\pgfqpoint{3.764226in}{0.804765in}}%
\pgfpathlineto{\pgfqpoint{3.775520in}{0.798507in}}%
\pgfpathlineto{\pgfqpoint{3.786814in}{0.788670in}}%
\pgfpathlineto{\pgfqpoint{3.789637in}{0.788199in}}%
\pgfpathlineto{\pgfqpoint{3.798108in}{0.791316in}}%
\pgfpathlineto{\pgfqpoint{3.802343in}{0.791311in}}%
\pgfpathlineto{\pgfqpoint{3.805167in}{0.791917in}}%
\pgfpathlineto{\pgfqpoint{3.809402in}{0.792391in}}%
\pgfpathlineto{\pgfqpoint{3.812225in}{0.794136in}}%
\pgfpathlineto{\pgfqpoint{3.816460in}{0.797485in}}%
\pgfpathlineto{\pgfqpoint{3.819284in}{0.800194in}}%
\pgfpathlineto{\pgfqpoint{3.824931in}{0.805550in}}%
\pgfpathlineto{\pgfqpoint{3.826343in}{0.805794in}}%
\pgfpathlineto{\pgfqpoint{3.836225in}{0.818231in}}%
\pgfpathlineto{\pgfqpoint{3.848930in}{0.835708in}}%
\pgfpathlineto{\pgfqpoint{3.855989in}{0.839665in}}%
\pgfpathlineto{\pgfqpoint{3.863048in}{0.838841in}}%
\pgfpathlineto{\pgfqpoint{3.870106in}{0.833452in}}%
\pgfpathlineto{\pgfqpoint{3.887047in}{0.810344in}}%
\pgfpathlineto{\pgfqpoint{3.888459in}{0.809317in}}%
\pgfpathlineto{\pgfqpoint{3.889871in}{0.811145in}}%
\pgfpathlineto{\pgfqpoint{3.892694in}{0.809853in}}%
\pgfpathlineto{\pgfqpoint{3.896929in}{0.809389in}}%
\pgfpathlineto{\pgfqpoint{3.898341in}{0.807923in}}%
\pgfpathlineto{\pgfqpoint{3.903988in}{0.808113in}}%
\pgfpathlineto{\pgfqpoint{3.905400in}{0.807072in}}%
\pgfpathlineto{\pgfqpoint{3.906812in}{0.807581in}}%
\pgfpathlineto{\pgfqpoint{3.909635in}{0.806812in}}%
\pgfpathlineto{\pgfqpoint{3.919517in}{0.810846in}}%
\pgfpathlineto{\pgfqpoint{3.922341in}{0.813859in}}%
\pgfpathlineto{\pgfqpoint{3.925164in}{0.816437in}}%
\pgfpathlineto{\pgfqpoint{3.929399in}{0.818916in}}%
\pgfpathlineto{\pgfqpoint{3.932223in}{0.821384in}}%
\pgfpathlineto{\pgfqpoint{3.940693in}{0.826636in}}%
\pgfpathlineto{\pgfqpoint{3.951987in}{0.830961in}}%
\pgfpathlineto{\pgfqpoint{3.956222in}{0.831315in}}%
\pgfpathlineto{\pgfqpoint{3.966105in}{0.833776in}}%
\pgfpathlineto{\pgfqpoint{3.973163in}{0.832680in}}%
\pgfpathlineto{\pgfqpoint{3.980222in}{0.826023in}}%
\pgfpathlineto{\pgfqpoint{3.991516in}{0.811848in}}%
\pgfpathlineto{\pgfqpoint{3.997163in}{0.810472in}}%
\pgfpathlineto{\pgfqpoint{4.022574in}{0.811682in}}%
\pgfpathlineto{\pgfqpoint{4.023986in}{0.812698in}}%
\pgfpathlineto{\pgfqpoint{4.025398in}{0.812307in}}%
\pgfpathlineto{\pgfqpoint{4.026809in}{0.813744in}}%
\pgfpathlineto{\pgfqpoint{4.028221in}{0.813305in}}%
\pgfpathlineto{\pgfqpoint{4.033868in}{0.816821in}}%
\pgfpathlineto{\pgfqpoint{4.035280in}{0.816499in}}%
\pgfpathlineto{\pgfqpoint{4.057868in}{0.838738in}}%
\pgfpathlineto{\pgfqpoint{4.073397in}{0.840042in}}%
\pgfpathlineto{\pgfqpoint{4.077632in}{0.837277in}}%
\pgfpathlineto{\pgfqpoint{4.081867in}{0.832988in}}%
\pgfpathlineto{\pgfqpoint{4.090338in}{0.817929in}}%
\pgfpathlineto{\pgfqpoint{4.093161in}{0.809644in}}%
\pgfpathlineto{\pgfqpoint{4.100220in}{0.800997in}}%
\pgfpathlineto{\pgfqpoint{4.101631in}{0.802669in}}%
\pgfpathlineto{\pgfqpoint{4.104455in}{0.800219in}}%
\pgfpathlineto{\pgfqpoint{4.118572in}{0.799716in}}%
\pgfpathlineto{\pgfqpoint{4.124219in}{0.798024in}}%
\pgfpathlineto{\pgfqpoint{4.135513in}{0.811535in}}%
\pgfpathlineto{\pgfqpoint{4.139748in}{0.818357in}}%
\pgfpathlineto{\pgfqpoint{4.142572in}{0.820492in}}%
\pgfpathlineto{\pgfqpoint{4.146807in}{0.823266in}}%
\pgfpathlineto{\pgfqpoint{4.148219in}{0.822910in}}%
\pgfpathlineto{\pgfqpoint{4.151042in}{0.823493in}}%
\pgfpathlineto{\pgfqpoint{4.153866in}{0.823909in}}%
\pgfpathlineto{\pgfqpoint{4.155278in}{0.824122in}}%
\pgfpathlineto{\pgfqpoint{4.158101in}{0.826580in}}%
\pgfpathlineto{\pgfqpoint{4.170807in}{0.832846in}}%
\pgfpathlineto{\pgfqpoint{4.173630in}{0.832135in}}%
\pgfpathlineto{\pgfqpoint{4.180689in}{0.830405in}}%
\pgfpathlineto{\pgfqpoint{4.183512in}{0.829069in}}%
\pgfpathlineto{\pgfqpoint{4.194806in}{0.829227in}}%
\pgfpathlineto{\pgfqpoint{4.200453in}{0.825415in}}%
\pgfpathlineto{\pgfqpoint{4.207512in}{0.826360in}}%
\pgfpathlineto{\pgfqpoint{4.214571in}{0.828243in}}%
\pgfpathlineto{\pgfqpoint{4.217394in}{0.827905in}}%
\pgfpathlineto{\pgfqpoint{4.218806in}{0.828697in}}%
\pgfpathlineto{\pgfqpoint{4.220217in}{0.827724in}}%
\pgfpathlineto{\pgfqpoint{4.221629in}{0.828376in}}%
\pgfpathlineto{\pgfqpoint{4.224453in}{0.827885in}}%
\pgfpathlineto{\pgfqpoint{4.228688in}{0.830130in}}%
\pgfpathlineto{\pgfqpoint{4.231511in}{0.829673in}}%
\pgfpathlineto{\pgfqpoint{4.235747in}{0.829769in}}%
\pgfpathlineto{\pgfqpoint{4.241394in}{0.822988in}}%
\pgfpathlineto{\pgfqpoint{4.242805in}{0.822492in}}%
\pgfpathlineto{\pgfqpoint{4.251276in}{0.813028in}}%
\pgfpathlineto{\pgfqpoint{4.254099in}{0.814301in}}%
\pgfpathlineto{\pgfqpoint{4.258334in}{0.824540in}}%
\pgfpathlineto{\pgfqpoint{4.262570in}{0.835117in}}%
\pgfpathlineto{\pgfqpoint{4.265393in}{0.836455in}}%
\pgfpathlineto{\pgfqpoint{4.269628in}{0.836463in}}%
\pgfpathlineto{\pgfqpoint{4.273863in}{0.836577in}}%
\pgfpathlineto{\pgfqpoint{4.276687in}{0.837098in}}%
\pgfpathlineto{\pgfqpoint{4.280922in}{0.832778in}}%
\pgfpathlineto{\pgfqpoint{4.285157in}{0.829457in}}%
\pgfpathlineto{\pgfqpoint{4.287981in}{0.827612in}}%
\pgfpathlineto{\pgfqpoint{4.303510in}{0.812344in}}%
\pgfpathlineto{\pgfqpoint{4.310569in}{0.805457in}}%
\pgfpathlineto{\pgfqpoint{4.314804in}{0.802612in}}%
\pgfpathlineto{\pgfqpoint{4.316216in}{0.801441in}}%
\pgfpathlineto{\pgfqpoint{4.317627in}{0.801764in}}%
\pgfpathlineto{\pgfqpoint{4.320451in}{0.800649in}}%
\pgfpathlineto{\pgfqpoint{4.326098in}{0.799338in}}%
\pgfpathlineto{\pgfqpoint{4.330333in}{0.800560in}}%
\pgfpathlineto{\pgfqpoint{4.331745in}{0.801194in}}%
\pgfpathlineto{\pgfqpoint{4.333156in}{0.800305in}}%
\pgfpathlineto{\pgfqpoint{4.335980in}{0.800744in}}%
\pgfpathlineto{\pgfqpoint{4.337392in}{0.800696in}}%
\pgfpathlineto{\pgfqpoint{4.345862in}{0.806924in}}%
\pgfpathlineto{\pgfqpoint{4.347274in}{0.806109in}}%
\pgfpathlineto{\pgfqpoint{4.352921in}{0.811477in}}%
\pgfpathlineto{\pgfqpoint{4.362803in}{0.826039in}}%
\pgfpathlineto{\pgfqpoint{4.367038in}{0.827164in}}%
\pgfpathlineto{\pgfqpoint{4.372685in}{0.835559in}}%
\pgfpathlineto{\pgfqpoint{4.382567in}{0.835903in}}%
\pgfpathlineto{\pgfqpoint{4.400920in}{0.817671in}}%
\pgfpathlineto{\pgfqpoint{4.410802in}{0.804523in}}%
\pgfpathlineto{\pgfqpoint{4.415037in}{0.801493in}}%
\pgfpathlineto{\pgfqpoint{4.426331in}{0.800465in}}%
\pgfpathlineto{\pgfqpoint{4.427743in}{0.801141in}}%
\pgfpathlineto{\pgfqpoint{4.430566in}{0.800977in}}%
\pgfpathlineto{\pgfqpoint{4.433390in}{0.801172in}}%
\pgfpathlineto{\pgfqpoint{4.444684in}{0.811740in}}%
\pgfpathlineto{\pgfqpoint{4.464448in}{0.816413in}}%
\pgfpathlineto{\pgfqpoint{4.472919in}{0.818094in}}%
\pgfpathlineto{\pgfqpoint{4.477154in}{0.816448in}}%
\pgfpathlineto{\pgfqpoint{4.482801in}{0.813566in}}%
\pgfpathlineto{\pgfqpoint{4.491271in}{0.806424in}}%
\pgfpathlineto{\pgfqpoint{4.496918in}{0.802794in}}%
\pgfpathlineto{\pgfqpoint{4.509624in}{0.799910in}}%
\pgfpathlineto{\pgfqpoint{4.513859in}{0.798489in}}%
\pgfpathlineto{\pgfqpoint{4.526565in}{0.798456in}}%
\pgfpathlineto{\pgfqpoint{4.551976in}{0.787264in}}%
\pgfpathlineto{\pgfqpoint{4.557623in}{0.785182in}}%
\pgfpathlineto{\pgfqpoint{4.566093in}{0.778294in}}%
\pgfpathlineto{\pgfqpoint{4.580211in}{0.770107in}}%
\pgfpathlineto{\pgfqpoint{4.594328in}{0.761006in}}%
\pgfpathlineto{\pgfqpoint{4.619739in}{0.747525in}}%
\pgfpathlineto{\pgfqpoint{4.631033in}{0.743847in}}%
\pgfpathlineto{\pgfqpoint{4.647974in}{0.739307in}}%
\pgfpathlineto{\pgfqpoint{4.663503in}{0.734730in}}%
\pgfpathlineto{\pgfqpoint{4.687503in}{0.728976in}}%
\pgfpathlineto{\pgfqpoint{4.695973in}{0.726798in}}%
\pgfpathlineto{\pgfqpoint{4.704444in}{0.724413in}}%
\pgfpathlineto{\pgfqpoint{4.714326in}{0.722158in}}%
\pgfpathlineto{\pgfqpoint{4.732678in}{0.718778in}}%
\pgfpathlineto{\pgfqpoint{4.741149in}{0.716843in}}%
\pgfpathlineto{\pgfqpoint{4.749619in}{0.715083in}}%
\pgfpathlineto{\pgfqpoint{4.765148in}{0.712884in}}%
\pgfpathlineto{\pgfqpoint{4.777854in}{0.710065in}}%
\pgfpathlineto{\pgfqpoint{4.818794in}{0.704538in}}%
\pgfpathlineto{\pgfqpoint{4.820206in}{0.722991in}}%
\pgfpathlineto{\pgfqpoint{4.823030in}{0.721763in}}%
\pgfpathlineto{\pgfqpoint{4.824441in}{0.703918in}}%
\pgfpathlineto{\pgfqpoint{4.837147in}{0.702684in}}%
\pgfpathlineto{\pgfqpoint{4.838559in}{0.717321in}}%
\pgfpathlineto{\pgfqpoint{4.852676in}{0.714015in}}%
\pgfpathlineto{\pgfqpoint{4.854088in}{0.737238in}}%
\pgfpathlineto{\pgfqpoint{4.855500in}{0.713520in}}%
\pgfpathlineto{\pgfqpoint{4.861146in}{0.712574in}}%
\pgfpathlineto{\pgfqpoint{4.862558in}{0.733999in}}%
\pgfpathlineto{\pgfqpoint{4.865382in}{0.736401in}}%
\pgfpathlineto{\pgfqpoint{4.866793in}{0.736990in}}%
\pgfpathlineto{\pgfqpoint{4.868205in}{0.712124in}}%
\pgfpathlineto{\pgfqpoint{4.880911in}{0.711977in}}%
\pgfpathlineto{\pgfqpoint{4.882323in}{0.773594in}}%
\pgfpathlineto{\pgfqpoint{4.883734in}{0.772701in}}%
\pgfpathlineto{\pgfqpoint{4.885146in}{0.770270in}}%
\pgfpathlineto{\pgfqpoint{4.886558in}{0.712001in}}%
\pgfpathlineto{\pgfqpoint{4.895028in}{0.711833in}}%
\pgfpathlineto{\pgfqpoint{4.896440in}{0.701580in}}%
\pgfpathlineto{\pgfqpoint{4.913381in}{0.701491in}}%
\pgfpathlineto{\pgfqpoint{4.916204in}{0.701726in}}%
\pgfpathlineto{\pgfqpoint{4.917616in}{0.708871in}}%
\pgfpathlineto{\pgfqpoint{4.920439in}{0.708435in}}%
\pgfpathlineto{\pgfqpoint{4.921851in}{0.755742in}}%
\pgfpathlineto{\pgfqpoint{4.923263in}{0.757772in}}%
\pgfpathlineto{\pgfqpoint{4.927498in}{0.743890in}}%
\pgfpathlineto{\pgfqpoint{4.930322in}{0.731931in}}%
\pgfpathlineto{\pgfqpoint{4.933145in}{0.718765in}}%
\pgfpathlineto{\pgfqpoint{4.934557in}{0.718393in}}%
\pgfpathlineto{\pgfqpoint{4.937380in}{0.722156in}}%
\pgfpathlineto{\pgfqpoint{4.938792in}{0.729563in}}%
\pgfpathlineto{\pgfqpoint{4.943027in}{0.766581in}}%
\pgfpathlineto{\pgfqpoint{4.944439in}{0.768666in}}%
\pgfpathlineto{\pgfqpoint{4.945851in}{0.767521in}}%
\pgfpathlineto{\pgfqpoint{4.948674in}{0.754644in}}%
\pgfpathlineto{\pgfqpoint{4.951498in}{0.745377in}}%
\pgfpathlineto{\pgfqpoint{4.952909in}{0.745310in}}%
\pgfpathlineto{\pgfqpoint{4.955733in}{0.749553in}}%
\pgfpathlineto{\pgfqpoint{4.957145in}{0.753335in}}%
\pgfpathlineto{\pgfqpoint{4.959968in}{0.770957in}}%
\pgfpathlineto{\pgfqpoint{4.961380in}{0.775415in}}%
\pgfpathlineto{\pgfqpoint{4.964203in}{0.772136in}}%
\pgfpathlineto{\pgfqpoint{4.965615in}{0.771672in}}%
\pgfpathlineto{\pgfqpoint{4.968439in}{0.782092in}}%
\pgfpathlineto{\pgfqpoint{4.969850in}{0.784776in}}%
\pgfpathlineto{\pgfqpoint{4.972674in}{0.798906in}}%
\pgfpathlineto{\pgfqpoint{4.974086in}{0.800231in}}%
\pgfpathlineto{\pgfqpoint{4.976909in}{0.798011in}}%
\pgfpathlineto{\pgfqpoint{4.979732in}{0.793542in}}%
\pgfpathlineto{\pgfqpoint{4.988203in}{0.766378in}}%
\pgfpathlineto{\pgfqpoint{4.995262in}{0.726404in}}%
\pgfpathlineto{\pgfqpoint{4.999497in}{0.720783in}}%
\pgfpathlineto{\pgfqpoint{5.005144in}{0.734222in}}%
\pgfpathlineto{\pgfqpoint{5.012202in}{0.769746in}}%
\pgfpathlineto{\pgfqpoint{5.013614in}{0.770267in}}%
\pgfpathlineto{\pgfqpoint{5.015026in}{0.768125in}}%
\pgfpathlineto{\pgfqpoint{5.017849in}{0.753157in}}%
\pgfpathlineto{\pgfqpoint{5.020673in}{0.738199in}}%
\pgfpathlineto{\pgfqpoint{5.026320in}{0.725160in}}%
\pgfpathlineto{\pgfqpoint{5.030555in}{0.729667in}}%
\pgfpathlineto{\pgfqpoint{5.031967in}{0.727078in}}%
\pgfpathlineto{\pgfqpoint{5.033378in}{0.700027in}}%
\pgfpathlineto{\pgfqpoint{5.040437in}{0.700445in}}%
\pgfpathlineto{\pgfqpoint{5.041849in}{0.696000in}}%
\pgfpathlineto{\pgfqpoint{5.496429in}{0.696000in}}%
\pgfpathlineto{\pgfqpoint{5.497840in}{0.752739in}}%
\pgfpathlineto{\pgfqpoint{5.500664in}{0.754394in}}%
\pgfpathlineto{\pgfqpoint{5.503487in}{0.752755in}}%
\pgfpathlineto{\pgfqpoint{5.509134in}{0.745847in}}%
\pgfpathlineto{\pgfqpoint{5.513369in}{0.739973in}}%
\pgfpathlineto{\pgfqpoint{5.516193in}{0.741061in}}%
\pgfpathlineto{\pgfqpoint{5.519016in}{0.743699in}}%
\pgfpathlineto{\pgfqpoint{5.526075in}{0.753735in}}%
\pgfpathlineto{\pgfqpoint{5.528899in}{0.754169in}}%
\pgfpathlineto{\pgfqpoint{5.533134in}{0.760233in}}%
\pgfpathlineto{\pgfqpoint{5.534545in}{0.696000in}}%
\pgfpathlineto{\pgfqpoint{5.534545in}{0.696000in}}%
\pgfusepath{stroke}%
\end{pgfscope}%
\begin{pgfscope}%
\pgfpathrectangle{\pgfqpoint{0.800000in}{0.528000in}}{\pgfqpoint{4.960000in}{3.696000in}} %
\pgfusepath{clip}%
\pgfsetrectcap%
\pgfsetroundjoin%
\pgfsetlinewidth{1.505625pt}%
\definecolor{currentstroke}{rgb}{1.000000,0.498039,0.054902}%
\pgfsetstrokecolor{currentstroke}%
\pgfsetdash{}{0pt}%
\pgfpathmoveto{\pgfqpoint{1.025455in}{0.696000in}}%
\pgfpathlineto{\pgfqpoint{1.122864in}{0.696000in}}%
\pgfpathlineto{\pgfqpoint{1.125688in}{0.698358in}}%
\pgfpathlineto{\pgfqpoint{1.132747in}{0.697706in}}%
\pgfpathlineto{\pgfqpoint{1.136982in}{0.698686in}}%
\pgfpathlineto{\pgfqpoint{1.141217in}{0.699154in}}%
\pgfpathlineto{\pgfqpoint{1.142629in}{0.699590in}}%
\pgfpathlineto{\pgfqpoint{1.144041in}{0.696611in}}%
\pgfpathlineto{\pgfqpoint{1.145452in}{0.699579in}}%
\pgfpathlineto{\pgfqpoint{1.149687in}{0.696780in}}%
\pgfpathlineto{\pgfqpoint{1.156746in}{0.697478in}}%
\pgfpathlineto{\pgfqpoint{1.160981in}{0.697138in}}%
\pgfpathlineto{\pgfqpoint{1.162393in}{0.699438in}}%
\pgfpathlineto{\pgfqpoint{1.163805in}{0.699573in}}%
\pgfpathlineto{\pgfqpoint{1.165217in}{0.696529in}}%
\pgfpathlineto{\pgfqpoint{1.166628in}{0.699648in}}%
\pgfpathlineto{\pgfqpoint{1.169452in}{0.699425in}}%
\pgfpathlineto{\pgfqpoint{1.172275in}{0.791076in}}%
\pgfpathlineto{\pgfqpoint{1.177922in}{0.879597in}}%
\pgfpathlineto{\pgfqpoint{1.180746in}{0.916229in}}%
\pgfpathlineto{\pgfqpoint{1.187804in}{0.996607in}}%
\pgfpathlineto{\pgfqpoint{1.192040in}{1.024423in}}%
\pgfpathlineto{\pgfqpoint{1.194863in}{1.034958in}}%
\pgfpathlineto{\pgfqpoint{1.203334in}{1.105936in}}%
\pgfpathlineto{\pgfqpoint{1.207569in}{1.132161in}}%
\pgfpathlineto{\pgfqpoint{1.225921in}{1.198165in}}%
\pgfpathlineto{\pgfqpoint{1.238627in}{1.252602in}}%
\pgfpathlineto{\pgfqpoint{1.241450in}{1.257284in}}%
\pgfpathlineto{\pgfqpoint{1.242862in}{1.258434in}}%
\pgfpathlineto{\pgfqpoint{1.244274in}{1.256854in}}%
\pgfpathlineto{\pgfqpoint{1.245686in}{1.257068in}}%
\pgfpathlineto{\pgfqpoint{1.248509in}{1.259680in}}%
\pgfpathlineto{\pgfqpoint{1.255568in}{1.264269in}}%
\pgfpathlineto{\pgfqpoint{1.261215in}{1.264456in}}%
\pgfpathlineto{\pgfqpoint{1.265450in}{1.260057in}}%
\pgfpathlineto{\pgfqpoint{1.279567in}{1.238934in}}%
\pgfpathlineto{\pgfqpoint{1.282391in}{1.240190in}}%
\pgfpathlineto{\pgfqpoint{1.292273in}{1.251274in}}%
\pgfpathlineto{\pgfqpoint{1.297920in}{1.249399in}}%
\pgfpathlineto{\pgfqpoint{1.300743in}{1.246654in}}%
\pgfpathlineto{\pgfqpoint{1.304979in}{1.239583in}}%
\pgfpathlineto{\pgfqpoint{1.317684in}{1.209608in}}%
\pgfpathlineto{\pgfqpoint{1.323331in}{1.201078in}}%
\pgfpathlineto{\pgfqpoint{1.327566in}{1.200023in}}%
\pgfpathlineto{\pgfqpoint{1.330390in}{1.201024in}}%
\pgfpathlineto{\pgfqpoint{1.333213in}{1.205117in}}%
\pgfpathlineto{\pgfqpoint{1.337449in}{1.217610in}}%
\pgfpathlineto{\pgfqpoint{1.345919in}{1.253006in}}%
\pgfpathlineto{\pgfqpoint{1.350154in}{1.275640in}}%
\pgfpathlineto{\pgfqpoint{1.352978in}{1.284246in}}%
\pgfpathlineto{\pgfqpoint{1.357213in}{1.288174in}}%
\pgfpathlineto{\pgfqpoint{1.358625in}{1.287535in}}%
\pgfpathlineto{\pgfqpoint{1.362860in}{1.278669in}}%
\pgfpathlineto{\pgfqpoint{1.374154in}{1.232971in}}%
\pgfpathlineto{\pgfqpoint{1.375566in}{1.232838in}}%
\pgfpathlineto{\pgfqpoint{1.378389in}{1.237877in}}%
\pgfpathlineto{\pgfqpoint{1.381213in}{1.245922in}}%
\pgfpathlineto{\pgfqpoint{1.392506in}{1.290925in}}%
\pgfpathlineto{\pgfqpoint{1.396742in}{1.299238in}}%
\pgfpathlineto{\pgfqpoint{1.399565in}{1.299493in}}%
\pgfpathlineto{\pgfqpoint{1.400977in}{1.299393in}}%
\pgfpathlineto{\pgfqpoint{1.402389in}{1.300432in}}%
\pgfpathlineto{\pgfqpoint{1.406624in}{1.307016in}}%
\pgfpathlineto{\pgfqpoint{1.415094in}{1.318090in}}%
\pgfpathlineto{\pgfqpoint{1.417918in}{1.317978in}}%
\pgfpathlineto{\pgfqpoint{1.420741in}{1.315792in}}%
\pgfpathlineto{\pgfqpoint{1.423565in}{1.313461in}}%
\pgfpathlineto{\pgfqpoint{1.426388in}{1.315233in}}%
\pgfpathlineto{\pgfqpoint{1.429212in}{1.320699in}}%
\pgfpathlineto{\pgfqpoint{1.433447in}{1.334684in}}%
\pgfpathlineto{\pgfqpoint{1.440505in}{1.365837in}}%
\pgfpathlineto{\pgfqpoint{1.448976in}{1.399734in}}%
\pgfpathlineto{\pgfqpoint{1.451799in}{1.433224in}}%
\pgfpathlineto{\pgfqpoint{1.454623in}{1.461689in}}%
\pgfpathlineto{\pgfqpoint{1.456035in}{1.464270in}}%
\pgfpathlineto{\pgfqpoint{1.457446in}{1.453567in}}%
\pgfpathlineto{\pgfqpoint{1.460270in}{1.403144in}}%
\pgfpathlineto{\pgfqpoint{1.467329in}{1.245757in}}%
\pgfpathlineto{\pgfqpoint{1.468740in}{1.235899in}}%
\pgfpathlineto{\pgfqpoint{1.470152in}{1.236000in}}%
\pgfpathlineto{\pgfqpoint{1.475799in}{1.254564in}}%
\pgfpathlineto{\pgfqpoint{1.477211in}{1.251627in}}%
\pgfpathlineto{\pgfqpoint{1.480034in}{1.229833in}}%
\pgfpathlineto{\pgfqpoint{1.484269in}{1.188081in}}%
\pgfpathlineto{\pgfqpoint{1.485681in}{1.186866in}}%
\pgfpathlineto{\pgfqpoint{1.487093in}{1.196293in}}%
\pgfpathlineto{\pgfqpoint{1.494152in}{1.282263in}}%
\pgfpathlineto{\pgfqpoint{1.496975in}{1.291402in}}%
\pgfpathlineto{\pgfqpoint{1.498387in}{1.290995in}}%
\pgfpathlineto{\pgfqpoint{1.502622in}{1.283540in}}%
\pgfpathlineto{\pgfqpoint{1.504034in}{1.288366in}}%
\pgfpathlineto{\pgfqpoint{1.508269in}{1.326696in}}%
\pgfpathlineto{\pgfqpoint{1.509681in}{1.331431in}}%
\pgfpathlineto{\pgfqpoint{1.511092in}{1.328548in}}%
\pgfpathlineto{\pgfqpoint{1.513916in}{1.310542in}}%
\pgfpathlineto{\pgfqpoint{1.522386in}{1.233856in}}%
\pgfpathlineto{\pgfqpoint{1.523798in}{1.239243in}}%
\pgfpathlineto{\pgfqpoint{1.526622in}{1.276396in}}%
\pgfpathlineto{\pgfqpoint{1.535092in}{1.433090in}}%
\pgfpathlineto{\pgfqpoint{1.537915in}{1.444952in}}%
\pgfpathlineto{\pgfqpoint{1.539327in}{1.439691in}}%
\pgfpathlineto{\pgfqpoint{1.543562in}{1.413561in}}%
\pgfpathlineto{\pgfqpoint{1.554856in}{1.362490in}}%
\pgfpathlineto{\pgfqpoint{1.559091in}{1.327931in}}%
\pgfpathlineto{\pgfqpoint{1.564738in}{1.270323in}}%
\pgfpathlineto{\pgfqpoint{1.567562in}{1.260058in}}%
\pgfpathlineto{\pgfqpoint{1.568974in}{1.263692in}}%
\pgfpathlineto{\pgfqpoint{1.571797in}{1.291536in}}%
\pgfpathlineto{\pgfqpoint{1.576032in}{1.377821in}}%
\pgfpathlineto{\pgfqpoint{1.583091in}{1.540781in}}%
\pgfpathlineto{\pgfqpoint{1.587326in}{1.600146in}}%
\pgfpathlineto{\pgfqpoint{1.590150in}{1.620789in}}%
\pgfpathlineto{\pgfqpoint{1.591561in}{1.625233in}}%
\pgfpathlineto{\pgfqpoint{1.592973in}{1.625793in}}%
\pgfpathlineto{\pgfqpoint{1.594385in}{1.623097in}}%
\pgfpathlineto{\pgfqpoint{1.597208in}{1.606918in}}%
\pgfpathlineto{\pgfqpoint{1.601444in}{1.559453in}}%
\pgfpathlineto{\pgfqpoint{1.611326in}{1.421079in}}%
\pgfpathlineto{\pgfqpoint{1.618384in}{1.358266in}}%
\pgfpathlineto{\pgfqpoint{1.624031in}{1.323732in}}%
\pgfpathlineto{\pgfqpoint{1.628267in}{1.309269in}}%
\pgfpathlineto{\pgfqpoint{1.631090in}{1.307036in}}%
\pgfpathlineto{\pgfqpoint{1.633914in}{1.310582in}}%
\pgfpathlineto{\pgfqpoint{1.636737in}{1.315510in}}%
\pgfpathlineto{\pgfqpoint{1.638149in}{1.315933in}}%
\pgfpathlineto{\pgfqpoint{1.643796in}{1.337251in}}%
\pgfpathlineto{\pgfqpoint{1.646619in}{1.326766in}}%
\pgfpathlineto{\pgfqpoint{1.657913in}{1.241039in}}%
\pgfpathlineto{\pgfqpoint{1.659325in}{1.243874in}}%
\pgfpathlineto{\pgfqpoint{1.662148in}{1.271036in}}%
\pgfpathlineto{\pgfqpoint{1.666384in}{1.350913in}}%
\pgfpathlineto{\pgfqpoint{1.677677in}{1.596634in}}%
\pgfpathlineto{\pgfqpoint{1.681913in}{1.633946in}}%
\pgfpathlineto{\pgfqpoint{1.683324in}{1.637511in}}%
\pgfpathlineto{\pgfqpoint{1.684736in}{1.636451in}}%
\pgfpathlineto{\pgfqpoint{1.687560in}{1.625844in}}%
\pgfpathlineto{\pgfqpoint{1.691795in}{1.591988in}}%
\pgfpathlineto{\pgfqpoint{1.697442in}{1.525980in}}%
\pgfpathlineto{\pgfqpoint{1.704500in}{1.455924in}}%
\pgfpathlineto{\pgfqpoint{1.707324in}{1.441342in}}%
\pgfpathlineto{\pgfqpoint{1.708736in}{1.444587in}}%
\pgfpathlineto{\pgfqpoint{1.718618in}{1.492147in}}%
\pgfpathlineto{\pgfqpoint{1.720030in}{1.492857in}}%
\pgfpathlineto{\pgfqpoint{1.721441in}{1.490888in}}%
\pgfpathlineto{\pgfqpoint{1.724265in}{1.481181in}}%
\pgfpathlineto{\pgfqpoint{1.729912in}{1.457203in}}%
\pgfpathlineto{\pgfqpoint{1.731324in}{1.457883in}}%
\pgfpathlineto{\pgfqpoint{1.735559in}{1.462163in}}%
\pgfpathlineto{\pgfqpoint{1.736970in}{1.458487in}}%
\pgfpathlineto{\pgfqpoint{1.739794in}{1.441172in}}%
\pgfpathlineto{\pgfqpoint{1.751088in}{1.331471in}}%
\pgfpathlineto{\pgfqpoint{1.753911in}{1.323194in}}%
\pgfpathlineto{\pgfqpoint{1.755323in}{1.324185in}}%
\pgfpathlineto{\pgfqpoint{1.756735in}{1.328530in}}%
\pgfpathlineto{\pgfqpoint{1.759558in}{1.349103in}}%
\pgfpathlineto{\pgfqpoint{1.765205in}{1.418101in}}%
\pgfpathlineto{\pgfqpoint{1.772264in}{1.496370in}}%
\pgfpathlineto{\pgfqpoint{1.775087in}{1.510638in}}%
\pgfpathlineto{\pgfqpoint{1.776499in}{1.512203in}}%
\pgfpathlineto{\pgfqpoint{1.777911in}{1.511620in}}%
\pgfpathlineto{\pgfqpoint{1.779323in}{1.508427in}}%
\pgfpathlineto{\pgfqpoint{1.782146in}{1.494302in}}%
\pgfpathlineto{\pgfqpoint{1.784970in}{1.471436in}}%
\pgfpathlineto{\pgfqpoint{1.796263in}{1.349857in}}%
\pgfpathlineto{\pgfqpoint{1.803322in}{1.328606in}}%
\pgfpathlineto{\pgfqpoint{1.806146in}{1.324454in}}%
\pgfpathlineto{\pgfqpoint{1.807557in}{1.327622in}}%
\pgfpathlineto{\pgfqpoint{1.818851in}{1.380818in}}%
\pgfpathlineto{\pgfqpoint{1.821675in}{1.383085in}}%
\pgfpathlineto{\pgfqpoint{1.824498in}{1.381774in}}%
\pgfpathlineto{\pgfqpoint{1.827322in}{1.377939in}}%
\pgfpathlineto{\pgfqpoint{1.831557in}{1.371612in}}%
\pgfpathlineto{\pgfqpoint{1.832969in}{1.369789in}}%
\pgfpathlineto{\pgfqpoint{1.835792in}{1.360783in}}%
\pgfpathlineto{\pgfqpoint{1.840027in}{1.339990in}}%
\pgfpathlineto{\pgfqpoint{1.842851in}{1.315244in}}%
\pgfpathlineto{\pgfqpoint{1.849909in}{1.247531in}}%
\pgfpathlineto{\pgfqpoint{1.852733in}{1.239230in}}%
\pgfpathlineto{\pgfqpoint{1.854145in}{1.241279in}}%
\pgfpathlineto{\pgfqpoint{1.856968in}{1.262103in}}%
\pgfpathlineto{\pgfqpoint{1.861203in}{1.328118in}}%
\pgfpathlineto{\pgfqpoint{1.872497in}{1.532210in}}%
\pgfpathlineto{\pgfqpoint{1.875321in}{1.554531in}}%
\pgfpathlineto{\pgfqpoint{1.878144in}{1.560985in}}%
\pgfpathlineto{\pgfqpoint{1.879556in}{1.560611in}}%
\pgfpathlineto{\pgfqpoint{1.882379in}{1.549668in}}%
\pgfpathlineto{\pgfqpoint{1.885203in}{1.527184in}}%
\pgfpathlineto{\pgfqpoint{1.893673in}{1.447584in}}%
\pgfpathlineto{\pgfqpoint{1.897909in}{1.442667in}}%
\pgfpathlineto{\pgfqpoint{1.899320in}{1.442871in}}%
\pgfpathlineto{\pgfqpoint{1.902144in}{1.447387in}}%
\pgfpathlineto{\pgfqpoint{1.904967in}{1.458566in}}%
\pgfpathlineto{\pgfqpoint{1.909202in}{1.488808in}}%
\pgfpathlineto{\pgfqpoint{1.921908in}{1.600546in}}%
\pgfpathlineto{\pgfqpoint{1.926143in}{1.621028in}}%
\pgfpathlineto{\pgfqpoint{1.927555in}{1.622005in}}%
\pgfpathlineto{\pgfqpoint{1.928967in}{1.619341in}}%
\pgfpathlineto{\pgfqpoint{1.931790in}{1.601409in}}%
\pgfpathlineto{\pgfqpoint{1.940261in}{1.523682in}}%
\pgfpathlineto{\pgfqpoint{1.955790in}{1.319673in}}%
\pgfpathlineto{\pgfqpoint{1.960025in}{1.303075in}}%
\pgfpathlineto{\pgfqpoint{1.961437in}{1.300972in}}%
\pgfpathlineto{\pgfqpoint{1.962849in}{1.302027in}}%
\pgfpathlineto{\pgfqpoint{1.965672in}{1.311276in}}%
\pgfpathlineto{\pgfqpoint{1.974142in}{1.350432in}}%
\pgfpathlineto{\pgfqpoint{1.975554in}{1.353243in}}%
\pgfpathlineto{\pgfqpoint{1.978378in}{1.348781in}}%
\pgfpathlineto{\pgfqpoint{1.981201in}{1.336885in}}%
\pgfpathlineto{\pgfqpoint{1.984025in}{1.317025in}}%
\pgfpathlineto{\pgfqpoint{1.989672in}{1.262260in}}%
\pgfpathlineto{\pgfqpoint{1.993907in}{1.244279in}}%
\pgfpathlineto{\pgfqpoint{1.995318in}{1.241951in}}%
\pgfpathlineto{\pgfqpoint{1.996730in}{1.242144in}}%
\pgfpathlineto{\pgfqpoint{1.999554in}{1.248381in}}%
\pgfpathlineto{\pgfqpoint{2.003789in}{1.264281in}}%
\pgfpathlineto{\pgfqpoint{2.010848in}{1.307638in}}%
\pgfpathlineto{\pgfqpoint{2.020730in}{1.381384in}}%
\pgfpathlineto{\pgfqpoint{2.024965in}{1.398564in}}%
\pgfpathlineto{\pgfqpoint{2.026377in}{1.399792in}}%
\pgfpathlineto{\pgfqpoint{2.027788in}{1.398148in}}%
\pgfpathlineto{\pgfqpoint{2.030612in}{1.402737in}}%
\pgfpathlineto{\pgfqpoint{2.032024in}{1.401788in}}%
\pgfpathlineto{\pgfqpoint{2.034847in}{1.390517in}}%
\pgfpathlineto{\pgfqpoint{2.037671in}{1.369129in}}%
\pgfpathlineto{\pgfqpoint{2.043318in}{1.282806in}}%
\pgfpathlineto{\pgfqpoint{2.047553in}{1.214940in}}%
\pgfpathlineto{\pgfqpoint{2.050376in}{1.196964in}}%
\pgfpathlineto{\pgfqpoint{2.051788in}{1.198778in}}%
\pgfpathlineto{\pgfqpoint{2.054611in}{1.221730in}}%
\pgfpathlineto{\pgfqpoint{2.058847in}{1.290467in}}%
\pgfpathlineto{\pgfqpoint{2.064494in}{1.385550in}}%
\pgfpathlineto{\pgfqpoint{2.068729in}{1.424189in}}%
\pgfpathlineto{\pgfqpoint{2.072964in}{1.437354in}}%
\pgfpathlineto{\pgfqpoint{2.074376in}{1.439009in}}%
\pgfpathlineto{\pgfqpoint{2.075788in}{1.438725in}}%
\pgfpathlineto{\pgfqpoint{2.077199in}{1.436244in}}%
\pgfpathlineto{\pgfqpoint{2.080023in}{1.422028in}}%
\pgfpathlineto{\pgfqpoint{2.082846in}{1.392732in}}%
\pgfpathlineto{\pgfqpoint{2.087081in}{1.347836in}}%
\pgfpathlineto{\pgfqpoint{2.092728in}{1.307718in}}%
\pgfpathlineto{\pgfqpoint{2.095552in}{1.298125in}}%
\pgfpathlineto{\pgfqpoint{2.098375in}{1.297244in}}%
\pgfpathlineto{\pgfqpoint{2.102611in}{1.302277in}}%
\pgfpathlineto{\pgfqpoint{2.106846in}{1.311805in}}%
\pgfpathlineto{\pgfqpoint{2.109669in}{1.324244in}}%
\pgfpathlineto{\pgfqpoint{2.115316in}{1.367464in}}%
\pgfpathlineto{\pgfqpoint{2.119551in}{1.393708in}}%
\pgfpathlineto{\pgfqpoint{2.123787in}{1.401992in}}%
\pgfpathlineto{\pgfqpoint{2.125198in}{1.404321in}}%
\pgfpathlineto{\pgfqpoint{2.126610in}{1.403887in}}%
\pgfpathlineto{\pgfqpoint{2.129434in}{1.397148in}}%
\pgfpathlineto{\pgfqpoint{2.132257in}{1.384126in}}%
\pgfpathlineto{\pgfqpoint{2.135081in}{1.363755in}}%
\pgfpathlineto{\pgfqpoint{2.140728in}{1.290603in}}%
\pgfpathlineto{\pgfqpoint{2.146374in}{1.227009in}}%
\pgfpathlineto{\pgfqpoint{2.147786in}{1.221997in}}%
\pgfpathlineto{\pgfqpoint{2.149198in}{1.223600in}}%
\pgfpathlineto{\pgfqpoint{2.152021in}{1.244750in}}%
\pgfpathlineto{\pgfqpoint{2.156257in}{1.315895in}}%
\pgfpathlineto{\pgfqpoint{2.164727in}{1.476270in}}%
\pgfpathlineto{\pgfqpoint{2.167551in}{1.500845in}}%
\pgfpathlineto{\pgfqpoint{2.170374in}{1.509158in}}%
\pgfpathlineto{\pgfqpoint{2.174609in}{1.489668in}}%
\pgfpathlineto{\pgfqpoint{2.178844in}{1.454392in}}%
\pgfpathlineto{\pgfqpoint{2.190138in}{1.351307in}}%
\pgfpathlineto{\pgfqpoint{2.194374in}{1.329073in}}%
\pgfpathlineto{\pgfqpoint{2.198609in}{1.317793in}}%
\pgfpathlineto{\pgfqpoint{2.200020in}{1.317082in}}%
\pgfpathlineto{\pgfqpoint{2.201432in}{1.318275in}}%
\pgfpathlineto{\pgfqpoint{2.204256in}{1.325240in}}%
\pgfpathlineto{\pgfqpoint{2.208491in}{1.348589in}}%
\pgfpathlineto{\pgfqpoint{2.216961in}{1.407875in}}%
\pgfpathlineto{\pgfqpoint{2.221197in}{1.421120in}}%
\pgfpathlineto{\pgfqpoint{2.222608in}{1.422213in}}%
\pgfpathlineto{\pgfqpoint{2.224020in}{1.420532in}}%
\pgfpathlineto{\pgfqpoint{2.226844in}{1.408047in}}%
\pgfpathlineto{\pgfqpoint{2.229667in}{1.395789in}}%
\pgfpathlineto{\pgfqpoint{2.232490in}{1.385198in}}%
\pgfpathlineto{\pgfqpoint{2.239549in}{1.339154in}}%
\pgfpathlineto{\pgfqpoint{2.242373in}{1.311799in}}%
\pgfpathlineto{\pgfqpoint{2.249431in}{1.234877in}}%
\pgfpathlineto{\pgfqpoint{2.252255in}{1.222027in}}%
\pgfpathlineto{\pgfqpoint{2.253667in}{1.219132in}}%
\pgfpathlineto{\pgfqpoint{2.255078in}{1.221792in}}%
\pgfpathlineto{\pgfqpoint{2.257902in}{1.234977in}}%
\pgfpathlineto{\pgfqpoint{2.267784in}{1.308985in}}%
\pgfpathlineto{\pgfqpoint{2.273431in}{1.333453in}}%
\pgfpathlineto{\pgfqpoint{2.274843in}{1.334694in}}%
\pgfpathlineto{\pgfqpoint{2.276254in}{1.332039in}}%
\pgfpathlineto{\pgfqpoint{2.279078in}{1.315541in}}%
\pgfpathlineto{\pgfqpoint{2.284725in}{1.267747in}}%
\pgfpathlineto{\pgfqpoint{2.288960in}{1.237190in}}%
\pgfpathlineto{\pgfqpoint{2.290372in}{1.232647in}}%
\pgfpathlineto{\pgfqpoint{2.291783in}{1.233318in}}%
\pgfpathlineto{\pgfqpoint{2.293195in}{1.238756in}}%
\pgfpathlineto{\pgfqpoint{2.296019in}{1.263060in}}%
\pgfpathlineto{\pgfqpoint{2.301666in}{1.345008in}}%
\pgfpathlineto{\pgfqpoint{2.315783in}{1.547496in}}%
\pgfpathlineto{\pgfqpoint{2.318606in}{1.564659in}}%
\pgfpathlineto{\pgfqpoint{2.321430in}{1.569446in}}%
\pgfpathlineto{\pgfqpoint{2.322842in}{1.566820in}}%
\pgfpathlineto{\pgfqpoint{2.331312in}{1.529518in}}%
\pgfpathlineto{\pgfqpoint{2.334136in}{1.489474in}}%
\pgfpathlineto{\pgfqpoint{2.341194in}{1.336303in}}%
\pgfpathlineto{\pgfqpoint{2.348253in}{1.186562in}}%
\pgfpathlineto{\pgfqpoint{2.351076in}{1.158648in}}%
\pgfpathlineto{\pgfqpoint{2.352488in}{1.155429in}}%
\pgfpathlineto{\pgfqpoint{2.353900in}{1.159059in}}%
\pgfpathlineto{\pgfqpoint{2.356723in}{1.185456in}}%
\pgfpathlineto{\pgfqpoint{2.369429in}{1.368938in}}%
\pgfpathlineto{\pgfqpoint{2.370841in}{1.371713in}}%
\pgfpathlineto{\pgfqpoint{2.372253in}{1.371223in}}%
\pgfpathlineto{\pgfqpoint{2.375076in}{1.357322in}}%
\pgfpathlineto{\pgfqpoint{2.387782in}{1.260495in}}%
\pgfpathlineto{\pgfqpoint{2.393429in}{1.238863in}}%
\pgfpathlineto{\pgfqpoint{2.394840in}{1.238640in}}%
\pgfpathlineto{\pgfqpoint{2.396252in}{1.241315in}}%
\pgfpathlineto{\pgfqpoint{2.399076in}{1.254968in}}%
\pgfpathlineto{\pgfqpoint{2.403311in}{1.295386in}}%
\pgfpathlineto{\pgfqpoint{2.407546in}{1.359963in}}%
\pgfpathlineto{\pgfqpoint{2.416016in}{1.495252in}}%
\pgfpathlineto{\pgfqpoint{2.418840in}{1.517655in}}%
\pgfpathlineto{\pgfqpoint{2.421663in}{1.523817in}}%
\pgfpathlineto{\pgfqpoint{2.423075in}{1.522004in}}%
\pgfpathlineto{\pgfqpoint{2.425899in}{1.511227in}}%
\pgfpathlineto{\pgfqpoint{2.428722in}{1.501622in}}%
\pgfpathlineto{\pgfqpoint{2.431546in}{1.488288in}}%
\pgfpathlineto{\pgfqpoint{2.434369in}{1.461261in}}%
\pgfpathlineto{\pgfqpoint{2.438604in}{1.393741in}}%
\pgfpathlineto{\pgfqpoint{2.448486in}{1.234328in}}%
\pgfpathlineto{\pgfqpoint{2.451310in}{1.206709in}}%
\pgfpathlineto{\pgfqpoint{2.454133in}{1.195176in}}%
\pgfpathlineto{\pgfqpoint{2.455545in}{1.195129in}}%
\pgfpathlineto{\pgfqpoint{2.456957in}{1.199528in}}%
\pgfpathlineto{\pgfqpoint{2.459780in}{1.219517in}}%
\pgfpathlineto{\pgfqpoint{2.465427in}{1.270577in}}%
\pgfpathlineto{\pgfqpoint{2.468251in}{1.287850in}}%
\pgfpathlineto{\pgfqpoint{2.471074in}{1.290884in}}%
\pgfpathlineto{\pgfqpoint{2.472486in}{1.289633in}}%
\pgfpathlineto{\pgfqpoint{2.475309in}{1.278896in}}%
\pgfpathlineto{\pgfqpoint{2.479545in}{1.260235in}}%
\pgfpathlineto{\pgfqpoint{2.489427in}{1.237693in}}%
\pgfpathlineto{\pgfqpoint{2.490839in}{1.237355in}}%
\pgfpathlineto{\pgfqpoint{2.492250in}{1.238849in}}%
\pgfpathlineto{\pgfqpoint{2.493662in}{1.242623in}}%
\pgfpathlineto{\pgfqpoint{2.496485in}{1.259608in}}%
\pgfpathlineto{\pgfqpoint{2.500721in}{1.304325in}}%
\pgfpathlineto{\pgfqpoint{2.506368in}{1.392830in}}%
\pgfpathlineto{\pgfqpoint{2.514838in}{1.524748in}}%
\pgfpathlineto{\pgfqpoint{2.519073in}{1.566338in}}%
\pgfpathlineto{\pgfqpoint{2.521897in}{1.578629in}}%
\pgfpathlineto{\pgfqpoint{2.523308in}{1.580966in}}%
\pgfpathlineto{\pgfqpoint{2.524720in}{1.580816in}}%
\pgfpathlineto{\pgfqpoint{2.527544in}{1.572958in}}%
\pgfpathlineto{\pgfqpoint{2.537426in}{1.522422in}}%
\pgfpathlineto{\pgfqpoint{2.544485in}{1.454439in}}%
\pgfpathlineto{\pgfqpoint{2.550131in}{1.382386in}}%
\pgfpathlineto{\pgfqpoint{2.562837in}{1.195891in}}%
\pgfpathlineto{\pgfqpoint{2.569896in}{1.127106in}}%
\pgfpathlineto{\pgfqpoint{2.572719in}{1.114218in}}%
\pgfpathlineto{\pgfqpoint{2.575543in}{1.108025in}}%
\pgfpathlineto{\pgfqpoint{2.578366in}{1.106648in}}%
\pgfpathlineto{\pgfqpoint{2.584013in}{1.111985in}}%
\pgfpathlineto{\pgfqpoint{2.589660in}{1.123824in}}%
\pgfpathlineto{\pgfqpoint{2.591072in}{1.124566in}}%
\pgfpathlineto{\pgfqpoint{2.593895in}{1.121451in}}%
\pgfpathlineto{\pgfqpoint{2.598131in}{1.128614in}}%
\pgfpathlineto{\pgfqpoint{2.608013in}{1.134045in}}%
\pgfpathlineto{\pgfqpoint{2.612248in}{1.135189in}}%
\pgfpathlineto{\pgfqpoint{2.619307in}{1.141394in}}%
\pgfpathlineto{\pgfqpoint{2.623542in}{1.148817in}}%
\pgfpathlineto{\pgfqpoint{2.632012in}{1.171832in}}%
\pgfpathlineto{\pgfqpoint{2.636248in}{1.187105in}}%
\pgfpathlineto{\pgfqpoint{2.646130in}{1.212004in}}%
\pgfpathlineto{\pgfqpoint{2.647541in}{1.211011in}}%
\pgfpathlineto{\pgfqpoint{2.650365in}{1.203556in}}%
\pgfpathlineto{\pgfqpoint{2.653188in}{1.187600in}}%
\pgfpathlineto{\pgfqpoint{2.661659in}{1.107086in}}%
\pgfpathlineto{\pgfqpoint{2.668717in}{1.050660in}}%
\pgfpathlineto{\pgfqpoint{2.670129in}{1.047251in}}%
\pgfpathlineto{\pgfqpoint{2.671541in}{1.047985in}}%
\pgfpathlineto{\pgfqpoint{2.672953in}{1.051093in}}%
\pgfpathlineto{\pgfqpoint{2.675776in}{1.077749in}}%
\pgfpathlineto{\pgfqpoint{2.678600in}{1.130494in}}%
\pgfpathlineto{\pgfqpoint{2.681423in}{1.215736in}}%
\pgfpathlineto{\pgfqpoint{2.691305in}{1.651906in}}%
\pgfpathlineto{\pgfqpoint{2.692717in}{1.652549in}}%
\pgfpathlineto{\pgfqpoint{2.694129in}{1.649491in}}%
\pgfpathlineto{\pgfqpoint{2.695541in}{1.649766in}}%
\pgfpathlineto{\pgfqpoint{2.698364in}{1.620191in}}%
\pgfpathlineto{\pgfqpoint{2.706834in}{2.149064in}}%
\pgfpathlineto{\pgfqpoint{2.708246in}{2.168285in}}%
\pgfpathlineto{\pgfqpoint{2.709658in}{2.161886in}}%
\pgfpathlineto{\pgfqpoint{2.711070in}{2.147380in}}%
\pgfpathlineto{\pgfqpoint{2.715305in}{2.208764in}}%
\pgfpathlineto{\pgfqpoint{2.719540in}{2.231398in}}%
\pgfpathlineto{\pgfqpoint{2.720952in}{2.236196in}}%
\pgfpathlineto{\pgfqpoint{2.726599in}{2.434049in}}%
\pgfpathlineto{\pgfqpoint{2.730834in}{2.530816in}}%
\pgfpathlineto{\pgfqpoint{2.735069in}{2.578897in}}%
\pgfpathlineto{\pgfqpoint{2.736481in}{2.581490in}}%
\pgfpathlineto{\pgfqpoint{2.737893in}{2.587392in}}%
\pgfpathlineto{\pgfqpoint{2.740716in}{2.624086in}}%
\pgfpathlineto{\pgfqpoint{2.742128in}{2.616527in}}%
\pgfpathlineto{\pgfqpoint{2.743540in}{2.585689in}}%
\pgfpathlineto{\pgfqpoint{2.746363in}{2.443270in}}%
\pgfpathlineto{\pgfqpoint{2.752010in}{1.899524in}}%
\pgfpathlineto{\pgfqpoint{2.759069in}{1.341465in}}%
\pgfpathlineto{\pgfqpoint{2.763304in}{1.208031in}}%
\pgfpathlineto{\pgfqpoint{2.768951in}{1.118995in}}%
\pgfpathlineto{\pgfqpoint{2.771774in}{1.030242in}}%
\pgfpathlineto{\pgfqpoint{2.773186in}{1.054875in}}%
\pgfpathlineto{\pgfqpoint{2.774598in}{0.917087in}}%
\pgfpathlineto{\pgfqpoint{2.776010in}{0.926044in}}%
\pgfpathlineto{\pgfqpoint{2.778833in}{0.972247in}}%
\pgfpathlineto{\pgfqpoint{2.780245in}{0.977219in}}%
\pgfpathlineto{\pgfqpoint{2.781657in}{0.971429in}}%
\pgfpathlineto{\pgfqpoint{2.784480in}{0.948335in}}%
\pgfpathlineto{\pgfqpoint{2.785892in}{1.619743in}}%
\pgfpathlineto{\pgfqpoint{2.787303in}{1.624356in}}%
\pgfpathlineto{\pgfqpoint{2.790127in}{1.598338in}}%
\pgfpathlineto{\pgfqpoint{2.797186in}{1.517574in}}%
\pgfpathlineto{\pgfqpoint{2.798597in}{1.525964in}}%
\pgfpathlineto{\pgfqpoint{2.801421in}{1.585069in}}%
\pgfpathlineto{\pgfqpoint{2.812715in}{1.917258in}}%
\pgfpathlineto{\pgfqpoint{2.819773in}{2.018795in}}%
\pgfpathlineto{\pgfqpoint{2.824009in}{2.059677in}}%
\pgfpathlineto{\pgfqpoint{2.828244in}{2.081618in}}%
\pgfpathlineto{\pgfqpoint{2.833891in}{2.106695in}}%
\pgfpathlineto{\pgfqpoint{2.848008in}{2.186607in}}%
\pgfpathlineto{\pgfqpoint{2.849420in}{2.187000in}}%
\pgfpathlineto{\pgfqpoint{2.850832in}{2.185519in}}%
\pgfpathlineto{\pgfqpoint{2.853655in}{2.176305in}}%
\pgfpathlineto{\pgfqpoint{2.859302in}{2.143657in}}%
\pgfpathlineto{\pgfqpoint{2.866361in}{2.094793in}}%
\pgfpathlineto{\pgfqpoint{2.876243in}{2.021705in}}%
\pgfpathlineto{\pgfqpoint{2.887537in}{1.961807in}}%
\pgfpathlineto{\pgfqpoint{2.896007in}{1.902809in}}%
\pgfpathlineto{\pgfqpoint{2.917183in}{1.716789in}}%
\pgfpathlineto{\pgfqpoint{2.922830in}{1.640689in}}%
\pgfpathlineto{\pgfqpoint{2.929889in}{1.508897in}}%
\pgfpathlineto{\pgfqpoint{2.938359in}{1.342891in}}%
\pgfpathlineto{\pgfqpoint{2.944006in}{1.277305in}}%
\pgfpathlineto{\pgfqpoint{2.948242in}{1.246705in}}%
\pgfpathlineto{\pgfqpoint{2.951065in}{1.238115in}}%
\pgfpathlineto{\pgfqpoint{2.953889in}{1.251733in}}%
\pgfpathlineto{\pgfqpoint{2.958124in}{1.292218in}}%
\pgfpathlineto{\pgfqpoint{2.963771in}{1.373043in}}%
\pgfpathlineto{\pgfqpoint{2.972241in}{1.528459in}}%
\pgfpathlineto{\pgfqpoint{2.987770in}{1.813282in}}%
\pgfpathlineto{\pgfqpoint{2.993417in}{1.868690in}}%
\pgfpathlineto{\pgfqpoint{2.996241in}{1.884204in}}%
\pgfpathlineto{\pgfqpoint{2.997652in}{1.887354in}}%
\pgfpathlineto{\pgfqpoint{2.999064in}{1.887291in}}%
\pgfpathlineto{\pgfqpoint{3.000476in}{1.883295in}}%
\pgfpathlineto{\pgfqpoint{3.003299in}{1.864077in}}%
\pgfpathlineto{\pgfqpoint{3.007535in}{1.811523in}}%
\pgfpathlineto{\pgfqpoint{3.013182in}{1.709646in}}%
\pgfpathlineto{\pgfqpoint{3.018828in}{1.556298in}}%
\pgfpathlineto{\pgfqpoint{3.025887in}{1.291537in}}%
\pgfpathlineto{\pgfqpoint{3.034358in}{0.975571in}}%
\pgfpathlineto{\pgfqpoint{3.038593in}{0.877976in}}%
\pgfpathlineto{\pgfqpoint{3.041416in}{0.846617in}}%
\pgfpathlineto{\pgfqpoint{3.045652in}{0.830105in}}%
\pgfpathlineto{\pgfqpoint{3.047063in}{0.829881in}}%
\pgfpathlineto{\pgfqpoint{3.048475in}{0.827778in}}%
\pgfpathlineto{\pgfqpoint{3.052710in}{0.865287in}}%
\pgfpathlineto{\pgfqpoint{3.055534in}{0.911046in}}%
\pgfpathlineto{\pgfqpoint{3.059769in}{1.021771in}}%
\pgfpathlineto{\pgfqpoint{3.065416in}{1.239621in}}%
\pgfpathlineto{\pgfqpoint{3.086592in}{2.153132in}}%
\pgfpathlineto{\pgfqpoint{3.090827in}{2.252346in}}%
\pgfpathlineto{\pgfqpoint{3.095062in}{2.299893in}}%
\pgfpathlineto{\pgfqpoint{3.096474in}{2.305147in}}%
\pgfpathlineto{\pgfqpoint{3.097886in}{2.305310in}}%
\pgfpathlineto{\pgfqpoint{3.099298in}{2.299314in}}%
\pgfpathlineto{\pgfqpoint{3.103533in}{2.258263in}}%
\pgfpathlineto{\pgfqpoint{3.107768in}{2.173148in}}%
\pgfpathlineto{\pgfqpoint{3.113415in}{2.003480in}}%
\pgfpathlineto{\pgfqpoint{3.124709in}{1.582258in}}%
\pgfpathlineto{\pgfqpoint{3.137414in}{1.100208in}}%
\pgfpathlineto{\pgfqpoint{3.141650in}{1.004866in}}%
\pgfpathlineto{\pgfqpoint{3.144473in}{0.971556in}}%
\pgfpathlineto{\pgfqpoint{3.145885in}{0.964162in}}%
\pgfpathlineto{\pgfqpoint{3.147297in}{0.963684in}}%
\pgfpathlineto{\pgfqpoint{3.150120in}{0.989490in}}%
\pgfpathlineto{\pgfqpoint{3.154355in}{1.062704in}}%
\pgfpathlineto{\pgfqpoint{3.160002in}{1.228099in}}%
\pgfpathlineto{\pgfqpoint{3.186825in}{2.147033in}}%
\pgfpathlineto{\pgfqpoint{3.191061in}{2.216326in}}%
\pgfpathlineto{\pgfqpoint{3.193884in}{2.230968in}}%
\pgfpathlineto{\pgfqpoint{3.195296in}{2.228477in}}%
\pgfpathlineto{\pgfqpoint{3.198119in}{2.208875in}}%
\pgfpathlineto{\pgfqpoint{3.206590in}{2.130875in}}%
\pgfpathlineto{\pgfqpoint{3.212237in}{2.101669in}}%
\pgfpathlineto{\pgfqpoint{3.219295in}{2.072424in}}%
\pgfpathlineto{\pgfqpoint{3.222119in}{2.046308in}}%
\pgfpathlineto{\pgfqpoint{3.226354in}{1.977057in}}%
\pgfpathlineto{\pgfqpoint{3.233413in}{1.799708in}}%
\pgfpathlineto{\pgfqpoint{3.243295in}{1.551539in}}%
\pgfpathlineto{\pgfqpoint{3.247530in}{1.490892in}}%
\pgfpathlineto{\pgfqpoint{3.250354in}{1.472741in}}%
\pgfpathlineto{\pgfqpoint{3.251765in}{1.471908in}}%
\pgfpathlineto{\pgfqpoint{3.253177in}{1.477231in}}%
\pgfpathlineto{\pgfqpoint{3.254589in}{1.489313in}}%
\pgfpathlineto{\pgfqpoint{3.258824in}{1.580873in}}%
\pgfpathlineto{\pgfqpoint{3.265883in}{1.813463in}}%
\pgfpathlineto{\pgfqpoint{3.284235in}{2.439208in}}%
\pgfpathlineto{\pgfqpoint{3.288470in}{2.497231in}}%
\pgfpathlineto{\pgfqpoint{3.291294in}{2.507242in}}%
\pgfpathlineto{\pgfqpoint{3.292706in}{2.505765in}}%
\pgfpathlineto{\pgfqpoint{3.295529in}{2.491836in}}%
\pgfpathlineto{\pgfqpoint{3.301176in}{2.455299in}}%
\pgfpathlineto{\pgfqpoint{3.305411in}{2.419389in}}%
\pgfpathlineto{\pgfqpoint{3.309646in}{2.357138in}}%
\pgfpathlineto{\pgfqpoint{3.315293in}{2.236208in}}%
\pgfpathlineto{\pgfqpoint{3.320940in}{2.063336in}}%
\pgfpathlineto{\pgfqpoint{3.327999in}{1.770712in}}%
\pgfpathlineto{\pgfqpoint{3.342116in}{1.133082in}}%
\pgfpathlineto{\pgfqpoint{3.346352in}{1.030107in}}%
\pgfpathlineto{\pgfqpoint{3.349175in}{1.000468in}}%
\pgfpathlineto{\pgfqpoint{3.350587in}{0.998621in}}%
\pgfpathlineto{\pgfqpoint{3.351999in}{1.004981in}}%
\pgfpathlineto{\pgfqpoint{3.354822in}{1.043261in}}%
\pgfpathlineto{\pgfqpoint{3.357646in}{1.117125in}}%
\pgfpathlineto{\pgfqpoint{3.363293in}{1.342562in}}%
\pgfpathlineto{\pgfqpoint{3.384469in}{2.295358in}}%
\pgfpathlineto{\pgfqpoint{3.388704in}{2.383764in}}%
\pgfpathlineto{\pgfqpoint{3.392939in}{2.425425in}}%
\pgfpathlineto{\pgfqpoint{3.395763in}{2.431297in}}%
\pgfpathlineto{\pgfqpoint{3.397174in}{2.428292in}}%
\pgfpathlineto{\pgfqpoint{3.405645in}{2.389412in}}%
\pgfpathlineto{\pgfqpoint{3.409880in}{2.355047in}}%
\pgfpathlineto{\pgfqpoint{3.414115in}{2.298748in}}%
\pgfpathlineto{\pgfqpoint{3.418350in}{2.206783in}}%
\pgfpathlineto{\pgfqpoint{3.423997in}{2.026630in}}%
\pgfpathlineto{\pgfqpoint{3.432468in}{1.666512in}}%
\pgfpathlineto{\pgfqpoint{3.442350in}{1.247670in}}%
\pgfpathlineto{\pgfqpoint{3.446585in}{1.140645in}}%
\pgfpathlineto{\pgfqpoint{3.449409in}{1.099752in}}%
\pgfpathlineto{\pgfqpoint{3.452232in}{1.086250in}}%
\pgfpathlineto{\pgfqpoint{3.453644in}{1.091056in}}%
\pgfpathlineto{\pgfqpoint{3.456467in}{1.126603in}}%
\pgfpathlineto{\pgfqpoint{3.460702in}{1.240677in}}%
\pgfpathlineto{\pgfqpoint{3.466349in}{1.469005in}}%
\pgfpathlineto{\pgfqpoint{3.487525in}{2.408198in}}%
\pgfpathlineto{\pgfqpoint{3.491761in}{2.496404in}}%
\pgfpathlineto{\pgfqpoint{3.494584in}{2.528357in}}%
\pgfpathlineto{\pgfqpoint{3.497408in}{2.537499in}}%
\pgfpathlineto{\pgfqpoint{3.498819in}{2.533532in}}%
\pgfpathlineto{\pgfqpoint{3.501643in}{2.510167in}}%
\pgfpathlineto{\pgfqpoint{3.507290in}{2.442266in}}%
\pgfpathlineto{\pgfqpoint{3.514348in}{2.311165in}}%
\pgfpathlineto{\pgfqpoint{3.519995in}{2.170452in}}%
\pgfpathlineto{\pgfqpoint{3.525642in}{1.978179in}}%
\pgfpathlineto{\pgfqpoint{3.534113in}{1.601327in}}%
\pgfpathlineto{\pgfqpoint{3.545407in}{1.102867in}}%
\pgfpathlineto{\pgfqpoint{3.549642in}{0.992589in}}%
\pgfpathlineto{\pgfqpoint{3.553877in}{0.940457in}}%
\pgfpathlineto{\pgfqpoint{3.555289in}{0.936075in}}%
\pgfpathlineto{\pgfqpoint{3.556701in}{0.938480in}}%
\pgfpathlineto{\pgfqpoint{3.559524in}{0.963404in}}%
\pgfpathlineto{\pgfqpoint{3.562348in}{1.015210in}}%
\pgfpathlineto{\pgfqpoint{3.566583in}{1.139173in}}%
\pgfpathlineto{\pgfqpoint{3.573641in}{1.436794in}}%
\pgfpathlineto{\pgfqpoint{3.584935in}{1.902394in}}%
\pgfpathlineto{\pgfqpoint{3.590582in}{2.064138in}}%
\pgfpathlineto{\pgfqpoint{3.594818in}{2.138625in}}%
\pgfpathlineto{\pgfqpoint{3.599053in}{2.178497in}}%
\pgfpathlineto{\pgfqpoint{3.603288in}{2.200988in}}%
\pgfpathlineto{\pgfqpoint{3.606111in}{2.205839in}}%
\pgfpathlineto{\pgfqpoint{3.607523in}{2.207749in}}%
\pgfpathlineto{\pgfqpoint{3.608935in}{2.207023in}}%
\pgfpathlineto{\pgfqpoint{3.610347in}{2.202550in}}%
\pgfpathlineto{\pgfqpoint{3.613170in}{2.181419in}}%
\pgfpathlineto{\pgfqpoint{3.617405in}{2.121806in}}%
\pgfpathlineto{\pgfqpoint{3.624464in}{1.972135in}}%
\pgfpathlineto{\pgfqpoint{3.634346in}{1.715020in}}%
\pgfpathlineto{\pgfqpoint{3.642817in}{1.417016in}}%
\pgfpathlineto{\pgfqpoint{3.649875in}{1.179760in}}%
\pgfpathlineto{\pgfqpoint{3.654111in}{1.102258in}}%
\pgfpathlineto{\pgfqpoint{3.656934in}{1.080244in}}%
\pgfpathlineto{\pgfqpoint{3.658346in}{1.077302in}}%
\pgfpathlineto{\pgfqpoint{3.659757in}{1.079291in}}%
\pgfpathlineto{\pgfqpoint{3.661169in}{1.086558in}}%
\pgfpathlineto{\pgfqpoint{3.663993in}{1.126700in}}%
\pgfpathlineto{\pgfqpoint{3.668228in}{1.230355in}}%
\pgfpathlineto{\pgfqpoint{3.675287in}{1.473816in}}%
\pgfpathlineto{\pgfqpoint{3.686581in}{1.851691in}}%
\pgfpathlineto{\pgfqpoint{3.695051in}{2.064116in}}%
\pgfpathlineto{\pgfqpoint{3.699286in}{2.132167in}}%
\pgfpathlineto{\pgfqpoint{3.702110in}{2.156162in}}%
\pgfpathlineto{\pgfqpoint{3.704933in}{2.163565in}}%
\pgfpathlineto{\pgfqpoint{3.706345in}{2.161142in}}%
\pgfpathlineto{\pgfqpoint{3.713404in}{2.129193in}}%
\pgfpathlineto{\pgfqpoint{3.717639in}{2.087654in}}%
\pgfpathlineto{\pgfqpoint{3.723286in}{2.002397in}}%
\pgfpathlineto{\pgfqpoint{3.728933in}{1.880637in}}%
\pgfpathlineto{\pgfqpoint{3.734580in}{1.705917in}}%
\pgfpathlineto{\pgfqpoint{3.744462in}{1.318503in}}%
\pgfpathlineto{\pgfqpoint{3.752932in}{0.985828in}}%
\pgfpathlineto{\pgfqpoint{3.757167in}{0.882645in}}%
\pgfpathlineto{\pgfqpoint{3.759991in}{0.844753in}}%
\pgfpathlineto{\pgfqpoint{3.762814in}{0.831387in}}%
\pgfpathlineto{\pgfqpoint{3.764226in}{0.834541in}}%
\pgfpathlineto{\pgfqpoint{3.767050in}{0.866414in}}%
\pgfpathlineto{\pgfqpoint{3.771285in}{0.961834in}}%
\pgfpathlineto{\pgfqpoint{3.776932in}{1.164217in}}%
\pgfpathlineto{\pgfqpoint{3.791049in}{1.713159in}}%
\pgfpathlineto{\pgfqpoint{3.798108in}{1.899465in}}%
\pgfpathlineto{\pgfqpoint{3.802343in}{1.968842in}}%
\pgfpathlineto{\pgfqpoint{3.807990in}{2.021517in}}%
\pgfpathlineto{\pgfqpoint{3.810813in}{2.032745in}}%
\pgfpathlineto{\pgfqpoint{3.812225in}{2.033308in}}%
\pgfpathlineto{\pgfqpoint{3.813637in}{2.030591in}}%
\pgfpathlineto{\pgfqpoint{3.816460in}{2.014270in}}%
\pgfpathlineto{\pgfqpoint{3.820696in}{1.964602in}}%
\pgfpathlineto{\pgfqpoint{3.824931in}{1.888079in}}%
\pgfpathlineto{\pgfqpoint{3.829166in}{1.777351in}}%
\pgfpathlineto{\pgfqpoint{3.834813in}{1.567410in}}%
\pgfpathlineto{\pgfqpoint{3.850342in}{0.913591in}}%
\pgfpathlineto{\pgfqpoint{3.853166in}{0.872946in}}%
\pgfpathlineto{\pgfqpoint{3.854577in}{0.864826in}}%
\pgfpathlineto{\pgfqpoint{3.855989in}{0.863420in}}%
\pgfpathlineto{\pgfqpoint{3.857401in}{0.864145in}}%
\pgfpathlineto{\pgfqpoint{3.861636in}{0.910055in}}%
\pgfpathlineto{\pgfqpoint{3.864459in}{0.917712in}}%
\pgfpathlineto{\pgfqpoint{3.865871in}{0.922624in}}%
\pgfpathlineto{\pgfqpoint{3.868695in}{0.920858in}}%
\pgfpathlineto{\pgfqpoint{3.871518in}{0.928489in}}%
\pgfpathlineto{\pgfqpoint{3.874342in}{0.964917in}}%
\pgfpathlineto{\pgfqpoint{3.877165in}{1.044426in}}%
\pgfpathlineto{\pgfqpoint{3.881400in}{1.224133in}}%
\pgfpathlineto{\pgfqpoint{3.899753in}{2.080124in}}%
\pgfpathlineto{\pgfqpoint{3.905400in}{2.240382in}}%
\pgfpathlineto{\pgfqpoint{3.909635in}{2.303417in}}%
\pgfpathlineto{\pgfqpoint{3.912459in}{2.323736in}}%
\pgfpathlineto{\pgfqpoint{3.915282in}{2.332404in}}%
\pgfpathlineto{\pgfqpoint{3.916694in}{2.333283in}}%
\pgfpathlineto{\pgfqpoint{3.919517in}{2.339129in}}%
\pgfpathlineto{\pgfqpoint{3.920929in}{2.337122in}}%
\pgfpathlineto{\pgfqpoint{3.923752in}{2.320443in}}%
\pgfpathlineto{\pgfqpoint{3.926576in}{2.284631in}}%
\pgfpathlineto{\pgfqpoint{3.930811in}{2.188893in}}%
\pgfpathlineto{\pgfqpoint{3.936458in}{1.986005in}}%
\pgfpathlineto{\pgfqpoint{3.944929in}{1.572431in}}%
\pgfpathlineto{\pgfqpoint{3.959046in}{0.842630in}}%
\pgfpathlineto{\pgfqpoint{3.961869in}{0.796622in}}%
\pgfpathlineto{\pgfqpoint{3.963281in}{0.792273in}}%
\pgfpathlineto{\pgfqpoint{3.964693in}{0.793133in}}%
\pgfpathlineto{\pgfqpoint{3.971752in}{0.854449in}}%
\pgfpathlineto{\pgfqpoint{3.974575in}{0.865900in}}%
\pgfpathlineto{\pgfqpoint{3.977399in}{0.906547in}}%
\pgfpathlineto{\pgfqpoint{3.980222in}{0.972810in}}%
\pgfpathlineto{\pgfqpoint{3.985869in}{1.210953in}}%
\pgfpathlineto{\pgfqpoint{4.002810in}{1.954705in}}%
\pgfpathlineto{\pgfqpoint{4.008457in}{2.118739in}}%
\pgfpathlineto{\pgfqpoint{4.012692in}{2.199374in}}%
\pgfpathlineto{\pgfqpoint{4.016927in}{2.241089in}}%
\pgfpathlineto{\pgfqpoint{4.022574in}{2.260500in}}%
\pgfpathlineto{\pgfqpoint{4.023986in}{2.258649in}}%
\pgfpathlineto{\pgfqpoint{4.026809in}{2.243126in}}%
\pgfpathlineto{\pgfqpoint{4.029633in}{2.210466in}}%
\pgfpathlineto{\pgfqpoint{4.033868in}{2.126636in}}%
\pgfpathlineto{\pgfqpoint{4.039515in}{1.949357in}}%
\pgfpathlineto{\pgfqpoint{4.047985in}{1.588141in}}%
\pgfpathlineto{\pgfqpoint{4.062103in}{0.950458in}}%
\pgfpathlineto{\pgfqpoint{4.066338in}{0.826110in}}%
\pgfpathlineto{\pgfqpoint{4.069161in}{0.786198in}}%
\pgfpathlineto{\pgfqpoint{4.071985in}{0.777597in}}%
\pgfpathlineto{\pgfqpoint{4.073397in}{0.782290in}}%
\pgfpathlineto{\pgfqpoint{4.076220in}{0.817648in}}%
\pgfpathlineto{\pgfqpoint{4.080455in}{0.918821in}}%
\pgfpathlineto{\pgfqpoint{4.084691in}{1.090267in}}%
\pgfpathlineto{\pgfqpoint{4.104455in}{2.037308in}}%
\pgfpathlineto{\pgfqpoint{4.111514in}{2.244683in}}%
\pgfpathlineto{\pgfqpoint{4.117161in}{2.347729in}}%
\pgfpathlineto{\pgfqpoint{4.119984in}{2.371843in}}%
\pgfpathlineto{\pgfqpoint{4.121396in}{2.376398in}}%
\pgfpathlineto{\pgfqpoint{4.122808in}{2.375911in}}%
\pgfpathlineto{\pgfqpoint{4.124219in}{2.371105in}}%
\pgfpathlineto{\pgfqpoint{4.132690in}{2.311942in}}%
\pgfpathlineto{\pgfqpoint{4.136925in}{2.253754in}}%
\pgfpathlineto{\pgfqpoint{4.141160in}{2.163258in}}%
\pgfpathlineto{\pgfqpoint{4.146807in}{1.984605in}}%
\pgfpathlineto{\pgfqpoint{4.155278in}{1.636700in}}%
\pgfpathlineto{\pgfqpoint{4.166571in}{1.167825in}}%
\pgfpathlineto{\pgfqpoint{4.172218in}{1.033728in}}%
\pgfpathlineto{\pgfqpoint{4.175042in}{1.000320in}}%
\pgfpathlineto{\pgfqpoint{4.177865in}{0.987354in}}%
\pgfpathlineto{\pgfqpoint{4.179277in}{0.990596in}}%
\pgfpathlineto{\pgfqpoint{4.180689in}{1.000141in}}%
\pgfpathlineto{\pgfqpoint{4.183512in}{1.050585in}}%
\pgfpathlineto{\pgfqpoint{4.187747in}{1.171863in}}%
\pgfpathlineto{\pgfqpoint{4.194806in}{1.463658in}}%
\pgfpathlineto{\pgfqpoint{4.204688in}{1.862775in}}%
\pgfpathlineto{\pgfqpoint{4.213159in}{2.117314in}}%
\pgfpathlineto{\pgfqpoint{4.218806in}{2.224964in}}%
\pgfpathlineto{\pgfqpoint{4.221629in}{2.251831in}}%
\pgfpathlineto{\pgfqpoint{4.223041in}{2.257673in}}%
\pgfpathlineto{\pgfqpoint{4.224453in}{2.258494in}}%
\pgfpathlineto{\pgfqpoint{4.225864in}{2.256877in}}%
\pgfpathlineto{\pgfqpoint{4.227276in}{2.253564in}}%
\pgfpathlineto{\pgfqpoint{4.230100in}{2.230641in}}%
\pgfpathlineto{\pgfqpoint{4.232923in}{2.186954in}}%
\pgfpathlineto{\pgfqpoint{4.237158in}{2.072053in}}%
\pgfpathlineto{\pgfqpoint{4.244217in}{1.777989in}}%
\pgfpathlineto{\pgfqpoint{4.256923in}{1.210609in}}%
\pgfpathlineto{\pgfqpoint{4.269628in}{0.784645in}}%
\pgfpathlineto{\pgfqpoint{4.272452in}{0.753283in}}%
\pgfpathlineto{\pgfqpoint{4.275275in}{0.743429in}}%
\pgfpathlineto{\pgfqpoint{4.280922in}{0.736954in}}%
\pgfpathlineto{\pgfqpoint{4.282334in}{0.736959in}}%
\pgfpathlineto{\pgfqpoint{4.283746in}{0.740802in}}%
\pgfpathlineto{\pgfqpoint{4.286569in}{0.763946in}}%
\pgfpathlineto{\pgfqpoint{4.289393in}{0.807406in}}%
\pgfpathlineto{\pgfqpoint{4.293628in}{0.926356in}}%
\pgfpathlineto{\pgfqpoint{4.300687in}{1.217226in}}%
\pgfpathlineto{\pgfqpoint{4.311980in}{1.693341in}}%
\pgfpathlineto{\pgfqpoint{4.319039in}{1.904058in}}%
\pgfpathlineto{\pgfqpoint{4.323274in}{1.977165in}}%
\pgfpathlineto{\pgfqpoint{4.328921in}{2.027530in}}%
\pgfpathlineto{\pgfqpoint{4.330333in}{2.029291in}}%
\pgfpathlineto{\pgfqpoint{4.331745in}{2.024855in}}%
\pgfpathlineto{\pgfqpoint{4.334568in}{1.998129in}}%
\pgfpathlineto{\pgfqpoint{4.338803in}{1.914145in}}%
\pgfpathlineto{\pgfqpoint{4.344450in}{1.752184in}}%
\pgfpathlineto{\pgfqpoint{4.355744in}{1.409941in}}%
\pgfpathlineto{\pgfqpoint{4.367038in}{1.057123in}}%
\pgfpathlineto{\pgfqpoint{4.374097in}{0.872321in}}%
\pgfpathlineto{\pgfqpoint{4.378332in}{0.829186in}}%
\pgfpathlineto{\pgfqpoint{4.379744in}{0.829798in}}%
\pgfpathlineto{\pgfqpoint{4.382567in}{0.827535in}}%
\pgfpathlineto{\pgfqpoint{4.383979in}{0.828388in}}%
\pgfpathlineto{\pgfqpoint{4.385391in}{0.826495in}}%
\pgfpathlineto{\pgfqpoint{4.391038in}{0.851875in}}%
\pgfpathlineto{\pgfqpoint{4.393861in}{0.885284in}}%
\pgfpathlineto{\pgfqpoint{4.398096in}{0.979464in}}%
\pgfpathlineto{\pgfqpoint{4.405155in}{1.229386in}}%
\pgfpathlineto{\pgfqpoint{4.415037in}{1.572658in}}%
\pgfpathlineto{\pgfqpoint{4.422096in}{1.747231in}}%
\pgfpathlineto{\pgfqpoint{4.426331in}{1.810838in}}%
\pgfpathlineto{\pgfqpoint{4.430566in}{1.847929in}}%
\pgfpathlineto{\pgfqpoint{4.434802in}{1.865962in}}%
\pgfpathlineto{\pgfqpoint{4.436213in}{1.867578in}}%
\pgfpathlineto{\pgfqpoint{4.437625in}{1.866286in}}%
\pgfpathlineto{\pgfqpoint{4.440449in}{1.854283in}}%
\pgfpathlineto{\pgfqpoint{4.443272in}{1.828195in}}%
\pgfpathlineto{\pgfqpoint{4.447507in}{1.755880in}}%
\pgfpathlineto{\pgfqpoint{4.453154in}{1.597385in}}%
\pgfpathlineto{\pgfqpoint{4.464448in}{1.175310in}}%
\pgfpathlineto{\pgfqpoint{4.472919in}{0.887512in}}%
\pgfpathlineto{\pgfqpoint{4.477154in}{0.805935in}}%
\pgfpathlineto{\pgfqpoint{4.481389in}{0.766452in}}%
\pgfpathlineto{\pgfqpoint{4.482801in}{0.767165in}}%
\pgfpathlineto{\pgfqpoint{4.484212in}{0.769288in}}%
\pgfpathlineto{\pgfqpoint{4.487036in}{0.781946in}}%
\pgfpathlineto{\pgfqpoint{4.491271in}{0.816749in}}%
\pgfpathlineto{\pgfqpoint{4.499742in}{0.914493in}}%
\pgfpathlineto{\pgfqpoint{4.522329in}{1.213873in}}%
\pgfpathlineto{\pgfqpoint{4.527976in}{1.256095in}}%
\pgfpathlineto{\pgfqpoint{4.530800in}{1.264915in}}%
\pgfpathlineto{\pgfqpoint{4.532212in}{1.265224in}}%
\pgfpathlineto{\pgfqpoint{4.533623in}{1.262265in}}%
\pgfpathlineto{\pgfqpoint{4.536447in}{1.245512in}}%
\pgfpathlineto{\pgfqpoint{4.540682in}{1.200012in}}%
\pgfpathlineto{\pgfqpoint{4.553388in}{1.036203in}}%
\pgfpathlineto{\pgfqpoint{4.557623in}{1.009285in}}%
\pgfpathlineto{\pgfqpoint{4.559035in}{1.006707in}}%
\pgfpathlineto{\pgfqpoint{4.560446in}{1.007607in}}%
\pgfpathlineto{\pgfqpoint{4.563270in}{1.017234in}}%
\pgfpathlineto{\pgfqpoint{4.568917in}{1.042191in}}%
\pgfpathlineto{\pgfqpoint{4.571740in}{1.046751in}}%
\pgfpathlineto{\pgfqpoint{4.574564in}{1.046400in}}%
\pgfpathlineto{\pgfqpoint{4.577387in}{1.042340in}}%
\pgfpathlineto{\pgfqpoint{4.581622in}{1.029662in}}%
\pgfpathlineto{\pgfqpoint{4.587269in}{1.000414in}}%
\pgfpathlineto{\pgfqpoint{4.614092in}{0.839758in}}%
\pgfpathlineto{\pgfqpoint{4.623974in}{0.788812in}}%
\pgfpathlineto{\pgfqpoint{4.629621in}{0.771082in}}%
\pgfpathlineto{\pgfqpoint{4.633857in}{0.764744in}}%
\pgfpathlineto{\pgfqpoint{4.638092in}{0.762690in}}%
\pgfpathlineto{\pgfqpoint{4.645151in}{0.763535in}}%
\pgfpathlineto{\pgfqpoint{4.653621in}{0.764857in}}%
\pgfpathlineto{\pgfqpoint{4.660680in}{0.763194in}}%
\pgfpathlineto{\pgfqpoint{4.707267in}{0.745535in}}%
\pgfpathlineto{\pgfqpoint{4.717149in}{0.740272in}}%
\pgfpathlineto{\pgfqpoint{4.728443in}{0.734446in}}%
\pgfpathlineto{\pgfqpoint{4.741149in}{0.731691in}}%
\pgfpathlineto{\pgfqpoint{4.749619in}{0.731765in}}%
\pgfpathlineto{\pgfqpoint{4.762325in}{0.732004in}}%
\pgfpathlineto{\pgfqpoint{4.776442in}{0.730988in}}%
\pgfpathlineto{\pgfqpoint{4.793383in}{0.729640in}}%
\pgfpathlineto{\pgfqpoint{4.804677in}{0.727279in}}%
\pgfpathlineto{\pgfqpoint{4.818794in}{0.723545in}}%
\pgfpathlineto{\pgfqpoint{4.820206in}{0.703636in}}%
\pgfpathlineto{\pgfqpoint{4.823030in}{0.703490in}}%
\pgfpathlineto{\pgfqpoint{4.824441in}{0.721164in}}%
\pgfpathlineto{\pgfqpoint{4.834323in}{0.718464in}}%
\pgfpathlineto{\pgfqpoint{4.837147in}{0.717740in}}%
\pgfpathlineto{\pgfqpoint{4.838559in}{0.703111in}}%
\pgfpathlineto{\pgfqpoint{4.852676in}{0.702960in}}%
\pgfpathlineto{\pgfqpoint{4.855500in}{0.702907in}}%
\pgfpathlineto{\pgfqpoint{4.865382in}{0.702947in}}%
\pgfpathlineto{\pgfqpoint{4.866793in}{0.701614in}}%
\pgfpathlineto{\pgfqpoint{4.869617in}{0.702848in}}%
\pgfpathlineto{\pgfqpoint{4.880911in}{0.702418in}}%
\pgfpathlineto{\pgfqpoint{4.885146in}{0.699556in}}%
\pgfpathlineto{\pgfqpoint{4.886558in}{0.701961in}}%
\pgfpathlineto{\pgfqpoint{4.895028in}{0.701648in}}%
\pgfpathlineto{\pgfqpoint{4.896440in}{0.711767in}}%
\pgfpathlineto{\pgfqpoint{4.913381in}{0.709522in}}%
\pgfpathlineto{\pgfqpoint{4.916204in}{0.709086in}}%
\pgfpathlineto{\pgfqpoint{4.917616in}{0.701354in}}%
\pgfpathlineto{\pgfqpoint{4.920439in}{0.701400in}}%
\pgfpathlineto{\pgfqpoint{4.921851in}{0.710766in}}%
\pgfpathlineto{\pgfqpoint{4.924675in}{0.708182in}}%
\pgfpathlineto{\pgfqpoint{4.927498in}{0.705737in}}%
\pgfpathlineto{\pgfqpoint{4.930322in}{0.707721in}}%
\pgfpathlineto{\pgfqpoint{4.933145in}{0.709053in}}%
\pgfpathlineto{\pgfqpoint{4.937380in}{0.706877in}}%
\pgfpathlineto{\pgfqpoint{4.944439in}{0.706839in}}%
\pgfpathlineto{\pgfqpoint{4.951498in}{0.701402in}}%
\pgfpathlineto{\pgfqpoint{4.954321in}{0.700058in}}%
\pgfpathlineto{\pgfqpoint{4.959968in}{0.701907in}}%
\pgfpathlineto{\pgfqpoint{4.961380in}{0.700413in}}%
\pgfpathlineto{\pgfqpoint{4.967027in}{0.700470in}}%
\pgfpathlineto{\pgfqpoint{4.969850in}{0.706824in}}%
\pgfpathlineto{\pgfqpoint{4.975497in}{0.714535in}}%
\pgfpathlineto{\pgfqpoint{4.979732in}{0.713874in}}%
\pgfpathlineto{\pgfqpoint{4.985379in}{0.711248in}}%
\pgfpathlineto{\pgfqpoint{4.989615in}{0.703966in}}%
\pgfpathlineto{\pgfqpoint{4.991026in}{0.703761in}}%
\pgfpathlineto{\pgfqpoint{4.996673in}{0.709801in}}%
\pgfpathlineto{\pgfqpoint{5.000909in}{0.705408in}}%
\pgfpathlineto{\pgfqpoint{5.012202in}{0.703186in}}%
\pgfpathlineto{\pgfqpoint{5.013614in}{0.699495in}}%
\pgfpathlineto{\pgfqpoint{5.016438in}{0.700839in}}%
\pgfpathlineto{\pgfqpoint{5.017849in}{0.700766in}}%
\pgfpathlineto{\pgfqpoint{5.019261in}{0.699188in}}%
\pgfpathlineto{\pgfqpoint{5.022085in}{0.705471in}}%
\pgfpathlineto{\pgfqpoint{5.026320in}{0.708664in}}%
\pgfpathlineto{\pgfqpoint{5.027732in}{0.708351in}}%
\pgfpathlineto{\pgfqpoint{5.029143in}{0.709840in}}%
\pgfpathlineto{\pgfqpoint{5.033378in}{0.704893in}}%
\pgfpathlineto{\pgfqpoint{5.040437in}{0.704761in}}%
\pgfpathlineto{\pgfqpoint{5.041849in}{0.696000in}}%
\pgfpathlineto{\pgfqpoint{5.496429in}{0.696000in}}%
\pgfpathlineto{\pgfqpoint{5.499252in}{0.707185in}}%
\pgfpathlineto{\pgfqpoint{5.503487in}{0.717936in}}%
\pgfpathlineto{\pgfqpoint{5.506311in}{0.719957in}}%
\pgfpathlineto{\pgfqpoint{5.509134in}{0.718932in}}%
\pgfpathlineto{\pgfqpoint{5.514781in}{0.713721in}}%
\pgfpathlineto{\pgfqpoint{5.517605in}{0.697642in}}%
\pgfpathlineto{\pgfqpoint{5.520428in}{0.699578in}}%
\pgfpathlineto{\pgfqpoint{5.523252in}{0.700596in}}%
\pgfpathlineto{\pgfqpoint{5.526075in}{0.713933in}}%
\pgfpathlineto{\pgfqpoint{5.528899in}{0.716344in}}%
\pgfpathlineto{\pgfqpoint{5.533134in}{0.711032in}}%
\pgfpathlineto{\pgfqpoint{5.534545in}{0.696000in}}%
\pgfpathlineto{\pgfqpoint{5.534545in}{0.696000in}}%
\pgfusepath{stroke}%
\end{pgfscope}%
\begin{pgfscope}%
\pgfpathrectangle{\pgfqpoint{0.800000in}{0.528000in}}{\pgfqpoint{4.960000in}{3.696000in}} %
\pgfusepath{clip}%
\pgfsetrectcap%
\pgfsetroundjoin%
\pgfsetlinewidth{1.505625pt}%
\definecolor{currentstroke}{rgb}{0.172549,0.627451,0.172549}%
\pgfsetstrokecolor{currentstroke}%
\pgfsetdash{}{0pt}%
\pgfpathmoveto{\pgfqpoint{1.025455in}{0.696000in}}%
\pgfpathlineto{\pgfqpoint{1.122864in}{0.696000in}}%
\pgfpathlineto{\pgfqpoint{1.125688in}{0.697292in}}%
\pgfpathlineto{\pgfqpoint{1.132747in}{0.696971in}}%
\pgfpathlineto{\pgfqpoint{1.134158in}{0.698163in}}%
\pgfpathlineto{\pgfqpoint{1.136982in}{0.697984in}}%
\pgfpathlineto{\pgfqpoint{1.138394in}{0.696000in}}%
\pgfpathlineto{\pgfqpoint{1.148276in}{0.696635in}}%
\pgfpathlineto{\pgfqpoint{1.155334in}{0.696749in}}%
\pgfpathlineto{\pgfqpoint{1.166628in}{0.697799in}}%
\pgfpathlineto{\pgfqpoint{1.168040in}{0.698960in}}%
\pgfpathlineto{\pgfqpoint{1.169452in}{0.696915in}}%
\pgfpathlineto{\pgfqpoint{1.180746in}{0.822012in}}%
\pgfpathlineto{\pgfqpoint{1.193451in}{0.963910in}}%
\pgfpathlineto{\pgfqpoint{1.197687in}{0.992172in}}%
\pgfpathlineto{\pgfqpoint{1.200510in}{1.012979in}}%
\pgfpathlineto{\pgfqpoint{1.207569in}{1.070678in}}%
\pgfpathlineto{\pgfqpoint{1.214627in}{1.110993in}}%
\pgfpathlineto{\pgfqpoint{1.224510in}{1.159858in}}%
\pgfpathlineto{\pgfqpoint{1.235803in}{1.220641in}}%
\pgfpathlineto{\pgfqpoint{1.242862in}{1.230103in}}%
\pgfpathlineto{\pgfqpoint{1.247097in}{1.230114in}}%
\pgfpathlineto{\pgfqpoint{1.248509in}{1.231064in}}%
\pgfpathlineto{\pgfqpoint{1.259803in}{1.265717in}}%
\pgfpathlineto{\pgfqpoint{1.266862in}{1.295881in}}%
\pgfpathlineto{\pgfqpoint{1.271097in}{1.316908in}}%
\pgfpathlineto{\pgfqpoint{1.273920in}{1.325128in}}%
\pgfpathlineto{\pgfqpoint{1.276744in}{1.330556in}}%
\pgfpathlineto{\pgfqpoint{1.278156in}{1.330959in}}%
\pgfpathlineto{\pgfqpoint{1.279567in}{1.329524in}}%
\pgfpathlineto{\pgfqpoint{1.282391in}{1.320412in}}%
\pgfpathlineto{\pgfqpoint{1.286626in}{1.295065in}}%
\pgfpathlineto{\pgfqpoint{1.295096in}{1.227238in}}%
\pgfpathlineto{\pgfqpoint{1.302155in}{1.156102in}}%
\pgfpathlineto{\pgfqpoint{1.309214in}{1.060055in}}%
\pgfpathlineto{\pgfqpoint{1.319096in}{0.906148in}}%
\pgfpathlineto{\pgfqpoint{1.327566in}{0.776605in}}%
\pgfpathlineto{\pgfqpoint{1.328978in}{0.771794in}}%
\pgfpathlineto{\pgfqpoint{1.330390in}{0.771376in}}%
\pgfpathlineto{\pgfqpoint{1.331802in}{0.775387in}}%
\pgfpathlineto{\pgfqpoint{1.334625in}{0.803202in}}%
\pgfpathlineto{\pgfqpoint{1.343096in}{0.926790in}}%
\pgfpathlineto{\pgfqpoint{1.347331in}{0.958090in}}%
\pgfpathlineto{\pgfqpoint{1.350154in}{0.969425in}}%
\pgfpathlineto{\pgfqpoint{1.352978in}{0.971370in}}%
\pgfpathlineto{\pgfqpoint{1.354389in}{0.972153in}}%
\pgfpathlineto{\pgfqpoint{1.360036in}{0.963517in}}%
\pgfpathlineto{\pgfqpoint{1.361448in}{0.964568in}}%
\pgfpathlineto{\pgfqpoint{1.364272in}{0.972790in}}%
\pgfpathlineto{\pgfqpoint{1.367095in}{0.990852in}}%
\pgfpathlineto{\pgfqpoint{1.371330in}{1.037109in}}%
\pgfpathlineto{\pgfqpoint{1.378389in}{1.153403in}}%
\pgfpathlineto{\pgfqpoint{1.384036in}{1.241664in}}%
\pgfpathlineto{\pgfqpoint{1.388271in}{1.267754in}}%
\pgfpathlineto{\pgfqpoint{1.389683in}{1.269919in}}%
\pgfpathlineto{\pgfqpoint{1.391095in}{1.269105in}}%
\pgfpathlineto{\pgfqpoint{1.392506in}{1.265318in}}%
\pgfpathlineto{\pgfqpoint{1.395330in}{1.246184in}}%
\pgfpathlineto{\pgfqpoint{1.403800in}{1.145261in}}%
\pgfpathlineto{\pgfqpoint{1.417918in}{0.915099in}}%
\pgfpathlineto{\pgfqpoint{1.419329in}{0.902516in}}%
\pgfpathlineto{\pgfqpoint{1.422153in}{0.900194in}}%
\pgfpathlineto{\pgfqpoint{1.423565in}{0.906085in}}%
\pgfpathlineto{\pgfqpoint{1.426388in}{0.931713in}}%
\pgfpathlineto{\pgfqpoint{1.429212in}{0.994349in}}%
\pgfpathlineto{\pgfqpoint{1.436270in}{1.145759in}}%
\pgfpathlineto{\pgfqpoint{1.440505in}{1.193446in}}%
\pgfpathlineto{\pgfqpoint{1.443329in}{1.212508in}}%
\pgfpathlineto{\pgfqpoint{1.444741in}{1.216620in}}%
\pgfpathlineto{\pgfqpoint{1.446152in}{1.217028in}}%
\pgfpathlineto{\pgfqpoint{1.447564in}{1.213495in}}%
\pgfpathlineto{\pgfqpoint{1.450388in}{1.194434in}}%
\pgfpathlineto{\pgfqpoint{1.457446in}{1.135144in}}%
\pgfpathlineto{\pgfqpoint{1.461682in}{1.092613in}}%
\pgfpathlineto{\pgfqpoint{1.463093in}{1.085398in}}%
\pgfpathlineto{\pgfqpoint{1.464505in}{1.084382in}}%
\pgfpathlineto{\pgfqpoint{1.465917in}{1.090471in}}%
\pgfpathlineto{\pgfqpoint{1.468740in}{1.122110in}}%
\pgfpathlineto{\pgfqpoint{1.481446in}{1.312610in}}%
\pgfpathlineto{\pgfqpoint{1.484269in}{1.327779in}}%
\pgfpathlineto{\pgfqpoint{1.485681in}{1.330365in}}%
\pgfpathlineto{\pgfqpoint{1.487093in}{1.329948in}}%
\pgfpathlineto{\pgfqpoint{1.488505in}{1.326718in}}%
\pgfpathlineto{\pgfqpoint{1.491328in}{1.310813in}}%
\pgfpathlineto{\pgfqpoint{1.495563in}{1.264995in}}%
\pgfpathlineto{\pgfqpoint{1.499798in}{1.204299in}}%
\pgfpathlineto{\pgfqpoint{1.505445in}{1.089884in}}%
\pgfpathlineto{\pgfqpoint{1.512504in}{0.923390in}}%
\pgfpathlineto{\pgfqpoint{1.515328in}{0.906421in}}%
\pgfpathlineto{\pgfqpoint{1.516739in}{0.905696in}}%
\pgfpathlineto{\pgfqpoint{1.518151in}{0.908715in}}%
\pgfpathlineto{\pgfqpoint{1.519563in}{0.917312in}}%
\pgfpathlineto{\pgfqpoint{1.522386in}{0.967355in}}%
\pgfpathlineto{\pgfqpoint{1.533680in}{1.179172in}}%
\pgfpathlineto{\pgfqpoint{1.536504in}{1.202349in}}%
\pgfpathlineto{\pgfqpoint{1.537915in}{1.206588in}}%
\pgfpathlineto{\pgfqpoint{1.539327in}{1.205164in}}%
\pgfpathlineto{\pgfqpoint{1.542151in}{1.185920in}}%
\pgfpathlineto{\pgfqpoint{1.554856in}{1.044125in}}%
\pgfpathlineto{\pgfqpoint{1.556268in}{1.045783in}}%
\pgfpathlineto{\pgfqpoint{1.559091in}{1.076154in}}%
\pgfpathlineto{\pgfqpoint{1.563327in}{1.178885in}}%
\pgfpathlineto{\pgfqpoint{1.573209in}{1.440705in}}%
\pgfpathlineto{\pgfqpoint{1.578856in}{1.539202in}}%
\pgfpathlineto{\pgfqpoint{1.583091in}{1.573282in}}%
\pgfpathlineto{\pgfqpoint{1.584503in}{1.576290in}}%
\pgfpathlineto{\pgfqpoint{1.587326in}{1.566476in}}%
\pgfpathlineto{\pgfqpoint{1.590150in}{1.545894in}}%
\pgfpathlineto{\pgfqpoint{1.594385in}{1.493260in}}%
\pgfpathlineto{\pgfqpoint{1.600032in}{1.390218in}}%
\pgfpathlineto{\pgfqpoint{1.607091in}{1.211059in}}%
\pgfpathlineto{\pgfqpoint{1.616973in}{0.933083in}}%
\pgfpathlineto{\pgfqpoint{1.621208in}{0.840362in}}%
\pgfpathlineto{\pgfqpoint{1.624031in}{0.816068in}}%
\pgfpathlineto{\pgfqpoint{1.625443in}{0.812885in}}%
\pgfpathlineto{\pgfqpoint{1.626855in}{0.814306in}}%
\pgfpathlineto{\pgfqpoint{1.629678in}{0.845232in}}%
\pgfpathlineto{\pgfqpoint{1.635325in}{0.913656in}}%
\pgfpathlineto{\pgfqpoint{1.638149in}{0.929551in}}%
\pgfpathlineto{\pgfqpoint{1.640972in}{0.935083in}}%
\pgfpathlineto{\pgfqpoint{1.643796in}{0.914501in}}%
\pgfpathlineto{\pgfqpoint{1.646619in}{0.887248in}}%
\pgfpathlineto{\pgfqpoint{1.649443in}{0.893620in}}%
\pgfpathlineto{\pgfqpoint{1.652266in}{0.922214in}}%
\pgfpathlineto{\pgfqpoint{1.655090in}{0.976579in}}%
\pgfpathlineto{\pgfqpoint{1.667795in}{1.396642in}}%
\pgfpathlineto{\pgfqpoint{1.672031in}{1.474617in}}%
\pgfpathlineto{\pgfqpoint{1.676266in}{1.513625in}}%
\pgfpathlineto{\pgfqpoint{1.677677in}{1.517942in}}%
\pgfpathlineto{\pgfqpoint{1.679089in}{1.518217in}}%
\pgfpathlineto{\pgfqpoint{1.680501in}{1.514954in}}%
\pgfpathlineto{\pgfqpoint{1.683324in}{1.500084in}}%
\pgfpathlineto{\pgfqpoint{1.688971in}{1.447048in}}%
\pgfpathlineto{\pgfqpoint{1.705912in}{1.254538in}}%
\pgfpathlineto{\pgfqpoint{1.712971in}{1.142074in}}%
\pgfpathlineto{\pgfqpoint{1.725677in}{0.859018in}}%
\pgfpathlineto{\pgfqpoint{1.729912in}{0.808249in}}%
\pgfpathlineto{\pgfqpoint{1.734147in}{0.779004in}}%
\pgfpathlineto{\pgfqpoint{1.735559in}{0.780703in}}%
\pgfpathlineto{\pgfqpoint{1.738382in}{0.800672in}}%
\pgfpathlineto{\pgfqpoint{1.741206in}{0.833657in}}%
\pgfpathlineto{\pgfqpoint{1.744029in}{0.900370in}}%
\pgfpathlineto{\pgfqpoint{1.749676in}{1.100657in}}%
\pgfpathlineto{\pgfqpoint{1.759558in}{1.466662in}}%
\pgfpathlineto{\pgfqpoint{1.763793in}{1.550385in}}%
\pgfpathlineto{\pgfqpoint{1.766617in}{1.576564in}}%
\pgfpathlineto{\pgfqpoint{1.768029in}{1.581527in}}%
\pgfpathlineto{\pgfqpoint{1.769440in}{1.581298in}}%
\pgfpathlineto{\pgfqpoint{1.770852in}{1.576274in}}%
\pgfpathlineto{\pgfqpoint{1.773676in}{1.553534in}}%
\pgfpathlineto{\pgfqpoint{1.779323in}{1.472427in}}%
\pgfpathlineto{\pgfqpoint{1.793440in}{1.253331in}}%
\pgfpathlineto{\pgfqpoint{1.799087in}{1.202135in}}%
\pgfpathlineto{\pgfqpoint{1.817440in}{1.039951in}}%
\pgfpathlineto{\pgfqpoint{1.825910in}{0.955821in}}%
\pgfpathlineto{\pgfqpoint{1.832969in}{0.897726in}}%
\pgfpathlineto{\pgfqpoint{1.834380in}{0.895276in}}%
\pgfpathlineto{\pgfqpoint{1.837204in}{0.895699in}}%
\pgfpathlineto{\pgfqpoint{1.838616in}{0.900293in}}%
\pgfpathlineto{\pgfqpoint{1.841439in}{0.924128in}}%
\pgfpathlineto{\pgfqpoint{1.845674in}{0.998146in}}%
\pgfpathlineto{\pgfqpoint{1.852733in}{1.192221in}}%
\pgfpathlineto{\pgfqpoint{1.859792in}{1.378970in}}%
\pgfpathlineto{\pgfqpoint{1.864027in}{1.452125in}}%
\pgfpathlineto{\pgfqpoint{1.866850in}{1.477621in}}%
\pgfpathlineto{\pgfqpoint{1.868262in}{1.482892in}}%
\pgfpathlineto{\pgfqpoint{1.869674in}{1.483629in}}%
\pgfpathlineto{\pgfqpoint{1.871086in}{1.480554in}}%
\pgfpathlineto{\pgfqpoint{1.873909in}{1.464292in}}%
\pgfpathlineto{\pgfqpoint{1.879556in}{1.410509in}}%
\pgfpathlineto{\pgfqpoint{1.885203in}{1.337194in}}%
\pgfpathlineto{\pgfqpoint{1.907791in}{0.974072in}}%
\pgfpathlineto{\pgfqpoint{1.909202in}{0.972247in}}%
\pgfpathlineto{\pgfqpoint{1.910614in}{0.975338in}}%
\pgfpathlineto{\pgfqpoint{1.913438in}{0.998141in}}%
\pgfpathlineto{\pgfqpoint{1.916261in}{1.039987in}}%
\pgfpathlineto{\pgfqpoint{1.926143in}{1.260225in}}%
\pgfpathlineto{\pgfqpoint{1.927555in}{1.272007in}}%
\pgfpathlineto{\pgfqpoint{1.928967in}{1.273672in}}%
\pgfpathlineto{\pgfqpoint{1.930379in}{1.268338in}}%
\pgfpathlineto{\pgfqpoint{1.933202in}{1.234267in}}%
\pgfpathlineto{\pgfqpoint{1.945908in}{1.004553in}}%
\pgfpathlineto{\pgfqpoint{1.948731in}{0.990961in}}%
\pgfpathlineto{\pgfqpoint{1.950143in}{0.996629in}}%
\pgfpathlineto{\pgfqpoint{1.952966in}{1.029607in}}%
\pgfpathlineto{\pgfqpoint{1.968495in}{1.310419in}}%
\pgfpathlineto{\pgfqpoint{1.971319in}{1.332443in}}%
\pgfpathlineto{\pgfqpoint{1.974142in}{1.340649in}}%
\pgfpathlineto{\pgfqpoint{1.975554in}{1.339707in}}%
\pgfpathlineto{\pgfqpoint{1.976966in}{1.335485in}}%
\pgfpathlineto{\pgfqpoint{1.979789in}{1.315971in}}%
\pgfpathlineto{\pgfqpoint{1.984025in}{1.262029in}}%
\pgfpathlineto{\pgfqpoint{1.989672in}{1.147663in}}%
\pgfpathlineto{\pgfqpoint{1.996730in}{0.979017in}}%
\pgfpathlineto{\pgfqpoint{2.000965in}{0.879509in}}%
\pgfpathlineto{\pgfqpoint{2.003789in}{0.860973in}}%
\pgfpathlineto{\pgfqpoint{2.005201in}{0.865357in}}%
\pgfpathlineto{\pgfqpoint{2.008024in}{0.903196in}}%
\pgfpathlineto{\pgfqpoint{2.010848in}{0.977894in}}%
\pgfpathlineto{\pgfqpoint{2.016495in}{1.130691in}}%
\pgfpathlineto{\pgfqpoint{2.020730in}{1.194135in}}%
\pgfpathlineto{\pgfqpoint{2.023553in}{1.209812in}}%
\pgfpathlineto{\pgfqpoint{2.024965in}{1.209427in}}%
\pgfpathlineto{\pgfqpoint{2.026377in}{1.203770in}}%
\pgfpathlineto{\pgfqpoint{2.032024in}{1.154495in}}%
\pgfpathlineto{\pgfqpoint{2.040494in}{1.062248in}}%
\pgfpathlineto{\pgfqpoint{2.041906in}{1.059203in}}%
\pgfpathlineto{\pgfqpoint{2.043318in}{1.063676in}}%
\pgfpathlineto{\pgfqpoint{2.046141in}{1.095320in}}%
\pgfpathlineto{\pgfqpoint{2.051788in}{1.214324in}}%
\pgfpathlineto{\pgfqpoint{2.058847in}{1.354931in}}%
\pgfpathlineto{\pgfqpoint{2.063082in}{1.406771in}}%
\pgfpathlineto{\pgfqpoint{2.065905in}{1.420904in}}%
\pgfpathlineto{\pgfqpoint{2.067317in}{1.421380in}}%
\pgfpathlineto{\pgfqpoint{2.070141in}{1.407251in}}%
\pgfpathlineto{\pgfqpoint{2.074376in}{1.366812in}}%
\pgfpathlineto{\pgfqpoint{2.080023in}{1.284904in}}%
\pgfpathlineto{\pgfqpoint{2.087081in}{1.138476in}}%
\pgfpathlineto{\pgfqpoint{2.098375in}{0.896439in}}%
\pgfpathlineto{\pgfqpoint{2.101199in}{0.889326in}}%
\pgfpathlineto{\pgfqpoint{2.104022in}{0.902781in}}%
\pgfpathlineto{\pgfqpoint{2.105434in}{0.917653in}}%
\pgfpathlineto{\pgfqpoint{2.116728in}{1.148006in}}%
\pgfpathlineto{\pgfqpoint{2.119551in}{1.166609in}}%
\pgfpathlineto{\pgfqpoint{2.120963in}{1.166614in}}%
\pgfpathlineto{\pgfqpoint{2.122375in}{1.160833in}}%
\pgfpathlineto{\pgfqpoint{2.129434in}{1.102184in}}%
\pgfpathlineto{\pgfqpoint{2.137904in}{1.029379in}}%
\pgfpathlineto{\pgfqpoint{2.142139in}{1.072744in}}%
\pgfpathlineto{\pgfqpoint{2.144963in}{1.128061in}}%
\pgfpathlineto{\pgfqpoint{2.153433in}{1.312909in}}%
\pgfpathlineto{\pgfqpoint{2.159080in}{1.392574in}}%
\pgfpathlineto{\pgfqpoint{2.161904in}{1.412809in}}%
\pgfpathlineto{\pgfqpoint{2.163315in}{1.417064in}}%
\pgfpathlineto{\pgfqpoint{2.164727in}{1.417166in}}%
\pgfpathlineto{\pgfqpoint{2.166139in}{1.413648in}}%
\pgfpathlineto{\pgfqpoint{2.168962in}{1.397981in}}%
\pgfpathlineto{\pgfqpoint{2.173197in}{1.359968in}}%
\pgfpathlineto{\pgfqpoint{2.181668in}{1.249083in}}%
\pgfpathlineto{\pgfqpoint{2.191550in}{1.063073in}}%
\pgfpathlineto{\pgfqpoint{2.200020in}{0.902850in}}%
\pgfpathlineto{\pgfqpoint{2.201432in}{0.900963in}}%
\pgfpathlineto{\pgfqpoint{2.202844in}{0.903076in}}%
\pgfpathlineto{\pgfqpoint{2.205667in}{0.927956in}}%
\pgfpathlineto{\pgfqpoint{2.208491in}{0.974496in}}%
\pgfpathlineto{\pgfqpoint{2.214138in}{1.115688in}}%
\pgfpathlineto{\pgfqpoint{2.218373in}{1.187277in}}%
\pgfpathlineto{\pgfqpoint{2.221197in}{1.214392in}}%
\pgfpathlineto{\pgfqpoint{2.222608in}{1.218202in}}%
\pgfpathlineto{\pgfqpoint{2.224020in}{1.218388in}}%
\pgfpathlineto{\pgfqpoint{2.225432in}{1.213194in}}%
\pgfpathlineto{\pgfqpoint{2.228255in}{1.187140in}}%
\pgfpathlineto{\pgfqpoint{2.238137in}{1.066459in}}%
\pgfpathlineto{\pgfqpoint{2.240961in}{1.045491in}}%
\pgfpathlineto{\pgfqpoint{2.242373in}{1.038489in}}%
\pgfpathlineto{\pgfqpoint{2.243784in}{1.039227in}}%
\pgfpathlineto{\pgfqpoint{2.245196in}{1.045781in}}%
\pgfpathlineto{\pgfqpoint{2.248020in}{1.075503in}}%
\pgfpathlineto{\pgfqpoint{2.255078in}{1.205478in}}%
\pgfpathlineto{\pgfqpoint{2.262137in}{1.326413in}}%
\pgfpathlineto{\pgfqpoint{2.264960in}{1.352628in}}%
\pgfpathlineto{\pgfqpoint{2.267784in}{1.365121in}}%
\pgfpathlineto{\pgfqpoint{2.269196in}{1.365490in}}%
\pgfpathlineto{\pgfqpoint{2.270607in}{1.362191in}}%
\pgfpathlineto{\pgfqpoint{2.273431in}{1.345407in}}%
\pgfpathlineto{\pgfqpoint{2.276254in}{1.313869in}}%
\pgfpathlineto{\pgfqpoint{2.281901in}{1.213735in}}%
\pgfpathlineto{\pgfqpoint{2.287548in}{1.058678in}}%
\pgfpathlineto{\pgfqpoint{2.290372in}{0.988987in}}%
\pgfpathlineto{\pgfqpoint{2.293195in}{0.977276in}}%
\pgfpathlineto{\pgfqpoint{2.294607in}{0.984815in}}%
\pgfpathlineto{\pgfqpoint{2.297430in}{1.027408in}}%
\pgfpathlineto{\pgfqpoint{2.300254in}{1.104136in}}%
\pgfpathlineto{\pgfqpoint{2.307313in}{1.321299in}}%
\pgfpathlineto{\pgfqpoint{2.311548in}{1.389067in}}%
\pgfpathlineto{\pgfqpoint{2.314371in}{1.405700in}}%
\pgfpathlineto{\pgfqpoint{2.315783in}{1.406054in}}%
\pgfpathlineto{\pgfqpoint{2.317195in}{1.401888in}}%
\pgfpathlineto{\pgfqpoint{2.320018in}{1.379143in}}%
\pgfpathlineto{\pgfqpoint{2.324253in}{1.315125in}}%
\pgfpathlineto{\pgfqpoint{2.332724in}{1.161226in}}%
\pgfpathlineto{\pgfqpoint{2.336959in}{1.112357in}}%
\pgfpathlineto{\pgfqpoint{2.338371in}{1.105746in}}%
\pgfpathlineto{\pgfqpoint{2.339783in}{1.108989in}}%
\pgfpathlineto{\pgfqpoint{2.342606in}{1.137344in}}%
\pgfpathlineto{\pgfqpoint{2.346841in}{1.219661in}}%
\pgfpathlineto{\pgfqpoint{2.358135in}{1.459924in}}%
\pgfpathlineto{\pgfqpoint{2.362370in}{1.510755in}}%
\pgfpathlineto{\pgfqpoint{2.365194in}{1.523639in}}%
\pgfpathlineto{\pgfqpoint{2.366606in}{1.525676in}}%
\pgfpathlineto{\pgfqpoint{2.368017in}{1.523858in}}%
\pgfpathlineto{\pgfqpoint{2.370841in}{1.505764in}}%
\pgfpathlineto{\pgfqpoint{2.373664in}{1.467675in}}%
\pgfpathlineto{\pgfqpoint{2.380723in}{1.319081in}}%
\pgfpathlineto{\pgfqpoint{2.386370in}{1.159542in}}%
\pgfpathlineto{\pgfqpoint{2.393429in}{0.950630in}}%
\pgfpathlineto{\pgfqpoint{2.394840in}{0.936792in}}%
\pgfpathlineto{\pgfqpoint{2.396252in}{0.934642in}}%
\pgfpathlineto{\pgfqpoint{2.397664in}{0.943638in}}%
\pgfpathlineto{\pgfqpoint{2.400487in}{0.992149in}}%
\pgfpathlineto{\pgfqpoint{2.403311in}{1.084413in}}%
\pgfpathlineto{\pgfqpoint{2.410369in}{1.343757in}}%
\pgfpathlineto{\pgfqpoint{2.414605in}{1.431368in}}%
\pgfpathlineto{\pgfqpoint{2.417428in}{1.450349in}}%
\pgfpathlineto{\pgfqpoint{2.418840in}{1.446335in}}%
\pgfpathlineto{\pgfqpoint{2.421663in}{1.422637in}}%
\pgfpathlineto{\pgfqpoint{2.425899in}{1.357100in}}%
\pgfpathlineto{\pgfqpoint{2.434369in}{1.206773in}}%
\pgfpathlineto{\pgfqpoint{2.440016in}{1.137441in}}%
\pgfpathlineto{\pgfqpoint{2.441428in}{1.131883in}}%
\pgfpathlineto{\pgfqpoint{2.442839in}{1.132337in}}%
\pgfpathlineto{\pgfqpoint{2.444251in}{1.138787in}}%
\pgfpathlineto{\pgfqpoint{2.447075in}{1.168446in}}%
\pgfpathlineto{\pgfqpoint{2.452722in}{1.270655in}}%
\pgfpathlineto{\pgfqpoint{2.459780in}{1.399254in}}%
\pgfpathlineto{\pgfqpoint{2.464015in}{1.443246in}}%
\pgfpathlineto{\pgfqpoint{2.466839in}{1.452586in}}%
\pgfpathlineto{\pgfqpoint{2.468251in}{1.449935in}}%
\pgfpathlineto{\pgfqpoint{2.471074in}{1.428522in}}%
\pgfpathlineto{\pgfqpoint{2.475309in}{1.360412in}}%
\pgfpathlineto{\pgfqpoint{2.483780in}{1.169715in}}%
\pgfpathlineto{\pgfqpoint{2.490839in}{0.974217in}}%
\pgfpathlineto{\pgfqpoint{2.493662in}{0.955306in}}%
\pgfpathlineto{\pgfqpoint{2.495074in}{0.958904in}}%
\pgfpathlineto{\pgfqpoint{2.497897in}{0.994196in}}%
\pgfpathlineto{\pgfqpoint{2.500721in}{1.076082in}}%
\pgfpathlineto{\pgfqpoint{2.510603in}{1.421513in}}%
\pgfpathlineto{\pgfqpoint{2.514838in}{1.489348in}}%
\pgfpathlineto{\pgfqpoint{2.517662in}{1.500809in}}%
\pgfpathlineto{\pgfqpoint{2.519073in}{1.497479in}}%
\pgfpathlineto{\pgfqpoint{2.520485in}{1.488640in}}%
\pgfpathlineto{\pgfqpoint{2.524720in}{1.433064in}}%
\pgfpathlineto{\pgfqpoint{2.530367in}{1.324171in}}%
\pgfpathlineto{\pgfqpoint{2.537426in}{1.186066in}}%
\pgfpathlineto{\pgfqpoint{2.543073in}{1.114980in}}%
\pgfpathlineto{\pgfqpoint{2.545896in}{1.098597in}}%
\pgfpathlineto{\pgfqpoint{2.547308in}{1.096774in}}%
\pgfpathlineto{\pgfqpoint{2.548720in}{1.099033in}}%
\pgfpathlineto{\pgfqpoint{2.551543in}{1.113201in}}%
\pgfpathlineto{\pgfqpoint{2.557190in}{1.163415in}}%
\pgfpathlineto{\pgfqpoint{2.562837in}{1.210796in}}%
\pgfpathlineto{\pgfqpoint{2.565661in}{1.223071in}}%
\pgfpathlineto{\pgfqpoint{2.568484in}{1.230443in}}%
\pgfpathlineto{\pgfqpoint{2.571308in}{1.232628in}}%
\pgfpathlineto{\pgfqpoint{2.574131in}{1.232395in}}%
\pgfpathlineto{\pgfqpoint{2.576955in}{1.227518in}}%
\pgfpathlineto{\pgfqpoint{2.579778in}{1.215820in}}%
\pgfpathlineto{\pgfqpoint{2.584013in}{1.186842in}}%
\pgfpathlineto{\pgfqpoint{2.591072in}{1.121652in}}%
\pgfpathlineto{\pgfqpoint{2.595307in}{1.064667in}}%
\pgfpathlineto{\pgfqpoint{2.599542in}{0.983127in}}%
\pgfpathlineto{\pgfqpoint{2.606601in}{0.850581in}}%
\pgfpathlineto{\pgfqpoint{2.609424in}{0.823534in}}%
\pgfpathlineto{\pgfqpoint{2.612248in}{0.816371in}}%
\pgfpathlineto{\pgfqpoint{2.613660in}{0.819408in}}%
\pgfpathlineto{\pgfqpoint{2.619307in}{0.860746in}}%
\pgfpathlineto{\pgfqpoint{2.629189in}{0.953110in}}%
\pgfpathlineto{\pgfqpoint{2.632012in}{0.964510in}}%
\pgfpathlineto{\pgfqpoint{2.640483in}{0.981943in}}%
\pgfpathlineto{\pgfqpoint{2.643306in}{0.986934in}}%
\pgfpathlineto{\pgfqpoint{2.644718in}{0.987462in}}%
\pgfpathlineto{\pgfqpoint{2.646130in}{0.986235in}}%
\pgfpathlineto{\pgfqpoint{2.650365in}{0.977019in}}%
\pgfpathlineto{\pgfqpoint{2.658835in}{0.943713in}}%
\pgfpathlineto{\pgfqpoint{2.661659in}{0.939640in}}%
\pgfpathlineto{\pgfqpoint{2.663071in}{0.940136in}}%
\pgfpathlineto{\pgfqpoint{2.664482in}{0.942664in}}%
\pgfpathlineto{\pgfqpoint{2.667306in}{0.956074in}}%
\pgfpathlineto{\pgfqpoint{2.670129in}{0.983380in}}%
\pgfpathlineto{\pgfqpoint{2.674364in}{1.054401in}}%
\pgfpathlineto{\pgfqpoint{2.685658in}{1.303217in}}%
\pgfpathlineto{\pgfqpoint{2.687070in}{1.311257in}}%
\pgfpathlineto{\pgfqpoint{2.688482in}{1.302600in}}%
\pgfpathlineto{\pgfqpoint{2.691305in}{1.211335in}}%
\pgfpathlineto{\pgfqpoint{2.694129in}{1.052009in}}%
\pgfpathlineto{\pgfqpoint{2.696952in}{1.016472in}}%
\pgfpathlineto{\pgfqpoint{2.698364in}{1.019772in}}%
\pgfpathlineto{\pgfqpoint{2.699776in}{1.014195in}}%
\pgfpathlineto{\pgfqpoint{2.702599in}{0.989033in}}%
\pgfpathlineto{\pgfqpoint{2.705423in}{0.965682in}}%
\pgfpathlineto{\pgfqpoint{2.706834in}{0.961782in}}%
\pgfpathlineto{\pgfqpoint{2.708246in}{0.951921in}}%
\pgfpathlineto{\pgfqpoint{2.709658in}{0.953681in}}%
\pgfpathlineto{\pgfqpoint{2.713893in}{0.916085in}}%
\pgfpathlineto{\pgfqpoint{2.719540in}{0.819170in}}%
\pgfpathlineto{\pgfqpoint{2.720952in}{0.812648in}}%
\pgfpathlineto{\pgfqpoint{2.722364in}{0.818313in}}%
\pgfpathlineto{\pgfqpoint{2.725187in}{0.850928in}}%
\pgfpathlineto{\pgfqpoint{2.728010in}{0.901749in}}%
\pgfpathlineto{\pgfqpoint{2.736481in}{1.079732in}}%
\pgfpathlineto{\pgfqpoint{2.739304in}{1.105787in}}%
\pgfpathlineto{\pgfqpoint{2.740716in}{1.106877in}}%
\pgfpathlineto{\pgfqpoint{2.743540in}{1.131169in}}%
\pgfpathlineto{\pgfqpoint{2.744951in}{1.133748in}}%
\pgfpathlineto{\pgfqpoint{2.746363in}{1.125799in}}%
\pgfpathlineto{\pgfqpoint{2.747775in}{1.128817in}}%
\pgfpathlineto{\pgfqpoint{2.749187in}{1.136546in}}%
\pgfpathlineto{\pgfqpoint{2.750598in}{1.135776in}}%
\pgfpathlineto{\pgfqpoint{2.759069in}{1.088569in}}%
\pgfpathlineto{\pgfqpoint{2.761892in}{1.051976in}}%
\pgfpathlineto{\pgfqpoint{2.773186in}{0.832970in}}%
\pgfpathlineto{\pgfqpoint{2.776010in}{0.868504in}}%
\pgfpathlineto{\pgfqpoint{2.778833in}{0.912594in}}%
\pgfpathlineto{\pgfqpoint{2.780245in}{0.919904in}}%
\pgfpathlineto{\pgfqpoint{2.781657in}{0.919943in}}%
\pgfpathlineto{\pgfqpoint{2.785892in}{0.905645in}}%
\pgfpathlineto{\pgfqpoint{2.790127in}{0.946398in}}%
\pgfpathlineto{\pgfqpoint{2.791539in}{0.954047in}}%
\pgfpathlineto{\pgfqpoint{2.792950in}{0.954875in}}%
\pgfpathlineto{\pgfqpoint{2.797186in}{0.949363in}}%
\pgfpathlineto{\pgfqpoint{2.798597in}{0.950505in}}%
\pgfpathlineto{\pgfqpoint{2.800009in}{0.949274in}}%
\pgfpathlineto{\pgfqpoint{2.802833in}{0.953134in}}%
\pgfpathlineto{\pgfqpoint{2.814126in}{0.993376in}}%
\pgfpathlineto{\pgfqpoint{2.816950in}{0.995635in}}%
\pgfpathlineto{\pgfqpoint{2.818362in}{0.995845in}}%
\pgfpathlineto{\pgfqpoint{2.828244in}{0.984151in}}%
\pgfpathlineto{\pgfqpoint{2.831067in}{0.980875in}}%
\pgfpathlineto{\pgfqpoint{2.833891in}{0.979043in}}%
\pgfpathlineto{\pgfqpoint{2.835303in}{0.979105in}}%
\pgfpathlineto{\pgfqpoint{2.836714in}{0.976282in}}%
\pgfpathlineto{\pgfqpoint{2.838126in}{0.977243in}}%
\pgfpathlineto{\pgfqpoint{2.839538in}{0.976788in}}%
\pgfpathlineto{\pgfqpoint{2.845185in}{0.981603in}}%
\pgfpathlineto{\pgfqpoint{2.846596in}{0.981702in}}%
\pgfpathlineto{\pgfqpoint{2.848008in}{0.983479in}}%
\pgfpathlineto{\pgfqpoint{2.850832in}{0.984083in}}%
\pgfpathlineto{\pgfqpoint{2.855067in}{0.984794in}}%
\pgfpathlineto{\pgfqpoint{2.856479in}{0.983647in}}%
\pgfpathlineto{\pgfqpoint{2.857890in}{0.984298in}}%
\pgfpathlineto{\pgfqpoint{2.860714in}{0.983486in}}%
\pgfpathlineto{\pgfqpoint{2.864949in}{0.983181in}}%
\pgfpathlineto{\pgfqpoint{2.869184in}{0.979771in}}%
\pgfpathlineto{\pgfqpoint{2.870596in}{0.980388in}}%
\pgfpathlineto{\pgfqpoint{2.877655in}{0.972249in}}%
\pgfpathlineto{\pgfqpoint{2.896007in}{0.961608in}}%
\pgfpathlineto{\pgfqpoint{2.900243in}{0.949619in}}%
\pgfpathlineto{\pgfqpoint{2.907301in}{0.932926in}}%
\pgfpathlineto{\pgfqpoint{2.911536in}{0.931446in}}%
\pgfpathlineto{\pgfqpoint{2.918595in}{0.932840in}}%
\pgfpathlineto{\pgfqpoint{2.921419in}{0.932525in}}%
\pgfpathlineto{\pgfqpoint{2.927066in}{0.931210in}}%
\pgfpathlineto{\pgfqpoint{2.929889in}{0.929009in}}%
\pgfpathlineto{\pgfqpoint{2.932712in}{0.928758in}}%
\pgfpathlineto{\pgfqpoint{2.935536in}{0.931708in}}%
\pgfpathlineto{\pgfqpoint{2.948242in}{0.963319in}}%
\pgfpathlineto{\pgfqpoint{2.953889in}{0.984713in}}%
\pgfpathlineto{\pgfqpoint{2.956712in}{0.988057in}}%
\pgfpathlineto{\pgfqpoint{2.963771in}{0.990253in}}%
\pgfpathlineto{\pgfqpoint{2.975065in}{1.014370in}}%
\pgfpathlineto{\pgfqpoint{2.977888in}{1.015841in}}%
\pgfpathlineto{\pgfqpoint{2.980712in}{1.018697in}}%
\pgfpathlineto{\pgfqpoint{2.987770in}{1.031210in}}%
\pgfpathlineto{\pgfqpoint{2.989182in}{1.032815in}}%
\pgfpathlineto{\pgfqpoint{2.993417in}{1.030562in}}%
\pgfpathlineto{\pgfqpoint{2.996241in}{1.026131in}}%
\pgfpathlineto{\pgfqpoint{3.011770in}{0.983741in}}%
\pgfpathlineto{\pgfqpoint{3.017417in}{0.971499in}}%
\pgfpathlineto{\pgfqpoint{3.021652in}{0.964060in}}%
\pgfpathlineto{\pgfqpoint{3.025887in}{0.958661in}}%
\pgfpathlineto{\pgfqpoint{3.028711in}{0.959395in}}%
\pgfpathlineto{\pgfqpoint{3.032946in}{0.969462in}}%
\pgfpathlineto{\pgfqpoint{3.040005in}{0.992237in}}%
\pgfpathlineto{\pgfqpoint{3.045652in}{1.004497in}}%
\pgfpathlineto{\pgfqpoint{3.052710in}{1.019124in}}%
\pgfpathlineto{\pgfqpoint{3.056945in}{1.028325in}}%
\pgfpathlineto{\pgfqpoint{3.068239in}{1.043634in}}%
\pgfpathlineto{\pgfqpoint{3.075298in}{1.041235in}}%
\pgfpathlineto{\pgfqpoint{3.080945in}{1.033202in}}%
\pgfpathlineto{\pgfqpoint{3.085180in}{1.020634in}}%
\pgfpathlineto{\pgfqpoint{3.093651in}{0.976668in}}%
\pgfpathlineto{\pgfqpoint{3.100709in}{0.932510in}}%
\pgfpathlineto{\pgfqpoint{3.107768in}{0.884529in}}%
\pgfpathlineto{\pgfqpoint{3.112003in}{0.871090in}}%
\pgfpathlineto{\pgfqpoint{3.114827in}{0.866331in}}%
\pgfpathlineto{\pgfqpoint{3.119062in}{0.860166in}}%
\pgfpathlineto{\pgfqpoint{3.120474in}{0.858980in}}%
\pgfpathlineto{\pgfqpoint{3.124709in}{0.860797in}}%
\pgfpathlineto{\pgfqpoint{3.130356in}{0.872012in}}%
\pgfpathlineto{\pgfqpoint{3.136003in}{0.889254in}}%
\pgfpathlineto{\pgfqpoint{3.143061in}{0.908759in}}%
\pgfpathlineto{\pgfqpoint{3.151532in}{0.924001in}}%
\pgfpathlineto{\pgfqpoint{3.161414in}{0.936691in}}%
\pgfpathlineto{\pgfqpoint{3.168473in}{0.952924in}}%
\pgfpathlineto{\pgfqpoint{3.179767in}{0.987463in}}%
\pgfpathlineto{\pgfqpoint{3.181178in}{0.987908in}}%
\pgfpathlineto{\pgfqpoint{3.184002in}{0.983497in}}%
\pgfpathlineto{\pgfqpoint{3.188237in}{0.960521in}}%
\pgfpathlineto{\pgfqpoint{3.193884in}{0.910614in}}%
\pgfpathlineto{\pgfqpoint{3.199531in}{0.862428in}}%
\pgfpathlineto{\pgfqpoint{3.203766in}{0.841179in}}%
\pgfpathlineto{\pgfqpoint{3.206590in}{0.833434in}}%
\pgfpathlineto{\pgfqpoint{3.212237in}{0.826956in}}%
\pgfpathlineto{\pgfqpoint{3.215060in}{0.827224in}}%
\pgfpathlineto{\pgfqpoint{3.223530in}{0.839356in}}%
\pgfpathlineto{\pgfqpoint{3.229177in}{0.845349in}}%
\pgfpathlineto{\pgfqpoint{3.233413in}{0.854692in}}%
\pgfpathlineto{\pgfqpoint{3.240471in}{0.879205in}}%
\pgfpathlineto{\pgfqpoint{3.246118in}{0.901288in}}%
\pgfpathlineto{\pgfqpoint{3.250354in}{0.920908in}}%
\pgfpathlineto{\pgfqpoint{3.265883in}{0.999879in}}%
\pgfpathlineto{\pgfqpoint{3.277177in}{1.035994in}}%
\pgfpathlineto{\pgfqpoint{3.280000in}{1.038592in}}%
\pgfpathlineto{\pgfqpoint{3.282823in}{1.034406in}}%
\pgfpathlineto{\pgfqpoint{3.287059in}{1.015351in}}%
\pgfpathlineto{\pgfqpoint{3.296941in}{0.948550in}}%
\pgfpathlineto{\pgfqpoint{3.301176in}{0.919525in}}%
\pgfpathlineto{\pgfqpoint{3.306823in}{0.897243in}}%
\pgfpathlineto{\pgfqpoint{3.309646in}{0.891108in}}%
\pgfpathlineto{\pgfqpoint{3.320940in}{0.873863in}}%
\pgfpathlineto{\pgfqpoint{3.322352in}{0.872943in}}%
\pgfpathlineto{\pgfqpoint{3.330823in}{0.858333in}}%
\pgfpathlineto{\pgfqpoint{3.333646in}{0.858301in}}%
\pgfpathlineto{\pgfqpoint{3.336470in}{0.862232in}}%
\pgfpathlineto{\pgfqpoint{3.340705in}{0.876603in}}%
\pgfpathlineto{\pgfqpoint{3.347763in}{0.906034in}}%
\pgfpathlineto{\pgfqpoint{3.354822in}{0.937983in}}%
\pgfpathlineto{\pgfqpoint{3.359057in}{0.954375in}}%
\pgfpathlineto{\pgfqpoint{3.366116in}{0.977583in}}%
\pgfpathlineto{\pgfqpoint{3.375998in}{1.011316in}}%
\pgfpathlineto{\pgfqpoint{3.378822in}{1.014555in}}%
\pgfpathlineto{\pgfqpoint{3.381645in}{1.011572in}}%
\pgfpathlineto{\pgfqpoint{3.385880in}{0.997653in}}%
\pgfpathlineto{\pgfqpoint{3.401409in}{0.923823in}}%
\pgfpathlineto{\pgfqpoint{3.416939in}{0.873168in}}%
\pgfpathlineto{\pgfqpoint{3.425409in}{0.851392in}}%
\pgfpathlineto{\pgfqpoint{3.428232in}{0.849641in}}%
\pgfpathlineto{\pgfqpoint{3.432468in}{0.854007in}}%
\pgfpathlineto{\pgfqpoint{3.438115in}{0.868405in}}%
\pgfpathlineto{\pgfqpoint{3.452232in}{0.921669in}}%
\pgfpathlineto{\pgfqpoint{3.462114in}{0.970309in}}%
\pgfpathlineto{\pgfqpoint{3.480467in}{1.043346in}}%
\pgfpathlineto{\pgfqpoint{3.481879in}{1.043669in}}%
\pgfpathlineto{\pgfqpoint{3.484702in}{1.040166in}}%
\pgfpathlineto{\pgfqpoint{3.488937in}{1.023920in}}%
\pgfpathlineto{\pgfqpoint{3.497408in}{0.973520in}}%
\pgfpathlineto{\pgfqpoint{3.510113in}{0.894947in}}%
\pgfpathlineto{\pgfqpoint{3.517172in}{0.872242in}}%
\pgfpathlineto{\pgfqpoint{3.524231in}{0.854411in}}%
\pgfpathlineto{\pgfqpoint{3.529878in}{0.843046in}}%
\pgfpathlineto{\pgfqpoint{3.532701in}{0.841057in}}%
\pgfpathlineto{\pgfqpoint{3.535525in}{0.840899in}}%
\pgfpathlineto{\pgfqpoint{3.538348in}{0.842761in}}%
\pgfpathlineto{\pgfqpoint{3.542583in}{0.850854in}}%
\pgfpathlineto{\pgfqpoint{3.549642in}{0.874464in}}%
\pgfpathlineto{\pgfqpoint{3.553877in}{0.888878in}}%
\pgfpathlineto{\pgfqpoint{3.556701in}{0.895226in}}%
\pgfpathlineto{\pgfqpoint{3.580700in}{0.970971in}}%
\pgfpathlineto{\pgfqpoint{3.584935in}{0.975146in}}%
\pgfpathlineto{\pgfqpoint{3.587759in}{0.973100in}}%
\pgfpathlineto{\pgfqpoint{3.591994in}{0.961590in}}%
\pgfpathlineto{\pgfqpoint{3.601876in}{0.924484in}}%
\pgfpathlineto{\pgfqpoint{3.610347in}{0.883598in}}%
\pgfpathlineto{\pgfqpoint{3.615994in}{0.860718in}}%
\pgfpathlineto{\pgfqpoint{3.621641in}{0.846751in}}%
\pgfpathlineto{\pgfqpoint{3.625876in}{0.840681in}}%
\pgfpathlineto{\pgfqpoint{3.630111in}{0.835923in}}%
\pgfpathlineto{\pgfqpoint{3.631523in}{0.836514in}}%
\pgfpathlineto{\pgfqpoint{3.637170in}{0.826655in}}%
\pgfpathlineto{\pgfqpoint{3.639993in}{0.824796in}}%
\pgfpathlineto{\pgfqpoint{3.642817in}{0.827737in}}%
\pgfpathlineto{\pgfqpoint{3.645640in}{0.835208in}}%
\pgfpathlineto{\pgfqpoint{3.649875in}{0.854331in}}%
\pgfpathlineto{\pgfqpoint{3.656934in}{0.885418in}}%
\pgfpathlineto{\pgfqpoint{3.659757in}{0.896469in}}%
\pgfpathlineto{\pgfqpoint{3.669640in}{0.933527in}}%
\pgfpathlineto{\pgfqpoint{3.676698in}{0.951909in}}%
\pgfpathlineto{\pgfqpoint{3.687992in}{0.966943in}}%
\pgfpathlineto{\pgfqpoint{3.690816in}{0.969664in}}%
\pgfpathlineto{\pgfqpoint{3.693639in}{0.968572in}}%
\pgfpathlineto{\pgfqpoint{3.697874in}{0.957905in}}%
\pgfpathlineto{\pgfqpoint{3.703521in}{0.934081in}}%
\pgfpathlineto{\pgfqpoint{3.714815in}{0.880437in}}%
\pgfpathlineto{\pgfqpoint{3.723286in}{0.852971in}}%
\pgfpathlineto{\pgfqpoint{3.728933in}{0.841166in}}%
\pgfpathlineto{\pgfqpoint{3.734580in}{0.830790in}}%
\pgfpathlineto{\pgfqpoint{3.737403in}{0.828278in}}%
\pgfpathlineto{\pgfqpoint{3.740227in}{0.827505in}}%
\pgfpathlineto{\pgfqpoint{3.744462in}{0.830434in}}%
\pgfpathlineto{\pgfqpoint{3.748697in}{0.837834in}}%
\pgfpathlineto{\pgfqpoint{3.752932in}{0.850034in}}%
\pgfpathlineto{\pgfqpoint{3.767050in}{0.899888in}}%
\pgfpathlineto{\pgfqpoint{3.769873in}{0.913039in}}%
\pgfpathlineto{\pgfqpoint{3.778343in}{0.935702in}}%
\pgfpathlineto{\pgfqpoint{3.782579in}{0.940839in}}%
\pgfpathlineto{\pgfqpoint{3.786814in}{0.943752in}}%
\pgfpathlineto{\pgfqpoint{3.793873in}{0.949130in}}%
\pgfpathlineto{\pgfqpoint{3.796696in}{0.948413in}}%
\pgfpathlineto{\pgfqpoint{3.803755in}{0.937586in}}%
\pgfpathlineto{\pgfqpoint{3.806578in}{0.929784in}}%
\pgfpathlineto{\pgfqpoint{3.822107in}{0.889878in}}%
\pgfpathlineto{\pgfqpoint{3.829166in}{0.877692in}}%
\pgfpathlineto{\pgfqpoint{3.836225in}{0.864608in}}%
\pgfpathlineto{\pgfqpoint{3.839048in}{0.864233in}}%
\pgfpathlineto{\pgfqpoint{3.843283in}{0.868733in}}%
\pgfpathlineto{\pgfqpoint{3.850342in}{0.882789in}}%
\pgfpathlineto{\pgfqpoint{3.855989in}{0.900798in}}%
\pgfpathlineto{\pgfqpoint{3.863048in}{0.926411in}}%
\pgfpathlineto{\pgfqpoint{3.865871in}{0.936330in}}%
\pgfpathlineto{\pgfqpoint{3.878577in}{0.985560in}}%
\pgfpathlineto{\pgfqpoint{3.884224in}{0.995931in}}%
\pgfpathlineto{\pgfqpoint{3.887047in}{0.997703in}}%
\pgfpathlineto{\pgfqpoint{3.889871in}{0.997864in}}%
\pgfpathlineto{\pgfqpoint{3.892694in}{0.996824in}}%
\pgfpathlineto{\pgfqpoint{3.896929in}{0.989574in}}%
\pgfpathlineto{\pgfqpoint{3.899753in}{0.981603in}}%
\pgfpathlineto{\pgfqpoint{3.905400in}{0.958530in}}%
\pgfpathlineto{\pgfqpoint{3.911047in}{0.932506in}}%
\pgfpathlineto{\pgfqpoint{3.925164in}{0.885349in}}%
\pgfpathlineto{\pgfqpoint{3.930811in}{0.862646in}}%
\pgfpathlineto{\pgfqpoint{3.940693in}{0.838216in}}%
\pgfpathlineto{\pgfqpoint{3.944929in}{0.833555in}}%
\pgfpathlineto{\pgfqpoint{3.950576in}{0.832567in}}%
\pgfpathlineto{\pgfqpoint{3.953399in}{0.833578in}}%
\pgfpathlineto{\pgfqpoint{3.956222in}{0.837593in}}%
\pgfpathlineto{\pgfqpoint{3.959046in}{0.846825in}}%
\pgfpathlineto{\pgfqpoint{3.981634in}{0.946782in}}%
\pgfpathlineto{\pgfqpoint{3.992928in}{0.965883in}}%
\pgfpathlineto{\pgfqpoint{3.997163in}{0.974060in}}%
\pgfpathlineto{\pgfqpoint{3.999986in}{0.976095in}}%
\pgfpathlineto{\pgfqpoint{4.001398in}{0.976651in}}%
\pgfpathlineto{\pgfqpoint{4.005633in}{0.972896in}}%
\pgfpathlineto{\pgfqpoint{4.012692in}{0.956661in}}%
\pgfpathlineto{\pgfqpoint{4.022574in}{0.920424in}}%
\pgfpathlineto{\pgfqpoint{4.035280in}{0.869297in}}%
\pgfpathlineto{\pgfqpoint{4.039515in}{0.856756in}}%
\pgfpathlineto{\pgfqpoint{4.042338in}{0.853293in}}%
\pgfpathlineto{\pgfqpoint{4.045162in}{0.852959in}}%
\pgfpathlineto{\pgfqpoint{4.057868in}{0.863320in}}%
\pgfpathlineto{\pgfqpoint{4.062103in}{0.872717in}}%
\pgfpathlineto{\pgfqpoint{4.074808in}{0.917654in}}%
\pgfpathlineto{\pgfqpoint{4.084691in}{0.965877in}}%
\pgfpathlineto{\pgfqpoint{4.088926in}{0.976164in}}%
\pgfpathlineto{\pgfqpoint{4.091749in}{0.978979in}}%
\pgfpathlineto{\pgfqpoint{4.094573in}{0.978797in}}%
\pgfpathlineto{\pgfqpoint{4.098808in}{0.972761in}}%
\pgfpathlineto{\pgfqpoint{4.103043in}{0.961483in}}%
\pgfpathlineto{\pgfqpoint{4.108690in}{0.942460in}}%
\pgfpathlineto{\pgfqpoint{4.121396in}{0.891681in}}%
\pgfpathlineto{\pgfqpoint{4.127043in}{0.870817in}}%
\pgfpathlineto{\pgfqpoint{4.136925in}{0.845857in}}%
\pgfpathlineto{\pgfqpoint{4.151042in}{0.823095in}}%
\pgfpathlineto{\pgfqpoint{4.158101in}{0.817400in}}%
\pgfpathlineto{\pgfqpoint{4.160924in}{0.818333in}}%
\pgfpathlineto{\pgfqpoint{4.163748in}{0.823765in}}%
\pgfpathlineto{\pgfqpoint{4.167983in}{0.839100in}}%
\pgfpathlineto{\pgfqpoint{4.183512in}{0.913515in}}%
\pgfpathlineto{\pgfqpoint{4.189159in}{0.943906in}}%
\pgfpathlineto{\pgfqpoint{4.196218in}{0.971619in}}%
\pgfpathlineto{\pgfqpoint{4.200453in}{0.981246in}}%
\pgfpathlineto{\pgfqpoint{4.210335in}{0.992513in}}%
\pgfpathlineto{\pgfqpoint{4.214571in}{0.991834in}}%
\pgfpathlineto{\pgfqpoint{4.217394in}{0.987994in}}%
\pgfpathlineto{\pgfqpoint{4.224453in}{0.962803in}}%
\pgfpathlineto{\pgfqpoint{4.232923in}{0.928577in}}%
\pgfpathlineto{\pgfqpoint{4.247040in}{0.911797in}}%
\pgfpathlineto{\pgfqpoint{4.249864in}{0.901259in}}%
\pgfpathlineto{\pgfqpoint{4.254099in}{0.875748in}}%
\pgfpathlineto{\pgfqpoint{4.258334in}{0.848668in}}%
\pgfpathlineto{\pgfqpoint{4.261158in}{0.843442in}}%
\pgfpathlineto{\pgfqpoint{4.262570in}{0.845745in}}%
\pgfpathlineto{\pgfqpoint{4.265393in}{0.857371in}}%
\pgfpathlineto{\pgfqpoint{4.272452in}{0.890005in}}%
\pgfpathlineto{\pgfqpoint{4.273863in}{0.894323in}}%
\pgfpathlineto{\pgfqpoint{4.275275in}{0.894895in}}%
\pgfpathlineto{\pgfqpoint{4.279510in}{0.899793in}}%
\pgfpathlineto{\pgfqpoint{4.283746in}{0.902376in}}%
\pgfpathlineto{\pgfqpoint{4.286569in}{0.911153in}}%
\pgfpathlineto{\pgfqpoint{4.297863in}{0.948733in}}%
\pgfpathlineto{\pgfqpoint{4.300687in}{0.953439in}}%
\pgfpathlineto{\pgfqpoint{4.303510in}{0.954918in}}%
\pgfpathlineto{\pgfqpoint{4.309157in}{0.949807in}}%
\pgfpathlineto{\pgfqpoint{4.313392in}{0.942295in}}%
\pgfpathlineto{\pgfqpoint{4.320451in}{0.919797in}}%
\pgfpathlineto{\pgfqpoint{4.324686in}{0.902851in}}%
\pgfpathlineto{\pgfqpoint{4.326098in}{0.901973in}}%
\pgfpathlineto{\pgfqpoint{4.333156in}{0.912134in}}%
\pgfpathlineto{\pgfqpoint{4.334568in}{0.909372in}}%
\pgfpathlineto{\pgfqpoint{4.340215in}{0.879453in}}%
\pgfpathlineto{\pgfqpoint{4.345862in}{0.866588in}}%
\pgfpathlineto{\pgfqpoint{4.348686in}{0.861423in}}%
\pgfpathlineto{\pgfqpoint{4.358568in}{0.833645in}}%
\pgfpathlineto{\pgfqpoint{4.359980in}{0.833050in}}%
\pgfpathlineto{\pgfqpoint{4.362803in}{0.837278in}}%
\pgfpathlineto{\pgfqpoint{4.375509in}{0.871287in}}%
\pgfpathlineto{\pgfqpoint{4.376920in}{0.871734in}}%
\pgfpathlineto{\pgfqpoint{4.383979in}{0.883807in}}%
\pgfpathlineto{\pgfqpoint{4.388214in}{0.889551in}}%
\pgfpathlineto{\pgfqpoint{4.393861in}{0.900400in}}%
\pgfpathlineto{\pgfqpoint{4.396685in}{0.908666in}}%
\pgfpathlineto{\pgfqpoint{4.402332in}{0.917908in}}%
\pgfpathlineto{\pgfqpoint{4.405155in}{0.918224in}}%
\pgfpathlineto{\pgfqpoint{4.407979in}{0.915332in}}%
\pgfpathlineto{\pgfqpoint{4.412214in}{0.909230in}}%
\pgfpathlineto{\pgfqpoint{4.419272in}{0.900352in}}%
\pgfpathlineto{\pgfqpoint{4.424919in}{0.894074in}}%
\pgfpathlineto{\pgfqpoint{4.426331in}{0.893520in}}%
\pgfpathlineto{\pgfqpoint{4.433390in}{0.883131in}}%
\pgfpathlineto{\pgfqpoint{4.448919in}{0.846952in}}%
\pgfpathlineto{\pgfqpoint{4.450331in}{0.846887in}}%
\pgfpathlineto{\pgfqpoint{4.454566in}{0.838817in}}%
\pgfpathlineto{\pgfqpoint{4.458801in}{0.833672in}}%
\pgfpathlineto{\pgfqpoint{4.463036in}{0.831742in}}%
\pgfpathlineto{\pgfqpoint{4.467272in}{0.833361in}}%
\pgfpathlineto{\pgfqpoint{4.471507in}{0.839897in}}%
\pgfpathlineto{\pgfqpoint{4.481389in}{0.855664in}}%
\pgfpathlineto{\pgfqpoint{4.488448in}{0.860489in}}%
\pgfpathlineto{\pgfqpoint{4.496918in}{0.860286in}}%
\pgfpathlineto{\pgfqpoint{4.508212in}{0.869726in}}%
\pgfpathlineto{\pgfqpoint{4.516682in}{0.872981in}}%
\pgfpathlineto{\pgfqpoint{4.520918in}{0.873893in}}%
\pgfpathlineto{\pgfqpoint{4.523741in}{0.872674in}}%
\pgfpathlineto{\pgfqpoint{4.530800in}{0.865493in}}%
\pgfpathlineto{\pgfqpoint{4.533623in}{0.862799in}}%
\pgfpathlineto{\pgfqpoint{4.537858in}{0.861570in}}%
\pgfpathlineto{\pgfqpoint{4.544917in}{0.864203in}}%
\pgfpathlineto{\pgfqpoint{4.550564in}{0.867572in}}%
\pgfpathlineto{\pgfqpoint{4.557623in}{0.866986in}}%
\pgfpathlineto{\pgfqpoint{4.561858in}{0.861643in}}%
\pgfpathlineto{\pgfqpoint{4.571740in}{0.845910in}}%
\pgfpathlineto{\pgfqpoint{4.581622in}{0.837680in}}%
\pgfpathlineto{\pgfqpoint{4.587269in}{0.826475in}}%
\pgfpathlineto{\pgfqpoint{4.604210in}{0.787529in}}%
\pgfpathlineto{\pgfqpoint{4.623974in}{0.760601in}}%
\pgfpathlineto{\pgfqpoint{4.633857in}{0.751151in}}%
\pgfpathlineto{\pgfqpoint{4.647974in}{0.741309in}}%
\pgfpathlineto{\pgfqpoint{4.664915in}{0.733025in}}%
\pgfpathlineto{\pgfqpoint{4.683267in}{0.726393in}}%
\pgfpathlineto{\pgfqpoint{4.745384in}{0.710280in}}%
\pgfpathlineto{\pgfqpoint{4.823030in}{0.698609in}}%
\pgfpathlineto{\pgfqpoint{4.825853in}{0.699865in}}%
\pgfpathlineto{\pgfqpoint{4.837147in}{0.699319in}}%
\pgfpathlineto{\pgfqpoint{4.839970in}{0.697650in}}%
\pgfpathlineto{\pgfqpoint{4.863970in}{0.696494in}}%
\pgfpathlineto{\pgfqpoint{4.866793in}{0.696000in}}%
\pgfpathlineto{\pgfqpoint{4.868205in}{0.698086in}}%
\pgfpathlineto{\pgfqpoint{4.882323in}{0.698426in}}%
\pgfpathlineto{\pgfqpoint{4.885146in}{0.696583in}}%
\pgfpathlineto{\pgfqpoint{4.887969in}{0.698012in}}%
\pgfpathlineto{\pgfqpoint{4.895028in}{0.698073in}}%
\pgfpathlineto{\pgfqpoint{4.897852in}{0.699131in}}%
\pgfpathlineto{\pgfqpoint{4.916204in}{0.699024in}}%
\pgfpathlineto{\pgfqpoint{4.919028in}{0.697698in}}%
\pgfpathlineto{\pgfqpoint{4.920439in}{0.697719in}}%
\pgfpathlineto{\pgfqpoint{4.923263in}{0.700251in}}%
\pgfpathlineto{\pgfqpoint{4.924675in}{0.700584in}}%
\pgfpathlineto{\pgfqpoint{4.928910in}{0.698795in}}%
\pgfpathlineto{\pgfqpoint{4.930322in}{0.699433in}}%
\pgfpathlineto{\pgfqpoint{4.931733in}{0.697420in}}%
\pgfpathlineto{\pgfqpoint{4.933145in}{0.697343in}}%
\pgfpathlineto{\pgfqpoint{4.935969in}{0.698544in}}%
\pgfpathlineto{\pgfqpoint{4.940204in}{0.696849in}}%
\pgfpathlineto{\pgfqpoint{4.945851in}{0.696388in}}%
\pgfpathlineto{\pgfqpoint{4.955733in}{0.697388in}}%
\pgfpathlineto{\pgfqpoint{4.958556in}{0.697681in}}%
\pgfpathlineto{\pgfqpoint{4.962792in}{0.696636in}}%
\pgfpathlineto{\pgfqpoint{4.965615in}{0.696751in}}%
\pgfpathlineto{\pgfqpoint{4.969850in}{0.697673in}}%
\pgfpathlineto{\pgfqpoint{4.972674in}{0.697051in}}%
\pgfpathlineto{\pgfqpoint{4.975497in}{0.696928in}}%
\pgfpathlineto{\pgfqpoint{4.985379in}{0.698238in}}%
\pgfpathlineto{\pgfqpoint{4.988203in}{0.696876in}}%
\pgfpathlineto{\pgfqpoint{4.991026in}{0.698558in}}%
\pgfpathlineto{\pgfqpoint{4.996673in}{0.697198in}}%
\pgfpathlineto{\pgfqpoint{5.003732in}{0.697808in}}%
\pgfpathlineto{\pgfqpoint{5.007967in}{0.696643in}}%
\pgfpathlineto{\pgfqpoint{5.030555in}{0.697093in}}%
\pgfpathlineto{\pgfqpoint{5.031967in}{0.696680in}}%
\pgfpathlineto{\pgfqpoint{5.034790in}{0.697938in}}%
\pgfpathlineto{\pgfqpoint{5.040437in}{0.698122in}}%
\pgfpathlineto{\pgfqpoint{5.041849in}{0.696000in}}%
\pgfpathlineto{\pgfqpoint{5.496429in}{0.696000in}}%
\pgfpathlineto{\pgfqpoint{5.497840in}{0.698149in}}%
\pgfpathlineto{\pgfqpoint{5.500664in}{0.696285in}}%
\pgfpathlineto{\pgfqpoint{5.520428in}{0.697132in}}%
\pgfpathlineto{\pgfqpoint{5.526075in}{0.696655in}}%
\pgfpathlineto{\pgfqpoint{5.527487in}{0.696823in}}%
\pgfpathlineto{\pgfqpoint{5.528899in}{0.698157in}}%
\pgfpathlineto{\pgfqpoint{5.531722in}{0.697239in}}%
\pgfpathlineto{\pgfqpoint{5.534545in}{0.696000in}}%
\pgfpathlineto{\pgfqpoint{5.534545in}{0.696000in}}%
\pgfusepath{stroke}%
\end{pgfscope}%
\begin{pgfscope}%
\pgfpathrectangle{\pgfqpoint{0.800000in}{0.528000in}}{\pgfqpoint{4.960000in}{3.696000in}} %
\pgfusepath{clip}%
\pgfsetrectcap%
\pgfsetroundjoin%
\pgfsetlinewidth{1.505625pt}%
\definecolor{currentstroke}{rgb}{0.839216,0.152941,0.156863}%
\pgfsetstrokecolor{currentstroke}%
\pgfsetdash{}{0pt}%
\pgfpathmoveto{\pgfqpoint{1.025455in}{0.696000in}}%
\pgfpathlineto{\pgfqpoint{1.165217in}{0.696254in}}%
\pgfpathlineto{\pgfqpoint{1.168040in}{0.703406in}}%
\pgfpathlineto{\pgfqpoint{1.169452in}{0.709552in}}%
\pgfpathlineto{\pgfqpoint{1.170864in}{0.710678in}}%
\pgfpathlineto{\pgfqpoint{1.173687in}{0.717193in}}%
\pgfpathlineto{\pgfqpoint{1.179334in}{0.766603in}}%
\pgfpathlineto{\pgfqpoint{1.192040in}{0.860385in}}%
\pgfpathlineto{\pgfqpoint{1.199098in}{0.931269in}}%
\pgfpathlineto{\pgfqpoint{1.227333in}{1.181888in}}%
\pgfpathlineto{\pgfqpoint{1.235803in}{1.232724in}}%
\pgfpathlineto{\pgfqpoint{1.241450in}{1.255828in}}%
\pgfpathlineto{\pgfqpoint{1.251333in}{1.290217in}}%
\pgfpathlineto{\pgfqpoint{1.259803in}{1.325649in}}%
\pgfpathlineto{\pgfqpoint{1.266862in}{1.342852in}}%
\pgfpathlineto{\pgfqpoint{1.272509in}{1.348826in}}%
\pgfpathlineto{\pgfqpoint{1.275332in}{1.348177in}}%
\pgfpathlineto{\pgfqpoint{1.278156in}{1.344643in}}%
\pgfpathlineto{\pgfqpoint{1.280979in}{1.336835in}}%
\pgfpathlineto{\pgfqpoint{1.288038in}{1.307969in}}%
\pgfpathlineto{\pgfqpoint{1.302155in}{1.241498in}}%
\pgfpathlineto{\pgfqpoint{1.306390in}{1.209015in}}%
\pgfpathlineto{\pgfqpoint{1.313449in}{1.131270in}}%
\pgfpathlineto{\pgfqpoint{1.321920in}{1.017749in}}%
\pgfpathlineto{\pgfqpoint{1.330390in}{0.915440in}}%
\pgfpathlineto{\pgfqpoint{1.333213in}{0.897347in}}%
\pgfpathlineto{\pgfqpoint{1.336037in}{0.891966in}}%
\pgfpathlineto{\pgfqpoint{1.337449in}{0.893952in}}%
\pgfpathlineto{\pgfqpoint{1.340272in}{0.905080in}}%
\pgfpathlineto{\pgfqpoint{1.347331in}{0.952788in}}%
\pgfpathlineto{\pgfqpoint{1.351566in}{0.977322in}}%
\pgfpathlineto{\pgfqpoint{1.358625in}{1.005617in}}%
\pgfpathlineto{\pgfqpoint{1.361448in}{1.027293in}}%
\pgfpathlineto{\pgfqpoint{1.364272in}{1.058005in}}%
\pgfpathlineto{\pgfqpoint{1.368507in}{1.136073in}}%
\pgfpathlineto{\pgfqpoint{1.375566in}{1.336839in}}%
\pgfpathlineto{\pgfqpoint{1.382624in}{1.532221in}}%
\pgfpathlineto{\pgfqpoint{1.386859in}{1.602364in}}%
\pgfpathlineto{\pgfqpoint{1.389683in}{1.628292in}}%
\pgfpathlineto{\pgfqpoint{1.391095in}{1.634410in}}%
\pgfpathlineto{\pgfqpoint{1.392506in}{1.634734in}}%
\pgfpathlineto{\pgfqpoint{1.393918in}{1.629205in}}%
\pgfpathlineto{\pgfqpoint{1.399565in}{1.574450in}}%
\pgfpathlineto{\pgfqpoint{1.403800in}{1.489013in}}%
\pgfpathlineto{\pgfqpoint{1.423565in}{1.014992in}}%
\pgfpathlineto{\pgfqpoint{1.426388in}{0.989893in}}%
\pgfpathlineto{\pgfqpoint{1.429212in}{0.984480in}}%
\pgfpathlineto{\pgfqpoint{1.430623in}{0.985822in}}%
\pgfpathlineto{\pgfqpoint{1.433447in}{1.005435in}}%
\pgfpathlineto{\pgfqpoint{1.436270in}{1.051219in}}%
\pgfpathlineto{\pgfqpoint{1.446152in}{1.237216in}}%
\pgfpathlineto{\pgfqpoint{1.448976in}{1.265554in}}%
\pgfpathlineto{\pgfqpoint{1.450388in}{1.272652in}}%
\pgfpathlineto{\pgfqpoint{1.451799in}{1.274049in}}%
\pgfpathlineto{\pgfqpoint{1.453211in}{1.269042in}}%
\pgfpathlineto{\pgfqpoint{1.456035in}{1.232090in}}%
\pgfpathlineto{\pgfqpoint{1.460270in}{1.161692in}}%
\pgfpathlineto{\pgfqpoint{1.464505in}{1.108535in}}%
\pgfpathlineto{\pgfqpoint{1.465917in}{1.108030in}}%
\pgfpathlineto{\pgfqpoint{1.467329in}{1.123952in}}%
\pgfpathlineto{\pgfqpoint{1.470152in}{1.207911in}}%
\pgfpathlineto{\pgfqpoint{1.482858in}{1.735730in}}%
\pgfpathlineto{\pgfqpoint{1.485681in}{1.777725in}}%
\pgfpathlineto{\pgfqpoint{1.488505in}{1.789276in}}%
\pgfpathlineto{\pgfqpoint{1.489916in}{1.787063in}}%
\pgfpathlineto{\pgfqpoint{1.492740in}{1.761134in}}%
\pgfpathlineto{\pgfqpoint{1.496975in}{1.677112in}}%
\pgfpathlineto{\pgfqpoint{1.502622in}{1.505604in}}%
\pgfpathlineto{\pgfqpoint{1.511092in}{1.164039in}}%
\pgfpathlineto{\pgfqpoint{1.515328in}{1.026132in}}%
\pgfpathlineto{\pgfqpoint{1.518151in}{0.991445in}}%
\pgfpathlineto{\pgfqpoint{1.519563in}{0.988323in}}%
\pgfpathlineto{\pgfqpoint{1.520975in}{0.995628in}}%
\pgfpathlineto{\pgfqpoint{1.523798in}{1.033060in}}%
\pgfpathlineto{\pgfqpoint{1.526622in}{1.102685in}}%
\pgfpathlineto{\pgfqpoint{1.536504in}{1.368289in}}%
\pgfpathlineto{\pgfqpoint{1.540739in}{1.440786in}}%
\pgfpathlineto{\pgfqpoint{1.543562in}{1.467662in}}%
\pgfpathlineto{\pgfqpoint{1.544974in}{1.472372in}}%
\pgfpathlineto{\pgfqpoint{1.546386in}{1.467723in}}%
\pgfpathlineto{\pgfqpoint{1.547798in}{1.455961in}}%
\pgfpathlineto{\pgfqpoint{1.550621in}{1.402989in}}%
\pgfpathlineto{\pgfqpoint{1.554856in}{1.310032in}}%
\pgfpathlineto{\pgfqpoint{1.559091in}{1.220167in}}%
\pgfpathlineto{\pgfqpoint{1.560503in}{1.216749in}}%
\pgfpathlineto{\pgfqpoint{1.563327in}{1.269999in}}%
\pgfpathlineto{\pgfqpoint{1.567562in}{1.496608in}}%
\pgfpathlineto{\pgfqpoint{1.576032in}{1.965540in}}%
\pgfpathlineto{\pgfqpoint{1.580268in}{2.072257in}}%
\pgfpathlineto{\pgfqpoint{1.581679in}{2.084671in}}%
\pgfpathlineto{\pgfqpoint{1.583091in}{2.085970in}}%
\pgfpathlineto{\pgfqpoint{1.584503in}{2.076682in}}%
\pgfpathlineto{\pgfqpoint{1.587326in}{2.033098in}}%
\pgfpathlineto{\pgfqpoint{1.592973in}{1.891846in}}%
\pgfpathlineto{\pgfqpoint{1.604267in}{1.550517in}}%
\pgfpathlineto{\pgfqpoint{1.621208in}{0.959912in}}%
\pgfpathlineto{\pgfqpoint{1.624031in}{0.922542in}}%
\pgfpathlineto{\pgfqpoint{1.625443in}{0.913659in}}%
\pgfpathlineto{\pgfqpoint{1.626855in}{0.914103in}}%
\pgfpathlineto{\pgfqpoint{1.628267in}{0.919995in}}%
\pgfpathlineto{\pgfqpoint{1.631090in}{0.955811in}}%
\pgfpathlineto{\pgfqpoint{1.638149in}{1.083094in}}%
\pgfpathlineto{\pgfqpoint{1.640972in}{1.102920in}}%
\pgfpathlineto{\pgfqpoint{1.642384in}{1.102052in}}%
\pgfpathlineto{\pgfqpoint{1.643796in}{1.093146in}}%
\pgfpathlineto{\pgfqpoint{1.645207in}{1.092960in}}%
\pgfpathlineto{\pgfqpoint{1.646619in}{1.089343in}}%
\pgfpathlineto{\pgfqpoint{1.650854in}{1.061493in}}%
\pgfpathlineto{\pgfqpoint{1.652266in}{1.066610in}}%
\pgfpathlineto{\pgfqpoint{1.655090in}{1.123788in}}%
\pgfpathlineto{\pgfqpoint{1.657913in}{1.250037in}}%
\pgfpathlineto{\pgfqpoint{1.669207in}{1.849067in}}%
\pgfpathlineto{\pgfqpoint{1.672031in}{1.894547in}}%
\pgfpathlineto{\pgfqpoint{1.673442in}{1.896243in}}%
\pgfpathlineto{\pgfqpoint{1.674854in}{1.885112in}}%
\pgfpathlineto{\pgfqpoint{1.677677in}{1.829662in}}%
\pgfpathlineto{\pgfqpoint{1.688971in}{1.523259in}}%
\pgfpathlineto{\pgfqpoint{1.691795in}{1.500008in}}%
\pgfpathlineto{\pgfqpoint{1.693207in}{1.497120in}}%
\pgfpathlineto{\pgfqpoint{1.694618in}{1.497414in}}%
\pgfpathlineto{\pgfqpoint{1.696030in}{1.500173in}}%
\pgfpathlineto{\pgfqpoint{1.698854in}{1.498049in}}%
\pgfpathlineto{\pgfqpoint{1.700265in}{1.495123in}}%
\pgfpathlineto{\pgfqpoint{1.703089in}{1.471906in}}%
\pgfpathlineto{\pgfqpoint{1.708736in}{1.383548in}}%
\pgfpathlineto{\pgfqpoint{1.712971in}{1.292720in}}%
\pgfpathlineto{\pgfqpoint{1.720030in}{1.108432in}}%
\pgfpathlineto{\pgfqpoint{1.722853in}{1.071472in}}%
\pgfpathlineto{\pgfqpoint{1.724265in}{1.063935in}}%
\pgfpathlineto{\pgfqpoint{1.725677in}{1.062735in}}%
\pgfpathlineto{\pgfqpoint{1.727088in}{1.066278in}}%
\pgfpathlineto{\pgfqpoint{1.735559in}{1.106650in}}%
\pgfpathlineto{\pgfqpoint{1.738382in}{1.091339in}}%
\pgfpathlineto{\pgfqpoint{1.739794in}{1.094337in}}%
\pgfpathlineto{\pgfqpoint{1.741206in}{1.094284in}}%
\pgfpathlineto{\pgfqpoint{1.742617in}{1.091884in}}%
\pgfpathlineto{\pgfqpoint{1.744029in}{1.091781in}}%
\pgfpathlineto{\pgfqpoint{1.745441in}{1.099175in}}%
\pgfpathlineto{\pgfqpoint{1.746853in}{1.118726in}}%
\pgfpathlineto{\pgfqpoint{1.749676in}{1.205491in}}%
\pgfpathlineto{\pgfqpoint{1.755323in}{1.513822in}}%
\pgfpathlineto{\pgfqpoint{1.759558in}{1.707005in}}%
\pgfpathlineto{\pgfqpoint{1.762382in}{1.764080in}}%
\pgfpathlineto{\pgfqpoint{1.763793in}{1.767217in}}%
\pgfpathlineto{\pgfqpoint{1.765205in}{1.751614in}}%
\pgfpathlineto{\pgfqpoint{1.768029in}{1.680900in}}%
\pgfpathlineto{\pgfqpoint{1.776499in}{1.388196in}}%
\pgfpathlineto{\pgfqpoint{1.780734in}{1.328636in}}%
\pgfpathlineto{\pgfqpoint{1.782146in}{1.324319in}}%
\pgfpathlineto{\pgfqpoint{1.789205in}{1.383827in}}%
\pgfpathlineto{\pgfqpoint{1.790616in}{1.383886in}}%
\pgfpathlineto{\pgfqpoint{1.792028in}{1.378912in}}%
\pgfpathlineto{\pgfqpoint{1.794852in}{1.355343in}}%
\pgfpathlineto{\pgfqpoint{1.807557in}{1.200554in}}%
\pgfpathlineto{\pgfqpoint{1.817440in}{1.019070in}}%
\pgfpathlineto{\pgfqpoint{1.820263in}{1.003402in}}%
\pgfpathlineto{\pgfqpoint{1.821675in}{1.004976in}}%
\pgfpathlineto{\pgfqpoint{1.824498in}{1.020458in}}%
\pgfpathlineto{\pgfqpoint{1.831557in}{1.067546in}}%
\pgfpathlineto{\pgfqpoint{1.832969in}{1.070931in}}%
\pgfpathlineto{\pgfqpoint{1.834380in}{1.071231in}}%
\pgfpathlineto{\pgfqpoint{1.837204in}{1.055066in}}%
\pgfpathlineto{\pgfqpoint{1.840027in}{1.034799in}}%
\pgfpathlineto{\pgfqpoint{1.844263in}{1.023861in}}%
\pgfpathlineto{\pgfqpoint{1.845674in}{1.031167in}}%
\pgfpathlineto{\pgfqpoint{1.848498in}{1.086182in}}%
\pgfpathlineto{\pgfqpoint{1.852733in}{1.277006in}}%
\pgfpathlineto{\pgfqpoint{1.862615in}{1.766520in}}%
\pgfpathlineto{\pgfqpoint{1.865439in}{1.815739in}}%
\pgfpathlineto{\pgfqpoint{1.866850in}{1.818483in}}%
\pgfpathlineto{\pgfqpoint{1.868262in}{1.807752in}}%
\pgfpathlineto{\pgfqpoint{1.871086in}{1.752910in}}%
\pgfpathlineto{\pgfqpoint{1.880968in}{1.465872in}}%
\pgfpathlineto{\pgfqpoint{1.885203in}{1.411407in}}%
\pgfpathlineto{\pgfqpoint{1.889438in}{1.367058in}}%
\pgfpathlineto{\pgfqpoint{1.899320in}{1.205241in}}%
\pgfpathlineto{\pgfqpoint{1.906379in}{1.052960in}}%
\pgfpathlineto{\pgfqpoint{1.909202in}{1.040410in}}%
\pgfpathlineto{\pgfqpoint{1.910614in}{1.043263in}}%
\pgfpathlineto{\pgfqpoint{1.913438in}{1.071839in}}%
\pgfpathlineto{\pgfqpoint{1.917673in}{1.154051in}}%
\pgfpathlineto{\pgfqpoint{1.933202in}{1.588450in}}%
\pgfpathlineto{\pgfqpoint{1.936026in}{1.617780in}}%
\pgfpathlineto{\pgfqpoint{1.937437in}{1.619160in}}%
\pgfpathlineto{\pgfqpoint{1.938849in}{1.609244in}}%
\pgfpathlineto{\pgfqpoint{1.941672in}{1.548694in}}%
\pgfpathlineto{\pgfqpoint{1.947319in}{1.314063in}}%
\pgfpathlineto{\pgfqpoint{1.951555in}{1.164402in}}%
\pgfpathlineto{\pgfqpoint{1.952966in}{1.141636in}}%
\pgfpathlineto{\pgfqpoint{1.954378in}{1.148556in}}%
\pgfpathlineto{\pgfqpoint{1.955790in}{1.177404in}}%
\pgfpathlineto{\pgfqpoint{1.960025in}{1.380010in}}%
\pgfpathlineto{\pgfqpoint{1.967084in}{1.717523in}}%
\pgfpathlineto{\pgfqpoint{1.971319in}{1.813298in}}%
\pgfpathlineto{\pgfqpoint{1.974142in}{1.829783in}}%
\pgfpathlineto{\pgfqpoint{1.975554in}{1.824990in}}%
\pgfpathlineto{\pgfqpoint{1.978378in}{1.791023in}}%
\pgfpathlineto{\pgfqpoint{1.982613in}{1.692763in}}%
\pgfpathlineto{\pgfqpoint{1.992495in}{1.378766in}}%
\pgfpathlineto{\pgfqpoint{2.002377in}{1.041085in}}%
\pgfpathlineto{\pgfqpoint{2.005201in}{1.012542in}}%
\pgfpathlineto{\pgfqpoint{2.006612in}{1.009080in}}%
\pgfpathlineto{\pgfqpoint{2.008024in}{1.015109in}}%
\pgfpathlineto{\pgfqpoint{2.010848in}{1.053184in}}%
\pgfpathlineto{\pgfqpoint{2.013671in}{1.106803in}}%
\pgfpathlineto{\pgfqpoint{2.024965in}{1.426688in}}%
\pgfpathlineto{\pgfqpoint{2.029200in}{1.492611in}}%
\pgfpathlineto{\pgfqpoint{2.032024in}{1.505625in}}%
\pgfpathlineto{\pgfqpoint{2.033435in}{1.499431in}}%
\pgfpathlineto{\pgfqpoint{2.036259in}{1.449804in}}%
\pgfpathlineto{\pgfqpoint{2.046141in}{1.169662in}}%
\pgfpathlineto{\pgfqpoint{2.047553in}{1.180592in}}%
\pgfpathlineto{\pgfqpoint{2.050376in}{1.235263in}}%
\pgfpathlineto{\pgfqpoint{2.063082in}{1.876379in}}%
\pgfpathlineto{\pgfqpoint{2.065905in}{1.904629in}}%
\pgfpathlineto{\pgfqpoint{2.067317in}{1.899526in}}%
\pgfpathlineto{\pgfqpoint{2.070141in}{1.858626in}}%
\pgfpathlineto{\pgfqpoint{2.075788in}{1.709791in}}%
\pgfpathlineto{\pgfqpoint{2.091317in}{1.217945in}}%
\pgfpathlineto{\pgfqpoint{2.098375in}{1.001403in}}%
\pgfpathlineto{\pgfqpoint{2.101199in}{0.975152in}}%
\pgfpathlineto{\pgfqpoint{2.102611in}{0.972898in}}%
\pgfpathlineto{\pgfqpoint{2.104022in}{0.979886in}}%
\pgfpathlineto{\pgfqpoint{2.106846in}{1.014275in}}%
\pgfpathlineto{\pgfqpoint{2.111081in}{1.110117in}}%
\pgfpathlineto{\pgfqpoint{2.120963in}{1.371917in}}%
\pgfpathlineto{\pgfqpoint{2.125198in}{1.430376in}}%
\pgfpathlineto{\pgfqpoint{2.126610in}{1.436565in}}%
\pgfpathlineto{\pgfqpoint{2.128022in}{1.434993in}}%
\pgfpathlineto{\pgfqpoint{2.130845in}{1.400416in}}%
\pgfpathlineto{\pgfqpoint{2.140728in}{1.176292in}}%
\pgfpathlineto{\pgfqpoint{2.143551in}{1.223095in}}%
\pgfpathlineto{\pgfqpoint{2.146374in}{1.316574in}}%
\pgfpathlineto{\pgfqpoint{2.154845in}{1.805523in}}%
\pgfpathlineto{\pgfqpoint{2.157668in}{1.878768in}}%
\pgfpathlineto{\pgfqpoint{2.159080in}{1.890296in}}%
\pgfpathlineto{\pgfqpoint{2.160492in}{1.885979in}}%
\pgfpathlineto{\pgfqpoint{2.163315in}{1.835710in}}%
\pgfpathlineto{\pgfqpoint{2.177433in}{1.445728in}}%
\pgfpathlineto{\pgfqpoint{2.184491in}{1.341269in}}%
\pgfpathlineto{\pgfqpoint{2.195785in}{1.093517in}}%
\pgfpathlineto{\pgfqpoint{2.198609in}{1.014831in}}%
\pgfpathlineto{\pgfqpoint{2.201432in}{0.982945in}}%
\pgfpathlineto{\pgfqpoint{2.202844in}{0.977094in}}%
\pgfpathlineto{\pgfqpoint{2.204256in}{0.980426in}}%
\pgfpathlineto{\pgfqpoint{2.207079in}{1.013456in}}%
\pgfpathlineto{\pgfqpoint{2.209903in}{1.067698in}}%
\pgfpathlineto{\pgfqpoint{2.222608in}{1.409929in}}%
\pgfpathlineto{\pgfqpoint{2.226844in}{1.477288in}}%
\pgfpathlineto{\pgfqpoint{2.229667in}{1.501328in}}%
\pgfpathlineto{\pgfqpoint{2.231079in}{1.504509in}}%
\pgfpathlineto{\pgfqpoint{2.232490in}{1.500684in}}%
\pgfpathlineto{\pgfqpoint{2.235314in}{1.469089in}}%
\pgfpathlineto{\pgfqpoint{2.238137in}{1.396340in}}%
\pgfpathlineto{\pgfqpoint{2.245196in}{1.172219in}}%
\pgfpathlineto{\pgfqpoint{2.248020in}{1.126651in}}%
\pgfpathlineto{\pgfqpoint{2.249431in}{1.134219in}}%
\pgfpathlineto{\pgfqpoint{2.252255in}{1.219431in}}%
\pgfpathlineto{\pgfqpoint{2.264960in}{1.794313in}}%
\pgfpathlineto{\pgfqpoint{2.267784in}{1.828430in}}%
\pgfpathlineto{\pgfqpoint{2.269196in}{1.829338in}}%
\pgfpathlineto{\pgfqpoint{2.270607in}{1.821348in}}%
\pgfpathlineto{\pgfqpoint{2.273431in}{1.784281in}}%
\pgfpathlineto{\pgfqpoint{2.279078in}{1.659884in}}%
\pgfpathlineto{\pgfqpoint{2.283313in}{1.525172in}}%
\pgfpathlineto{\pgfqpoint{2.288960in}{1.253568in}}%
\pgfpathlineto{\pgfqpoint{2.291783in}{1.133452in}}%
\pgfpathlineto{\pgfqpoint{2.294607in}{1.082838in}}%
\pgfpathlineto{\pgfqpoint{2.296019in}{1.073970in}}%
\pgfpathlineto{\pgfqpoint{2.297430in}{1.084512in}}%
\pgfpathlineto{\pgfqpoint{2.300254in}{1.139924in}}%
\pgfpathlineto{\pgfqpoint{2.304489in}{1.286455in}}%
\pgfpathlineto{\pgfqpoint{2.310136in}{1.476096in}}%
\pgfpathlineto{\pgfqpoint{2.315783in}{1.598024in}}%
\pgfpathlineto{\pgfqpoint{2.324253in}{1.727768in}}%
\pgfpathlineto{\pgfqpoint{2.327077in}{1.754361in}}%
\pgfpathlineto{\pgfqpoint{2.328489in}{1.759978in}}%
\pgfpathlineto{\pgfqpoint{2.329900in}{1.755609in}}%
\pgfpathlineto{\pgfqpoint{2.332724in}{1.719534in}}%
\pgfpathlineto{\pgfqpoint{2.335547in}{1.632126in}}%
\pgfpathlineto{\pgfqpoint{2.345429in}{1.231013in}}%
\pgfpathlineto{\pgfqpoint{2.346841in}{1.235955in}}%
\pgfpathlineto{\pgfqpoint{2.348253in}{1.252546in}}%
\pgfpathlineto{\pgfqpoint{2.349665in}{1.288552in}}%
\pgfpathlineto{\pgfqpoint{2.362370in}{1.906990in}}%
\pgfpathlineto{\pgfqpoint{2.365194in}{1.947348in}}%
\pgfpathlineto{\pgfqpoint{2.366606in}{1.952096in}}%
\pgfpathlineto{\pgfqpoint{2.368017in}{1.946424in}}%
\pgfpathlineto{\pgfqpoint{2.370841in}{1.905939in}}%
\pgfpathlineto{\pgfqpoint{2.375076in}{1.783583in}}%
\pgfpathlineto{\pgfqpoint{2.393429in}{1.103358in}}%
\pgfpathlineto{\pgfqpoint{2.396252in}{1.082174in}}%
\pgfpathlineto{\pgfqpoint{2.397664in}{1.087695in}}%
\pgfpathlineto{\pgfqpoint{2.400487in}{1.130566in}}%
\pgfpathlineto{\pgfqpoint{2.404722in}{1.238336in}}%
\pgfpathlineto{\pgfqpoint{2.413193in}{1.505305in}}%
\pgfpathlineto{\pgfqpoint{2.420252in}{1.656642in}}%
\pgfpathlineto{\pgfqpoint{2.425899in}{1.733115in}}%
\pgfpathlineto{\pgfqpoint{2.427310in}{1.743637in}}%
\pgfpathlineto{\pgfqpoint{2.428722in}{1.745339in}}%
\pgfpathlineto{\pgfqpoint{2.430134in}{1.740377in}}%
\pgfpathlineto{\pgfqpoint{2.432957in}{1.708736in}}%
\pgfpathlineto{\pgfqpoint{2.435781in}{1.642702in}}%
\pgfpathlineto{\pgfqpoint{2.440016in}{1.459003in}}%
\pgfpathlineto{\pgfqpoint{2.445663in}{1.230479in}}%
\pgfpathlineto{\pgfqpoint{2.448486in}{1.177572in}}%
\pgfpathlineto{\pgfqpoint{2.449898in}{1.191275in}}%
\pgfpathlineto{\pgfqpoint{2.452722in}{1.288129in}}%
\pgfpathlineto{\pgfqpoint{2.464015in}{1.843517in}}%
\pgfpathlineto{\pgfqpoint{2.466839in}{1.894134in}}%
\pgfpathlineto{\pgfqpoint{2.468251in}{1.901922in}}%
\pgfpathlineto{\pgfqpoint{2.469662in}{1.898534in}}%
\pgfpathlineto{\pgfqpoint{2.472486in}{1.862046in}}%
\pgfpathlineto{\pgfqpoint{2.478133in}{1.730914in}}%
\pgfpathlineto{\pgfqpoint{2.483780in}{1.522760in}}%
\pgfpathlineto{\pgfqpoint{2.492250in}{1.156316in}}%
\pgfpathlineto{\pgfqpoint{2.495074in}{1.114002in}}%
\pgfpathlineto{\pgfqpoint{2.496485in}{1.108274in}}%
\pgfpathlineto{\pgfqpoint{2.497897in}{1.115431in}}%
\pgfpathlineto{\pgfqpoint{2.500721in}{1.160750in}}%
\pgfpathlineto{\pgfqpoint{2.506368in}{1.320188in}}%
\pgfpathlineto{\pgfqpoint{2.513426in}{1.537887in}}%
\pgfpathlineto{\pgfqpoint{2.520485in}{1.682975in}}%
\pgfpathlineto{\pgfqpoint{2.526132in}{1.757706in}}%
\pgfpathlineto{\pgfqpoint{2.528955in}{1.773897in}}%
\pgfpathlineto{\pgfqpoint{2.530367in}{1.775083in}}%
\pgfpathlineto{\pgfqpoint{2.531779in}{1.771568in}}%
\pgfpathlineto{\pgfqpoint{2.534602in}{1.749797in}}%
\pgfpathlineto{\pgfqpoint{2.537426in}{1.700858in}}%
\pgfpathlineto{\pgfqpoint{2.541661in}{1.579362in}}%
\pgfpathlineto{\pgfqpoint{2.552955in}{1.155346in}}%
\pgfpathlineto{\pgfqpoint{2.554367in}{1.135640in}}%
\pgfpathlineto{\pgfqpoint{2.555778in}{1.132834in}}%
\pgfpathlineto{\pgfqpoint{2.557190in}{1.147978in}}%
\pgfpathlineto{\pgfqpoint{2.560014in}{1.221697in}}%
\pgfpathlineto{\pgfqpoint{2.569896in}{1.540403in}}%
\pgfpathlineto{\pgfqpoint{2.574131in}{1.607972in}}%
\pgfpathlineto{\pgfqpoint{2.576955in}{1.629352in}}%
\pgfpathlineto{\pgfqpoint{2.578366in}{1.632874in}}%
\pgfpathlineto{\pgfqpoint{2.579778in}{1.631432in}}%
\pgfpathlineto{\pgfqpoint{2.582601in}{1.614514in}}%
\pgfpathlineto{\pgfqpoint{2.586837in}{1.565735in}}%
\pgfpathlineto{\pgfqpoint{2.592484in}{1.463052in}}%
\pgfpathlineto{\pgfqpoint{2.598131in}{1.311401in}}%
\pgfpathlineto{\pgfqpoint{2.608013in}{1.027649in}}%
\pgfpathlineto{\pgfqpoint{2.612248in}{0.964945in}}%
\pgfpathlineto{\pgfqpoint{2.615071in}{0.948054in}}%
\pgfpathlineto{\pgfqpoint{2.616483in}{0.946189in}}%
\pgfpathlineto{\pgfqpoint{2.617895in}{0.948044in}}%
\pgfpathlineto{\pgfqpoint{2.620718in}{0.958355in}}%
\pgfpathlineto{\pgfqpoint{2.626365in}{0.994077in}}%
\pgfpathlineto{\pgfqpoint{2.630601in}{1.017571in}}%
\pgfpathlineto{\pgfqpoint{2.633424in}{1.027239in}}%
\pgfpathlineto{\pgfqpoint{2.634836in}{1.028626in}}%
\pgfpathlineto{\pgfqpoint{2.636248in}{1.027970in}}%
\pgfpathlineto{\pgfqpoint{2.639071in}{1.022723in}}%
\pgfpathlineto{\pgfqpoint{2.641894in}{1.011586in}}%
\pgfpathlineto{\pgfqpoint{2.644718in}{0.987599in}}%
\pgfpathlineto{\pgfqpoint{2.650365in}{0.930966in}}%
\pgfpathlineto{\pgfqpoint{2.654600in}{0.904581in}}%
\pgfpathlineto{\pgfqpoint{2.656012in}{0.904160in}}%
\pgfpathlineto{\pgfqpoint{2.657424in}{0.909770in}}%
\pgfpathlineto{\pgfqpoint{2.660247in}{0.939876in}}%
\pgfpathlineto{\pgfqpoint{2.664482in}{1.033268in}}%
\pgfpathlineto{\pgfqpoint{2.670129in}{1.214146in}}%
\pgfpathlineto{\pgfqpoint{2.682835in}{1.667330in}}%
\pgfpathlineto{\pgfqpoint{2.684247in}{1.657834in}}%
\pgfpathlineto{\pgfqpoint{2.687070in}{1.559813in}}%
\pgfpathlineto{\pgfqpoint{2.696952in}{1.000152in}}%
\pgfpathlineto{\pgfqpoint{2.698364in}{1.011961in}}%
\pgfpathlineto{\pgfqpoint{2.704011in}{1.111184in}}%
\pgfpathlineto{\pgfqpoint{2.708246in}{1.073186in}}%
\pgfpathlineto{\pgfqpoint{2.712481in}{0.951465in}}%
\pgfpathlineto{\pgfqpoint{2.719540in}{0.771605in}}%
\pgfpathlineto{\pgfqpoint{2.722364in}{0.757986in}}%
\pgfpathlineto{\pgfqpoint{2.725187in}{0.754584in}}%
\pgfpathlineto{\pgfqpoint{2.728010in}{0.758953in}}%
\pgfpathlineto{\pgfqpoint{2.732246in}{0.774934in}}%
\pgfpathlineto{\pgfqpoint{2.735069in}{0.795673in}}%
\pgfpathlineto{\pgfqpoint{2.740716in}{0.881141in}}%
\pgfpathlineto{\pgfqpoint{2.750598in}{1.053319in}}%
\pgfpathlineto{\pgfqpoint{2.752010in}{1.061027in}}%
\pgfpathlineto{\pgfqpoint{2.754833in}{1.050878in}}%
\pgfpathlineto{\pgfqpoint{2.756245in}{1.044986in}}%
\pgfpathlineto{\pgfqpoint{2.759069in}{1.004977in}}%
\pgfpathlineto{\pgfqpoint{2.764716in}{0.942453in}}%
\pgfpathlineto{\pgfqpoint{2.766127in}{0.947386in}}%
\pgfpathlineto{\pgfqpoint{2.767539in}{0.944387in}}%
\pgfpathlineto{\pgfqpoint{2.768951in}{0.947919in}}%
\pgfpathlineto{\pgfqpoint{2.771774in}{0.933828in}}%
\pgfpathlineto{\pgfqpoint{2.773186in}{0.915921in}}%
\pgfpathlineto{\pgfqpoint{2.774598in}{0.738843in}}%
\pgfpathlineto{\pgfqpoint{2.777421in}{0.742535in}}%
\pgfpathlineto{\pgfqpoint{2.781657in}{0.748585in}}%
\pgfpathlineto{\pgfqpoint{2.784480in}{0.747994in}}%
\pgfpathlineto{\pgfqpoint{2.785892in}{0.936795in}}%
\pgfpathlineto{\pgfqpoint{2.787303in}{0.926895in}}%
\pgfpathlineto{\pgfqpoint{2.792950in}{0.959707in}}%
\pgfpathlineto{\pgfqpoint{2.795774in}{0.974804in}}%
\pgfpathlineto{\pgfqpoint{2.800009in}{1.038391in}}%
\pgfpathlineto{\pgfqpoint{2.802833in}{1.056947in}}%
\pgfpathlineto{\pgfqpoint{2.808480in}{1.075693in}}%
\pgfpathlineto{\pgfqpoint{2.809891in}{1.075832in}}%
\pgfpathlineto{\pgfqpoint{2.815538in}{1.086790in}}%
\pgfpathlineto{\pgfqpoint{2.819773in}{1.092725in}}%
\pgfpathlineto{\pgfqpoint{2.822597in}{1.093806in}}%
\pgfpathlineto{\pgfqpoint{2.826832in}{1.091790in}}%
\pgfpathlineto{\pgfqpoint{2.833891in}{1.089154in}}%
\pgfpathlineto{\pgfqpoint{2.836714in}{1.092444in}}%
\pgfpathlineto{\pgfqpoint{2.846596in}{1.112204in}}%
\pgfpathlineto{\pgfqpoint{2.849420in}{1.112877in}}%
\pgfpathlineto{\pgfqpoint{2.852243in}{1.109638in}}%
\pgfpathlineto{\pgfqpoint{2.862126in}{1.089809in}}%
\pgfpathlineto{\pgfqpoint{2.863537in}{1.088120in}}%
\pgfpathlineto{\pgfqpoint{2.864949in}{1.089360in}}%
\pgfpathlineto{\pgfqpoint{2.870596in}{1.084453in}}%
\pgfpathlineto{\pgfqpoint{2.874831in}{1.089884in}}%
\pgfpathlineto{\pgfqpoint{2.877655in}{1.104941in}}%
\pgfpathlineto{\pgfqpoint{2.881890in}{1.125866in}}%
\pgfpathlineto{\pgfqpoint{2.887537in}{1.141560in}}%
\pgfpathlineto{\pgfqpoint{2.893184in}{1.147896in}}%
\pgfpathlineto{\pgfqpoint{2.896007in}{1.151226in}}%
\pgfpathlineto{\pgfqpoint{2.901654in}{1.158099in}}%
\pgfpathlineto{\pgfqpoint{2.903066in}{1.158607in}}%
\pgfpathlineto{\pgfqpoint{2.907301in}{1.155118in}}%
\pgfpathlineto{\pgfqpoint{2.911536in}{1.151390in}}%
\pgfpathlineto{\pgfqpoint{2.915772in}{1.155818in}}%
\pgfpathlineto{\pgfqpoint{2.917183in}{1.156401in}}%
\pgfpathlineto{\pgfqpoint{2.920007in}{1.154200in}}%
\pgfpathlineto{\pgfqpoint{2.922830in}{1.147044in}}%
\pgfpathlineto{\pgfqpoint{2.927066in}{1.129316in}}%
\pgfpathlineto{\pgfqpoint{2.939771in}{1.056488in}}%
\pgfpathlineto{\pgfqpoint{2.941183in}{1.048153in}}%
\pgfpathlineto{\pgfqpoint{2.942595in}{1.046938in}}%
\pgfpathlineto{\pgfqpoint{2.945418in}{1.048114in}}%
\pgfpathlineto{\pgfqpoint{2.949653in}{1.053823in}}%
\pgfpathlineto{\pgfqpoint{2.955300in}{1.063808in}}%
\pgfpathlineto{\pgfqpoint{2.958124in}{1.065625in}}%
\pgfpathlineto{\pgfqpoint{2.959535in}{1.067442in}}%
\pgfpathlineto{\pgfqpoint{2.962359in}{1.077153in}}%
\pgfpathlineto{\pgfqpoint{2.965182in}{1.097522in}}%
\pgfpathlineto{\pgfqpoint{2.969418in}{1.148633in}}%
\pgfpathlineto{\pgfqpoint{2.984947in}{1.370470in}}%
\pgfpathlineto{\pgfqpoint{2.989182in}{1.402723in}}%
\pgfpathlineto{\pgfqpoint{2.992005in}{1.414657in}}%
\pgfpathlineto{\pgfqpoint{2.994829in}{1.417846in}}%
\pgfpathlineto{\pgfqpoint{2.996241in}{1.416280in}}%
\pgfpathlineto{\pgfqpoint{2.999064in}{1.406903in}}%
\pgfpathlineto{\pgfqpoint{3.003299in}{1.380063in}}%
\pgfpathlineto{\pgfqpoint{3.011770in}{1.306445in}}%
\pgfpathlineto{\pgfqpoint{3.016005in}{1.273255in}}%
\pgfpathlineto{\pgfqpoint{3.020240in}{1.236249in}}%
\pgfpathlineto{\pgfqpoint{3.027299in}{1.141179in}}%
\pgfpathlineto{\pgfqpoint{3.034358in}{1.057081in}}%
\pgfpathlineto{\pgfqpoint{3.045652in}{0.964637in}}%
\pgfpathlineto{\pgfqpoint{3.051298in}{0.945353in}}%
\pgfpathlineto{\pgfqpoint{3.052710in}{0.944203in}}%
\pgfpathlineto{\pgfqpoint{3.055534in}{0.949594in}}%
\pgfpathlineto{\pgfqpoint{3.058357in}{0.973572in}}%
\pgfpathlineto{\pgfqpoint{3.062592in}{1.044963in}}%
\pgfpathlineto{\pgfqpoint{3.079533in}{1.403950in}}%
\pgfpathlineto{\pgfqpoint{3.082357in}{1.423709in}}%
\pgfpathlineto{\pgfqpoint{3.083768in}{1.426577in}}%
\pgfpathlineto{\pgfqpoint{3.085180in}{1.425393in}}%
\pgfpathlineto{\pgfqpoint{3.088004in}{1.411431in}}%
\pgfpathlineto{\pgfqpoint{3.092239in}{1.370258in}}%
\pgfpathlineto{\pgfqpoint{3.102121in}{1.236904in}}%
\pgfpathlineto{\pgfqpoint{3.106356in}{1.188525in}}%
\pgfpathlineto{\pgfqpoint{3.110591in}{1.159788in}}%
\pgfpathlineto{\pgfqpoint{3.112003in}{1.157613in}}%
\pgfpathlineto{\pgfqpoint{3.117650in}{1.184851in}}%
\pgfpathlineto{\pgfqpoint{3.128944in}{1.241539in}}%
\pgfpathlineto{\pgfqpoint{3.131768in}{1.244419in}}%
\pgfpathlineto{\pgfqpoint{3.133179in}{1.242180in}}%
\pgfpathlineto{\pgfqpoint{3.136003in}{1.229960in}}%
\pgfpathlineto{\pgfqpoint{3.140238in}{1.185218in}}%
\pgfpathlineto{\pgfqpoint{3.145885in}{1.118306in}}%
\pgfpathlineto{\pgfqpoint{3.152944in}{1.056568in}}%
\pgfpathlineto{\pgfqpoint{3.155767in}{1.044630in}}%
\pgfpathlineto{\pgfqpoint{3.157179in}{1.044338in}}%
\pgfpathlineto{\pgfqpoint{3.158591in}{1.047962in}}%
\pgfpathlineto{\pgfqpoint{3.161414in}{1.066909in}}%
\pgfpathlineto{\pgfqpoint{3.167061in}{1.135860in}}%
\pgfpathlineto{\pgfqpoint{3.174120in}{1.221649in}}%
\pgfpathlineto{\pgfqpoint{3.176943in}{1.235261in}}%
\pgfpathlineto{\pgfqpoint{3.178355in}{1.236185in}}%
\pgfpathlineto{\pgfqpoint{3.179767in}{1.231725in}}%
\pgfpathlineto{\pgfqpoint{3.182590in}{1.207462in}}%
\pgfpathlineto{\pgfqpoint{3.186825in}{1.137532in}}%
\pgfpathlineto{\pgfqpoint{3.198119in}{0.916346in}}%
\pgfpathlineto{\pgfqpoint{3.200943in}{0.894740in}}%
\pgfpathlineto{\pgfqpoint{3.203766in}{0.889022in}}%
\pgfpathlineto{\pgfqpoint{3.208001in}{0.900910in}}%
\pgfpathlineto{\pgfqpoint{3.210825in}{0.917886in}}%
\pgfpathlineto{\pgfqpoint{3.219295in}{0.991902in}}%
\pgfpathlineto{\pgfqpoint{3.223530in}{1.029891in}}%
\pgfpathlineto{\pgfqpoint{3.230589in}{1.076316in}}%
\pgfpathlineto{\pgfqpoint{3.236236in}{1.100258in}}%
\pgfpathlineto{\pgfqpoint{3.239060in}{1.106621in}}%
\pgfpathlineto{\pgfqpoint{3.240471in}{1.107516in}}%
\pgfpathlineto{\pgfqpoint{3.243295in}{1.095747in}}%
\pgfpathlineto{\pgfqpoint{3.248942in}{1.062428in}}%
\pgfpathlineto{\pgfqpoint{3.257412in}{1.007734in}}%
\pgfpathlineto{\pgfqpoint{3.258824in}{1.008294in}}%
\pgfpathlineto{\pgfqpoint{3.260236in}{1.014195in}}%
\pgfpathlineto{\pgfqpoint{3.263059in}{1.043183in}}%
\pgfpathlineto{\pgfqpoint{3.268706in}{1.144146in}}%
\pgfpathlineto{\pgfqpoint{3.277177in}{1.296695in}}%
\pgfpathlineto{\pgfqpoint{3.281412in}{1.335333in}}%
\pgfpathlineto{\pgfqpoint{3.282823in}{1.339450in}}%
\pgfpathlineto{\pgfqpoint{3.284235in}{1.339193in}}%
\pgfpathlineto{\pgfqpoint{3.287059in}{1.327181in}}%
\pgfpathlineto{\pgfqpoint{3.289882in}{1.304677in}}%
\pgfpathlineto{\pgfqpoint{3.299764in}{1.209822in}}%
\pgfpathlineto{\pgfqpoint{3.304000in}{1.189026in}}%
\pgfpathlineto{\pgfqpoint{3.308235in}{1.180809in}}%
\pgfpathlineto{\pgfqpoint{3.313882in}{1.178186in}}%
\pgfpathlineto{\pgfqpoint{3.318117in}{1.190167in}}%
\pgfpathlineto{\pgfqpoint{3.319529in}{1.190988in}}%
\pgfpathlineto{\pgfqpoint{3.320940in}{1.190066in}}%
\pgfpathlineto{\pgfqpoint{3.326587in}{1.178363in}}%
\pgfpathlineto{\pgfqpoint{3.330823in}{1.165768in}}%
\pgfpathlineto{\pgfqpoint{3.335058in}{1.159363in}}%
\pgfpathlineto{\pgfqpoint{3.340705in}{1.155446in}}%
\pgfpathlineto{\pgfqpoint{3.342116in}{1.152415in}}%
\pgfpathlineto{\pgfqpoint{3.344940in}{1.134588in}}%
\pgfpathlineto{\pgfqpoint{3.357646in}{1.026450in}}%
\pgfpathlineto{\pgfqpoint{3.359057in}{1.019699in}}%
\pgfpathlineto{\pgfqpoint{3.361881in}{1.038456in}}%
\pgfpathlineto{\pgfqpoint{3.366116in}{1.091483in}}%
\pgfpathlineto{\pgfqpoint{3.375998in}{1.278932in}}%
\pgfpathlineto{\pgfqpoint{3.380233in}{1.312713in}}%
\pgfpathlineto{\pgfqpoint{3.381645in}{1.315428in}}%
\pgfpathlineto{\pgfqpoint{3.383057in}{1.314183in}}%
\pgfpathlineto{\pgfqpoint{3.385880in}{1.300356in}}%
\pgfpathlineto{\pgfqpoint{3.390116in}{1.259922in}}%
\pgfpathlineto{\pgfqpoint{3.399998in}{1.154252in}}%
\pgfpathlineto{\pgfqpoint{3.404233in}{1.129354in}}%
\pgfpathlineto{\pgfqpoint{3.407056in}{1.122402in}}%
\pgfpathlineto{\pgfqpoint{3.412703in}{1.118027in}}%
\pgfpathlineto{\pgfqpoint{3.415527in}{1.115534in}}%
\pgfpathlineto{\pgfqpoint{3.416939in}{1.113033in}}%
\pgfpathlineto{\pgfqpoint{3.426821in}{1.165288in}}%
\pgfpathlineto{\pgfqpoint{3.438115in}{1.206700in}}%
\pgfpathlineto{\pgfqpoint{3.439526in}{1.207981in}}%
\pgfpathlineto{\pgfqpoint{3.440938in}{1.206643in}}%
\pgfpathlineto{\pgfqpoint{3.442350in}{1.203602in}}%
\pgfpathlineto{\pgfqpoint{3.446585in}{1.171918in}}%
\pgfpathlineto{\pgfqpoint{3.450820in}{1.135690in}}%
\pgfpathlineto{\pgfqpoint{3.455056in}{1.099323in}}%
\pgfpathlineto{\pgfqpoint{3.460702in}{1.051966in}}%
\pgfpathlineto{\pgfqpoint{3.462114in}{1.054891in}}%
\pgfpathlineto{\pgfqpoint{3.464938in}{1.081069in}}%
\pgfpathlineto{\pgfqpoint{3.469173in}{1.150138in}}%
\pgfpathlineto{\pgfqpoint{3.477643in}{1.308516in}}%
\pgfpathlineto{\pgfqpoint{3.480467in}{1.335336in}}%
\pgfpathlineto{\pgfqpoint{3.483290in}{1.345527in}}%
\pgfpathlineto{\pgfqpoint{3.484702in}{1.345055in}}%
\pgfpathlineto{\pgfqpoint{3.487525in}{1.334329in}}%
\pgfpathlineto{\pgfqpoint{3.491761in}{1.300975in}}%
\pgfpathlineto{\pgfqpoint{3.501643in}{1.211347in}}%
\pgfpathlineto{\pgfqpoint{3.505878in}{1.186869in}}%
\pgfpathlineto{\pgfqpoint{3.519995in}{1.126430in}}%
\pgfpathlineto{\pgfqpoint{3.527054in}{1.132272in}}%
\pgfpathlineto{\pgfqpoint{3.534113in}{1.128230in}}%
\pgfpathlineto{\pgfqpoint{3.536936in}{1.129295in}}%
\pgfpathlineto{\pgfqpoint{3.541172in}{1.136811in}}%
\pgfpathlineto{\pgfqpoint{3.543995in}{1.141971in}}%
\pgfpathlineto{\pgfqpoint{3.545407in}{1.142827in}}%
\pgfpathlineto{\pgfqpoint{3.546818in}{1.140935in}}%
\pgfpathlineto{\pgfqpoint{3.549642in}{1.123962in}}%
\pgfpathlineto{\pgfqpoint{3.562348in}{1.008880in}}%
\pgfpathlineto{\pgfqpoint{3.565171in}{0.993036in}}%
\pgfpathlineto{\pgfqpoint{3.567995in}{1.009613in}}%
\pgfpathlineto{\pgfqpoint{3.572230in}{1.055054in}}%
\pgfpathlineto{\pgfqpoint{3.576465in}{1.135162in}}%
\pgfpathlineto{\pgfqpoint{3.582112in}{1.227671in}}%
\pgfpathlineto{\pgfqpoint{3.586347in}{1.259589in}}%
\pgfpathlineto{\pgfqpoint{3.589171in}{1.264812in}}%
\pgfpathlineto{\pgfqpoint{3.591994in}{1.261855in}}%
\pgfpathlineto{\pgfqpoint{3.596229in}{1.238093in}}%
\pgfpathlineto{\pgfqpoint{3.603288in}{1.177532in}}%
\pgfpathlineto{\pgfqpoint{3.608935in}{1.135412in}}%
\pgfpathlineto{\pgfqpoint{3.611758in}{1.124853in}}%
\pgfpathlineto{\pgfqpoint{3.621641in}{1.105347in}}%
\pgfpathlineto{\pgfqpoint{3.628699in}{1.083072in}}%
\pgfpathlineto{\pgfqpoint{3.631523in}{1.080374in}}%
\pgfpathlineto{\pgfqpoint{3.632934in}{1.080009in}}%
\pgfpathlineto{\pgfqpoint{3.635758in}{1.082392in}}%
\pgfpathlineto{\pgfqpoint{3.642817in}{1.102403in}}%
\pgfpathlineto{\pgfqpoint{3.648464in}{1.125584in}}%
\pgfpathlineto{\pgfqpoint{3.651287in}{1.129355in}}%
\pgfpathlineto{\pgfqpoint{3.654111in}{1.115338in}}%
\pgfpathlineto{\pgfqpoint{3.656934in}{1.098606in}}%
\pgfpathlineto{\pgfqpoint{3.659757in}{1.084296in}}%
\pgfpathlineto{\pgfqpoint{3.668228in}{1.025145in}}%
\pgfpathlineto{\pgfqpoint{3.669640in}{1.028425in}}%
\pgfpathlineto{\pgfqpoint{3.672463in}{1.050833in}}%
\pgfpathlineto{\pgfqpoint{3.676698in}{1.104787in}}%
\pgfpathlineto{\pgfqpoint{3.685169in}{1.248882in}}%
\pgfpathlineto{\pgfqpoint{3.689404in}{1.285757in}}%
\pgfpathlineto{\pgfqpoint{3.692227in}{1.295350in}}%
\pgfpathlineto{\pgfqpoint{3.693639in}{1.296188in}}%
\pgfpathlineto{\pgfqpoint{3.696463in}{1.292306in}}%
\pgfpathlineto{\pgfqpoint{3.700698in}{1.266706in}}%
\pgfpathlineto{\pgfqpoint{3.711992in}{1.181522in}}%
\pgfpathlineto{\pgfqpoint{3.719050in}{1.152547in}}%
\pgfpathlineto{\pgfqpoint{3.723286in}{1.138541in}}%
\pgfpathlineto{\pgfqpoint{3.726109in}{1.128187in}}%
\pgfpathlineto{\pgfqpoint{3.728933in}{1.129441in}}%
\pgfpathlineto{\pgfqpoint{3.731756in}{1.129761in}}%
\pgfpathlineto{\pgfqpoint{3.733168in}{1.131882in}}%
\pgfpathlineto{\pgfqpoint{3.737403in}{1.131306in}}%
\pgfpathlineto{\pgfqpoint{3.740227in}{1.132533in}}%
\pgfpathlineto{\pgfqpoint{3.743050in}{1.135221in}}%
\pgfpathlineto{\pgfqpoint{3.747285in}{1.144202in}}%
\pgfpathlineto{\pgfqpoint{3.751520in}{1.153725in}}%
\pgfpathlineto{\pgfqpoint{3.752932in}{1.154083in}}%
\pgfpathlineto{\pgfqpoint{3.754344in}{1.152360in}}%
\pgfpathlineto{\pgfqpoint{3.755756in}{1.147412in}}%
\pgfpathlineto{\pgfqpoint{3.758579in}{1.123064in}}%
\pgfpathlineto{\pgfqpoint{3.764226in}{1.065280in}}%
\pgfpathlineto{\pgfqpoint{3.771285in}{1.000472in}}%
\pgfpathlineto{\pgfqpoint{3.772697in}{0.996135in}}%
\pgfpathlineto{\pgfqpoint{3.775520in}{1.011938in}}%
\pgfpathlineto{\pgfqpoint{3.779755in}{1.054853in}}%
\pgfpathlineto{\pgfqpoint{3.793873in}{1.246134in}}%
\pgfpathlineto{\pgfqpoint{3.796696in}{1.257768in}}%
\pgfpathlineto{\pgfqpoint{3.799520in}{1.261894in}}%
\pgfpathlineto{\pgfqpoint{3.800931in}{1.261592in}}%
\pgfpathlineto{\pgfqpoint{3.803755in}{1.257278in}}%
\pgfpathlineto{\pgfqpoint{3.809402in}{1.237744in}}%
\pgfpathlineto{\pgfqpoint{3.816460in}{1.215881in}}%
\pgfpathlineto{\pgfqpoint{3.822107in}{1.204598in}}%
\pgfpathlineto{\pgfqpoint{3.826343in}{1.191892in}}%
\pgfpathlineto{\pgfqpoint{3.830578in}{1.186838in}}%
\pgfpathlineto{\pgfqpoint{3.831990in}{1.185839in}}%
\pgfpathlineto{\pgfqpoint{3.833401in}{1.187348in}}%
\pgfpathlineto{\pgfqpoint{3.837636in}{1.183449in}}%
\pgfpathlineto{\pgfqpoint{3.844695in}{1.178311in}}%
\pgfpathlineto{\pgfqpoint{3.854577in}{1.173181in}}%
\pgfpathlineto{\pgfqpoint{3.855989in}{1.170315in}}%
\pgfpathlineto{\pgfqpoint{3.858813in}{1.153786in}}%
\pgfpathlineto{\pgfqpoint{3.872930in}{1.042828in}}%
\pgfpathlineto{\pgfqpoint{3.874342in}{1.040824in}}%
\pgfpathlineto{\pgfqpoint{3.875753in}{1.046999in}}%
\pgfpathlineto{\pgfqpoint{3.879989in}{1.096287in}}%
\pgfpathlineto{\pgfqpoint{3.884224in}{1.173896in}}%
\pgfpathlineto{\pgfqpoint{3.889871in}{1.278548in}}%
\pgfpathlineto{\pgfqpoint{3.894106in}{1.318660in}}%
\pgfpathlineto{\pgfqpoint{3.895518in}{1.323184in}}%
\pgfpathlineto{\pgfqpoint{3.896929in}{1.323734in}}%
\pgfpathlineto{\pgfqpoint{3.899753in}{1.316120in}}%
\pgfpathlineto{\pgfqpoint{3.903988in}{1.291640in}}%
\pgfpathlineto{\pgfqpoint{3.919517in}{1.184750in}}%
\pgfpathlineto{\pgfqpoint{3.922341in}{1.177698in}}%
\pgfpathlineto{\pgfqpoint{3.923752in}{1.175517in}}%
\pgfpathlineto{\pgfqpoint{3.929399in}{1.156583in}}%
\pgfpathlineto{\pgfqpoint{3.933635in}{1.168684in}}%
\pgfpathlineto{\pgfqpoint{3.936458in}{1.170747in}}%
\pgfpathlineto{\pgfqpoint{3.943517in}{1.162672in}}%
\pgfpathlineto{\pgfqpoint{3.946340in}{1.161702in}}%
\pgfpathlineto{\pgfqpoint{3.947752in}{1.162211in}}%
\pgfpathlineto{\pgfqpoint{3.950576in}{1.167684in}}%
\pgfpathlineto{\pgfqpoint{3.953399in}{1.179476in}}%
\pgfpathlineto{\pgfqpoint{3.960458in}{1.215193in}}%
\pgfpathlineto{\pgfqpoint{3.961869in}{1.216579in}}%
\pgfpathlineto{\pgfqpoint{3.964693in}{1.200612in}}%
\pgfpathlineto{\pgfqpoint{3.968928in}{1.169522in}}%
\pgfpathlineto{\pgfqpoint{3.973163in}{1.132021in}}%
\pgfpathlineto{\pgfqpoint{3.978810in}{1.070793in}}%
\pgfpathlineto{\pgfqpoint{3.981634in}{1.079436in}}%
\pgfpathlineto{\pgfqpoint{3.984457in}{1.098598in}}%
\pgfpathlineto{\pgfqpoint{3.987281in}{1.128572in}}%
\pgfpathlineto{\pgfqpoint{3.998575in}{1.330990in}}%
\pgfpathlineto{\pgfqpoint{4.002810in}{1.372001in}}%
\pgfpathlineto{\pgfqpoint{4.005633in}{1.380951in}}%
\pgfpathlineto{\pgfqpoint{4.007045in}{1.381230in}}%
\pgfpathlineto{\pgfqpoint{4.008457in}{1.380011in}}%
\pgfpathlineto{\pgfqpoint{4.012692in}{1.369090in}}%
\pgfpathlineto{\pgfqpoint{4.018339in}{1.346820in}}%
\pgfpathlineto{\pgfqpoint{4.026809in}{1.298405in}}%
\pgfpathlineto{\pgfqpoint{4.033868in}{1.253299in}}%
\pgfpathlineto{\pgfqpoint{4.035280in}{1.251230in}}%
\pgfpathlineto{\pgfqpoint{4.038103in}{1.255944in}}%
\pgfpathlineto{\pgfqpoint{4.039515in}{1.255859in}}%
\pgfpathlineto{\pgfqpoint{4.043750in}{1.247205in}}%
\pgfpathlineto{\pgfqpoint{4.055044in}{1.204439in}}%
\pgfpathlineto{\pgfqpoint{4.063515in}{1.186820in}}%
\pgfpathlineto{\pgfqpoint{4.064926in}{1.180872in}}%
\pgfpathlineto{\pgfqpoint{4.069161in}{1.143214in}}%
\pgfpathlineto{\pgfqpoint{4.071985in}{1.119542in}}%
\pgfpathlineto{\pgfqpoint{4.076220in}{1.087503in}}%
\pgfpathlineto{\pgfqpoint{4.080455in}{1.052469in}}%
\pgfpathlineto{\pgfqpoint{4.081867in}{1.052703in}}%
\pgfpathlineto{\pgfqpoint{4.086102in}{1.097537in}}%
\pgfpathlineto{\pgfqpoint{4.091749in}{1.192014in}}%
\pgfpathlineto{\pgfqpoint{4.097396in}{1.285263in}}%
\pgfpathlineto{\pgfqpoint{4.100220in}{1.307062in}}%
\pgfpathlineto{\pgfqpoint{4.101631in}{1.310860in}}%
\pgfpathlineto{\pgfqpoint{4.103043in}{1.309412in}}%
\pgfpathlineto{\pgfqpoint{4.105867in}{1.293770in}}%
\pgfpathlineto{\pgfqpoint{4.110102in}{1.246953in}}%
\pgfpathlineto{\pgfqpoint{4.124219in}{1.066613in}}%
\pgfpathlineto{\pgfqpoint{4.125631in}{1.060816in}}%
\pgfpathlineto{\pgfqpoint{4.127043in}{1.060254in}}%
\pgfpathlineto{\pgfqpoint{4.128454in}{1.061302in}}%
\pgfpathlineto{\pgfqpoint{4.131278in}{1.071935in}}%
\pgfpathlineto{\pgfqpoint{4.135513in}{1.088033in}}%
\pgfpathlineto{\pgfqpoint{4.136925in}{1.089878in}}%
\pgfpathlineto{\pgfqpoint{4.138337in}{1.089188in}}%
\pgfpathlineto{\pgfqpoint{4.139748in}{1.089798in}}%
\pgfpathlineto{\pgfqpoint{4.143984in}{1.104267in}}%
\pgfpathlineto{\pgfqpoint{4.145395in}{1.103743in}}%
\pgfpathlineto{\pgfqpoint{4.148219in}{1.099587in}}%
\pgfpathlineto{\pgfqpoint{4.153866in}{1.085776in}}%
\pgfpathlineto{\pgfqpoint{4.155278in}{1.085548in}}%
\pgfpathlineto{\pgfqpoint{4.156689in}{1.087677in}}%
\pgfpathlineto{\pgfqpoint{4.159513in}{1.100883in}}%
\pgfpathlineto{\pgfqpoint{4.169395in}{1.163966in}}%
\pgfpathlineto{\pgfqpoint{4.170807in}{1.162866in}}%
\pgfpathlineto{\pgfqpoint{4.172218in}{1.157918in}}%
\pgfpathlineto{\pgfqpoint{4.182101in}{1.090639in}}%
\pgfpathlineto{\pgfqpoint{4.187747in}{1.046001in}}%
\pgfpathlineto{\pgfqpoint{4.189159in}{1.050151in}}%
\pgfpathlineto{\pgfqpoint{4.191983in}{1.073169in}}%
\pgfpathlineto{\pgfqpoint{4.196218in}{1.128211in}}%
\pgfpathlineto{\pgfqpoint{4.206100in}{1.305742in}}%
\pgfpathlineto{\pgfqpoint{4.210335in}{1.349275in}}%
\pgfpathlineto{\pgfqpoint{4.214571in}{1.371716in}}%
\pgfpathlineto{\pgfqpoint{4.217394in}{1.377617in}}%
\pgfpathlineto{\pgfqpoint{4.220217in}{1.378116in}}%
\pgfpathlineto{\pgfqpoint{4.223041in}{1.374438in}}%
\pgfpathlineto{\pgfqpoint{4.227276in}{1.365381in}}%
\pgfpathlineto{\pgfqpoint{4.230100in}{1.356678in}}%
\pgfpathlineto{\pgfqpoint{4.234335in}{1.346466in}}%
\pgfpathlineto{\pgfqpoint{4.237158in}{1.332198in}}%
\pgfpathlineto{\pgfqpoint{4.239982in}{1.302646in}}%
\pgfpathlineto{\pgfqpoint{4.247040in}{1.176768in}}%
\pgfpathlineto{\pgfqpoint{4.252687in}{1.085994in}}%
\pgfpathlineto{\pgfqpoint{4.255511in}{1.074815in}}%
\pgfpathlineto{\pgfqpoint{4.256923in}{1.078568in}}%
\pgfpathlineto{\pgfqpoint{4.262570in}{1.112011in}}%
\pgfpathlineto{\pgfqpoint{4.263981in}{1.114729in}}%
\pgfpathlineto{\pgfqpoint{4.265393in}{1.113469in}}%
\pgfpathlineto{\pgfqpoint{4.268217in}{1.102313in}}%
\pgfpathlineto{\pgfqpoint{4.271040in}{1.083557in}}%
\pgfpathlineto{\pgfqpoint{4.275275in}{1.056255in}}%
\pgfpathlineto{\pgfqpoint{4.283746in}{1.024177in}}%
\pgfpathlineto{\pgfqpoint{4.286569in}{1.018689in}}%
\pgfpathlineto{\pgfqpoint{4.289393in}{1.018315in}}%
\pgfpathlineto{\pgfqpoint{4.290804in}{1.020297in}}%
\pgfpathlineto{\pgfqpoint{4.293628in}{1.034869in}}%
\pgfpathlineto{\pgfqpoint{4.296451in}{1.068442in}}%
\pgfpathlineto{\pgfqpoint{4.302098in}{1.177033in}}%
\pgfpathlineto{\pgfqpoint{4.307745in}{1.277424in}}%
\pgfpathlineto{\pgfqpoint{4.311980in}{1.322895in}}%
\pgfpathlineto{\pgfqpoint{4.314804in}{1.337193in}}%
\pgfpathlineto{\pgfqpoint{4.317627in}{1.341676in}}%
\pgfpathlineto{\pgfqpoint{4.320451in}{1.337236in}}%
\pgfpathlineto{\pgfqpoint{4.326098in}{1.313631in}}%
\pgfpathlineto{\pgfqpoint{4.331745in}{1.292069in}}%
\pgfpathlineto{\pgfqpoint{4.334568in}{1.282869in}}%
\pgfpathlineto{\pgfqpoint{4.340215in}{1.241926in}}%
\pgfpathlineto{\pgfqpoint{4.348686in}{1.173516in}}%
\pgfpathlineto{\pgfqpoint{4.352921in}{1.156245in}}%
\pgfpathlineto{\pgfqpoint{4.355744in}{1.151262in}}%
\pgfpathlineto{\pgfqpoint{4.358568in}{1.153975in}}%
\pgfpathlineto{\pgfqpoint{4.361391in}{1.156562in}}%
\pgfpathlineto{\pgfqpoint{4.362803in}{1.155154in}}%
\pgfpathlineto{\pgfqpoint{4.365626in}{1.146147in}}%
\pgfpathlineto{\pgfqpoint{4.376920in}{1.091448in}}%
\pgfpathlineto{\pgfqpoint{4.388214in}{1.012243in}}%
\pgfpathlineto{\pgfqpoint{4.391038in}{0.999158in}}%
\pgfpathlineto{\pgfqpoint{4.392449in}{0.996541in}}%
\pgfpathlineto{\pgfqpoint{4.395273in}{0.997448in}}%
\pgfpathlineto{\pgfqpoint{4.398096in}{1.010318in}}%
\pgfpathlineto{\pgfqpoint{4.400920in}{1.036202in}}%
\pgfpathlineto{\pgfqpoint{4.406567in}{1.120981in}}%
\pgfpathlineto{\pgfqpoint{4.413626in}{1.224621in}}%
\pgfpathlineto{\pgfqpoint{4.417861in}{1.257433in}}%
\pgfpathlineto{\pgfqpoint{4.422096in}{1.272483in}}%
\pgfpathlineto{\pgfqpoint{4.424919in}{1.275995in}}%
\pgfpathlineto{\pgfqpoint{4.426331in}{1.276195in}}%
\pgfpathlineto{\pgfqpoint{4.429155in}{1.271767in}}%
\pgfpathlineto{\pgfqpoint{4.433390in}{1.257043in}}%
\pgfpathlineto{\pgfqpoint{4.440449in}{1.231360in}}%
\pgfpathlineto{\pgfqpoint{4.444684in}{1.219894in}}%
\pgfpathlineto{\pgfqpoint{4.454566in}{1.174634in}}%
\pgfpathlineto{\pgfqpoint{4.464448in}{1.107619in}}%
\pgfpathlineto{\pgfqpoint{4.477154in}{1.004445in}}%
\pgfpathlineto{\pgfqpoint{4.479977in}{0.978563in}}%
\pgfpathlineto{\pgfqpoint{4.487036in}{0.938328in}}%
\pgfpathlineto{\pgfqpoint{4.489859in}{0.931441in}}%
\pgfpathlineto{\pgfqpoint{4.491271in}{0.931214in}}%
\pgfpathlineto{\pgfqpoint{4.492683in}{0.932476in}}%
\pgfpathlineto{\pgfqpoint{4.495506in}{0.940317in}}%
\pgfpathlineto{\pgfqpoint{4.499742in}{0.961644in}}%
\pgfpathlineto{\pgfqpoint{4.506800in}{1.012689in}}%
\pgfpathlineto{\pgfqpoint{4.518094in}{1.096969in}}%
\pgfpathlineto{\pgfqpoint{4.522329in}{1.116983in}}%
\pgfpathlineto{\pgfqpoint{4.525153in}{1.122023in}}%
\pgfpathlineto{\pgfqpoint{4.529388in}{1.127268in}}%
\pgfpathlineto{\pgfqpoint{4.532212in}{1.126694in}}%
\pgfpathlineto{\pgfqpoint{4.535035in}{1.123288in}}%
\pgfpathlineto{\pgfqpoint{4.537858in}{1.117490in}}%
\pgfpathlineto{\pgfqpoint{4.542094in}{1.104130in}}%
\pgfpathlineto{\pgfqpoint{4.550564in}{1.069504in}}%
\pgfpathlineto{\pgfqpoint{4.557623in}{1.043504in}}%
\pgfpathlineto{\pgfqpoint{4.564682in}{1.016150in}}%
\pgfpathlineto{\pgfqpoint{4.577387in}{0.945218in}}%
\pgfpathlineto{\pgfqpoint{4.584446in}{0.894874in}}%
\pgfpathlineto{\pgfqpoint{4.591505in}{0.841862in}}%
\pgfpathlineto{\pgfqpoint{4.595740in}{0.822251in}}%
\pgfpathlineto{\pgfqpoint{4.599975in}{0.810535in}}%
\pgfpathlineto{\pgfqpoint{4.604210in}{0.804547in}}%
\pgfpathlineto{\pgfqpoint{4.608445in}{0.802647in}}%
\pgfpathlineto{\pgfqpoint{4.615504in}{0.800592in}}%
\pgfpathlineto{\pgfqpoint{4.626798in}{0.791288in}}%
\pgfpathlineto{\pgfqpoint{4.693150in}{0.737743in}}%
\pgfpathlineto{\pgfqpoint{4.712914in}{0.728032in}}%
\pgfpathlineto{\pgfqpoint{4.756678in}{0.709585in}}%
\pgfpathlineto{\pgfqpoint{4.767972in}{0.707415in}}%
\pgfpathlineto{\pgfqpoint{4.801853in}{0.703792in}}%
\pgfpathlineto{\pgfqpoint{4.818794in}{0.703704in}}%
\pgfpathlineto{\pgfqpoint{4.820206in}{0.696295in}}%
\pgfpathlineto{\pgfqpoint{4.823030in}{0.696265in}}%
\pgfpathlineto{\pgfqpoint{4.824441in}{0.703414in}}%
\pgfpathlineto{\pgfqpoint{4.837147in}{0.703066in}}%
\pgfpathlineto{\pgfqpoint{4.838559in}{0.696390in}}%
\pgfpathlineto{\pgfqpoint{4.852676in}{0.696282in}}%
\pgfpathlineto{\pgfqpoint{4.854088in}{0.701844in}}%
\pgfpathlineto{\pgfqpoint{4.855500in}{0.696298in}}%
\pgfpathlineto{\pgfqpoint{4.861146in}{0.696294in}}%
\pgfpathlineto{\pgfqpoint{4.862558in}{0.702086in}}%
\pgfpathlineto{\pgfqpoint{4.866793in}{0.701973in}}%
\pgfpathlineto{\pgfqpoint{4.868205in}{0.696306in}}%
\pgfpathlineto{\pgfqpoint{4.880911in}{0.696365in}}%
\pgfpathlineto{\pgfqpoint{4.882323in}{0.702397in}}%
\pgfpathlineto{\pgfqpoint{4.885146in}{0.702578in}}%
\pgfpathlineto{\pgfqpoint{4.886558in}{0.696403in}}%
\pgfpathlineto{\pgfqpoint{4.895028in}{0.696361in}}%
\pgfpathlineto{\pgfqpoint{4.896440in}{0.701621in}}%
\pgfpathlineto{\pgfqpoint{4.916204in}{0.701339in}}%
\pgfpathlineto{\pgfqpoint{4.917616in}{0.696351in}}%
\pgfpathlineto{\pgfqpoint{4.957145in}{0.698003in}}%
\pgfpathlineto{\pgfqpoint{4.958556in}{0.698198in}}%
\pgfpathlineto{\pgfqpoint{4.959968in}{0.696022in}}%
\pgfpathlineto{\pgfqpoint{4.961380in}{0.698577in}}%
\pgfpathlineto{\pgfqpoint{4.962792in}{0.697085in}}%
\pgfpathlineto{\pgfqpoint{4.964203in}{0.698520in}}%
\pgfpathlineto{\pgfqpoint{4.965615in}{0.696898in}}%
\pgfpathlineto{\pgfqpoint{4.971262in}{0.697275in}}%
\pgfpathlineto{\pgfqpoint{4.983968in}{0.696081in}}%
\pgfpathlineto{\pgfqpoint{4.989615in}{0.696379in}}%
\pgfpathlineto{\pgfqpoint{4.993850in}{0.696461in}}%
\pgfpathlineto{\pgfqpoint{5.012202in}{0.696758in}}%
\pgfpathlineto{\pgfqpoint{5.013614in}{0.696106in}}%
\pgfpathlineto{\pgfqpoint{5.016438in}{0.697596in}}%
\pgfpathlineto{\pgfqpoint{5.017849in}{0.696019in}}%
\pgfpathlineto{\pgfqpoint{5.020673in}{0.696407in}}%
\pgfpathlineto{\pgfqpoint{5.031967in}{0.696420in}}%
\pgfpathlineto{\pgfqpoint{5.033378in}{0.698532in}}%
\pgfpathlineto{\pgfqpoint{5.040437in}{0.698555in}}%
\pgfpathlineto{\pgfqpoint{5.041849in}{0.696000in}}%
\pgfpathlineto{\pgfqpoint{5.496429in}{0.696000in}}%
\pgfpathlineto{\pgfqpoint{5.499252in}{0.697232in}}%
\pgfpathlineto{\pgfqpoint{5.511958in}{0.696752in}}%
\pgfpathlineto{\pgfqpoint{5.513369in}{0.698953in}}%
\pgfpathlineto{\pgfqpoint{5.514781in}{0.696000in}}%
\pgfpathlineto{\pgfqpoint{5.517605in}{0.697616in}}%
\pgfpathlineto{\pgfqpoint{5.520428in}{0.697133in}}%
\pgfpathlineto{\pgfqpoint{5.534545in}{0.696000in}}%
\pgfpathlineto{\pgfqpoint{5.534545in}{0.696000in}}%
\pgfusepath{stroke}%
\end{pgfscope}%
\begin{pgfscope}%
\pgfpathrectangle{\pgfqpoint{0.800000in}{0.528000in}}{\pgfqpoint{4.960000in}{3.696000in}} %
\pgfusepath{clip}%
\pgfsetrectcap%
\pgfsetroundjoin%
\pgfsetlinewidth{1.505625pt}%
\definecolor{currentstroke}{rgb}{0.580392,0.403922,0.741176}%
\pgfsetstrokecolor{currentstroke}%
\pgfsetdash{}{0pt}%
\pgfpathmoveto{\pgfqpoint{1.025455in}{0.696000in}}%
\pgfpathlineto{\pgfqpoint{1.162393in}{0.696826in}}%
\pgfpathlineto{\pgfqpoint{1.163805in}{0.697479in}}%
\pgfpathlineto{\pgfqpoint{1.165217in}{0.696292in}}%
\pgfpathlineto{\pgfqpoint{1.169452in}{0.714581in}}%
\pgfpathlineto{\pgfqpoint{1.170864in}{0.706972in}}%
\pgfpathlineto{\pgfqpoint{1.173687in}{0.714114in}}%
\pgfpathlineto{\pgfqpoint{1.179334in}{0.739862in}}%
\pgfpathlineto{\pgfqpoint{1.184981in}{0.763894in}}%
\pgfpathlineto{\pgfqpoint{1.201922in}{0.854725in}}%
\pgfpathlineto{\pgfqpoint{1.204745in}{0.861709in}}%
\pgfpathlineto{\pgfqpoint{1.208980in}{0.877594in}}%
\pgfpathlineto{\pgfqpoint{1.211804in}{0.887979in}}%
\pgfpathlineto{\pgfqpoint{1.220274in}{0.904906in}}%
\pgfpathlineto{\pgfqpoint{1.241450in}{0.943963in}}%
\pgfpathlineto{\pgfqpoint{1.248509in}{0.956384in}}%
\pgfpathlineto{\pgfqpoint{1.254156in}{0.966083in}}%
\pgfpathlineto{\pgfqpoint{1.256980in}{0.966760in}}%
\pgfpathlineto{\pgfqpoint{1.259803in}{0.964538in}}%
\pgfpathlineto{\pgfqpoint{1.268273in}{0.952207in}}%
\pgfpathlineto{\pgfqpoint{1.275332in}{0.937524in}}%
\pgfpathlineto{\pgfqpoint{1.283803in}{0.919204in}}%
\pgfpathlineto{\pgfqpoint{1.289450in}{0.916185in}}%
\pgfpathlineto{\pgfqpoint{1.292273in}{0.916434in}}%
\pgfpathlineto{\pgfqpoint{1.295096in}{0.918425in}}%
\pgfpathlineto{\pgfqpoint{1.297920in}{0.922987in}}%
\pgfpathlineto{\pgfqpoint{1.300743in}{0.923737in}}%
\pgfpathlineto{\pgfqpoint{1.303567in}{0.922301in}}%
\pgfpathlineto{\pgfqpoint{1.306390in}{0.917720in}}%
\pgfpathlineto{\pgfqpoint{1.316273in}{0.897126in}}%
\pgfpathlineto{\pgfqpoint{1.323331in}{0.890547in}}%
\pgfpathlineto{\pgfqpoint{1.324743in}{0.889235in}}%
\pgfpathlineto{\pgfqpoint{1.327566in}{0.883324in}}%
\pgfpathlineto{\pgfqpoint{1.334625in}{0.875713in}}%
\pgfpathlineto{\pgfqpoint{1.338860in}{0.872592in}}%
\pgfpathlineto{\pgfqpoint{1.341684in}{0.872681in}}%
\pgfpathlineto{\pgfqpoint{1.343096in}{0.873989in}}%
\pgfpathlineto{\pgfqpoint{1.355801in}{0.911842in}}%
\pgfpathlineto{\pgfqpoint{1.369919in}{0.978450in}}%
\pgfpathlineto{\pgfqpoint{1.378389in}{1.021929in}}%
\pgfpathlineto{\pgfqpoint{1.381213in}{1.026464in}}%
\pgfpathlineto{\pgfqpoint{1.382624in}{1.025762in}}%
\pgfpathlineto{\pgfqpoint{1.385448in}{1.018075in}}%
\pgfpathlineto{\pgfqpoint{1.393918in}{0.985453in}}%
\pgfpathlineto{\pgfqpoint{1.402389in}{0.966417in}}%
\pgfpathlineto{\pgfqpoint{1.406624in}{0.958468in}}%
\pgfpathlineto{\pgfqpoint{1.409447in}{0.956977in}}%
\pgfpathlineto{\pgfqpoint{1.412271in}{0.958557in}}%
\pgfpathlineto{\pgfqpoint{1.416506in}{0.963293in}}%
\pgfpathlineto{\pgfqpoint{1.422153in}{0.954049in}}%
\pgfpathlineto{\pgfqpoint{1.424976in}{0.946757in}}%
\pgfpathlineto{\pgfqpoint{1.429212in}{0.939094in}}%
\pgfpathlineto{\pgfqpoint{1.436270in}{0.924948in}}%
\pgfpathlineto{\pgfqpoint{1.439094in}{0.921707in}}%
\pgfpathlineto{\pgfqpoint{1.440505in}{0.922324in}}%
\pgfpathlineto{\pgfqpoint{1.450388in}{0.941295in}}%
\pgfpathlineto{\pgfqpoint{1.458858in}{0.952270in}}%
\pgfpathlineto{\pgfqpoint{1.471564in}{1.055563in}}%
\pgfpathlineto{\pgfqpoint{1.475799in}{1.064892in}}%
\pgfpathlineto{\pgfqpoint{1.478622in}{1.065611in}}%
\pgfpathlineto{\pgfqpoint{1.481446in}{1.063125in}}%
\pgfpathlineto{\pgfqpoint{1.484269in}{1.058089in}}%
\pgfpathlineto{\pgfqpoint{1.491328in}{1.062101in}}%
\pgfpathlineto{\pgfqpoint{1.492740in}{1.062394in}}%
\pgfpathlineto{\pgfqpoint{1.494152in}{1.064530in}}%
\pgfpathlineto{\pgfqpoint{1.496975in}{1.063363in}}%
\pgfpathlineto{\pgfqpoint{1.499798in}{1.059872in}}%
\pgfpathlineto{\pgfqpoint{1.504034in}{1.050149in}}%
\pgfpathlineto{\pgfqpoint{1.512504in}{1.012106in}}%
\pgfpathlineto{\pgfqpoint{1.522386in}{0.995488in}}%
\pgfpathlineto{\pgfqpoint{1.535092in}{0.977272in}}%
\pgfpathlineto{\pgfqpoint{1.536504in}{0.974206in}}%
\pgfpathlineto{\pgfqpoint{1.543562in}{0.986513in}}%
\pgfpathlineto{\pgfqpoint{1.547798in}{0.989378in}}%
\pgfpathlineto{\pgfqpoint{1.550621in}{0.987109in}}%
\pgfpathlineto{\pgfqpoint{1.552033in}{0.988349in}}%
\pgfpathlineto{\pgfqpoint{1.564738in}{1.072523in}}%
\pgfpathlineto{\pgfqpoint{1.571797in}{1.100677in}}%
\pgfpathlineto{\pgfqpoint{1.576032in}{1.111268in}}%
\pgfpathlineto{\pgfqpoint{1.577444in}{1.112555in}}%
\pgfpathlineto{\pgfqpoint{1.578856in}{1.112336in}}%
\pgfpathlineto{\pgfqpoint{1.581679in}{1.107914in}}%
\pgfpathlineto{\pgfqpoint{1.584503in}{1.097776in}}%
\pgfpathlineto{\pgfqpoint{1.588738in}{1.073445in}}%
\pgfpathlineto{\pgfqpoint{1.597208in}{1.025772in}}%
\pgfpathlineto{\pgfqpoint{1.604267in}{1.001403in}}%
\pgfpathlineto{\pgfqpoint{1.619796in}{0.925524in}}%
\pgfpathlineto{\pgfqpoint{1.629678in}{0.900105in}}%
\pgfpathlineto{\pgfqpoint{1.633914in}{0.891098in}}%
\pgfpathlineto{\pgfqpoint{1.635325in}{0.890137in}}%
\pgfpathlineto{\pgfqpoint{1.636737in}{0.891519in}}%
\pgfpathlineto{\pgfqpoint{1.639561in}{0.891695in}}%
\pgfpathlineto{\pgfqpoint{1.642384in}{0.893825in}}%
\pgfpathlineto{\pgfqpoint{1.643796in}{0.894718in}}%
\pgfpathlineto{\pgfqpoint{1.646619in}{0.910508in}}%
\pgfpathlineto{\pgfqpoint{1.656501in}{1.006477in}}%
\pgfpathlineto{\pgfqpoint{1.660737in}{1.045074in}}%
\pgfpathlineto{\pgfqpoint{1.663560in}{1.057675in}}%
\pgfpathlineto{\pgfqpoint{1.664972in}{1.059060in}}%
\pgfpathlineto{\pgfqpoint{1.666384in}{1.057461in}}%
\pgfpathlineto{\pgfqpoint{1.670619in}{1.033125in}}%
\pgfpathlineto{\pgfqpoint{1.676266in}{0.996911in}}%
\pgfpathlineto{\pgfqpoint{1.679089in}{0.990628in}}%
\pgfpathlineto{\pgfqpoint{1.680501in}{0.991789in}}%
\pgfpathlineto{\pgfqpoint{1.683324in}{0.999799in}}%
\pgfpathlineto{\pgfqpoint{1.690383in}{1.027009in}}%
\pgfpathlineto{\pgfqpoint{1.693207in}{1.030828in}}%
\pgfpathlineto{\pgfqpoint{1.694618in}{1.029983in}}%
\pgfpathlineto{\pgfqpoint{1.697442in}{1.023180in}}%
\pgfpathlineto{\pgfqpoint{1.701677in}{0.998950in}}%
\pgfpathlineto{\pgfqpoint{1.704500in}{0.998679in}}%
\pgfpathlineto{\pgfqpoint{1.710147in}{0.977968in}}%
\pgfpathlineto{\pgfqpoint{1.714383in}{0.962905in}}%
\pgfpathlineto{\pgfqpoint{1.717206in}{0.959034in}}%
\pgfpathlineto{\pgfqpoint{1.722853in}{0.958082in}}%
\pgfpathlineto{\pgfqpoint{1.724265in}{0.956466in}}%
\pgfpathlineto{\pgfqpoint{1.725677in}{0.956810in}}%
\pgfpathlineto{\pgfqpoint{1.729912in}{0.951848in}}%
\pgfpathlineto{\pgfqpoint{1.731324in}{0.951642in}}%
\pgfpathlineto{\pgfqpoint{1.732735in}{0.952652in}}%
\pgfpathlineto{\pgfqpoint{1.738382in}{0.961181in}}%
\pgfpathlineto{\pgfqpoint{1.749676in}{1.042762in}}%
\pgfpathlineto{\pgfqpoint{1.752500in}{1.051089in}}%
\pgfpathlineto{\pgfqpoint{1.753911in}{1.051653in}}%
\pgfpathlineto{\pgfqpoint{1.755323in}{1.048920in}}%
\pgfpathlineto{\pgfqpoint{1.756735in}{1.049976in}}%
\pgfpathlineto{\pgfqpoint{1.758147in}{1.048989in}}%
\pgfpathlineto{\pgfqpoint{1.760970in}{1.040286in}}%
\pgfpathlineto{\pgfqpoint{1.768029in}{1.014996in}}%
\pgfpathlineto{\pgfqpoint{1.770852in}{1.012865in}}%
\pgfpathlineto{\pgfqpoint{1.775087in}{1.013280in}}%
\pgfpathlineto{\pgfqpoint{1.783558in}{1.033897in}}%
\pgfpathlineto{\pgfqpoint{1.786381in}{1.027985in}}%
\pgfpathlineto{\pgfqpoint{1.790616in}{1.006272in}}%
\pgfpathlineto{\pgfqpoint{1.793440in}{1.008769in}}%
\pgfpathlineto{\pgfqpoint{1.799087in}{1.000646in}}%
\pgfpathlineto{\pgfqpoint{1.807557in}{0.995226in}}%
\pgfpathlineto{\pgfqpoint{1.808969in}{0.992956in}}%
\pgfpathlineto{\pgfqpoint{1.814616in}{0.975087in}}%
\pgfpathlineto{\pgfqpoint{1.818851in}{0.968599in}}%
\pgfpathlineto{\pgfqpoint{1.823086in}{0.966371in}}%
\pgfpathlineto{\pgfqpoint{1.824498in}{0.967568in}}%
\pgfpathlineto{\pgfqpoint{1.831557in}{0.961461in}}%
\pgfpathlineto{\pgfqpoint{1.834380in}{0.962569in}}%
\pgfpathlineto{\pgfqpoint{1.840027in}{0.969278in}}%
\pgfpathlineto{\pgfqpoint{1.841439in}{0.974318in}}%
\pgfpathlineto{\pgfqpoint{1.851321in}{1.050859in}}%
\pgfpathlineto{\pgfqpoint{1.854145in}{1.062110in}}%
\pgfpathlineto{\pgfqpoint{1.855556in}{1.064734in}}%
\pgfpathlineto{\pgfqpoint{1.858380in}{1.058709in}}%
\pgfpathlineto{\pgfqpoint{1.869674in}{1.016320in}}%
\pgfpathlineto{\pgfqpoint{1.872497in}{1.010964in}}%
\pgfpathlineto{\pgfqpoint{1.878144in}{1.023789in}}%
\pgfpathlineto{\pgfqpoint{1.880968in}{1.022923in}}%
\pgfpathlineto{\pgfqpoint{1.882379in}{1.021145in}}%
\pgfpathlineto{\pgfqpoint{1.888026in}{0.998015in}}%
\pgfpathlineto{\pgfqpoint{1.889438in}{0.996904in}}%
\pgfpathlineto{\pgfqpoint{1.899320in}{0.947119in}}%
\pgfpathlineto{\pgfqpoint{1.902144in}{0.934802in}}%
\pgfpathlineto{\pgfqpoint{1.909202in}{0.919685in}}%
\pgfpathlineto{\pgfqpoint{1.913438in}{0.910890in}}%
\pgfpathlineto{\pgfqpoint{1.916261in}{0.910327in}}%
\pgfpathlineto{\pgfqpoint{1.919085in}{0.914326in}}%
\pgfpathlineto{\pgfqpoint{1.923320in}{0.926814in}}%
\pgfpathlineto{\pgfqpoint{1.924732in}{0.931898in}}%
\pgfpathlineto{\pgfqpoint{1.927555in}{0.934025in}}%
\pgfpathlineto{\pgfqpoint{1.933202in}{0.956858in}}%
\pgfpathlineto{\pgfqpoint{1.937437in}{0.960424in}}%
\pgfpathlineto{\pgfqpoint{1.940261in}{0.956134in}}%
\pgfpathlineto{\pgfqpoint{1.941672in}{0.955570in}}%
\pgfpathlineto{\pgfqpoint{1.947319in}{0.992711in}}%
\pgfpathlineto{\pgfqpoint{1.962849in}{1.093526in}}%
\pgfpathlineto{\pgfqpoint{1.964260in}{1.095359in}}%
\pgfpathlineto{\pgfqpoint{1.965672in}{1.095322in}}%
\pgfpathlineto{\pgfqpoint{1.968495in}{1.091218in}}%
\pgfpathlineto{\pgfqpoint{1.971319in}{1.082212in}}%
\pgfpathlineto{\pgfqpoint{1.976966in}{1.052750in}}%
\pgfpathlineto{\pgfqpoint{1.979789in}{1.040322in}}%
\pgfpathlineto{\pgfqpoint{1.984025in}{1.023616in}}%
\pgfpathlineto{\pgfqpoint{1.986848in}{1.008426in}}%
\pgfpathlineto{\pgfqpoint{1.995318in}{0.961243in}}%
\pgfpathlineto{\pgfqpoint{2.000965in}{0.938645in}}%
\pgfpathlineto{\pgfqpoint{2.005201in}{0.932604in}}%
\pgfpathlineto{\pgfqpoint{2.009436in}{0.921044in}}%
\pgfpathlineto{\pgfqpoint{2.012259in}{0.917944in}}%
\pgfpathlineto{\pgfqpoint{2.015083in}{0.918310in}}%
\pgfpathlineto{\pgfqpoint{2.017906in}{0.921091in}}%
\pgfpathlineto{\pgfqpoint{2.020730in}{0.919703in}}%
\pgfpathlineto{\pgfqpoint{2.023553in}{0.919310in}}%
\pgfpathlineto{\pgfqpoint{2.034847in}{0.942000in}}%
\pgfpathlineto{\pgfqpoint{2.036259in}{0.943815in}}%
\pgfpathlineto{\pgfqpoint{2.044729in}{1.014223in}}%
\pgfpathlineto{\pgfqpoint{2.051788in}{1.068647in}}%
\pgfpathlineto{\pgfqpoint{2.056023in}{1.085090in}}%
\pgfpathlineto{\pgfqpoint{2.058847in}{1.087045in}}%
\pgfpathlineto{\pgfqpoint{2.061670in}{1.082382in}}%
\pgfpathlineto{\pgfqpoint{2.065905in}{1.064617in}}%
\pgfpathlineto{\pgfqpoint{2.077199in}{1.010113in}}%
\pgfpathlineto{\pgfqpoint{2.080023in}{1.005702in}}%
\pgfpathlineto{\pgfqpoint{2.082846in}{0.998054in}}%
\pgfpathlineto{\pgfqpoint{2.096964in}{0.927941in}}%
\pgfpathlineto{\pgfqpoint{2.102611in}{0.918482in}}%
\pgfpathlineto{\pgfqpoint{2.104022in}{0.918559in}}%
\pgfpathlineto{\pgfqpoint{2.106846in}{0.915417in}}%
\pgfpathlineto{\pgfqpoint{2.109669in}{0.915067in}}%
\pgfpathlineto{\pgfqpoint{2.112493in}{0.917590in}}%
\pgfpathlineto{\pgfqpoint{2.113904in}{0.920082in}}%
\pgfpathlineto{\pgfqpoint{2.115316in}{0.920128in}}%
\pgfpathlineto{\pgfqpoint{2.116728in}{0.921340in}}%
\pgfpathlineto{\pgfqpoint{2.118140in}{0.924724in}}%
\pgfpathlineto{\pgfqpoint{2.122375in}{0.944601in}}%
\pgfpathlineto{\pgfqpoint{2.125198in}{0.949448in}}%
\pgfpathlineto{\pgfqpoint{2.129434in}{0.953076in}}%
\pgfpathlineto{\pgfqpoint{2.130845in}{0.950900in}}%
\pgfpathlineto{\pgfqpoint{2.132257in}{0.950809in}}%
\pgfpathlineto{\pgfqpoint{2.135081in}{0.970848in}}%
\pgfpathlineto{\pgfqpoint{2.144963in}{1.062760in}}%
\pgfpathlineto{\pgfqpoint{2.147786in}{1.077031in}}%
\pgfpathlineto{\pgfqpoint{2.149198in}{1.080281in}}%
\pgfpathlineto{\pgfqpoint{2.150610in}{1.080716in}}%
\pgfpathlineto{\pgfqpoint{2.152021in}{1.078877in}}%
\pgfpathlineto{\pgfqpoint{2.157668in}{1.048293in}}%
\pgfpathlineto{\pgfqpoint{2.163315in}{1.018432in}}%
\pgfpathlineto{\pgfqpoint{2.166139in}{1.009769in}}%
\pgfpathlineto{\pgfqpoint{2.167551in}{1.008500in}}%
\pgfpathlineto{\pgfqpoint{2.178844in}{0.975618in}}%
\pgfpathlineto{\pgfqpoint{2.181668in}{0.970455in}}%
\pgfpathlineto{\pgfqpoint{2.183080in}{0.966552in}}%
\pgfpathlineto{\pgfqpoint{2.190138in}{0.925578in}}%
\pgfpathlineto{\pgfqpoint{2.194374in}{0.895135in}}%
\pgfpathlineto{\pgfqpoint{2.198609in}{0.882837in}}%
\pgfpathlineto{\pgfqpoint{2.201432in}{0.875202in}}%
\pgfpathlineto{\pgfqpoint{2.202844in}{0.873680in}}%
\pgfpathlineto{\pgfqpoint{2.204256in}{0.873822in}}%
\pgfpathlineto{\pgfqpoint{2.207079in}{0.878828in}}%
\pgfpathlineto{\pgfqpoint{2.211314in}{0.895091in}}%
\pgfpathlineto{\pgfqpoint{2.216961in}{0.915778in}}%
\pgfpathlineto{\pgfqpoint{2.221197in}{0.922469in}}%
\pgfpathlineto{\pgfqpoint{2.222608in}{0.924162in}}%
\pgfpathlineto{\pgfqpoint{2.226844in}{0.937496in}}%
\pgfpathlineto{\pgfqpoint{2.231079in}{0.940270in}}%
\pgfpathlineto{\pgfqpoint{2.233902in}{0.942118in}}%
\pgfpathlineto{\pgfqpoint{2.235314in}{0.939759in}}%
\pgfpathlineto{\pgfqpoint{2.238137in}{0.940178in}}%
\pgfpathlineto{\pgfqpoint{2.250843in}{1.018981in}}%
\pgfpathlineto{\pgfqpoint{2.257902in}{1.057675in}}%
\pgfpathlineto{\pgfqpoint{2.260725in}{1.065359in}}%
\pgfpathlineto{\pgfqpoint{2.264960in}{1.060346in}}%
\pgfpathlineto{\pgfqpoint{2.267784in}{1.049976in}}%
\pgfpathlineto{\pgfqpoint{2.272019in}{1.033817in}}%
\pgfpathlineto{\pgfqpoint{2.274843in}{1.028113in}}%
\pgfpathlineto{\pgfqpoint{2.279078in}{1.010681in}}%
\pgfpathlineto{\pgfqpoint{2.283313in}{0.978539in}}%
\pgfpathlineto{\pgfqpoint{2.287548in}{0.936611in}}%
\pgfpathlineto{\pgfqpoint{2.291783in}{0.922368in}}%
\pgfpathlineto{\pgfqpoint{2.294607in}{0.919837in}}%
\pgfpathlineto{\pgfqpoint{2.296019in}{0.921130in}}%
\pgfpathlineto{\pgfqpoint{2.298842in}{0.928003in}}%
\pgfpathlineto{\pgfqpoint{2.307313in}{0.954838in}}%
\pgfpathlineto{\pgfqpoint{2.310136in}{0.958967in}}%
\pgfpathlineto{\pgfqpoint{2.312960in}{0.959837in}}%
\pgfpathlineto{\pgfqpoint{2.314371in}{0.959293in}}%
\pgfpathlineto{\pgfqpoint{2.321430in}{0.941895in}}%
\pgfpathlineto{\pgfqpoint{2.329900in}{0.947446in}}%
\pgfpathlineto{\pgfqpoint{2.331312in}{0.945604in}}%
\pgfpathlineto{\pgfqpoint{2.332724in}{0.946050in}}%
\pgfpathlineto{\pgfqpoint{2.334136in}{0.949020in}}%
\pgfpathlineto{\pgfqpoint{2.339783in}{0.987198in}}%
\pgfpathlineto{\pgfqpoint{2.351076in}{1.066212in}}%
\pgfpathlineto{\pgfqpoint{2.353900in}{1.074910in}}%
\pgfpathlineto{\pgfqpoint{2.356723in}{1.076344in}}%
\pgfpathlineto{\pgfqpoint{2.359547in}{1.072457in}}%
\pgfpathlineto{\pgfqpoint{2.363782in}{1.059134in}}%
\pgfpathlineto{\pgfqpoint{2.368017in}{1.041889in}}%
\pgfpathlineto{\pgfqpoint{2.373664in}{1.004849in}}%
\pgfpathlineto{\pgfqpoint{2.376488in}{0.997429in}}%
\pgfpathlineto{\pgfqpoint{2.384958in}{0.946960in}}%
\pgfpathlineto{\pgfqpoint{2.387782in}{0.930731in}}%
\pgfpathlineto{\pgfqpoint{2.390605in}{0.924346in}}%
\pgfpathlineto{\pgfqpoint{2.392017in}{0.923815in}}%
\pgfpathlineto{\pgfqpoint{2.394840in}{0.919286in}}%
\pgfpathlineto{\pgfqpoint{2.396252in}{0.918853in}}%
\pgfpathlineto{\pgfqpoint{2.399076in}{0.921725in}}%
\pgfpathlineto{\pgfqpoint{2.403311in}{0.932704in}}%
\pgfpathlineto{\pgfqpoint{2.407546in}{0.951027in}}%
\pgfpathlineto{\pgfqpoint{2.414605in}{0.983034in}}%
\pgfpathlineto{\pgfqpoint{2.417428in}{0.982693in}}%
\pgfpathlineto{\pgfqpoint{2.421663in}{0.976783in}}%
\pgfpathlineto{\pgfqpoint{2.425899in}{0.971425in}}%
\pgfpathlineto{\pgfqpoint{2.430134in}{0.962126in}}%
\pgfpathlineto{\pgfqpoint{2.435781in}{0.958448in}}%
\pgfpathlineto{\pgfqpoint{2.438604in}{0.971059in}}%
\pgfpathlineto{\pgfqpoint{2.445663in}{1.013457in}}%
\pgfpathlineto{\pgfqpoint{2.454133in}{1.074113in}}%
\pgfpathlineto{\pgfqpoint{2.456957in}{1.082905in}}%
\pgfpathlineto{\pgfqpoint{2.459780in}{1.084905in}}%
\pgfpathlineto{\pgfqpoint{2.462604in}{1.080971in}}%
\pgfpathlineto{\pgfqpoint{2.473898in}{1.033227in}}%
\pgfpathlineto{\pgfqpoint{2.475309in}{1.030201in}}%
\pgfpathlineto{\pgfqpoint{2.483780in}{0.988176in}}%
\pgfpathlineto{\pgfqpoint{2.488015in}{0.962961in}}%
\pgfpathlineto{\pgfqpoint{2.493662in}{0.951306in}}%
\pgfpathlineto{\pgfqpoint{2.496485in}{0.947717in}}%
\pgfpathlineto{\pgfqpoint{2.499309in}{0.948339in}}%
\pgfpathlineto{\pgfqpoint{2.502132in}{0.952713in}}%
\pgfpathlineto{\pgfqpoint{2.506368in}{0.966046in}}%
\pgfpathlineto{\pgfqpoint{2.513426in}{0.989440in}}%
\pgfpathlineto{\pgfqpoint{2.516250in}{0.994125in}}%
\pgfpathlineto{\pgfqpoint{2.521897in}{0.983377in}}%
\pgfpathlineto{\pgfqpoint{2.523308in}{0.979505in}}%
\pgfpathlineto{\pgfqpoint{2.526132in}{0.979932in}}%
\pgfpathlineto{\pgfqpoint{2.527544in}{0.979293in}}%
\pgfpathlineto{\pgfqpoint{2.528955in}{0.980070in}}%
\pgfpathlineto{\pgfqpoint{2.533191in}{0.974156in}}%
\pgfpathlineto{\pgfqpoint{2.537426in}{0.972202in}}%
\pgfpathlineto{\pgfqpoint{2.538838in}{0.972937in}}%
\pgfpathlineto{\pgfqpoint{2.540249in}{0.971510in}}%
\pgfpathlineto{\pgfqpoint{2.543073in}{0.974347in}}%
\pgfpathlineto{\pgfqpoint{2.547308in}{0.986437in}}%
\pgfpathlineto{\pgfqpoint{2.550131in}{0.999370in}}%
\pgfpathlineto{\pgfqpoint{2.555778in}{1.032129in}}%
\pgfpathlineto{\pgfqpoint{2.560014in}{1.046361in}}%
\pgfpathlineto{\pgfqpoint{2.564249in}{1.053325in}}%
\pgfpathlineto{\pgfqpoint{2.567072in}{1.053982in}}%
\pgfpathlineto{\pgfqpoint{2.572719in}{1.052524in}}%
\pgfpathlineto{\pgfqpoint{2.576955in}{1.047631in}}%
\pgfpathlineto{\pgfqpoint{2.582601in}{1.037020in}}%
\pgfpathlineto{\pgfqpoint{2.591072in}{1.017838in}}%
\pgfpathlineto{\pgfqpoint{2.595307in}{1.002512in}}%
\pgfpathlineto{\pgfqpoint{2.600954in}{0.973129in}}%
\pgfpathlineto{\pgfqpoint{2.605189in}{0.947302in}}%
\pgfpathlineto{\pgfqpoint{2.609424in}{0.930682in}}%
\pgfpathlineto{\pgfqpoint{2.615071in}{0.915311in}}%
\pgfpathlineto{\pgfqpoint{2.620718in}{0.903231in}}%
\pgfpathlineto{\pgfqpoint{2.624954in}{0.890091in}}%
\pgfpathlineto{\pgfqpoint{2.632012in}{0.877776in}}%
\pgfpathlineto{\pgfqpoint{2.637659in}{0.870292in}}%
\pgfpathlineto{\pgfqpoint{2.641894in}{0.868219in}}%
\pgfpathlineto{\pgfqpoint{2.644718in}{0.869735in}}%
\pgfpathlineto{\pgfqpoint{2.647541in}{0.874651in}}%
\pgfpathlineto{\pgfqpoint{2.650365in}{0.883987in}}%
\pgfpathlineto{\pgfqpoint{2.672953in}{0.991447in}}%
\pgfpathlineto{\pgfqpoint{2.675776in}{1.001690in}}%
\pgfpathlineto{\pgfqpoint{2.677188in}{1.002124in}}%
\pgfpathlineto{\pgfqpoint{2.678600in}{0.995892in}}%
\pgfpathlineto{\pgfqpoint{2.685658in}{0.941922in}}%
\pgfpathlineto{\pgfqpoint{2.688482in}{0.927229in}}%
\pgfpathlineto{\pgfqpoint{2.689894in}{0.922829in}}%
\pgfpathlineto{\pgfqpoint{2.691305in}{0.908201in}}%
\pgfpathlineto{\pgfqpoint{2.694129in}{0.906278in}}%
\pgfpathlineto{\pgfqpoint{2.695541in}{0.907167in}}%
\pgfpathlineto{\pgfqpoint{2.696952in}{0.916060in}}%
\pgfpathlineto{\pgfqpoint{2.698364in}{0.915039in}}%
\pgfpathlineto{\pgfqpoint{2.699776in}{0.901676in}}%
\pgfpathlineto{\pgfqpoint{2.701187in}{0.903708in}}%
\pgfpathlineto{\pgfqpoint{2.702599in}{0.900054in}}%
\pgfpathlineto{\pgfqpoint{2.704011in}{0.906395in}}%
\pgfpathlineto{\pgfqpoint{2.706834in}{0.953703in}}%
\pgfpathlineto{\pgfqpoint{2.708246in}{0.958787in}}%
\pgfpathlineto{\pgfqpoint{2.711070in}{0.948555in}}%
\pgfpathlineto{\pgfqpoint{2.712481in}{0.950702in}}%
\pgfpathlineto{\pgfqpoint{2.722364in}{1.115066in}}%
\pgfpathlineto{\pgfqpoint{2.723775in}{1.115688in}}%
\pgfpathlineto{\pgfqpoint{2.728010in}{1.071935in}}%
\pgfpathlineto{\pgfqpoint{2.730834in}{1.034249in}}%
\pgfpathlineto{\pgfqpoint{2.733657in}{0.993601in}}%
\pgfpathlineto{\pgfqpoint{2.737893in}{1.008718in}}%
\pgfpathlineto{\pgfqpoint{2.739304in}{1.009914in}}%
\pgfpathlineto{\pgfqpoint{2.740716in}{1.008806in}}%
\pgfpathlineto{\pgfqpoint{2.742128in}{1.016897in}}%
\pgfpathlineto{\pgfqpoint{2.743540in}{1.016858in}}%
\pgfpathlineto{\pgfqpoint{2.744951in}{1.007891in}}%
\pgfpathlineto{\pgfqpoint{2.753422in}{0.894894in}}%
\pgfpathlineto{\pgfqpoint{2.757657in}{0.844481in}}%
\pgfpathlineto{\pgfqpoint{2.764716in}{0.793148in}}%
\pgfpathlineto{\pgfqpoint{2.766127in}{0.789146in}}%
\pgfpathlineto{\pgfqpoint{2.767539in}{0.792544in}}%
\pgfpathlineto{\pgfqpoint{2.773186in}{0.824667in}}%
\pgfpathlineto{\pgfqpoint{2.774598in}{0.749808in}}%
\pgfpathlineto{\pgfqpoint{2.778833in}{0.760496in}}%
\pgfpathlineto{\pgfqpoint{2.784480in}{0.776610in}}%
\pgfpathlineto{\pgfqpoint{2.785892in}{1.004196in}}%
\pgfpathlineto{\pgfqpoint{2.788715in}{1.028828in}}%
\pgfpathlineto{\pgfqpoint{2.790127in}{1.027910in}}%
\pgfpathlineto{\pgfqpoint{2.792950in}{1.012964in}}%
\pgfpathlineto{\pgfqpoint{2.797186in}{1.005163in}}%
\pgfpathlineto{\pgfqpoint{2.798597in}{1.006496in}}%
\pgfpathlineto{\pgfqpoint{2.801421in}{1.018720in}}%
\pgfpathlineto{\pgfqpoint{2.812715in}{1.104225in}}%
\pgfpathlineto{\pgfqpoint{2.816950in}{1.116048in}}%
\pgfpathlineto{\pgfqpoint{2.822597in}{1.123458in}}%
\pgfpathlineto{\pgfqpoint{2.825420in}{1.124900in}}%
\pgfpathlineto{\pgfqpoint{2.828244in}{1.124394in}}%
\pgfpathlineto{\pgfqpoint{2.836714in}{1.116399in}}%
\pgfpathlineto{\pgfqpoint{2.839538in}{1.116702in}}%
\pgfpathlineto{\pgfqpoint{2.848008in}{1.125372in}}%
\pgfpathlineto{\pgfqpoint{2.850832in}{1.125202in}}%
\pgfpathlineto{\pgfqpoint{2.853655in}{1.120602in}}%
\pgfpathlineto{\pgfqpoint{2.855067in}{1.119717in}}%
\pgfpathlineto{\pgfqpoint{2.856479in}{1.120581in}}%
\pgfpathlineto{\pgfqpoint{2.864949in}{1.116814in}}%
\pgfpathlineto{\pgfqpoint{2.869184in}{1.114254in}}%
\pgfpathlineto{\pgfqpoint{2.873419in}{1.114865in}}%
\pgfpathlineto{\pgfqpoint{2.877655in}{1.117803in}}%
\pgfpathlineto{\pgfqpoint{2.879066in}{1.117894in}}%
\pgfpathlineto{\pgfqpoint{2.887537in}{1.096112in}}%
\pgfpathlineto{\pgfqpoint{2.888949in}{1.096421in}}%
\pgfpathlineto{\pgfqpoint{2.896007in}{1.092855in}}%
\pgfpathlineto{\pgfqpoint{2.897419in}{1.092350in}}%
\pgfpathlineto{\pgfqpoint{2.898831in}{1.093995in}}%
\pgfpathlineto{\pgfqpoint{2.901654in}{1.091622in}}%
\pgfpathlineto{\pgfqpoint{2.910125in}{1.086071in}}%
\pgfpathlineto{\pgfqpoint{2.912948in}{1.088605in}}%
\pgfpathlineto{\pgfqpoint{2.917183in}{1.095814in}}%
\pgfpathlineto{\pgfqpoint{2.925654in}{1.114760in}}%
\pgfpathlineto{\pgfqpoint{2.928477in}{1.116525in}}%
\pgfpathlineto{\pgfqpoint{2.931301in}{1.116175in}}%
\pgfpathlineto{\pgfqpoint{2.934124in}{1.114613in}}%
\pgfpathlineto{\pgfqpoint{2.938359in}{1.109518in}}%
\pgfpathlineto{\pgfqpoint{2.939771in}{1.107480in}}%
\pgfpathlineto{\pgfqpoint{2.941183in}{1.108057in}}%
\pgfpathlineto{\pgfqpoint{2.946830in}{1.121736in}}%
\pgfpathlineto{\pgfqpoint{2.955300in}{1.134445in}}%
\pgfpathlineto{\pgfqpoint{2.958124in}{1.133577in}}%
\pgfpathlineto{\pgfqpoint{2.960947in}{1.129066in}}%
\pgfpathlineto{\pgfqpoint{2.963771in}{1.116814in}}%
\pgfpathlineto{\pgfqpoint{2.973653in}{1.072995in}}%
\pgfpathlineto{\pgfqpoint{2.976476in}{1.068289in}}%
\pgfpathlineto{\pgfqpoint{2.979300in}{1.069250in}}%
\pgfpathlineto{\pgfqpoint{2.982123in}{1.075737in}}%
\pgfpathlineto{\pgfqpoint{2.996241in}{1.101444in}}%
\pgfpathlineto{\pgfqpoint{2.999064in}{1.102935in}}%
\pgfpathlineto{\pgfqpoint{3.003299in}{1.102089in}}%
\pgfpathlineto{\pgfqpoint{3.007535in}{1.096509in}}%
\pgfpathlineto{\pgfqpoint{3.016005in}{1.091712in}}%
\pgfpathlineto{\pgfqpoint{3.023064in}{1.112398in}}%
\pgfpathlineto{\pgfqpoint{3.025887in}{1.118143in}}%
\pgfpathlineto{\pgfqpoint{3.031534in}{1.134322in}}%
\pgfpathlineto{\pgfqpoint{3.034358in}{1.140995in}}%
\pgfpathlineto{\pgfqpoint{3.040005in}{1.157139in}}%
\pgfpathlineto{\pgfqpoint{3.041416in}{1.159281in}}%
\pgfpathlineto{\pgfqpoint{3.051298in}{1.191110in}}%
\pgfpathlineto{\pgfqpoint{3.054122in}{1.190738in}}%
\pgfpathlineto{\pgfqpoint{3.058357in}{1.187756in}}%
\pgfpathlineto{\pgfqpoint{3.062592in}{1.177337in}}%
\pgfpathlineto{\pgfqpoint{3.066828in}{1.166688in}}%
\pgfpathlineto{\pgfqpoint{3.071063in}{1.172724in}}%
\pgfpathlineto{\pgfqpoint{3.073886in}{1.173177in}}%
\pgfpathlineto{\pgfqpoint{3.082357in}{1.168803in}}%
\pgfpathlineto{\pgfqpoint{3.085180in}{1.167127in}}%
\pgfpathlineto{\pgfqpoint{3.088004in}{1.164501in}}%
\pgfpathlineto{\pgfqpoint{3.090827in}{1.159293in}}%
\pgfpathlineto{\pgfqpoint{3.096474in}{1.143606in}}%
\pgfpathlineto{\pgfqpoint{3.097886in}{1.143119in}}%
\pgfpathlineto{\pgfqpoint{3.102121in}{1.132942in}}%
\pgfpathlineto{\pgfqpoint{3.107768in}{1.102261in}}%
\pgfpathlineto{\pgfqpoint{3.112003in}{1.070718in}}%
\pgfpathlineto{\pgfqpoint{3.113415in}{1.069786in}}%
\pgfpathlineto{\pgfqpoint{3.117650in}{1.078535in}}%
\pgfpathlineto{\pgfqpoint{3.134591in}{1.148789in}}%
\pgfpathlineto{\pgfqpoint{3.137414in}{1.147753in}}%
\pgfpathlineto{\pgfqpoint{3.138826in}{1.146149in}}%
\pgfpathlineto{\pgfqpoint{3.141650in}{1.138397in}}%
\pgfpathlineto{\pgfqpoint{3.148708in}{1.153503in}}%
\pgfpathlineto{\pgfqpoint{3.151532in}{1.148924in}}%
\pgfpathlineto{\pgfqpoint{3.154355in}{1.141094in}}%
\pgfpathlineto{\pgfqpoint{3.162826in}{1.102309in}}%
\pgfpathlineto{\pgfqpoint{3.168473in}{1.100041in}}%
\pgfpathlineto{\pgfqpoint{3.171296in}{1.101337in}}%
\pgfpathlineto{\pgfqpoint{3.176943in}{1.111595in}}%
\pgfpathlineto{\pgfqpoint{3.178355in}{1.115534in}}%
\pgfpathlineto{\pgfqpoint{3.179767in}{1.115520in}}%
\pgfpathlineto{\pgfqpoint{3.182590in}{1.111485in}}%
\pgfpathlineto{\pgfqpoint{3.185414in}{1.102024in}}%
\pgfpathlineto{\pgfqpoint{3.189649in}{1.078592in}}%
\pgfpathlineto{\pgfqpoint{3.202354in}{0.990085in}}%
\pgfpathlineto{\pgfqpoint{3.205178in}{0.988620in}}%
\pgfpathlineto{\pgfqpoint{3.206590in}{0.990533in}}%
\pgfpathlineto{\pgfqpoint{3.208001in}{0.990548in}}%
\pgfpathlineto{\pgfqpoint{3.210825in}{0.996155in}}%
\pgfpathlineto{\pgfqpoint{3.219295in}{1.025750in}}%
\pgfpathlineto{\pgfqpoint{3.226354in}{1.080842in}}%
\pgfpathlineto{\pgfqpoint{3.233413in}{1.114707in}}%
\pgfpathlineto{\pgfqpoint{3.239060in}{1.134093in}}%
\pgfpathlineto{\pgfqpoint{3.241883in}{1.141395in}}%
\pgfpathlineto{\pgfqpoint{3.243295in}{1.143180in}}%
\pgfpathlineto{\pgfqpoint{3.244707in}{1.149056in}}%
\pgfpathlineto{\pgfqpoint{3.250354in}{1.187345in}}%
\pgfpathlineto{\pgfqpoint{3.254589in}{1.197763in}}%
\pgfpathlineto{\pgfqpoint{3.256000in}{1.199752in}}%
\pgfpathlineto{\pgfqpoint{3.261647in}{1.192948in}}%
\pgfpathlineto{\pgfqpoint{3.263059in}{1.189770in}}%
\pgfpathlineto{\pgfqpoint{3.264471in}{1.190138in}}%
\pgfpathlineto{\pgfqpoint{3.267294in}{1.194517in}}%
\pgfpathlineto{\pgfqpoint{3.270118in}{1.194964in}}%
\pgfpathlineto{\pgfqpoint{3.275765in}{1.196920in}}%
\pgfpathlineto{\pgfqpoint{3.277177in}{1.198270in}}%
\pgfpathlineto{\pgfqpoint{3.278588in}{1.202260in}}%
\pgfpathlineto{\pgfqpoint{3.282823in}{1.203368in}}%
\pgfpathlineto{\pgfqpoint{3.285647in}{1.199511in}}%
\pgfpathlineto{\pgfqpoint{3.291294in}{1.163535in}}%
\pgfpathlineto{\pgfqpoint{3.294117in}{1.142085in}}%
\pgfpathlineto{\pgfqpoint{3.304000in}{1.116858in}}%
\pgfpathlineto{\pgfqpoint{3.306823in}{1.117336in}}%
\pgfpathlineto{\pgfqpoint{3.308235in}{1.115738in}}%
\pgfpathlineto{\pgfqpoint{3.315293in}{1.121690in}}%
\pgfpathlineto{\pgfqpoint{3.316705in}{1.123506in}}%
\pgfpathlineto{\pgfqpoint{3.322352in}{1.137283in}}%
\pgfpathlineto{\pgfqpoint{3.325176in}{1.138552in}}%
\pgfpathlineto{\pgfqpoint{3.327999in}{1.138932in}}%
\pgfpathlineto{\pgfqpoint{3.332234in}{1.131572in}}%
\pgfpathlineto{\pgfqpoint{3.335058in}{1.130269in}}%
\pgfpathlineto{\pgfqpoint{3.337881in}{1.134145in}}%
\pgfpathlineto{\pgfqpoint{3.342116in}{1.141356in}}%
\pgfpathlineto{\pgfqpoint{3.346352in}{1.146964in}}%
\pgfpathlineto{\pgfqpoint{3.350587in}{1.172263in}}%
\pgfpathlineto{\pgfqpoint{3.353410in}{1.179972in}}%
\pgfpathlineto{\pgfqpoint{3.356234in}{1.181854in}}%
\pgfpathlineto{\pgfqpoint{3.360469in}{1.168136in}}%
\pgfpathlineto{\pgfqpoint{3.361881in}{1.161529in}}%
\pgfpathlineto{\pgfqpoint{3.370351in}{1.174602in}}%
\pgfpathlineto{\pgfqpoint{3.373175in}{1.179580in}}%
\pgfpathlineto{\pgfqpoint{3.377410in}{1.184358in}}%
\pgfpathlineto{\pgfqpoint{3.381645in}{1.185265in}}%
\pgfpathlineto{\pgfqpoint{3.384469in}{1.182638in}}%
\pgfpathlineto{\pgfqpoint{3.385880in}{1.179661in}}%
\pgfpathlineto{\pgfqpoint{3.394351in}{1.137292in}}%
\pgfpathlineto{\pgfqpoint{3.397174in}{1.119428in}}%
\pgfpathlineto{\pgfqpoint{3.402821in}{1.112269in}}%
\pgfpathlineto{\pgfqpoint{3.404233in}{1.114075in}}%
\pgfpathlineto{\pgfqpoint{3.407056in}{1.113373in}}%
\pgfpathlineto{\pgfqpoint{3.412703in}{1.119538in}}%
\pgfpathlineto{\pgfqpoint{3.416939in}{1.113360in}}%
\pgfpathlineto{\pgfqpoint{3.419762in}{1.106542in}}%
\pgfpathlineto{\pgfqpoint{3.431056in}{1.145539in}}%
\pgfpathlineto{\pgfqpoint{3.435291in}{1.157971in}}%
\pgfpathlineto{\pgfqpoint{3.439526in}{1.164593in}}%
\pgfpathlineto{\pgfqpoint{3.442350in}{1.162397in}}%
\pgfpathlineto{\pgfqpoint{3.445173in}{1.158718in}}%
\pgfpathlineto{\pgfqpoint{3.446585in}{1.157049in}}%
\pgfpathlineto{\pgfqpoint{3.452232in}{1.177508in}}%
\pgfpathlineto{\pgfqpoint{3.455056in}{1.183165in}}%
\pgfpathlineto{\pgfqpoint{3.457879in}{1.186959in}}%
\pgfpathlineto{\pgfqpoint{3.463526in}{1.179894in}}%
\pgfpathlineto{\pgfqpoint{3.464938in}{1.182943in}}%
\pgfpathlineto{\pgfqpoint{3.470585in}{1.206364in}}%
\pgfpathlineto{\pgfqpoint{3.476232in}{1.219222in}}%
\pgfpathlineto{\pgfqpoint{3.479055in}{1.220580in}}%
\pgfpathlineto{\pgfqpoint{3.480467in}{1.221131in}}%
\pgfpathlineto{\pgfqpoint{3.484702in}{1.218251in}}%
\pgfpathlineto{\pgfqpoint{3.488937in}{1.210652in}}%
\pgfpathlineto{\pgfqpoint{3.495996in}{1.175328in}}%
\pgfpathlineto{\pgfqpoint{3.498819in}{1.158835in}}%
\pgfpathlineto{\pgfqpoint{3.501643in}{1.156915in}}%
\pgfpathlineto{\pgfqpoint{3.505878in}{1.149460in}}%
\pgfpathlineto{\pgfqpoint{3.511525in}{1.133277in}}%
\pgfpathlineto{\pgfqpoint{3.515760in}{1.117573in}}%
\pgfpathlineto{\pgfqpoint{3.521407in}{1.100146in}}%
\pgfpathlineto{\pgfqpoint{3.527054in}{1.108545in}}%
\pgfpathlineto{\pgfqpoint{3.531289in}{1.112017in}}%
\pgfpathlineto{\pgfqpoint{3.539760in}{1.108729in}}%
\pgfpathlineto{\pgfqpoint{3.542583in}{1.111108in}}%
\pgfpathlineto{\pgfqpoint{3.545407in}{1.113152in}}%
\pgfpathlineto{\pgfqpoint{3.548230in}{1.113836in}}%
\pgfpathlineto{\pgfqpoint{3.549642in}{1.113039in}}%
\pgfpathlineto{\pgfqpoint{3.551054in}{1.114642in}}%
\pgfpathlineto{\pgfqpoint{3.555289in}{1.132720in}}%
\pgfpathlineto{\pgfqpoint{3.558112in}{1.139307in}}%
\pgfpathlineto{\pgfqpoint{3.562348in}{1.142242in}}%
\pgfpathlineto{\pgfqpoint{3.567995in}{1.127951in}}%
\pgfpathlineto{\pgfqpoint{3.573641in}{1.137804in}}%
\pgfpathlineto{\pgfqpoint{3.580700in}{1.139216in}}%
\pgfpathlineto{\pgfqpoint{3.584935in}{1.134410in}}%
\pgfpathlineto{\pgfqpoint{3.587759in}{1.129353in}}%
\pgfpathlineto{\pgfqpoint{3.589171in}{1.126357in}}%
\pgfpathlineto{\pgfqpoint{3.600465in}{1.084126in}}%
\pgfpathlineto{\pgfqpoint{3.603288in}{1.078583in}}%
\pgfpathlineto{\pgfqpoint{3.606111in}{1.076478in}}%
\pgfpathlineto{\pgfqpoint{3.608935in}{1.077226in}}%
\pgfpathlineto{\pgfqpoint{3.610347in}{1.080269in}}%
\pgfpathlineto{\pgfqpoint{3.613170in}{1.079281in}}%
\pgfpathlineto{\pgfqpoint{3.617405in}{1.078057in}}%
\pgfpathlineto{\pgfqpoint{3.620229in}{1.071932in}}%
\pgfpathlineto{\pgfqpoint{3.625876in}{1.065505in}}%
\pgfpathlineto{\pgfqpoint{3.628699in}{1.062227in}}%
\pgfpathlineto{\pgfqpoint{3.630111in}{1.059811in}}%
\pgfpathlineto{\pgfqpoint{3.632934in}{1.061998in}}%
\pgfpathlineto{\pgfqpoint{3.645640in}{1.085368in}}%
\pgfpathlineto{\pgfqpoint{3.652699in}{1.107523in}}%
\pgfpathlineto{\pgfqpoint{3.654111in}{1.109754in}}%
\pgfpathlineto{\pgfqpoint{3.658346in}{1.140723in}}%
\pgfpathlineto{\pgfqpoint{3.661169in}{1.150748in}}%
\pgfpathlineto{\pgfqpoint{3.663993in}{1.152944in}}%
\pgfpathlineto{\pgfqpoint{3.665404in}{1.153138in}}%
\pgfpathlineto{\pgfqpoint{3.672463in}{1.135528in}}%
\pgfpathlineto{\pgfqpoint{3.675287in}{1.138718in}}%
\pgfpathlineto{\pgfqpoint{3.678110in}{1.136599in}}%
\pgfpathlineto{\pgfqpoint{3.682345in}{1.128264in}}%
\pgfpathlineto{\pgfqpoint{3.686581in}{1.119757in}}%
\pgfpathlineto{\pgfqpoint{3.690816in}{1.114707in}}%
\pgfpathlineto{\pgfqpoint{3.693639in}{1.112434in}}%
\pgfpathlineto{\pgfqpoint{3.696463in}{1.108419in}}%
\pgfpathlineto{\pgfqpoint{3.700698in}{1.095660in}}%
\pgfpathlineto{\pgfqpoint{3.704933in}{1.079548in}}%
\pgfpathlineto{\pgfqpoint{3.707757in}{1.076153in}}%
\pgfpathlineto{\pgfqpoint{3.713404in}{1.067488in}}%
\pgfpathlineto{\pgfqpoint{3.716227in}{1.063206in}}%
\pgfpathlineto{\pgfqpoint{3.719050in}{1.060866in}}%
\pgfpathlineto{\pgfqpoint{3.720462in}{1.060460in}}%
\pgfpathlineto{\pgfqpoint{3.728933in}{1.041612in}}%
\pgfpathlineto{\pgfqpoint{3.740227in}{1.060189in}}%
\pgfpathlineto{\pgfqpoint{3.757167in}{1.081091in}}%
\pgfpathlineto{\pgfqpoint{3.758579in}{1.080637in}}%
\pgfpathlineto{\pgfqpoint{3.761403in}{1.092820in}}%
\pgfpathlineto{\pgfqpoint{3.765638in}{1.108121in}}%
\pgfpathlineto{\pgfqpoint{3.769873in}{1.111239in}}%
\pgfpathlineto{\pgfqpoint{3.775520in}{1.096249in}}%
\pgfpathlineto{\pgfqpoint{3.778343in}{1.100214in}}%
\pgfpathlineto{\pgfqpoint{3.781167in}{1.100159in}}%
\pgfpathlineto{\pgfqpoint{3.785402in}{1.096309in}}%
\pgfpathlineto{\pgfqpoint{3.789637in}{1.092186in}}%
\pgfpathlineto{\pgfqpoint{3.796696in}{1.087268in}}%
\pgfpathlineto{\pgfqpoint{3.799520in}{1.082946in}}%
\pgfpathlineto{\pgfqpoint{3.800931in}{1.079635in}}%
\pgfpathlineto{\pgfqpoint{3.803755in}{1.081624in}}%
\pgfpathlineto{\pgfqpoint{3.806578in}{1.081808in}}%
\pgfpathlineto{\pgfqpoint{3.807990in}{1.081142in}}%
\pgfpathlineto{\pgfqpoint{3.809402in}{1.082938in}}%
\pgfpathlineto{\pgfqpoint{3.812225in}{1.078825in}}%
\pgfpathlineto{\pgfqpoint{3.815049in}{1.076747in}}%
\pgfpathlineto{\pgfqpoint{3.816460in}{1.076891in}}%
\pgfpathlineto{\pgfqpoint{3.817872in}{1.074377in}}%
\pgfpathlineto{\pgfqpoint{3.822107in}{1.074284in}}%
\pgfpathlineto{\pgfqpoint{3.824931in}{1.071900in}}%
\pgfpathlineto{\pgfqpoint{3.827754in}{1.067933in}}%
\pgfpathlineto{\pgfqpoint{3.839048in}{1.090371in}}%
\pgfpathlineto{\pgfqpoint{3.848930in}{1.125650in}}%
\pgfpathlineto{\pgfqpoint{3.853166in}{1.134498in}}%
\pgfpathlineto{\pgfqpoint{3.854577in}{1.134529in}}%
\pgfpathlineto{\pgfqpoint{3.860224in}{1.138658in}}%
\pgfpathlineto{\pgfqpoint{3.865871in}{1.172798in}}%
\pgfpathlineto{\pgfqpoint{3.868695in}{1.175612in}}%
\pgfpathlineto{\pgfqpoint{3.871518in}{1.175129in}}%
\pgfpathlineto{\pgfqpoint{3.877165in}{1.160090in}}%
\pgfpathlineto{\pgfqpoint{3.878577in}{1.161912in}}%
\pgfpathlineto{\pgfqpoint{3.881400in}{1.169152in}}%
\pgfpathlineto{\pgfqpoint{3.884224in}{1.169625in}}%
\pgfpathlineto{\pgfqpoint{3.885636in}{1.169595in}}%
\pgfpathlineto{\pgfqpoint{3.894106in}{1.152985in}}%
\pgfpathlineto{\pgfqpoint{3.896929in}{1.151265in}}%
\pgfpathlineto{\pgfqpoint{3.901165in}{1.151384in}}%
\pgfpathlineto{\pgfqpoint{3.902576in}{1.150580in}}%
\pgfpathlineto{\pgfqpoint{3.911047in}{1.123701in}}%
\pgfpathlineto{\pgfqpoint{3.915282in}{1.126878in}}%
\pgfpathlineto{\pgfqpoint{3.918106in}{1.127596in}}%
\pgfpathlineto{\pgfqpoint{3.919517in}{1.129632in}}%
\pgfpathlineto{\pgfqpoint{3.923752in}{1.127214in}}%
\pgfpathlineto{\pgfqpoint{3.925164in}{1.126266in}}%
\pgfpathlineto{\pgfqpoint{3.926576in}{1.123617in}}%
\pgfpathlineto{\pgfqpoint{3.932223in}{1.099490in}}%
\pgfpathlineto{\pgfqpoint{3.940693in}{1.106757in}}%
\pgfpathlineto{\pgfqpoint{3.946340in}{1.110549in}}%
\pgfpathlineto{\pgfqpoint{3.954811in}{1.109971in}}%
\pgfpathlineto{\pgfqpoint{3.959046in}{1.115458in}}%
\pgfpathlineto{\pgfqpoint{3.961869in}{1.120199in}}%
\pgfpathlineto{\pgfqpoint{3.966105in}{1.127123in}}%
\pgfpathlineto{\pgfqpoint{3.970340in}{1.156668in}}%
\pgfpathlineto{\pgfqpoint{3.973163in}{1.166806in}}%
\pgfpathlineto{\pgfqpoint{3.975987in}{1.170403in}}%
\pgfpathlineto{\pgfqpoint{3.981634in}{1.157238in}}%
\pgfpathlineto{\pgfqpoint{3.984457in}{1.162438in}}%
\pgfpathlineto{\pgfqpoint{3.987281in}{1.160962in}}%
\pgfpathlineto{\pgfqpoint{3.994339in}{1.150023in}}%
\pgfpathlineto{\pgfqpoint{3.997163in}{1.148201in}}%
\pgfpathlineto{\pgfqpoint{4.002810in}{1.142922in}}%
\pgfpathlineto{\pgfqpoint{4.005633in}{1.137009in}}%
\pgfpathlineto{\pgfqpoint{4.012692in}{1.116354in}}%
\pgfpathlineto{\pgfqpoint{4.014104in}{1.117288in}}%
\pgfpathlineto{\pgfqpoint{4.018339in}{1.119034in}}%
\pgfpathlineto{\pgfqpoint{4.021162in}{1.118296in}}%
\pgfpathlineto{\pgfqpoint{4.022574in}{1.118509in}}%
\pgfpathlineto{\pgfqpoint{4.026809in}{1.110601in}}%
\pgfpathlineto{\pgfqpoint{4.029633in}{1.103874in}}%
\pgfpathlineto{\pgfqpoint{4.031045in}{1.100394in}}%
\pgfpathlineto{\pgfqpoint{4.035280in}{1.076672in}}%
\pgfpathlineto{\pgfqpoint{4.036692in}{1.073474in}}%
\pgfpathlineto{\pgfqpoint{4.042338in}{1.086930in}}%
\pgfpathlineto{\pgfqpoint{4.047985in}{1.106350in}}%
\pgfpathlineto{\pgfqpoint{4.050809in}{1.116435in}}%
\pgfpathlineto{\pgfqpoint{4.053632in}{1.121290in}}%
\pgfpathlineto{\pgfqpoint{4.063515in}{1.132348in}}%
\pgfpathlineto{\pgfqpoint{4.066338in}{1.134088in}}%
\pgfpathlineto{\pgfqpoint{4.067750in}{1.133418in}}%
\pgfpathlineto{\pgfqpoint{4.069161in}{1.135962in}}%
\pgfpathlineto{\pgfqpoint{4.074808in}{1.162104in}}%
\pgfpathlineto{\pgfqpoint{4.079044in}{1.167236in}}%
\pgfpathlineto{\pgfqpoint{4.081867in}{1.161690in}}%
\pgfpathlineto{\pgfqpoint{4.083279in}{1.156821in}}%
\pgfpathlineto{\pgfqpoint{4.084691in}{1.156356in}}%
\pgfpathlineto{\pgfqpoint{4.087514in}{1.162696in}}%
\pgfpathlineto{\pgfqpoint{4.090338in}{1.160658in}}%
\pgfpathlineto{\pgfqpoint{4.108690in}{1.087402in}}%
\pgfpathlineto{\pgfqpoint{4.112925in}{1.084366in}}%
\pgfpathlineto{\pgfqpoint{4.117161in}{1.082679in}}%
\pgfpathlineto{\pgfqpoint{4.125631in}{1.053776in}}%
\pgfpathlineto{\pgfqpoint{4.129866in}{1.059292in}}%
\pgfpathlineto{\pgfqpoint{4.132690in}{1.060040in}}%
\pgfpathlineto{\pgfqpoint{4.135513in}{1.063348in}}%
\pgfpathlineto{\pgfqpoint{4.141160in}{1.061177in}}%
\pgfpathlineto{\pgfqpoint{4.146807in}{1.082640in}}%
\pgfpathlineto{\pgfqpoint{4.149631in}{1.087103in}}%
\pgfpathlineto{\pgfqpoint{4.155278in}{1.092717in}}%
\pgfpathlineto{\pgfqpoint{4.156689in}{1.092568in}}%
\pgfpathlineto{\pgfqpoint{4.160924in}{1.097884in}}%
\pgfpathlineto{\pgfqpoint{4.165160in}{1.111052in}}%
\pgfpathlineto{\pgfqpoint{4.169395in}{1.122619in}}%
\pgfpathlineto{\pgfqpoint{4.172218in}{1.125268in}}%
\pgfpathlineto{\pgfqpoint{4.173630in}{1.124287in}}%
\pgfpathlineto{\pgfqpoint{4.179277in}{1.154388in}}%
\pgfpathlineto{\pgfqpoint{4.184924in}{1.166822in}}%
\pgfpathlineto{\pgfqpoint{4.190571in}{1.164378in}}%
\pgfpathlineto{\pgfqpoint{4.191983in}{1.165673in}}%
\pgfpathlineto{\pgfqpoint{4.194806in}{1.175919in}}%
\pgfpathlineto{\pgfqpoint{4.197630in}{1.179613in}}%
\pgfpathlineto{\pgfqpoint{4.199041in}{1.180132in}}%
\pgfpathlineto{\pgfqpoint{4.203277in}{1.176351in}}%
\pgfpathlineto{\pgfqpoint{4.214571in}{1.161333in}}%
\pgfpathlineto{\pgfqpoint{4.215982in}{1.160886in}}%
\pgfpathlineto{\pgfqpoint{4.220217in}{1.161997in}}%
\pgfpathlineto{\pgfqpoint{4.224453in}{1.155560in}}%
\pgfpathlineto{\pgfqpoint{4.228688in}{1.143079in}}%
\pgfpathlineto{\pgfqpoint{4.232923in}{1.123205in}}%
\pgfpathlineto{\pgfqpoint{4.235747in}{1.102312in}}%
\pgfpathlineto{\pgfqpoint{4.241394in}{1.040148in}}%
\pgfpathlineto{\pgfqpoint{4.242805in}{1.037995in}}%
\pgfpathlineto{\pgfqpoint{4.244217in}{1.039905in}}%
\pgfpathlineto{\pgfqpoint{4.245629in}{1.045680in}}%
\pgfpathlineto{\pgfqpoint{4.255511in}{1.137787in}}%
\pgfpathlineto{\pgfqpoint{4.256923in}{1.140768in}}%
\pgfpathlineto{\pgfqpoint{4.261158in}{1.126325in}}%
\pgfpathlineto{\pgfqpoint{4.263981in}{1.119537in}}%
\pgfpathlineto{\pgfqpoint{4.265393in}{1.119473in}}%
\pgfpathlineto{\pgfqpoint{4.271040in}{1.127905in}}%
\pgfpathlineto{\pgfqpoint{4.272452in}{1.127626in}}%
\pgfpathlineto{\pgfqpoint{4.276687in}{1.135983in}}%
\pgfpathlineto{\pgfqpoint{4.283746in}{1.137539in}}%
\pgfpathlineto{\pgfqpoint{4.286569in}{1.142320in}}%
\pgfpathlineto{\pgfqpoint{4.290804in}{1.147559in}}%
\pgfpathlineto{\pgfqpoint{4.297863in}{1.135036in}}%
\pgfpathlineto{\pgfqpoint{4.300687in}{1.139095in}}%
\pgfpathlineto{\pgfqpoint{4.302098in}{1.139714in}}%
\pgfpathlineto{\pgfqpoint{4.304922in}{1.136850in}}%
\pgfpathlineto{\pgfqpoint{4.309157in}{1.126168in}}%
\pgfpathlineto{\pgfqpoint{4.317627in}{1.093207in}}%
\pgfpathlineto{\pgfqpoint{4.323274in}{1.066961in}}%
\pgfpathlineto{\pgfqpoint{4.326098in}{1.064150in}}%
\pgfpathlineto{\pgfqpoint{4.328921in}{1.062362in}}%
\pgfpathlineto{\pgfqpoint{4.330333in}{1.062948in}}%
\pgfpathlineto{\pgfqpoint{4.333156in}{1.057236in}}%
\pgfpathlineto{\pgfqpoint{4.337392in}{1.037429in}}%
\pgfpathlineto{\pgfqpoint{4.341627in}{1.010824in}}%
\pgfpathlineto{\pgfqpoint{4.343039in}{1.009366in}}%
\pgfpathlineto{\pgfqpoint{4.345862in}{1.014691in}}%
\pgfpathlineto{\pgfqpoint{4.351509in}{1.033938in}}%
\pgfpathlineto{\pgfqpoint{4.362803in}{1.079932in}}%
\pgfpathlineto{\pgfqpoint{4.376920in}{1.117309in}}%
\pgfpathlineto{\pgfqpoint{4.378332in}{1.118031in}}%
\pgfpathlineto{\pgfqpoint{4.383979in}{1.139536in}}%
\pgfpathlineto{\pgfqpoint{4.388214in}{1.146438in}}%
\pgfpathlineto{\pgfqpoint{4.393861in}{1.144759in}}%
\pgfpathlineto{\pgfqpoint{4.396685in}{1.142575in}}%
\pgfpathlineto{\pgfqpoint{4.403743in}{1.117430in}}%
\pgfpathlineto{\pgfqpoint{4.405155in}{1.116394in}}%
\pgfpathlineto{\pgfqpoint{4.407979in}{1.109333in}}%
\pgfpathlineto{\pgfqpoint{4.420684in}{1.064650in}}%
\pgfpathlineto{\pgfqpoint{4.426331in}{1.063324in}}%
\pgfpathlineto{\pgfqpoint{4.427743in}{1.062223in}}%
\pgfpathlineto{\pgfqpoint{4.429155in}{1.063524in}}%
\pgfpathlineto{\pgfqpoint{4.434802in}{1.054471in}}%
\pgfpathlineto{\pgfqpoint{4.437625in}{1.052017in}}%
\pgfpathlineto{\pgfqpoint{4.443272in}{1.054026in}}%
\pgfpathlineto{\pgfqpoint{4.446096in}{1.053137in}}%
\pgfpathlineto{\pgfqpoint{4.450331in}{1.044420in}}%
\pgfpathlineto{\pgfqpoint{4.455978in}{1.048391in}}%
\pgfpathlineto{\pgfqpoint{4.460213in}{1.050080in}}%
\pgfpathlineto{\pgfqpoint{4.461625in}{1.050614in}}%
\pgfpathlineto{\pgfqpoint{4.464448in}{1.055110in}}%
\pgfpathlineto{\pgfqpoint{4.474330in}{1.063634in}}%
\pgfpathlineto{\pgfqpoint{4.477154in}{1.061462in}}%
\pgfpathlineto{\pgfqpoint{4.481389in}{1.070776in}}%
\pgfpathlineto{\pgfqpoint{4.484212in}{1.073005in}}%
\pgfpathlineto{\pgfqpoint{4.501153in}{1.050004in}}%
\pgfpathlineto{\pgfqpoint{4.506800in}{1.045871in}}%
\pgfpathlineto{\pgfqpoint{4.511035in}{1.043037in}}%
\pgfpathlineto{\pgfqpoint{4.515271in}{1.037984in}}%
\pgfpathlineto{\pgfqpoint{4.519506in}{1.028575in}}%
\pgfpathlineto{\pgfqpoint{4.526565in}{1.004077in}}%
\pgfpathlineto{\pgfqpoint{4.533623in}{0.975973in}}%
\pgfpathlineto{\pgfqpoint{4.537858in}{0.951485in}}%
\pgfpathlineto{\pgfqpoint{4.553388in}{0.858318in}}%
\pgfpathlineto{\pgfqpoint{4.561858in}{0.820582in}}%
\pgfpathlineto{\pgfqpoint{4.568917in}{0.781526in}}%
\pgfpathlineto{\pgfqpoint{4.571740in}{0.771439in}}%
\pgfpathlineto{\pgfqpoint{4.574564in}{0.765895in}}%
\pgfpathlineto{\pgfqpoint{4.580211in}{0.747544in}}%
\pgfpathlineto{\pgfqpoint{4.590093in}{0.711757in}}%
\pgfpathlineto{\pgfqpoint{4.594328in}{0.705712in}}%
\pgfpathlineto{\pgfqpoint{4.597151in}{0.705599in}}%
\pgfpathlineto{\pgfqpoint{4.609857in}{0.711147in}}%
\pgfpathlineto{\pgfqpoint{4.616916in}{0.708906in}}%
\pgfpathlineto{\pgfqpoint{4.632445in}{0.702898in}}%
\pgfpathlineto{\pgfqpoint{4.643739in}{0.702186in}}%
\pgfpathlineto{\pgfqpoint{4.666327in}{0.701279in}}%
\pgfpathlineto{\pgfqpoint{4.686091in}{0.699833in}}%
\pgfpathlineto{\pgfqpoint{4.721384in}{0.697872in}}%
\pgfpathlineto{\pgfqpoint{4.743972in}{0.696621in}}%
\pgfpathlineto{\pgfqpoint{4.777854in}{0.696286in}}%
\pgfpathlineto{\pgfqpoint{4.858323in}{0.696340in}}%
\pgfpathlineto{\pgfqpoint{4.861146in}{0.696335in}}%
\pgfpathlineto{\pgfqpoint{4.862558in}{0.698607in}}%
\pgfpathlineto{\pgfqpoint{4.866793in}{0.698254in}}%
\pgfpathlineto{\pgfqpoint{4.869617in}{0.696597in}}%
\pgfpathlineto{\pgfqpoint{4.875264in}{0.696255in}}%
\pgfpathlineto{\pgfqpoint{4.880911in}{0.696305in}}%
\pgfpathlineto{\pgfqpoint{4.882323in}{0.698735in}}%
\pgfpathlineto{\pgfqpoint{4.885146in}{0.698779in}}%
\pgfpathlineto{\pgfqpoint{4.886558in}{0.696414in}}%
\pgfpathlineto{\pgfqpoint{4.920439in}{0.696384in}}%
\pgfpathlineto{\pgfqpoint{4.921851in}{0.701899in}}%
\pgfpathlineto{\pgfqpoint{4.943027in}{0.700966in}}%
\pgfpathlineto{\pgfqpoint{4.954321in}{0.701297in}}%
\pgfpathlineto{\pgfqpoint{4.962792in}{0.701974in}}%
\pgfpathlineto{\pgfqpoint{4.972674in}{0.700702in}}%
\pgfpathlineto{\pgfqpoint{4.991026in}{0.700806in}}%
\pgfpathlineto{\pgfqpoint{5.010791in}{0.700787in}}%
\pgfpathlineto{\pgfqpoint{5.023496in}{0.700516in}}%
\pgfpathlineto{\pgfqpoint{5.031967in}{0.700054in}}%
\pgfpathlineto{\pgfqpoint{5.033378in}{0.696329in}}%
\pgfpathlineto{\pgfqpoint{5.060202in}{0.696000in}}%
\pgfpathlineto{\pgfqpoint{5.519016in}{0.696809in}}%
\pgfpathlineto{\pgfqpoint{5.521840in}{0.696961in}}%
\pgfpathlineto{\pgfqpoint{5.528899in}{0.696308in}}%
\pgfpathlineto{\pgfqpoint{5.534545in}{0.696000in}}%
\pgfpathlineto{\pgfqpoint{5.534545in}{0.696000in}}%
\pgfusepath{stroke}%
\end{pgfscope}%
\begin{pgfscope}%
\pgfsetrectcap%
\pgfsetmiterjoin%
\pgfsetlinewidth{0.803000pt}%
\definecolor{currentstroke}{rgb}{0.000000,0.000000,0.000000}%
\pgfsetstrokecolor{currentstroke}%
\pgfsetdash{}{0pt}%
\pgfpathmoveto{\pgfqpoint{0.800000in}{0.528000in}}%
\pgfpathlineto{\pgfqpoint{0.800000in}{4.224000in}}%
\pgfusepath{stroke}%
\end{pgfscope}%
\begin{pgfscope}%
\pgfsetrectcap%
\pgfsetmiterjoin%
\pgfsetlinewidth{0.803000pt}%
\definecolor{currentstroke}{rgb}{0.000000,0.000000,0.000000}%
\pgfsetstrokecolor{currentstroke}%
\pgfsetdash{}{0pt}%
\pgfpathmoveto{\pgfqpoint{5.760000in}{0.528000in}}%
\pgfpathlineto{\pgfqpoint{5.760000in}{4.224000in}}%
\pgfusepath{stroke}%
\end{pgfscope}%
\begin{pgfscope}%
\pgfsetrectcap%
\pgfsetmiterjoin%
\pgfsetlinewidth{0.803000pt}%
\definecolor{currentstroke}{rgb}{0.000000,0.000000,0.000000}%
\pgfsetstrokecolor{currentstroke}%
\pgfsetdash{}{0pt}%
\pgfpathmoveto{\pgfqpoint{0.800000in}{0.528000in}}%
\pgfpathlineto{\pgfqpoint{5.760000in}{0.528000in}}%
\pgfusepath{stroke}%
\end{pgfscope}%
\begin{pgfscope}%
\pgfsetrectcap%
\pgfsetmiterjoin%
\pgfsetlinewidth{0.803000pt}%
\definecolor{currentstroke}{rgb}{0.000000,0.000000,0.000000}%
\pgfsetstrokecolor{currentstroke}%
\pgfsetdash{}{0pt}%
\pgfpathmoveto{\pgfqpoint{0.800000in}{4.224000in}}%
\pgfpathlineto{\pgfqpoint{5.760000in}{4.224000in}}%
\pgfusepath{stroke}%
\end{pgfscope}%
\begin{pgfscope}%
\pgfsetbuttcap%
\pgfsetmiterjoin%
\definecolor{currentfill}{rgb}{1.000000,1.000000,1.000000}%
\pgfsetfillcolor{currentfill}%
\pgfsetfillopacity{0.800000}%
\pgfsetlinewidth{1.003750pt}%
\definecolor{currentstroke}{rgb}{0.800000,0.800000,0.800000}%
\pgfsetstrokecolor{currentstroke}%
\pgfsetstrokeopacity{0.800000}%
\pgfsetdash{}{0pt}%
\pgfpathmoveto{\pgfqpoint{5.129968in}{3.093603in}}%
\pgfpathlineto{\pgfqpoint{5.662778in}{3.093603in}}%
\pgfpathquadraticcurveto{\pgfqpoint{5.690556in}{3.093603in}}{\pgfqpoint{5.690556in}{3.121381in}}%
\pgfpathlineto{\pgfqpoint{5.690556in}{4.126778in}}%
\pgfpathquadraticcurveto{\pgfqpoint{5.690556in}{4.154556in}}{\pgfqpoint{5.662778in}{4.154556in}}%
\pgfpathlineto{\pgfqpoint{5.129968in}{4.154556in}}%
\pgfpathquadraticcurveto{\pgfqpoint{5.102190in}{4.154556in}}{\pgfqpoint{5.102190in}{4.126778in}}%
\pgfpathlineto{\pgfqpoint{5.102190in}{3.121381in}}%
\pgfpathquadraticcurveto{\pgfqpoint{5.102190in}{3.093603in}}{\pgfqpoint{5.129968in}{3.093603in}}%
\pgfpathclose%
\pgfusepath{stroke,fill}%
\end{pgfscope}%
\begin{pgfscope}%
\pgfsetrectcap%
\pgfsetroundjoin%
\pgfsetlinewidth{1.505625pt}%
\definecolor{currentstroke}{rgb}{0.121569,0.466667,0.705882}%
\pgfsetstrokecolor{currentstroke}%
\pgfsetdash{}{0pt}%
\pgfpathmoveto{\pgfqpoint{5.157746in}{4.042088in}}%
\pgfpathlineto{\pgfqpoint{5.435524in}{4.042088in}}%
\pgfusepath{stroke}%
\end{pgfscope}%
\begin{pgfscope}%
\pgftext[x=5.546635in,y=3.993477in,left,base]{\sffamily\fontsize{10.000000}{12.000000}\selectfont 1}%
\end{pgfscope}%
\begin{pgfscope}%
\pgfsetrectcap%
\pgfsetroundjoin%
\pgfsetlinewidth{1.505625pt}%
\definecolor{currentstroke}{rgb}{1.000000,0.498039,0.054902}%
\pgfsetstrokecolor{currentstroke}%
\pgfsetdash{}{0pt}%
\pgfpathmoveto{\pgfqpoint{5.157746in}{3.838231in}}%
\pgfpathlineto{\pgfqpoint{5.435524in}{3.838231in}}%
\pgfusepath{stroke}%
\end{pgfscope}%
\begin{pgfscope}%
\pgftext[x=5.546635in,y=3.789620in,left,base]{\sffamily\fontsize{10.000000}{12.000000}\selectfont 2}%
\end{pgfscope}%
\begin{pgfscope}%
\pgfsetrectcap%
\pgfsetroundjoin%
\pgfsetlinewidth{1.505625pt}%
\definecolor{currentstroke}{rgb}{0.172549,0.627451,0.172549}%
\pgfsetstrokecolor{currentstroke}%
\pgfsetdash{}{0pt}%
\pgfpathmoveto{\pgfqpoint{5.157746in}{3.634374in}}%
\pgfpathlineto{\pgfqpoint{5.435524in}{3.634374in}}%
\pgfusepath{stroke}%
\end{pgfscope}%
\begin{pgfscope}%
\pgftext[x=5.546635in,y=3.585763in,left,base]{\sffamily\fontsize{10.000000}{12.000000}\selectfont 3}%
\end{pgfscope}%
\begin{pgfscope}%
\pgfsetrectcap%
\pgfsetroundjoin%
\pgfsetlinewidth{1.505625pt}%
\definecolor{currentstroke}{rgb}{0.839216,0.152941,0.156863}%
\pgfsetstrokecolor{currentstroke}%
\pgfsetdash{}{0pt}%
\pgfpathmoveto{\pgfqpoint{5.157746in}{3.430516in}}%
\pgfpathlineto{\pgfqpoint{5.435524in}{3.430516in}}%
\pgfusepath{stroke}%
\end{pgfscope}%
\begin{pgfscope}%
\pgftext[x=5.546635in,y=3.381905in,left,base]{\sffamily\fontsize{10.000000}{12.000000}\selectfont 4}%
\end{pgfscope}%
\begin{pgfscope}%
\pgfsetrectcap%
\pgfsetroundjoin%
\pgfsetlinewidth{1.505625pt}%
\definecolor{currentstroke}{rgb}{0.580392,0.403922,0.741176}%
\pgfsetstrokecolor{currentstroke}%
\pgfsetdash{}{0pt}%
\pgfpathmoveto{\pgfqpoint{5.157746in}{3.226659in}}%
\pgfpathlineto{\pgfqpoint{5.435524in}{3.226659in}}%
\pgfusepath{stroke}%
\end{pgfscope}%
\begin{pgfscope}%
\pgftext[x=5.546635in,y=3.178048in,left,base]{\sffamily\fontsize{10.000000}{12.000000}\selectfont 5}%
\end{pgfscope}%
\end{pgfpicture}%
\makeatother%
\endgroup%
}
        \label{fig:sub2}
    \end{subfigure}
    \caption{Amplitudenverlauf der ersten fünf Partialtöne}
    \label{fig:ampl}
\end{figure}

\subsection{}
Siehe Abbildung \ref{fig:freq}.

\begin{figure}[H]
    \centering
    \begin{subfigure}{.5\textwidth}
        \centering
        \caption{BuK\_04}
        \includegraphics[width=\linewidth]{Figures/Buk04_frequencies.pdf}
        \label{fig:sub1}
    \end{subfigure}%
    \begin{subfigure}{.5\textwidth}
        \centering
        \caption{BuK\_23}
        \includegraphics[width=\linewidth]{Figures/Buk23_frequencies.pdf}
        \label{fig:sub2}
    \end{subfigure}
    \caption{Frequenzverlauf der ersten fünf Partialtöne}
    \label{fig:freq}
\end{figure}


\section{Strikte Harmonische Sysnthese}
\label{sec:2}

\subsection{}


\subsection{}


\section{Freie Harmonische Sysnthese}
\label{sec:3}

\subsection{}

\subsection{}


\section{Berücksuchtugung der originalen Phasen}
\label{sec:4}